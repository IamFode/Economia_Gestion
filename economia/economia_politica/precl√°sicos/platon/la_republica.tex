\documentclass[10pt]{book}
\usepackage[text=17cm,left=4cm,right=4cm, headsep=20pt, top=2.5cm, bottom = 2cm,letterpaper,showframe = false]{geometry} %configuración página
\usepackage{latexsym,amsmath,amssymb,amsfonts} %(símbolos de la AMS).7
\parindent = 0cm  %sangria
\usepackage[T1]{fontenc} %acentos en español
\usepackage[spanish]{babel} %español capitulos y secciones
\usepackage{graphicx} %gráficos y figuras.

%---------------FORMATO de letra--------------------%

\usepackage{lmodern} % tipos de letras
\usepackage{titlesec} %formato de títulos
\usepackage[backref=page]{hyperref} %hipervinculos
\usepackage{multicol} %columnas
\usepackage{tcolorbox, empheq} %cajas
\usepackage{enumerate} %indice enumerado
\usepackage{marginnote}%notas en el margen
\tcbuselibrary{skins,breakable,listings,theorems}
\usepackage[Bjornstrup]{fncychap}%diseño de portada de capitulos
\usepackage[all]{xy}%flechas
\counterwithout{footnote}{chapter}
\usepackage{xcolor}
\usepackage[htt]{hyphenat}
%--------------------GRÀFICOS--------------------------

\usepackage{tkz-fct}

%---------------------------------

\titleformat*{\section}{\LARGE\bfseries\rmfamily}
\titleformat*{\subsection}{\Large\bfseries\rmfamily}
\titleformat*{\subsubsection}{\large\bfseries\rmfamily}
\titleformat*{\paragraph}{\normalsize\bfseries\rmfamily}
\titleformat*{\subparagraph}{\small\bfseries\rmfamily}

%------------------------------------------

\renewcommand{\labelenumi}{\Roman{enumi}.}%primer piso II) enumerate
\renewcommand{\labelenumii}{\arabic{enumii}$)$}%segundo piso 2)
\renewcommand{\labelenumiii}{\alph{enumiii}$)$}%tercer piso a)
\renewcommand{\labelenumiv}{$\bullet$}%cuarto piso (punto)

%----------Formato título de capítulos-------------

\usepackage{titlesec}
\renewcommand{\thechapter}{\arabic{chapter}}
\titleformat{\chapter}[display]
{\titlerule[2pt]
\vspace{4ex}\bfseries\sffamily\huge}
{\filleft\Huge\thechapter}
{2ex}
{\filleft}

\begin{document}

\normalfont
\input xy
\xyoption{all}
\author{\Large Apuntes por FODE}
\title{\small Platón \\ \vspace{1cm} \large LA REPÚBLICA}
\date{}
\pagestyle{empty}
\maketitle
\thispagestyle{empty}
\let\cleardoublepage\clearpage
\tableofcontents								%indice


%------------------------------------------
 
\let\cleardoublepage\clearpage

\chapter*{Libro Primero}
\addcontentsline{toc}{chapter}{Libro Primero}
\section*{Sócrates}
\addcontentsline{toc}{section}{Sócrates}
Céfalo dijo a Sócrates:  debes saber, que a medida que los placeres del cuerpo me abandonan, encuentro mayor encanto en la conversación.\\
La vejez, en efecto, es un estado de reposo y de libertad respecto de los sentidos. \\
si hemos de atenernos a lo que se ha dicho más arriba, la justicia hace bien a sus amigos y mal a sus enemigos. \\
El hombre justo, ¿en qué y en qué ocasión puede hacer mayor bien a sus amigos y mayor mal a sus enemigos? En la guerra, a mi parecer, atacando a los unos y defendiendo a los otros. \\
\textbf{Dime ahora en qué es útil la justicia durante la paz. Es útil en el comercio.}\\
\textbf{Trasimaco} dijo: Digo que la justicia no es otra cosa que lo que es provechoso al más fuerte. ¿No sabes que los diferentes Estados son monárquicos ó aristocráticos, ó populares? —Lo sé. —El que gobierna en cada Estado, ¿no es el más fuerte? —Seguramente. — ¿No hace leyes cada uno de ellos en ventaja suya, el pueblo leyes populares, el monarca leyes monárquicas, y asilos demás? Una vez hechas estas leyes, ¿no declaran que la justicia para los gobernados consiste en la observancia de las mismas? ¿No se castiga a los que las traspasan , como culpables de una acción injusta? Aquí tienes mi pensamiento. Encada Estado, la justicia no es más que la utilidad del que tiene la autoridad en sus manos, y, por consiguiente, del más fuerte. De donde se sigue para todo hombre que sabe discurrir, que la justicia y lo que es ventajoso al más fuerte en todas partes y siempre es una misma cosa.\\
En las transacciones y negocios particulares, hallarás siempre, que el injusto gana en el trato y que el hombre justo pierde. En los negocios públicos, si las necesidades del Estado exigen algunas contribuciones, el justo con fortuna igual suministrará más que el injusto. Tan cierto es, Sócrates, que la injusticia, cuando se la lleva hasta cierto punto, es más fuerte, más libre, más poderosa que la justicia; y que, como dije al principio, la justicia es el interés del más fuerte, y la injusticia es por sí misma útil y provechosa. 
En respuesta Sócrates menciona: Para comprometer a los hombres a que ejerzan el mando, a sido preciso proponerles alguna recompensa, como dinero, ó un castigo si rehúsan aceptarlo. \\
\textbf{Los sabios no quieren tomar parte en los negocios con ánimo de enriquecerse, porque temerían que se les mirara como mercenarios, si exigían manifiestamente algún salario por el mando, ó como ladrones si convertían los fondos públicos en su provecho.} Tampoco tienen en cuéntalos honores, porque no son ambiciosos.  Es preciso, pues, que algún motivo muy poderoso les obligue a tomar parte en el gobierno, como el temor de algún castigo. Y por esta razón se mira como cosa poco delicada el encargarse voluntariamente de la administración pública, sin verse comprometido a ello. Porque el mayor castigo para el hombre de bien, cuando rehúsa gobernar a los demás, es el verse gobernado por otro menos digno; y este temor es el que obliga a los sabios a encargarse del gobierno , no por su interés ni por su gusto, sino por verse precisados a ello a falta de otros, tanto ó más dignos de gobernar\\
Sócrates menciona a esto que es tal, pues, la naturaleza de la injusticia, ya se encuentre en un Estado, ya en un ejército ó en cualquiera otra sociedad, que, en primer lugar, la hace absolutamente  impotente para emprender nada a causa de las querellas y sediciones que provoca; y, en segundo lugar, se hace enemiga de sí misma y de todos los que son a ella contrarios, es decir, de los hombres de bien.\\ \textbf{La justicia es mucho mas ventajosa que la injusticia.}

\chapter*{Libro Segundo}
\addcontentsline{toc}{chapter}{Libro Segundo}
Glaucon menciona que la injusticia se trata de someterse a una necesidad y no tanto como un bien. \\
Se dice que es un bien en sí cometer la injusticia y un mal el padecerla. Se ha llegado a amar la justicia, no porque sea un bien en sí misma, sino en razón de la imposibilidad en que nos coloca de dañar a los demás. Hagamos una suposición. Demos al hombre de bien y al hombre malo un poder igual para hacer todo lo que quieran; sigámoslos, y veamos a dónde conduce la pasión al uno y al otro. No tardaremos en sorprender al hombre de bien, siguiendo los pasos del hombre malo. \\
Gijes después de extraer un anillo de Oro de las entrañas de un caballo se lo puso y dado la vuelta el anillo se hizo invisible, esta ventaja hace que llegue al palacio y se apodera el trono. Ahora si hubiera dos anillos y se diera al Malo uno y al Bueno otro, no se corremperían los dos al lado de la injusticia?. \textbf{El hombre se hace injusto tan pronto como cree poderlo ser sin temor}. Este hombre será elogiado de su virtud, pero con intención de engañarse mutuamente y por el temor de experimentar ellos mismos alguna injusticia.\\
\textbf{El gran mérito de la injusticia consiste en parecer justo sin serlo.} \\
Pindaro menciona: Subiré con trabajo al palacio, que habita la justicia, O marcharé por el torcido sendero del fraude, Para asegurar la felicidad de mi vida?\\
Todo lo que oigo me hace creer que de nada me servirá ser justo, si no adquiero la reputación de tal, y que  la virtud no tiene más que trabajos y penalidades que ofrecerme. \\
\textbf{El primero que consigue el poder de hacer mal, es el primero también en servirse de él hasta donde le es posible.} \\
¿ Cómo podremos encontrar en él la justicia y la injusticia? ¿Y dónde crees que tienen su origen en medio de todos estos diversos elementos? Dice Aristóteles. Si tenemos un estado y en ella hay muchas personas con distintos oficios, entre ellos la de los guardias que se rigen por su ferocidad, donde es poco probable que sea dulce y a la vez inclinado a la cólera. Por eso un niño que se forma sobre algún propósito oigan discursos que le conduzcan a la virtud. Por lo tanto se debe enseñar teorías a los jóvenes que sean dogmas en sus vidas.\\
nuestra segunda ley, que prohíbe hablar y escribir, respecto a los dioses, como si fueran encantadores, que toman diferentes formas y que intentan engañarnos con sus discursos y sus acciones? —La apruebo.

\chapter*{Libro Tercero}
\addcontentsline{toc}{chapter}{Libro Tercero}
\textbf{nuestro deber es estar muy en guardia respecto a los discursos que tengan esta tendencia, y recomendar a los poetas que conviertan en elogios todo lo malo que dicen ordinariamente de los infiernos, con tanto más motivo cuanto que lo que refieren ni es verdadero, ni propio para inspirar confianza a los guerreros.}\\
Sólo a los magistrados supremos pertenece el poder mentir, a fin de engañar al enemigo ó a los ciudadanos para bien de la república.La mentira no debe nunca permitirse a los demás hombres.\\
Guardémonos también de creer y de permitir que se diga, que Teseo, hijo de Neptuno, y Piritoo, hijo de Júpiter, hayan intentado el robo sacrilegio que se les atribuye (5), ni que ninguno otro hijo dé los dioses, ningún héroe se haya hecho culpable de las crueldades y de las impiedades de que les acusan falsamente los poetas.Obliguemos a estos a reconocer, que los héroes nunca han cometido semejantes acciones; ó que, si las han cometido, ya no son descendientes de los dioses.\\
 hay tres clases de narraciones. La primera es imitativa, y, como acabas de decir, pertenece a la tragedia y a la comedia. La segunda se hace en nombre del poeta; y la verás empleada en los ditirambos. La tercera es una mezcla de una y otra; y nos servimos de ella en la epopeya y en otras cosas.\\
 \textbf{ Lo mismo sucede con respecto a la imitación. Un hombre solo no puede imitar muchas cosas lo mismo que una sola.}\\
Nos resta hablar de esta otra parte de la música, que corresponde al canto y a la melodía. \textbf{la melodía se compone de tres elementos: las palabras, la armonía y el número.} Para los guerreros necesitamos la dórica y la frigia, que expresan el carácter de un hombre sabio y valiente. El número y la armonía están hechas para las palabras, y no las palabras para el número y la armonía. \\\\
 ¿No es por esta misma razón, mi querido Glaucon, la música la parte principal de la educación, porque insinuándose desde muy temprano en el alma, el número y la armonía se apoderan de ella, y consiguen que la gracia y lo bello entren como un resultado necesario en ella, siempre que se dé esta parte de educación como conviene darla, puesto que sucede todo lo contrario, cuando se la desatiende? Y también , porque educado un joven, cual conviene, en la música, advertirá con la mayor exactitud lo que vaya de imperfecto y de defectuoso en las obras de la naturaleza y del arte, y experimentará a su vista una impresión justa y penosa; alabará por la misma razón con entusiasmo la belleza que observe, la dará entrada en su alma, se alimentará con ella, y se formará por este medio en la virtud; mientras que en el caso opuesto mirará con desprecio y con una aversión natural lo que encuentre de vicioso; y como esto sucederá desde la edad más tierna, antes de que le ilumine la luz de la razón, apenas baya esta aparecido, invadirá su alma, y él se unirá con ella mediante la relación secreta que la música habrá creado de antemano entre la razón y él. Eh aquí, a mi parecer, las ventajas que se buscan al educar a los niños en la música. \\
Es preciso, por el contrario, que su alma sea pura, exenta de vicio, para que su bondad le haga discernir más seguramente lo que es justo.\\
Es evidente, que nuestros jóvenes, educados en los principios de esta sencilla música que hace nacer en el alma la templanza, obrarán de manera que no tendrán necesidad de los jueces. \\
cuando se hace demasiado tirante, degenera en dureza y brutalidad.\\
 Los dioses han hecho a los hombres el presente de la música y de la gimnasia, no con objeto de cultivar el alma y el cuerpo (porque si este último saca alguna ventaja, es sólo indirectamente), sino para cultivar el alma sola, y perfeccionar en ella la sabiduría y el valor , concertándolos, ya dándolos expansión, ya conteniéndolos dentro de justos límites.\\
Entre los ancianos deben escogerse los mejores. Para esto es preciso que además de la prudencia y de la energía necesaria, tengan mucho celo por el bien público. Escojamos, pues, entre todos los guardadores, aquellos que, previo un maduro examen, nos parezca que después de haber pasado toda su vida consagrados a procurar el bien público, nunca han perjudicado los intereses del Estado. \\
Que se les haga entender, que los dioses han puesto en su alma oro y plata divina, y, por consiguiente, que no tienen necesidad del oro y de la plata de los hombres; que no les es permitido manchar
la posesión de este oro inmortal con la liga del oro terrestre; que el oro, que ellos tienen, es puro, mientras que el de los hombres ha sido en todos tiempos origen de muchos crímenes; que igualmente son ellos los únicos, entre los demás ciudadadanos, a quienes está prohibido manejar y hasta tocar el oro y la plata.guardarlo para sí, adornar con ello sus vestidos, beber en copas de estos metales, y que este es el único medio de conservación así para ellos como para el Estado. Porque desde el momento en que se hicieran propietarios de tierras, de casas y de dinero, de guardadores que eran se convertirían en empresarios y labradores, y de defensores del Estado se convertirían en sus enemigos y sus tiranos; pasarían la vida aborreciéndose mutuamente y armándose lazos unos a otros; entonces, los enemigos, que más deben de temerse, son los de dentro, y la república y ellos mismos correrán rápidamente hacia su ruina.

\chapter*{Libro Cuarto}
\addcontentsline{toc}{chapter}{Libro Cuarto} ¿Qué responderás, Sócrates, si seta objeta, que tus guerreros no son muy dichosos, y esto por falta suya, pues son realmente dueños del Estado, y sin embargo están privados de todas las ventajas de la sociedad, no poseyendo como los demás, ni tierras, ni casas grandes, bellas y bien amuebladas; no pudiendo ni sacrificar a los dioses en una habitación doméstica, ni tener donde recibir huéspedes, ni poseer oro y plata, y en fin, nada de lo que, en opinión de los hombres, sirve para hacer una vida cómoda y agradable? En verdad se dirá, que los tratas como a extranjeros, que están a sueldo del Estado sin otro destino que el de guardarle.\\ Añade, le dije yo, que su sueldo sólo consiste en el alimento, y además de esto que no tienen paga como las tropas ordinarias, y por lo tanto, que no pueden ni salir de los límites del Estado, ni viajar, ni regalar a libertinas, ni disponer de nada a su gusto, como hacen los ricos y los que presumen de dichosos. \\
\textbf{nuestra tarea consiste en fundar un gobierno dichoso, a nuestro parecer por lo menos, un Estado, en el que la felicidad no sea patrimonio de un pequeño número de particulares, sino común a toda la sociedad.}\\
El alfarero, si se hace rico, ¿se ocupará mucho de su oficio? - No . — Se hará, por lo tanto, cada día más holgazán y más negligente. — Sin duda. \\
Las riquezas y la pobreza dañan igualmente a las artes y a los que las ejercen.\\
\textbf{La opulencia y la pobreza, porque la una engendra la molicie, la holgazanería y el amor a las novedades; y la otra este mismo amor a las novedades, la bajeza y el deseo de hacer mal.}\\
el Estado no parezca grande ni pequeño, sino que debe permanecer en un justo medio y siempre uno. \\
Quisimos por este medio hacerles entender, que cada ciudadano sólo debe aplicarse a una cosa, aquella para la que ha nacido, a fin de que cada particular, ajustándose a la profesión que le conviene, sea uno; para que el Estado sea también uno, y no haya ni muchos ciudadanos en un solo ciudadano, ni muchos Estados en un solo Estado.\\
Interesa solamente observar un punto, el único importante, ó más bien el único preciso. —¿Cuál es? —La educación de la juventud y de la infancia. Si nuestros ciudadanos son bien educados y se hacen hombres en regla, verán por sí mismos fácilmente la importancia de todos estos puntos y de muchos otros que omitimos aquí, como todo lo relativo a las mujeres, al matrimonio y a la procreación de los hijos; y verán, digo, que según el proverbio, todas las cosas deben de ser comunes entre los amigos. \\
\textbf{En un Estado todo depende de los principios.}\\
Una buena educación forma un buen carácter; los hijos siguiendo desde luego los pasos de sus padres, se hacen bien pronto mejores" que los que les han precedido, y tienen, entre otras ventajas, la de dar a luz hijos que les superan a ellos mismos en mérito, como sucede con los animales. \\
los que hayan de estar a la cabeza de nuestro Estado vigilarán especialmente para que la educación se mantenga pura.\\
\textbf{como dice Damon, y yo soy en esto de su dictamen, no se puede tocar a las reglas de la música sin conmover las leyes fundamentales del gobierno. }\\
\textbf{—Mientras que si los juegos de los niños se someten a regla desde el principio; si el amor al orden entra en su corazón con la música, sucederá, por un efecto contrario, que todo irá de mejor en mejor, de suerte que si la disciplina se relajase en algún punto, ellos mismos la repararían un día. }\\ mí querido Adimanto, que todas estas prácticas son un resultado natural de la educación, porque lo semejante ¿no atrae siempre a su semejante?\\
No te irrites contra nuestros políticos; son las gentes más divertidas del mundo con sus reglamentos, que modifican sin cesar, persuadidos de que remediarán así los abusos que se infiltran en las relaciones de la vida sobre todos los puntos de que he hablado. No pueden imaginarse que realmente no hacen más que cortar las cabezas de una hidra.\\
¿Hay en el Estado, que acabamos de formar, una ciencia que resida en algunos de sus miembros y cuyo fin es deliberar, no sobre alguna parte del Estado, sino sobre el Estado todo y sobre su gobierno, tanto interior como exterior? —Sin duda, la bay. —¿Qué ciencia es ésta y en quién reside? —Es la que tiene por objeto la conservación del Estado, y reside en aquellos magistrados que están encargados de su guarda. \\
 La templanza no es otra cosa que un cierto orden, un freno que el hombre pone a sus placeres y a sus pasiones. De aquí viene probablemente esta expresión, que no entiendo bastante bien: ser dueño de sí mismo.\\
Pero con respecto a los sentimientos sencillos y moderados, fundados sobre opiniones exactas y gobernados por la razón, sólo se encuentran en un pequeño número de personas, que unen a un excelente natural una excelente educación. \\
\textbf{no contribuye menos a la perfección de la sociedad civil, que la prudencia, la fortaleza y la templanza.}\\
\textbf{la justicia asegura a cada uno la posesión de lo que le pertenece y el ejercicio libre del  empleo que le conviene. }\\
\textbf{Luego el hombre justo, en tanto que justo, no se diferenciará en nada de un Estado justo, sino que será perfectamente semejante a él.}\\
El deseo de sed causa un efecto de buscar una bebida con todas sus fuerzas.
\textbf{lo que hace al Estado justo, hace igualmente justo al particular. } \\
En caso de un ataque exterior, tomarán las mejores medidas para la seguridad del alma y del cuerpo. La razón deliberará; la cólera combatirá, y secundada por el valor, ejecutará las órdenes de la razón. \\
La justicia no permitirá que ninguna de las partes del alma haga otra cosa que lo que le concierne y prohibiendo que las unas se entrometan en las funciones de las otras.\\
\textbf{ Si gobierna uno solo, se dará al gobierno el nombre de monarquía; y si la autoridad se divide entre muchos, se llamará aristocracia. }

\chapter*{Libro Quinto}
\addcontentsline{toc}{chapter}{Libro Quinto}
considero como un crimen menor matar a uno sin quererlo, que engañarle sobre lo bello, lo bueno, lo justo y las leyes. , si pedimos a las mujeres los mismos servicios que a los hombres, es preciso darles la misma educación. \\
hay y habrá siempre razón para decir que lo útil es bello, y que sólo es feo lo que es dañoso. \\
la mentira es útil cuando nos servimos de ella como de un remedio. \\
\textbf{El respeto y el temor son dos barreras poderosas}. La enemistad entre allegados se llama discordia; entre extraños, se llama guerra. \\
¿Quiénes son, en tu opinión, los verdaderos filósofos? — Los que gustan de contemplar la verdad. \\
Los que contemplan la esencia inmutable de las cosas tienen conocimientos y no opiniones.

\chapter*{Libro Sexto}
\addcontentsline{toc}{chapter}{Libro Sexto}
Es preciso ver a quiénes hemos de escoger para gobernar nuestro Estado. — ¿Y cuál será el mejor camino que para ello debamos tomar? — Designar para magistrados a los que nos parezcan más a propósito para mantener las leyes y las instituciones en todo su vigor. \\
Convengamos, por lo pronto, en que el primer signo del espíritu filosófico es amar con pasión la ciencia. Luego El horror a la mentira, a la que negarán toda entrada en el alma, al paso que habrán de tener un amor igual por la verdad\\
\textbf{el espíritu, verdaderamente ávido de ciencia, debe desde la primera juventud amar y buscar la verdad.}\\
Un hombre de tales condiciones es templado y enteramente extraño a la concupiscencia, porque las razones que obligan a los demás acorrer con tanto ardor tras las riquezas, no tienen ninguna influencia sobre él.\\  
A tales hombres, perfeccionados por la educación y por la experiencia, y sólo a ellos deberás confiar el gobierno del Estado. 
\textbf{lo natural es que el que tiene necesidad de ser gobernado vaya en busca del que puede gobernarle, y no que aquellos, cuyo gobierno pueda ser útil a los demás, supliquen a estos que se pongan en sus manos}\\
Antes bien a la verdad van unidas siempre costumbres puras y arregladas, siendo la templanza su compañera. \\
formase una ciencia que se pusiese a enseñar sin servirse por otra parte de ninguna regla segura para discernir lo que en estos hábitos y apetitos es honesto, bueno y justo, de lo que es vergonzoso, malo a injusto. Conformándose en sus juicios con el instinto del animal, llamando bien a todo lo que le halaga y le causa placer, mal a todo lo que le irrita; justo y bello a todo lo que satisface las necesidades de la naturaleza; sin hacer otra distinción.\\
\textbf{hombres así pervertidos son los que causan los mayores males al Estado y a los particulares, y los que, por el contrario, cuando cambian de dirección en buen sentido, producen los mayores bienes.}\\
 Si se encuentra con un gobierno, cuya perfección corresponda a la suya, entonces se verá que encierra verdaderamente en sí algo divino, mientras todos los demás caracteres y todas las demás profesiones sólo participan de lo humano. \\
Es preciso que los niños y los jóvenes se dediquen a los estudios propios de su edad, y que en este período de la vida, en que crece y se fortifica el cuerpo, se tenga un cuidado particular del mismo, a fin de que pueda en su día auxiliar mejor al espíritu en sus trabajos filosóficos. Con el tiempo y a medida que el espíritu se forma y se madura, se reforzarán los ejercicios a que haya de sujetársele. Y cuando gastadas las fuerzas, no les sea posible ir a la guerra, ni ocuparse de los negocios del Estado, entonces se les permitirá consagrarse por entero a la filosofía.\\
\textbf{el que mira como su único estudio la contemplación de la verdad, no tiene tiempo para hacer descender sus miradas sobre la conducta de los hombres para censurarla.}\\
Los magistrados mostrar un gran celo por el bien público, y que este celo debía probarse en medio del placer ó del dolor, de tal manera que ni los trabajos, ni el Platón, ni ninguna otra situación crítica les hiciese perder de vista esta máxima: que era preciso desechar aquel que hubiera sucumbido en estas pruebas, y escoger por magistrado al que saliera tan puro como el oro pasado por el fuego, colmándole de honores y de distinciones durante su vida y después de su muerte.\\
Pienso efectivamente que no será un seguro guardador de lo justo y de lo honesto el que no conozca las relaciones que mantienen con el bien; esto en el supuesto que pueda conocerse lo bello y lo justo sin conocer previamente el bien, lo cual me atrevo a negar. \\
Cuando fija sus miradas en objetos iluminados por la verdad y por el ser, los ve claramente, los conoce y muestra que está dotada de inteligencia; pero cuando vuelve sus miradas sobre lo que está envuelto en tinieblas, sobre lo que nace y perece, su vista se turba, se oscurece, y ya no tiene más que opiniones, que mudan a cada momento; en una palabra, parece completamente privada de inteligencia. \\

\chapter*{Libro Septimo}
\addcontentsline{toc}{chapter}{Libro Séptimo}
 No se trata de darle la facultad de ver, porque ya la tiene; sino que lo que sucede es que su órgano está mal dirigido y no mira a donde debía mirar, y esto es precisamente lo que debe corregirse.\\
 Vuelves, mi querido amigo, a olvidar que el legislador no debe proponerse por objeto la felicidad de una determinada clase de ciudadanos con exclusión de las demás, sino la felicidad de todos; que a este fin debe unir a todos los ciudadanos en los mismos intereses, comprometiéndolos por medio de la persuasión ó de la autoridad a que se comuniquen unos a otros todas las ventajas que están en posición de procurar a la comunidad; y que al formar con cuidado semejantes ciudadanos, no pretende dejarlos libres para que hagan de sus facultades el uso que les acomode, sino servirse de ellos con el fin de fortificar los lazos del Estado. \\
Si puedes encontrar para los que deben obtener el mando una condición que ellos prefieran al mando mismo, también podrás encontrar una república bien ordenada, porque en tal Estado sólo mandarán los que son verdaderamente ricos, no en oro, sino en sabiduría y en virtud, riquezas que constituyen la verdadera felicidad. Pero donde quiera que hombres pobres, hambrientos de bien, y que no tienen nada por sí mismos, aspiren al mando, creyendo encontrar en él la felicidad que buscan, el gobierno será siempre malo, se disputará y se usurpará la autoridad, y esta guerra doméstica a intestina arruinará al fin al Estado y a sus jefes. \\
Entiendo por objetos que no invitan al alma a la reflexión, aquellos que no excitan al mismo tiempo dos sensaciones contrarias; y por objetos que invitan al alma a reflexionar, entiendo aquellos que dan origen a dos sensaciones contrarias, cuando los sentidos no se dan cuenta de que sea tal cosa ó tal otra opuesta, ya hiera el objeto los sentidos de cerca ó de lejos.\\
Demos por lo tanto una ley a los que hemos destinado en nuestro plan a ocupar los primeros puestos, para
que se consagren a la ciencia del cálculo, para que la estudien, no superficialmente, sino hasta que por medio de la pura inteligencia hayan llegado a conocer la esencia de los números, no para servirse de esta ciencia en las compras y ventas, como hacen los mercaderes y negociantes, sino para aplicarla a las necesidades de la guerra y facilitar al alma el camino que debe conducirla desde la esfera de las cosas perecibles hasta la contemplación de la verdad y del ser. \\
Por consiguiente, la geometría atrae al alma hacia la verdad, forma en ella el espíritu filosófico, obligándola a dirigir a lo alto sus miradas, en lugar de abatirlas, como suele hacerse, sobre las cosas de este mundo. \\
los pitagóricos dicen, que estas dos ciencias, la astronomía y la música, son hermanas, y nosotros
somos de su opinión.\\
El estudio de las ciencias de que hemos hablado, produce el mismo efecto. Eleva la parte más noble del alma hasta la contemplación del más excelente de los seres.\\
\textbf{continuemos llamando ciencia a la primera y más perfecta manera de conocer; conocimiento razonado a la segunda; fe a la tercera; conjetura a la cuarta; comprendiendo las dos últimas bajo el nombre de opinión, y las dos primeras bajo el de inteligencia}.\\
de suerte que lo perecedero sea el objeto de la opinión, y lo permanente el de la inteligencia; y que la inteligencia sea a la opinión, la ciencia a la fe, el conocimiento razonado a la conjetura, lo que la esencia es a lo perecedero. \\
Tomemos, pues, todas las precauciones para hacer una buena elección, porque si sólo dedicamos a los estudios y ejercicios de esta importancia a personas a quienes nada falte ni con relación al cuerpo ni con relación al espíritu, la misma justicia nada-tendrá que echarnos en cara, y nuestro Estado y nuestras leyes se mantendrán firmes; pero si dedicamos a estos trabajos personas indignas, sucederá todo lo contrario, y pondremos en completo ridículo a la filosofía. \\
\textbf{ Cuando hayan concluido su curso de ejercicios gimnásticos, porque durante este tiempo, que será de dos a tres años, les es imposible dedicarse a otra cosa, porque no hay nada más enemigo de las ciencias que la fatiga y el sueño. }\\
El desorden reina en la  dialéctica\\
— Acabas, Sócrates, de fabricar, como un hábil escultor, perfectos hombres de Estado. — Di también mujeres, mi querido Glaucon; porque no creas que haya hablado yo más bien de hombres que de mujeres, siempre que estén dotadas de una aptitud conveniente. — Así debe ser, puesto que en nuestro sistema es preciso que todo sea común entre los dos sexos. \\
Y bien, amigos míos, ¿me concederéis ahora que nuestro proyecto de Estado y de gobierno no es un simple deseo? La ejecución es difícil sin duda, pero es posible; y sólo lo es, como se ha dicho, cuando estén a la cabeza de los gobiernos uno ó muchos verdaderos filósofos, que, mirando con desprecio los honores, que hoy con tanto ardor se solicitan, en la convicción de que no tienen ningún valor; no estimando sino el deber y los honores que son su recompensa; poniendo la justicia por encima de todo por su importancia y su necesidad; sometidos en todo a sus leyes y esforzándose en hacerlas prevalecer, emprendan la reforma del Estado. 

\chapter*{Libro Octavo}
\addcontentsline{toc}{chapter}{Libro Octavo}
Es,  pues,  cosa  reconocida  por  nosotros,  mi  querido  Glaucon,  que en un Estado  bien constituido  todo debe ser común,  mujeres,  bijos,  educación,   ejercicios  propios  de  la  paz  y  de la guerra,  y  que  deben  designarse  por jefes del  mismo  á  hombres  consumados  en  la filosofía y  en  la  ciencia  militar. \\
\textbf{Recuerdo que nos  pareció  conveniente  que  ninguno  de  ellos fuese  propietario  de nada,  que  es lo contrario  de  lo que  sucede actualmente  con los guerreros;  y que considerándose  como atletas destinados a combatir y vigilar por el bien público,  debían  proveer  a su  seguridad  y a la  de  sus conciudadanos,  y  recibir  de los  demás  en  recompensa  de  sus  servicios  lo  que  necesitaran  cada  año  para   su   manutención.}\\
Es  difícil que  la  constitución  de  un  Estado  como  el  vuestro,  se  altere;  pero  como  todo   lo  que  nace  está  destinado  a perecer,  vuestro  sistema  de  gobierno  no  subsistirá   eternamente. \textbf{Estos  cambios  tienen  lugar   cuando  cada especie termina y  vuelve  a  comenzar su revolución  circular,  la  cual  es  más  corta  ó  más  larga  según  que la  vida  de  cada  especie  es más larga  ó  más corta.  Vuestros  magistrados,  por hábiles  que sean y por mucho  que los  auxilien  la  experiencia  y  el  cálculo,  podrán  no  fijar  exactamente  el instante  favorable  ó contrario   a  la  propagación  de  su  especie.  Se les escapará  este  instante,  y  darán  al  Estado  tijos  en épocas  desfavorables. Habrá matrimonio donde nazcan hijos de la mala índole, pero  como  serán  indignos  de sucederles  en sus puestos, apenas se vean  elevados,  cuando ya  comenzarán  a  despreciarnos,  no   haciendo  de  la  música  el caso  que  debieran,  y  despreciando  en  igual  forma  la  gimnasia,  de donde resultará  que  la educación de vuestros jóvenes será mucho menos perfecta. Llegando,  pues,  a  mezclar  el  hierro  con  plata   y  el  bronce  con  el  oro,  resultará   de  esta  mezcla  una  falta  de conveniencia,  de regularidad y  de  armonía,  defecto  que  allí  donde  aparece  engendra  siempre  la  enemistad  y la  guerra. Una vez producida la excisión, las dos razas de hierro y  de bronce tratarán  de  enriquecerse  y   de  adquirir  tier-ras,  casas,  oro  y plata,  mientras  que  las  razas   de  oro y plata,  ricas por  naturaleza y no estando desprovistas, tenderán  a  la  virtud  y  al  sostenimiento  de  la   constitución   primitiva. Después  de  muchas  luchas  y   violencias,   las   gentes  de guerra  y  los magistrados  convendrán  en  dividir  entre  sí las tierras  y  las  casas,  destinarán   como  es-clavos  al  cuidado  de  sus tierras  y  sus  casas  el  resto  de  los ciudadanos, a quienes  consideraban  antes  como  hombres libres,  como sus  amigos  y  como  proveedores  de su mantenimiento,  y  continuarán   ellos  mismos haciendo  la  guerra  y  proveyendo  a la  común  seguridad}\\
Hombres de esta condición  estarán  ansiosos de riquezas,  como en los Estados  oligárquicos.  Ciegos  adoradores  del  oro y  de la plata,  los honrarán  en  la  oscuridad,  y  los  tendrán  secretamente  encerrados  en  cofres.  Ellos mismos, atrincherados  en  el recinto  de sus casas como en otros  tantos nidos, gastarán  en  mujeres  y  en  todo lo  que  halague  sus pasiones. Serán,  pues, avaros  de  su  dinero,  porque  lo  aman  y  lo  poseen  clandestinamente,  y   al   mismo   tiempo  serán  pródigos  de  los bienes  de los demás a causa  del  deseo  que  tienen  de satisfacer  sus  pasiones.\\
¿Qué entiendes  tú por  oligarquía?  — Entiendo  una  forma  de gobierno,  donde  el censo decide  de la  condición  de  cada  ciudadano;  donde  los  ricos,  por  consiguiente, ejercen  el mando sin que los pobres  participen  de  él.  \\
\textbf{ En  fin,  se dejan  dominar  más  y más por la  pasión  de  amontonar riquezas,  y  cuanto  más  aumenta  el  crédito  de  éstas,  tanto  más  disminuye  el  de  la  virtud. Por  consiguiente, la virtud  y  los hombres de  bien son menos estimados  en  un  Estado  a  proporción  que  se estiman  más  los ricos  y  las riquezas.}\\
\textbf{Es claro,   que  en   todo  Estado  en  que  veas pobres, hay ladronzuelos,  rateros, sacrilegios y  malvados de todas especies.  Si se  nos pregunta  quién  ha  creado  esta  mala  gente,  ¿no diremos  que  la  ignorancia,  la  mala  educación  y  el vicio mismo del gobierno ? — Sin  duda.Tal es el carácter del estado oligárquico}\\
cuando se trata  de gastar  bienes  ajenos,  entonces  es, mi querido amigo,  cuando descubrirás en los hombres de esta  condición  deseos  propios de la naturaleza  de  los  zánganos. \\\\
Me parece  que  corresponde ahora  examinar el origen y  las costumbres  de  la   democracia,  y  observar  después  estas  mismas cualidades  en  el  hombre  democrático, a fin de que  podamos compararlos  entre  sí y juzgarlos.\\
Es  evidente,  que  en  todo gobierno, cualquiera que  él  sea,  es  imposible  que  los ciudadanos  estimen  las  riquezas  y  practiquen  al  mismo  tiempo  la  templanza,  sino  que  es  una  necesidad  que  sacrifiquen  una  de estas  dos  cosas a la otra.
pag112
\chapter*{Libro Noveno}
\addcontentsline{toc}{chapter}{Libro Noveno}

\chapter*{Libro Désimo}
\addcontentsline{toc}{chapter}{Libro Décimo}
 

\end{document}

