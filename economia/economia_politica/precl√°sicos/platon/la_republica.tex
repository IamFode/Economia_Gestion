\documentclass[10pt]{book}
\usepackage[text=17cm,left=4cm,right=4cm, headsep=20pt, top=2.5cm, bottom = 2cm,letterpaper,showframe = false]{geometry} %configuración página
\usepackage{latexsym,amsmath,amssymb,amsfonts} %(símbolos de la AMS).7
\parindent = 0cm  %sangria
\usepackage[T1]{fontenc} %acentos en español
\usepackage[spanish]{babel} %español capitulos y secciones
\usepackage{graphicx} %gráficos y figuras.

%---------------FORMATO de letra--------------------%

\usepackage{lmodern} % tipos de letras
\usepackage{titlesec} %formato de títulos
\usepackage[backref=page]{hyperref} %hipervinculos
\usepackage{multicol} %columnas
\usepackage{tcolorbox, empheq} %cajas
\usepackage{enumerate} %indice enumerado
\usepackage{marginnote}%notas en el margen
\tcbuselibrary{skins,breakable,listings,theorems}
\usepackage[Bjornstrup]{fncychap}%diseño de portada de capitulos
\usepackage[all]{xy}%flechas
\counterwithout{footnote}{chapter}
\usepackage{xcolor}
\usepackage[htt]{hyphenat}
%--------------------GRÀFICOS--------------------------

\usepackage{tkz-fct}

%---------------------------------

\titleformat*{\section}{\LARGE\bfseries\rmfamily}
\titleformat*{\subsection}{\Large\bfseries\rmfamily}
\titleformat*{\subsubsection}{\large\bfseries\rmfamily}
\titleformat*{\paragraph}{\normalsize\bfseries\rmfamily}
\titleformat*{\subparagraph}{\small\bfseries\rmfamily}

%------------------------------------------

\renewcommand{\labelenumi}{\Roman{enumi}.}%primer piso II) enumerate
\renewcommand{\labelenumii}{\arabic{enumii}$)$}%segundo piso 2)
\renewcommand{\labelenumiii}{\alph{enumiii}$)$}%tercer piso a)
\renewcommand{\labelenumiv}{$\bullet$}%cuarto piso (punto)

%----------Formato título de capítulos-------------

\usepackage{titlesec}
\renewcommand{\thechapter}{\arabic{chapter}}
\titleformat{\chapter}[display]
{\titlerule[2pt]
\vspace{4ex}\bfseries\sffamily\huge}
{\filleft\Huge\thechapter}
{2ex}
{\filleft}

\begin{document}

\normalfont
\input xy
\xyoption{all}
\author{\Large Apuntes por FODE}
\title{\small Platón \\ \vspace{1cm} \large LA REPUBLICA}
\date{}
\pagestyle{empty}
\maketitle
\thispagestyle{empty}
\let\cleardoublepage\clearpage
\tableofcontents								%indice


%------------------------------------------
 
\let\cleardoublepage\clearpage

\chapter*{Libro Primero}
\addcontentsline{toc}{chapter}{Libro Primero}
\section*{Sócrates}
\addcontentsline{toc}{section}{Sócrates}
Céfalo dijo a Sócrates:  debes saber, que á medida que los placeres del cuerpo me abandonan, encuentro mayor encanto en la conversación.\\
La vejez, en efecto, es un estado de reposo y de libertad respecto de los sentidos. \\
si hemos de atenernos á lo que se ha dicho más arriba, la justicia hace bien á sus amigos y mal á sus enemigos. \\
El hombre justo, ¿en qué y en qué ocasión puede hacer mayor bien á sus amigos y mayor mal á sus enemigos? En la guerra, á mi parecer, atacando á los unos y defendiendo á los otros. \\
\textbf{Dime ahora en qué es útil la justicia durante la paz. Es útil en el comercio.}\\
\textbf{Trasimaco} dijo: Digo que la justicia no es otra cosa que lo que es provechoso al más fuerte. ¿No sabes que los diferentes Estados son monárquicos ó aristocráticos, ó populares? —Lo sé. —El que gobierna en cada Estado, ¿no es el más fuerte? —Seguramente. — ¿No hace leyes cada uno de ellos en ventaja suya, el pueblo leyes populares, el monarca leyes monárquicas, y asilos demás? Una vez hechas estas leyes, ¿no declaran que la justicia para los gobernados consiste en la observancia de las mismas? ¿No se castiga á los que las traspasan , como culpables de una acción injusta? Aquí tienes mi pensamiento. Encada Estado, la justicia no es más que la utilidad del que tiene la autoridad en sus manos, y, por consiguiente, del más fuerte. De donde se sigue para todo hombre que sabe discurrir, que la justicia y lo que es ventajoso al más fuerte en todas partes y siempre es una misma cosa.\\
En las transacciones y negocios particulares, hallarás siempre, que el injusto gana en el trato y que el hombre justo pierde. En los negocios públicos, si las necesidades del Estado exigen algunas contribuciones, el justo con fortuna igual suministrará más que el injusto. Tan cierto es, Sócrates, que la injusticia, cuando se la lleva hasta cierto punto, es más fuerte, más libre, más poderosa que la justicia; y que, como dije al principio, la justicia es el interés del más fuerte, y la injusticia es por sí misma útil y provechosa. 
En respuesta Sócrates menciona: Para comprometer a los hombres á que ejerzan el mando, ba sido preciso proponerles alguna recompensa, como dinero, bonores, ó un castigo si rehusan aceptarlo. \\
\textbf{Los sabios no quieren tomar parte en los negocios con ánimo de enriquecerse, porque temerían que se les mirara como mercenarios, si exigían manifiestamente algún salario por el mando, ó como ladrones si convertían los fondos públicos en su provecho.} Tampoco tienen en cuéntalos honores, porque no son ambiciosos.  Es preciso, pues, que algún motivo muy poderoso les obligue á tomar parte en el gobierno, como el temor de algún castigo. Y por esta razón se mira como cosa poco delicada el encargarse voluntariamente de la administración pública, sin verse comprometido á ello. Porque el mayor castigo para el hombre de bien, cuando rehusa gobernar á los demás, es el verse gobernado por otro menos digno; y este temor es el que obliga á los sabios á encargarse del gobierno , no por su interés ni por su gusto, sino por verse precisados á ello á falta de otros, tanto ó más dignos de gobernar\\
Sócrates menciona a esto que es tal, pues, la naturaleza de la injusticia, ya se encuentre en un Estado, ya en un ejército ó en cualquiera otra sociedad, que, en primer lugar, la hace absolutamente  impotente para emprender nada á causa de las querellas y sediciones que provoca; y, en segundo lugar, se hace enemiga de sí misma y de todos los que son á ella contrarios, es decir, de los hombres de bien.\\ \textbf{La justicia es mucho mas ventajosa que la injusticia.}


\chapter*{Libro Segundo}
\addcontentsline{toc}{chapter}{Libro Segundo}
Glaucon menciona que la injusticia se trata de someterse a una necesidad y no tanto como un bien. \\
Se dice que es un bien en sí cometer la injusticia y un mal el padecerla.
se ha llegado á amar la justicia, no porque sea un bien en sí misma, sino en razón de la imposibilidad en que nos coloca de dañar á los demás. Hagamos una suposición. Demos al hombre de bien y al hombre malo un poder igual para hacer todo lo que quieran; sigámoslos, y veamos á dónde conduce la pasión al uno y al otro. No tardaremos en sorprender al hombre de bien, siguiendo los pasos del hombre malo. \\
Gijes después de extraer un anillo de Oro de las entrañas de un caballo se lo puso y dado la vuelta el anillo se hizo invisible, esta ventaja hace que llegue al palacio y se apodera el trono. Ahora si hubiera dos anillos y se diera al Malo uno y al Bueno otro, no se corremperían los dos al ladod e la injusticia?. \textbf{El hombre se hace injusto tan pronto como cree poderlo ser sin temor}. Este hombre será elogiado de su virtud, pero con intención de engañarse mutuamente y por el temor de experimentar ellos mismos alguna injusticia.\\
\textbf{El gran mérito de la injusticia consiste en parecer justo sin serlo.} \\
Pindaro menciona: Subiré con trabajo al palacio, qiie habita lajiisticia, O marcharé por el torcido sendero del fraude, Para asegurar la felicidad de mi vida?\\
Todo lo que oigo me hace creer que de nada me servirá ser justo, si no adquiero la reputación de tal, y que  la virtud no tiene más que trabajos y penalidades que ofrecerme. \\
\textbf{El primero que consigue el poder de hacer mal, es el primero también en servirse de él hasta donde le es posible.} \\
¿ Cómo podremos encontrar en él la justicia y la injusticia? ¿Y dónde crees que tienen su origen en medio de todos estos diversos elementos? dice Aristóteles. Si tenemos un estado y en ella hay muchas personas con distintos oficios, entre ellos la de los guardias que se rigen por su ferocidad, donde es poco probable que sea dulce y a la vez inclinado a la cólera. Por eso un niño que se forma sobre algún propósito oigan discursos que le conduzcan a la virtud. Por lo tanto se debe enseñar teorías a los jóvenes que sean dogmas en sus vidas.\\
nuestra segunda ley, que prohibe hablar y escribir, respecto á los dioses, como si fueran encantadores, que toman diferentes formas y que intentan engañarnos con sus discursos y sus acciones? —La apruebo.

\chapter*{Libro Tercero}
\addcontentsline{toc}{chapter}{Libro Tercero}
\textbf{nuestro deber es estar muy en guardia respecto á los discursos que tengan esta tendencia, y recomendar á los poetas que conviertan en elogios todo lo malo que dicen ordinariamente de los infiernos, con tanto más motivo cuanto que lo que refieren ni es verdadero, ni propio para inspirar confianza á los guerreros.}\\
Sólo á los magistrados supremos pertenece el poder mentir, á fin de engañar al enemigo ó á los ciudadanos para bien de la república.La mentira no debe nunca permitirse á los demás hombres.\\
Guardémonos también de creer y de permitir que se diga, que Teseo, hijo de Neptuno, y Piritoo, hijo de Júpiter, hayan intentado el robo sacrilego que se les atribuye (5), ni que ninguno otro hijo dé los dioses, ningún héroe se haya hecho culpable de las crueldades y de las impiedades de que les acusan falsamente los poetas.Obliguemos á estos á reconocer, que los héroes nunca han cometido semejantes acciones; ó que, si las han cometido, ya no son descendientes de los dioses.\\
 hay tres clases de narraciones. La primera es imitativa, y, como acabas de decir, pertenece á la tragedia y á la comedia. La segunda se hace en nombre del poeta; y la verás empleada en los ditirambos. La tercera es una mezcla de una y otra; y nos servimos de ella en la epopeya y en otras cosas.\\
 \textbf{ Lo mismo sucede con respecto á la imitación. Un hombre solo no puede imitar muchas cosas lo mismo que una sola.}\\
Nos resta hablar de esta otra parte de la música, que corresponde al canto y á la melodía. \textbf{la melodía se compone de tres elementos: las palabras, la armonía y el número.} Para los guerreros necesitamos la dórica y la frigia, que expresan el carácter de un hombre sabio y valiente. el número y la armonía están hechas para las palabras, y no las palabras para el número y la armonía. \\\\
 ¿No es por esta misma razón, mi querido Glaucon, la música la parte principal de la educación, porque insinuándose desde muy temprano en el alma, el número y la armonía se apoderan de ella, y consiguen que la gracia y lo bello entren como un resultado necesario en ella, siempre que se dé esta parte de educación como conviene darla, puesto que sucede todo lo contrario, cuando se la desatiende? Y también , porque educado un joven, cual conviene, en la música, advertirá con la mayor exactitud lo quebaya de imperfecto yde defectuoso enlas obras de la naturaleza y del arte, y experimentará á su vista ima impresión justa y penosa; alabará por la misma razón con entusiasmo la belleza que observe, la dará entrada en su alma, se alimentará con ella, y se formará por este medio en la virtud; mientras que en el caso opuesto mirará con desprecio y con una aversión natural lo que encuentre de vicioso; y como esto sucederá desde la edad más tierna, antes de que le ilumine la luz de la razón, apenas baya esta aparecido, invadirá su alma, y él se unirá con ella mediante la relación secreta que la música habrá creado de antemano entre la razón y él. Hé aquí, á mi parecer, las ventajas que se buscan al educar á los niños en la música. \\
Es preciso, por el contrario, que su alma sea pura, exenta de vicio, para que su bondad le haga discernir más seguramente lo que es justo.\\
Es evidente, que nuestros jóvenes, educados en los principios de esta sencilla música que hace nacer en el alma la templanza, obrarán de manera que no tendrán necesidad de los jueces. \\
cuando se hace demasiado tirante, degenera en dureza y brutalidad.\\
 Los dioses han hecho á los hombres el presente de la música y de la gimnasia, no con objeto de cultivar el alma y el cuerpo (porque si este último saca alguna ventaja, es sólo indirectamente), sino para cultivar el alma sola, y perfeccionar en ella la sabiduría y el valor , concertándolos, ya dándolos expansión, ya conteniéndolos dentro de justos límites.\\
Entre los ancianos deben escogerse los mejores. Para esto es preciso que además de la prudencia y de la energía necesaria, tengan mucho celo por el bien público. Escojamos, pues, entre todos los guardadores, aquellos que, previo un maduro examen, nos parezca que después de haber pasado toda su vida consagrados á procurar el bien público, nunca han perjudicado los intereses del Estado. \\
Que se les haga entender, que los dioses han puesto en su alma oro y plata divina, y, por consiguiente, que no tienen necesidad del oro y de la plata de los hombres; que no les es permitido manchar
la posesión de este oro inmortal con la liga del oro terrestre; que el oro, que ellos tienen, es puro, mientras que el de los hombres ha sido en todos tiempos origen de muchos crímenes; que igualmente son ellos los únicos, entre los demás ciudadadanos, a quienes está prohibido manejar y hasta tocar el oro y la plata.guardarlo para sí, adornar con ello sus vestidos, beber en copas de estos metales, y que este es el único medio de conservación así para ellos como para el Estado. Porque desde el momento en que se hicieran propietarios de tierras, de casas y de dinero, de guardadores que eran se convertirían en empresarios y labradores, y de defensores del Estado se convertirían en sus enemigos y sus tiranos; pasarían la vida aborreciéndose mutuamente y armándose lazos unos á otros; entonces, los enemigos, que más deben de temerse, son los de dentro, y la república y ellos mismos correrán rápidamente hacia su ruina.

\chapter*{Libro Cuarto}
\addcontentsline{toc}{chapter}{Libro Cuarto}

 

\end{document}



