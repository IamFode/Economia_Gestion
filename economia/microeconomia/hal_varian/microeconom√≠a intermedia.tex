\documentclass[10pt]{article}
\usepackage[text=17cm,left=2.5cm,right=2.5cm, headsep=20pt, top=2.5cm, bottom = 2cm,letterpaper,showframe = false]{geometry} 	
\usepackage{latexsym,amsmath,amssymb,amsfonts}	%(símbolos de la AMS).7
\parindent = 0cm 								%sangria
\usepackage{lmodern}							% tipos de letras
\usepackage[T1]{fontenc}						%acentos en español
\usepackage[spanish]{babel}
\usepackage{titlesec} %formato de títulos
\pagestyle{empty}								%elimina numeración de página
\usepackage{multicol}
\usepackage{xcolor}



\begin{document}
\begin{center}
\huge Microeconomía intermedia\\
\vspace*{0.5cm}
\large Hal R. Varian\\
\vspace{1cm}
\Large Apuntes por Fode.
\vspace{1.5cm}
\end{center}
\part*{\center El marcado}
\section*{Cómo se construye un modelo}
La economía se basa en la construcción de modelos de los fenómenos sociales. Entendemos por modelo una representación simplificada de la realidad.\\
En general, lo mejor es adoptar el modelo más sencillo capaz de describir la situación económica que estemos examinando. 
\subsection*{Optimización y equilibro}
Siempre que tratamos de explicar la conducta de los seres humanos, necesitamos tener un modelo en el que basar el análisis. En economía se utiliza casi siempre un modelo basado en los dos principios siguientes.
\begin{itemize}
\item \textbf{El principio de la optimización:} los individuos tratan de elegir las mejores pautas de consumo que están a su alcance.
\item \textbf{El principio del equilibrio:} los precios se ajustan hasta que la cantidad que demandan los individuos de una cosa es igual a la que se ofrece.
\end{itemize}
 Si los individuos pueden decidir libremente sus actos, es razonable suponer que tratan de elegir las cosas que desean y no las que no desean. 
\section*{La curva de demanda}
La cantidad máxima que una determinada persona está dispuesta a pagar suele denominarse precio de reserva. En otras palabras, el precio de reserva de una persona es aquel al que le da exactamente igual comprar una cosa que no comprarla.\\
representa una curva de demanda, que relaciona la cantidad demandada y el precio de mercado.\\
\textbf{ La curva de demanda describe la cantidad demandada a
cada uno de los posibles precios.} La curva de demanda de apartamentos tiene pendiente negativa: los individuos están más dispuestos a alquilar apartamentos a medida que baja su precio.
\section*{La curva de oferta}
La curva de oferta a corto plazo es fija.
\section*{El equilibro de mercado}
El precio de equilibrio, $p$, se encuentra en la intersección de las curvas de oferta y de demanda.
\section*{Estática comparativa}
consiste en comparar dos equilibrios “estáticos”, sin preocuparse especialmente por la forma en que el mercado pasa de uno a otro.\\
El análisis de estática comparativa consiste solamente en comparar equilibrios, lo que ya plantea por el momento suficientes interrogantes que deben resolverse en este modelo.
\section*{Otras formas de asignar los apartamentos}
\subsection*{El monopolista discriminador}
 El caso en el que el mercado de un producto está dominado por un único vendedor se denomina monopolio.\\
\subsection*{Monopolista ordinario}
Se podrá restringir la producción a fin de maximizar el beneficio.
\subsection*{El control de los alquileres}
Se fija un precio máximo que se pueden cobrar.
\section*{La eficiencia en el sentido de Pareto}
 Un criterio útil para comparar los resultados de diferentes instituciones económicas es un concepto conocido con el nombre de eficiencia en el sentido de Pareto o eficiencia económica.\\
 Comenzamos con la siguiente definición: si podemos encontrar una forma de mejorar el bienestar de alguna persona sin empeorar el de ninguna otra, tenemos una mejora en el sentido de Pareto. Si una asignación puede ser mejorable en el sentido de Pareto, esta asignación se denomina ineficiente en el sentido de Pareto; si no puede ser mejorable en el sentido de Pareto, esta asignación se denomina eficiente en el sentido de Pareto.\\
 \textbf{Una situación económica es eficiente en el sentido de Pareto si no existe ninguna forma de mejorar el bienestar de un grupo de personas sin empeorar el de algún otro. El concepto de eficiencia en el sentido de Pareto puede utilizarse para evaluar las diferentes formas de asignar los recursos.}
\part*{\center La restricción presupuestaria}
La teoría económica del consumidor es muy sencilla: los economistas suponen que los consumidores eligen la mejor cesta de bienes que pueden adquirir.\\
\section*{La restricción presupuestaria}
La restricción presupuestaria es escrita algebraicamente como sigue:
$$p_1 x_1 + p_2 x_2 \leq m$$
\section*{Dos bienes suelen ser suficientes}
El supuesto de los dos bienes es más general de lo que parece a primera vista, ya que normalmente podemos considerar que uno de ellos representa todo lo demás que al individuo le gustaría consumir.\\
decimos que el bien 2 es un \textbf{bien compuesto} porque representa todo lo demás que podría consumir el individuo, aparte del bien 1. 
\section*{Propiedades del conjunto presupuestario}
La recta presupuestaria es el conjunto de cestas que cuestan exactamente $m$: $$p_1 x_1 + p_2 x_2 = m$$
Éstas son las cestas de bienes que agotan exactamente la renta del consumidor.\\
La restricción presupuestaria de la ecuación [2.3] también puede expresarse de la forma siguiente:
$$x_2=\dfrac{m}{p_2}-\dfrac{p_1}{p_2} x_1$$
Indica cuántas unidades del bien 2 necesita consumir el individuo para satisfacer exactamente la restricción presupuestaria si está consumiendo $x_1$ unidades del bien 1.\\
Debe preguntarse qué cantidad del bien 1 podría comprar si gastara todo el dinero en dicho bien. La respuesta es $m/p1$. Por lo tanto, las coordenadas en el origen miden la cantidad que podría comprar el consumidor si gastara todo el dinero en los bienes 1 y 2, respectivamente.\\
La pendiente de la recta presupuestaria tiene una bonita interpretación económica. Mide la relación en la que el mercado está dispuesto a sustituir el bien 2 por el 1.\\
Algunas veces los economistas dicen que la pendiente de la recta presupuestaria mide el\textbf{ coste de oportunidad} de consumir el bien 1. Para consumir una mayor cantidad de dicho bien hay que renunciar a alguna cantidad del 2.La renuncia a la oportunidad de consumir el bien 2 es el verdadero coste económico de consumir una mayor cantidad del 1, y ese coste está representado por la pendiente de la recta presupuestaria.
\section*{Cómo varía la recta presupuestaria}
Un incremento de la renta da lugar a un desplazamiento paralelo hacia fuera de la recta presupuestaria. En cambio, una reducción de la renta provoca un desplazamiento paralelo hacia dentro.\\



\end{document}