\documentclass[10pt]{book}
\usepackage[text=17cm,left=2.5cm,right=2.5cm, headsep=20pt, top=2.5cm, bottom = 2cm,letterpaper,showframe = false]{geometry} 	
\usepackage{latexsym,amsmath,amssymb,amsfonts}	%(símbolos de la AMS).7
\parindent = 0cm 								%sangria
\usepackage{lmodern}							% tipos de letras
\usepackage[T1]{fontenc}						%acentos en español
\usepackage[spanish]{babel}
\usepackage{titlesec} %formato de títulos
\pagestyle{empty}								%elimina numeración de página
\usepackage{multicol}
\usepackage{color} %resaltar texto
\usepackage{enumerate}

\begin{document}
\begin{center}
\huge The \$ $100$ startup\\
\vspace*{0.5cm}
\large Chris Guillebeau\\
\vspace{1cm}
\Large Apuntes por Fode.
\vspace{1.5cm}
\end{center}


\part{Emprendedores inesperados}
    \chapter{Renacimiento}
	\section{El modelo de inicio de \$ 100}	
	Todos los encuestados tenían el siguiente criterio en común:
	\begin{itemize}
	    \item \textbf{Modelo Sigue tu pasión}. 
	    \item \textbf{Costo inicial bajo}. 
	    \item \textbf{Al menos \$ 50,000 al año en ingresos netos.}
	    \item \textbf{Sin habilidades especiales}
	    \item \textbf{Divulgación financiera completa.}
	    \item \textbf{Menos de cinco empleados}
	\end{itemize}

	\section{Lección 1: Convergencia}        
	La forma más fácil de entender la convergencia es pensar en ello. Como el espacio superpuesto entre lo que le importa y en lo que otras personas están dispuestas a gastar dinero.	

	\section{Lección 2: Transformación de habilidades}
	Muchos de los proyectos que examinaremos fueron iniciados por personas con habilidades relacionadas, no necesariamente la habilidad más utilizada en el proyecto.\\
	Para tener éxito en un proyecto empresarial, especialmente en uno que le entusiasma, es útil pensar detenidamente en todas las habilidades que tiene que podrían ser útiles para los demás y, en particular, en la combinación de esas habilidades.
	\section{Lección 3: La fórmula mágica}	
	Pasión o habilidad + utilidad = éxito

	\section{El camino por delante: lo que aprenderemos}
	Solo necesita un producto o servicio, un grupo de personas dispuestas a pagar por él y una forma de cobrar. Esto se puede desglosar de la siguiente manera:

	\begin{enumerate}[\bfseries 1.]
	    \item Producto o servicio  que se vende.
	    \item Personas dispuestas a pagar por ello: sus clientes.
	    \item Una forma de cobrar: cómo cambiará un producto o servicio por dinero
	\end{enumerate}
	Si tiene un grupo de personas interesadas pero no tiene nada que vender, no tiene un negocio. Si tiene algo que vender pero nadie está dispuesto a comprarlo, no tiene un negocio. En ambos casos, sin una forma clara y sencilla para que los clientes paguen por lo que ofrece, no tiene un negocio. Junte los tres y felicitaciones: ahora es un emprendedor.\\
	Es útil tener una estrategia para generar interés y llamar la atención.

    \chapter{Dale el pescado}
    La forma más fácil es averiguar qué quiere la gente y luego encontrar la forma de dárselo.\\
    Una mejor manera es dar a las personas lo que realmente quieren, y la manera de hacerlo consiste en comprender algo muy simple sobre quiénes somos.
    
	\section{¿De dónde vienen las ideas?}
	A continuación, se muestran algunas fuentes habituales de inspiración.

	\begin{enumerate}[\bfseries 1.]
	    \item Una ineficiencia en el mercado.
	    \item Nueva tecnología o oportunidad.
	    \item Un espacio cambiante.
	    \item Un proyecto secundario o paralelo.
	\end{enumerate}
	    
	\section{¿Qué es el valor?}

	\textbf{El valor es algo deseable y valioso creado a través del intercambio o del esfuerzo, Valor significa ayudar a las personas. Si está tratando de construir una microempresa y comienza sus esfuerzos ayudando a las personas, está en el camino correcto.}

	    \subsection{Estrategia 1: Profundice para descubrir necesidades ocultas}
	    Ir mas allá de lo que la personas esperan.

	    \subsection{Estrategia 2: Haga de su cliente un héroe}
	    Se tiene que ayudar al cliente para que este solucione un problema y se convierta en héroe.
	    \subsection{Estrategia 3: Venda lo que compra la gente}
	    Este atento a lo que la gente quiere.

    \section{6 pasos para conseguir comenzar ya!}	
	\begin{enumerate}[\bfseries 1.]
	    \item Decidir sobre su producto o servicio. 
	    \item Configure un sitio web, incluso uno muy básico 
	    \item Desarrolle una oferta (una oferta es distinta de un producto o servicio).
	    \item Asegúrese de tener una forma de cobrar (obtenga una cuenta PayPal gratuita para comenzar). 
	    \item Anuncie su oferta al mundo 
	    \item Aprenda de los pasos 1 a 5, luego repita.
	\end{enumerate}

	\section{Que realmente quiere la gente}
	Concentrarse en lo que los clientes realmente quieren de una empresa es fundamental.\\
	Si su empresa se centra en dar a las personas más de lo que quieren o en quitarles algo que no quieren (o ambas cosas), está en el camino correcto.

    \chapter{Sigue tu pasión ... Tal vez}
    Puede iniciar un nuevo negocio como consultor en aproximadamente un día, si no antes. Siga estas dos reglas básicas:
    \begin{enumerate}[\bfseries 1.]
	\item Elija algo específico en lugar de algo general. No sea un \"consultor de negocios\" o un \"entrenador de vida\", consiga específico sobre lo que realmente puedes hacer para alguien.    
	\item Nadie valora a un consultor de \$ 15 la hora, así que no subestime su servicio. Como probablemente no tengas cuarenta horas de trabajo facturable cada semana, cobrar al menos \$ 100 por hora o una tarifa fija comparable por el beneficio que proporcionar.    
    \end{enumerate}
    Además de pasión, debes desarrollar una habilidad que proporciona una solución a un problema.
    \begin{center}
	(¿Pasión + habilidad)? (problema + mercado) = oportunidad
    \end{center}

    \chapter{El ascenso del emprendedor itinerante}
    Conviértete en tu propio editor. Siga estos pasos para ingresar la información negocio editorial. Cada paso se puede hacer más complicado, pero todos se relacionan con esto esquema básico.
    \begin{enumerate}[\bfseries 1.]
	    \item Encuentra un tema por el que la gente pague aprender acerca. Ayuda si eres un experto en el tema, pero si no, eso es lo que la investigación es para.    
	    \item Capture la información en una de las tres maneras:
		\begin{enumerate}[\bfseries a)]
		    \item Escríbelo.
		    \item Grabe audio o video.
		    \item Produzca una combinación.
		\end{enumerate}
	    \item Combine sus materiales en un producto: Un libro electrónico o un paquete digital que pueden descargar los compradores.
	    \item Crea una oferta. Que eres exactamente vendiendo y por qué la gente debería tomar acción en él?
	    \item Decidir un precio justo basado en el valor para tu oferta.
	    \item Encuentre una manera de que le paguen.
	    \item Publique la oferta y corra la voz.
	    \item Cobre y dirígete a la plaza.
    \end{enumerate}

    \chapter{La nueva demografía}
	\section{cambiando el $"$ Quien $"$} 
	Muchas veces debemos cambiar dos cosas: Lo que se ofrece y a quien ofrecer.

	\section{Desastre y recuperación}
	\begin{enumerate}[\bfseries 1.]
		\item Estrategia 1: Aferrarse a un pasatiempo, pasión o moda popular.
		\item Estrategia 2: Venda lo que la gente compra (y pregúnteles si no está seguro).\\
		A la mayoría de nosotros nos gusta comprar, pero no gusta ser vendido. El marketing de la vieja escuela es basado en la persuasión; nuevo marketing es basado en invitación.\\
		En términos generales, es bueno limitar las encuestas a menos de diez preguntas aproximadamente. Llegar respuestas más generales, haga menos preguntas.
	\end{enumerate}
	    
	\section{El cliente siempre tiene razón A menudo se equivoca}
	Un cliente individual no siempre sabe qué es lo mejor para todo tu negocio. Estos clientes pueden no ser los adecuados para su negocio, y no hay nada de malo en decir adiós a ellos para que pueda concentrarse en servir a otros personas.
	
	\section{La lista de posibilidades y la matriz de toma de decisiones}
	Probablemente termine sin escasez de ideas escrito en servilletas, garabateado en cuadernos, y flotando en tu cabeza. El problema es evaluar qué proyectos valen perseguir y luego decidir entre diferentes ideas. A veces, es posible que sepa intuitivamente cuál es el mejor movimiento. En esos casos, debe proceder sin dudarlo. Otras veces, sin embargo, se sentirá en conflicto. Qué
deberías hacer La matriz de toma de decisiones te ayudará evaluar una variedad de proyectos y separar los ganadores de los "quizás más tarde". Poniendo algo fuera de lugar no significa que nunca lo harás, pero la priorización lo ayudará a comenzar sobre lo que genera el mayor impacto. Ante todo, Tenga en cuenta las preguntas más básicas de cualquier microempresa de éxito:
	\begin{enumerate}[\bfseries 1.]
	    \item ¿El proyecto produce una ¿producto o servicio?
	    \item ¿Conoce personas que quieran Comprarlo? (O sabes donde encontrar ¿ellos?)
	    \item ¿Tiene alguna forma de que le paguen?
	\end{enumerate}
	En esta matriz, enumerará sus ideas en la columna de la izquierda y luego las calificará en una escala de 1 a 5, siendo 5 la más alta. Por supuesto, la puntuación será subjetiva, pero como estamos buscando tendencias, está bien hacer una estimación. Califique sus ideas de acuerdo con estos criterios:
	\begin{itemize}
	    \item \textbf{Impacto}. En general, ¿cuánto impacto tendrá este proyecto en su negocio y sus clientes?
	    \item \textbf{Esfuerzo}. ¿Cuánto tiempo y trabajo se necesitará para crear el proyecto? (En este caso, una puntuación más baja indica más esfuerzo, así que elija 1 para un proyecto que requiere una tonelada de trabajo y 5 para un proyecto que casi no requiere trabajo).
	    \item \textbf{Rentabilidad}. En relación con las otras ideas, ¿cuánto dinero aportará el proyecto?
	    \item \textbf{Visión}. ¿Qué tan cercano es este proyecto con su misión y visión general?
	\end{itemize}


\part{Llevarlo a las calles}

    \chapter{El plan de negocios de una sola página}   

	\section{El sesgo de la acción}
	En la batalla entre la planificación y la acción, la acción gana. Así es como lo haces. SELECCIONE UNA IDEA COMERCIAL.\\
	\textbf{Una idea comercial no tiene ser una idea grande e innovadora; solo tiene para proporcionar una solución a un problema o ser lo suficientemente útil como para que otras personas estén dispuestas a paga por ello.} No pienses en la innovación piensa en la utilidad.
	    \subsection{Siete pasos para las pruebas de mercado instantáneas}
		\begin{enumerate}[\bfseries 1.]
			\item Debe preocuparse por el problema que va a resolver, y debe haber un número considerable de personas a las que también les importe. Recuerde siempre la lección de la convergencia: la forma en que su idea se cruza con lo que otras personas valoran.
			\item Asegúrese de que el mercado sea lo suficientemente grande. Pruebe el tamaño verificando el número y la relevancia de las palabras clave de Google, las mismas palabras clave que usaría si estuviera tratando de encontrar su producto. Piense en las palabras clave que la gente usaría para encontrar una solución a un problema. Si estuviera buscando su propio producto en línea pero no supiera que existe, ¿qué palabras clave buscaría? Preste atención a los lados superior y derecho de las páginas de resultados, donde se muestran los anuncios.
			\item Concéntrese en eliminar el $"$dolor admitido flagrante$"$. El producto necesita resolver un problema que causa dolor que el mercado sabe que tiene. Es más fácil venderle a alguien que sabe que tiene un problema y está convencido de que necesita una solución que persuadir a alguien de que tiene un problema que necesita solución..
			\item Casi todo lo que se vende es por un profundo dolor o un profundo deseo.  Por ejemplo, la gente compra artículos de lujo por respeto y estatus, pero en un nivel más profundo quieren ser amados. Tener algo que elimine el dolor puede ser más efectivo que realizar un deseo. Debe mostrar a las personas cómo puede ayudar a eliminar o reducir el dolor.
			\item Piense siempre en términos de soluciones. Asegúrese de que su solución sea diferente y mejor. (Tenga en cuenta que no es necesario que sea más barato; competir por precio suele ser una propuesta con pérdidas). ¿Está frustrado el mercado con la solución actual? Ser diferente no es suficiente; la diferenciación que lo hace mejor es lo que se requiere. No tiene sentido introducir algo si el mercado ya está satisfecho con la Solución; su solución debe ser diferente o mejor. Es la importancia, no el tamaño, lo que importa.
			\item Pregunte a otros sobre la idea, pero asegúrese de que las personas a las que les pregunte sean su mercado objetivo potencial. Otros pueden proporcionar datos insignificantes y, por lo tanto, están sesgados y desinformados. Por lo tanto, cree una persona: la única persona que se beneficiaría más de su idea. Examine toda su red (comunidad, amigos, familia, redes sociales) y pregúntese si alguna de estas personas coincide con su personalidad. Lleve su idea a esta persona y discútala en detalle con ella. Esto le proporcionará datos mucho más relevantes que hablar con cualquiera.
			\item Cree un esquema de lo que está haciendo y muéstrelo a un subgrupo de su comunidad. Pídales que lo prueben gratis a cambio de comentarios y confidencialidad. Como beneficio adicional, el subgrupo se siente involucrado y actuará como evangelizador. Dar genera confianza y valor, y también le brinda la oportunidad de ofrecer la solución completa. Utilice un blog para desarrollar autoridad y experiencia en un tema. Deje comentarios en blogs donde su público objetivo se cuelgue.
		\end{enumerate}

		    \subsubsection{Mantenga los costos bajos}
		    Invirtiendo capital de sudor en lugar de dinero en su proyecto, evitará
endeudarse y minimizar el impacto de falla si no funciona.

		    \subsubsection{Consigue la venta lo mas antes posible}
		    El mayor problema es la inercia, por lo tanto se debe poner en funcionamiento el sitio para luego tenga su propia recompensa.

		    \subsubsection{Mercado antes de la fabricación}
		    Es bueno saber si la gente quiere lo que tienes para ofrecer antes de poner mucho trabajo en hacerlo.\\
		    Es importante comenzar lo más rápido posible y por qué la primera venta puede ser tan importante.

		    \subsubsection{Responder a los resultados iniciales}
		    En una microempresa construida sobre bajos costos y acción rápida, no es necesario hacer mucho planificación formal. Principalmente, necesita los elementos que hemos discutido a lo largo del libro: un producto o servicio, un grupo de clientes, y una forma de cobrar.
		    
	\section{Recibe libremente, da libremente}	    
	¿Cómo ayudará este negocio a la gente? Esta no se trata simplemente de ser generoso, porque a medida que una empresa ayuda a las personas, el propietario de la empresa recibe un pago.\\
	Algunas personas diseñan un negocios con fines de lucro en torno al componente social, otros cambian para centrarse en él a medida que avanzan y otros integran un proyecto social dentro de una empresa con fines de lucro.
	    \subsection{La declaración de misión de 140 caracteres}
	    Puede revisar la declaración de la siguiente manera: Ayudamos a clientes a hacer / lograr / a $"$Beneficio principal$"$

    \chapter{Una oferta que no puedes rechazar}
	    Recuerda, primero tu necesitan vender lo que la gente quiere comprar: dar a ellos el pescado. 
	    \begin{enumerate}[\bfseries 1.]
		\item \textbf{Comprenda que lo que queremos y lo que decimos que queremos no siempre es lo mismo.} Una buena oferta tiene que ser lo que la gente realmente quiera y esté dispuesta a pagar.
		\item \textbf{A la mayoría de nosotros nos gusta comprar} Una oferta que no puede rechazar puede ejercer una presión sutil, pero a nadie le gustan las ventas duras. En cambio, las ofertas atractivas a menudo crean la ilusión de que una compra es una invitación, no un lanzamiento.
		\item \textbf{Dale un empujón} 
	    \end{enumerate}

	    \begin{center}
		Proyecto de construcción de oferta FÓRMULA MÁGICA: EL PÚBLICO ADECUADO, LA PROMESA CORRECTA, EL MOMENTO CORRECTO = OFERTA QUE NO PUEDE RECHAZAR
	    \end{center}

	    \section{Kit de herramientas de oferta convincente}

		\subsection{Preguntas frecuentes, también conocido como Lo que quiero que sepas}
		El propósito de las preguntas frecuentes es brindar tranquilidad a los compradores potenciales y superar las objeciones. Tu misión, si decides aceptarla, es identificar las principales objeciones de sus compradores tendrá al considerar su oferta y responda con cuidado a ellos por adelantado. Estas objeciones suelen relacionarse con deseos humanos básicos, necesidades, preocupaciones y miedos.\\
		\textbf{La principal preocupación de cada una de estas objeciones se relaciona con la confianza y la autoridad. Debe crear la confianza del consumidor para superar las objeciones. Mientras elabora la oferta, piense en las objeciones y luego gírelas a su favor.}\\
		Un modelo que puede utilizar al describir su oferta se describe a continuación en lo que llamaremos $"$Formato impresionante en bruto$"$. Funciona así:

		    \begin{enumerate}
			    \item Esto es asombroso.
			    \item En serio, es realmente asombroso.
			    \item Por cierto, no es necesario.
			    \item Mira, es realmente asombroso.
		    \end{enumerate}

		\subsection{La increíble garantía, también conocida como $"$No tengas miedo$"$}
		    Independientemente de lo que venda, la principal preocupación de muchos clientes potenciales es, $"$¿Qué pasa si no me gusta?$"$. Puedo conseguir mi devolución de dinero? Una forma común y muy eficaz de combatir esta preocupación es ofrecer una satisfacción garantizada. Un consejo: haz. No haga que su garantía sea complicada, confusa o aburrida. ¡No quiere que su cliente lo piense demasiado! Mantenlo simple y fácil. Además, si hay alguna forma de atar el resultados prometidos de su oferta a la garantía, hágalo.

		\subsection{Entrega en exceso, también conocida como $"$Guau, mira todas estas cosas adicionales que no esperaba$"$}

		Déles más de lo que esperaban. Puede hacer esto actualizando su compra inesperadamente enviando una tarjeta de agradecimiento escrita a mano por correo o de cualquier manera que tenga más sentido para su negocio. El caso es que las pequeñas cosas cuentan.

    \chapter{Lanzamiento!}
    los lanzamientos se construyen principalmente a través de una serie de comunicaciones regulares con prospectos y clientes existentes. 

	\section{Era una noche oscura y tormentosa}
	La campaña generalmente  se despliega en una serie de mensajes que envías a tu audiencia, y debes tener en cuenta la analogía de Hollywood. \\
	Debemos pensar en Relatibilidad y oportunidad. Relatibilidad que puede ser o no una palabra real, se refiere a la necesidad de asegurar que las personas escuchen sobre el lanzamiento y puedan relacionarse.

	\section{No se trata de las ventas.}
	El objetivo de un buen lanzamiento no es solo convertir tantos prospectos como sea posible; es también para preservar su relación con otros prospectos y aumentar su influencia.\\
	Es mejor para construir relaciones a lo largo del tiempo.

	\section{Lista de verificación de lanzamiento de productos de treinta y nueve pasos}	

	\begin{enumerate}[\bfseries 1.]
		\item Asegúrese de que su producto o servicio tenga una propuesta de valor clara.  ¿Qué reciben los clientes cuando intercambian dinero por su oferta?
		\item Decidir sobre bonificaciones, incentivos o recompensas para los primeros compradores. ¿Cómo serán recompensados por actuar?
		\item ¿Has hecho que el lanzamiento sea divertido de alguna manera? (Recuerde pensar tanto en los no compradores como en los compradores. Si las personas no quieren comprar, ¿seguirán disfrutando de escuchar o leer sobre el lanzamiento?)
		\item Si su lanzamiento es en línea, ¿ha grabado un mensaje de video o audio para complementar la copia escrita?
		\item ¿Ha incorporado anticipación al lanzamiento? ¿Están entusiasmados los prospectos?
		\item ¿Ha incorporado la urgencia, no del tipo falso, sino una verdadera razón para la puntualidad, en el lanzamiento?
		\item Publique la hora y la fecha del lanzamiento con anticipación (si está en línea, unas personas acamparán en el sitio una hora antes, presionando el botón de actualización cada pocos minutos).
		\item Revise todos los materiales de ventas varias veces ... y pida a otra persona que los revise también.\\
		¡Esto es superimportante! USP significa $"$propuesta de venta única$"$ y se refiere a la única cosa que distingue su oferta de todas las demás. ¿Por qué la gente debería prestar atención a lo que vendes? Debes responder bien a esta pregunta.
		\item Si se trata de un producto en línea, ¿está configurado correctamente en su carrito de compras o con PayPal?
		\item Pruebe cada paso del proceso de pedido repetidamente. Siempre que cambie alguna variable (precio, componentes del pedido, texto, etc.), pruébela nuevamente.
		\item ¿Ha registrado todos los dominios asociados a su producto? (Los dominios son baratos; también puede obtener el .com, .net, .org y cualquier nombre muy similar si está disponible).
		\item ¿Están todos los archivos cargados y en el lugar correcto?
		\item Revise la página de pedidos detenidamente para detectar errores o mejoras fáciles de realizar. Imprímelo y compártelo con varios amigos para que lo revisen, incluidas algunas personas que no saben nada sobre tu negocio.
		\item Lea las comunicaciones importantes (mensaje de lanzamiento, página de pedido, página de ventas en voz alta. Probablemente notará un error o una oración mal redactada que se perdió al leerla en su cabeza.
		\item ¿Ha creado usted o su diseñador algún gráfico personalizado para la oferta, incluidos los anuncios necesarios para afiliados o socios?
		\item Establezca un objetivo monetario claro para el lanzamiento. ¿Cuántas ventas desea ver y cuántos ingresos netos? (En otras palabras, ¿cómo será el éxito?)
		\item Avise a la cuenta mercantil o al banco de los fondos entrantes.
		\item Cree un plan de respaldo para los fondos entrantes si es necesario (obtenga una cuenta de comerciante adicional, planifique cambiar todos los pagos a PayPal, etc.).
		\item ¿Puede agregar otra opción de pago para cualquier persona que tenga problemas para realizar un pedido?
		\item Para un producto de alto precio, ¿puede ofrecer un plan de pago? (Nota: es común ofrecer un pequeño descuento para los clientes que pagan en su totalidad. Esto sirve como un incentivo para los clientes que prefieren pagar todo a la vez, al mismo tiempo que brinda una alternativa para aquellos que necesitan pagar en el tiempo).
		\item Borre la mayor cantidad de correo electrónico posible además de cualquier otra tarea en línea para que pueda concentrarse en el gran día de mañana.
		\item Escriba un mensaje de lanzamiento contundente para sus listas de lectores, clientes y / o afiliados.
		\item Prepare una publicación de blog y cualquier publicación necesaria en las redes sociales (si corresponde).
		\item Configure dos relojes de alarma para asegurarse de estar completamente despierto y disponible al menos una hora antes del lanzamiento programado.
		\item Programe su hora de lanzamiento para adaptarse a su audiencia, no a usted. En igualdad de condiciones, normalmente es mejor realizar el lanzamiento temprano en la mañana.
		\item Realice un lanzamiento suave al menos diez minutos antes para asegurarse de que todo funcione. ¡Es mejor para usted encontrar los problemas que hacer que sus clientes los encuentren!
		\item Escriba a los primeros tres a cinco compradores para dar las gracias y preguntar: "¿Todo salió bien en el proceso de pedido?" (Beneficio adicional: estos compradores probablemente sean sus mayores admiradores de todos modos, por lo que apreciarán el registro personal). 
		\item Siempre que sea posible, envíe una nota personal rápida a cada comprador además del agradecimiento automático que se envía. (Si no es posible siempre, hágalo con tanta frecuencia como pueda).
		\item Lo más importante: pida ayuda para correr la voz. Muchos lectores, prospectos y conocidos ayudarán contándoles a sus amigos y seguidores, pero tienes que preguntarles.
		\item Escriba a los afiliados con un recordatorio sobre la nueva oferta.
		\item Escriba a periodistas o contactos de medios, si corresponde.
		\item Publica en Twitter, Facebook, LinkedIn y cualquier otra red social en la que ya participes. (Por lo general, no es una buena idea unirse a una nueva red solo para promocionar algo).
		\item Escriba el mensaje general de agradecimiento que todos los compradores recibirán al comprar.
		\item Si corresponde, escriba el primer mensaje de la serie de seguimiento de correo electrónico que recibirán los compradores.
		\item Describa contenido adicional para futuras comunicaciones y planifique programarlo después de recuperarse del lanzamiento.
		\item ¿Cómo puede sobreentregar y sorprender a sus clientes con este producto? ¿Puede incluir entregables adicionales o algún tipo de beneficio no anunciado?
		\item ¿Hay algo especial que pueda hacer para agradecer a sus clientes? (Para un lanzamiento de alto precio, envíe postales a cada comprador; para algo adicional, llame a algunos de sus clientes por teléfono).
		\item No olvide celebrar. Es un gran día en el que has trabajado durante mucho tiempo. Vaya a su restaurante favorito, tome una copa de vino, compre algo que haya estado mirando durante un tiempo o haga algo como recompensa personal. Te lo has ganado.
		\item Empiece a pensar en el próximo lanzamiento. ¿En qué puedes construir a partir de este? ¿Qué aprendió que puede ayudarlo a crear algo aún mejor la próxima vez?.

	\end{enumerate}
	    Recuerde, muchos clientes lo apoyarán de por vida siempre y cuando siga brindándoles un gran valor. Es mucho más fácil venderle a un cliente existente que a uno nuevo, así que esfuércese por sobreentregar y planifique con anticipación el próximo proyecto. (Por ejemplo, cuando prometa una lista de verificación de treinta y nueve pasos, agregue un paso adicional).

    \section{Después del lanzamiento: no ha terminado}
    \textbf{Un buen lanzamiento es como una película de Hollywood}

    \chapter{Hustling: el suave arte de la autopromoción}
	\section{Que es hustling} 
	    A veces el mejor esfuerzo consiste en crear una gran oferta y hacer que la gente hable de ella.\\
	    Hustling es mucha creación y mucha conexión.

	\section{Si lo construyes podrías venir}	
	    El primer paso para construir una comunidad es saber \textbf{que decir}

	\section{El plan de marketing de donaciones estratégicas}
	    Dar libremente, recibir gratuitamente: funciona. Cuanto más centre su negocio en proporcionar un servicio valioso y ayudar a las personas, más crecerá su negocio.

	\section{Construir relaciones no es una táctica si no una estrategia}
	    Puede que no sea una venta o una asociación, pero la construcción de esas relaciones hoy siempre regresa para 351/617 nuevas oportunidades mañana

	\section{Regala algo y mira a la gente saltar}
	    Los sorteos son una herramienta muy importante. Un concurso generalmente requiere más trabajo tanto para los aspirantes a ganadores como para la empresa que organiza el concurso, pero puede generar más interés. Un sorteo es rápido y fácil y puede generar una gran cantidad de entradas, pero dado que generalmente no hay nada que hacer más que escribir su nombre, el sorteo típico no genera mucha participación real. Para obtener los mejores resultados, experimente con el tiempo con ambos métodos.

	\section{El experimento sexual y de marketing de diez horas de \$ 10,000}
	\textbf{En el futuro, el marketing será como el sexo: solo los perdedores lo pagan.}\\
	La gran mayoría de los sujetos de los estudios de caso con los que hablé construyeron su base de clientes sin ningún tipo de publicidad paga; lo hicieron en gran parte a través del boca a boca.

	    \subsection{El plan de promoción de una página}
	    \begin{itemize} 
		\item Diariamente Mantenga una presencia regular en las redes sociales sin desviarse ni abrumarse. Publique de uno a tres elementos útiles, responda preguntas y toque la base con cualquier persona que necesite ayuda.
		\item Supervisar una o dos métricas clave\\
	    \textbf{Semanalmente}
		\item Pida ayuda o promociones conjuntas de sus colegas y asegúrese de que también les está ayudando.
		\item Mantener una comunicación regular con prospectos y clientes.\\
	    \textbf{Al menos una vez al mes}
		\item Conéctese con los clientes existentes para asegurarse de que estén contentos. (Pregunte: "¿Hay algo más que pueda hacer por usted?")
		\item Prepárese para un próximo evento, concurso o lanzamiento de producto.\\
	    \textbf{De vez en cuando}
		\item Realice su propia auditoría empresarial (consulte el Capítulo 12) para encontrar oportunidades perdidas que puedan convertirse en proyectos activos.
		\item Asegúrese de trabajar regularmente para construir algo significativo, no solo reaccionar a las cosas tal como aparecen.
	    \end{itemize}

    \chapter{Muestra el dinero}
	\textbf{Muchos aspirantes a dueños de negocios cometen dos errores comunes relacionados: pensar demasiado en dónde obtener dinero para comenzar su proyecto y pensar muy poco acerca de dónde vendrán los ingresos comerciales. Solucionar estos problemas (o evitarlos en primer lugar) requiere una solución simple: gaste la menor cantidad de dinero posible y gane tanto dinero como pueda.}

	\section{Gane más dinero (tres principios clave para centrarse en las ganancias)}
	    \begin{enumerate}[\bfseries 1.]	
		    \item Ponga el precio de su producto o servicio en relación con el beneficio que brinda, no con el costo de producción.
		    \item Ofrezca a los clientes una gama limitada de precios.
		    \item Cobra más de una vez por lo mismo.
	    \end{enumerate}
		
		\subsection{Principio 1: Basar los precios en los beneficios, no en los costos}
		    Cuando base sus precios en los beneficios que brinda, esté preparado para defender su posición, porque algunas personas siempre se quejarán de que el precio es demasiado alto, sin importar cuál sea. 

		\subsection{Principio 2: Ofrezca un rango (limitado) de precios}
		    Elegir un precio inicial para su servicio que se base en el beneficio brindado a los clientes es el principio más importante para asegurar la rentabilidad. Pero para crear una rentabilidad óptima o al menos para crear más protección en su modelo de negocio, a continuación, querrá presentar más de un precio para su oferta. Esta práctica generalmente marca una gran diferencia en el resultado final, porque le permite aumentar los ingresos sin aumentar su base de clientes.
		
		\subsection{Principio 3: Cobre más de una vez}

	    \section{Tienes más de lo que piensas}

\part{APALANCAMIENTO Y PRÓXIMOS PASOS}

    \chapter{Continuando para arriba}

	\section{Modificando su camino hacia el banco: el panorama general}
	    El no tan secreto para mejorar los ingresos en un negocio existente es a través de ajustes: pequeños cambios que crean un gran impacto.

	    \paragraph{Aumentar el tráfico}
		Tráfico significa atención. ¿Cuánta atención está recibiendo su negocio?

	    \paragraph{Aumentar la conversión}
		Una vez que tenga una base estable de atención (ya sea que se mida en el tráfico del sitio o de otra manera), querrá observar de cerca la tasa de conversión: el porcentaje de prospectos que se convierten en clientes. La forma clásica de aumentar la tasa de conversión es a través de pruebas midiendo un intento de redacción publicitaria 

	    \paragraph{Aumentar el precio medio de venta}
		Si puede aumentar el precio de venta promedio por pedido, esto aumentará su resultado final, al igual que lo hará el aumento del tráfico o la conversión. Puede hacer esto más fácilmente a través de ventas adicionales, ventas cruzadas y ventas después de la venta. 

	    \paragraph{Vende más a clientes existentes}
		Es probable que sus clientes existentes respondan a ventas, promociones u ofertas adicionales de cualquier tipo. Si se comunica con ellos con más frecuencia, es casi seguro que obtendrá ingresos adicionales. Deberá tener cuidado de no presionarlos demasiado, pero la clave es el equilibrio: sus clientes quieren saber de usted. Te han dado dinero a cambio de algo que valoran. Haz que sea fácil para ellos hacerlo una y otra vez.

	\section{Modificando su camino hacia el banco: todos los detalles}
		A continuación, se muestran algunos ejemplos comunes de ajustes basados en acciones.

	    \paragraph{Crea un salón de fama}
		Destaque a sus mejores clientes; Permítales que cuenten sus propias historias.

	    \paragraph{Instituir una nueva venta superior}
		Agregar una buena oferta de venta adicional, o varias, es probablemente la estrategia más fácil y poderosa que puede usar para aumentar el tamaño promedio de su pedido.

	    \paragraph{Fomentar las referencias}
		La mayoría de la gente sabe que el boca a boca es la mayor fuente de nuevos negocios, pero en lugar de esperar a que suceda algo, puede alentar a sus clientes a correr la voz. 
		
	    \paragraph{Realiza un concurso}
		Algunas personas se motivan mucho con los concursos y los sorteos. Encuentra una forma de regalar un premio e invitar a la gente a competir. Cuanto mayor sea el premio o más exclusivo sea el concurso, mejor. Es posible que no obtenga muchas ventas de un concurso, pero atraerá más atención y una mayor audiencia para futuras ventas.

	    \paragraph{Introduzca la garantía mas poderosa que pueda pensar}
		La mayoría de las empresas tienen garantías aburridas: si no le gusta esto, le devolveremos su dinero. Pero cuando compramos algo, nuestro dinero no es lo único que nos preocupa. También nos preocupa el tiempo y la validación. Si tengo que devolver algo, ¿será un dolor de cabeza? Haga que sea lo opuesto a un dolor en el trasero: algunas empresas ofrecen una garantía del 110 por ciento, lo que garantiza que la carga recaiga sobre la empresa.

	    \paragraph{Alternativamente, haga un gran acuerdo sobre no ofrecer garantía}
		Tenga en cuenta que esta estrategia suele funcionar mejor para productos de alta gama. Probablemente disminuirá las ventas generales, pero aumentará el nivel de compromiso de quienes compran. Irónicamente, las personas que pagan por productos de alta gama tienden a ser mejores clientes en general. $"$Los compradores que pagan poco son los peores$"$

	\section{Producto a servicio, servicio a producto}
	    Otra cosa fácil que pueden hacer muchas empresas existentes para agregar una nueva fuente de ingresos rápidamente es crear un servicio a partir de una empresa basada en productos o crear un producto a partir de una empresa basada en servicios.\\
	    Puede decirles a los clientes potenciales: Oigan, mi servicio cuesta mucho dinero porque todo está personalizado. Pero si solo necesita una solución general, puede obtener esta versión por mucho menos. Algunos clientes seguirán queriendo la solución personalizada, pero de esta manera no cerrará la puerta a otros a quienes les gusta la idea pero que no pueden pagar el trabajo de alta gama.\\
	    Sea amable con la gente y brinde un gran servicio puede que no parezca una gran diferencia, pero todas estas cosas se suman. Ya sea que tenga una tienda minorista o no,

	\section{Nota para los proveedores de servicios: aumente los precios con regularidad}
	    Es de esperar que un aumento de precio tenga una tendencia a filtrar a algunos clientes del negocio mientras compensa la pérdida con un ingreso general más alto. A veces, este es el caso, pero muchos de los proveedores de servicios con los que hablé se sorprendieron de que casi nadie se fuera después de un aumento.\\
	    \textbf{Recuerde fijar el precio en función del valor, no del tiempo.}

	\section{La mejor estrategia de redes sociales: habla de ti mismo}	
	    ¿De qué debería hablar en línea? Es simple: hable sobre usted y su negocio. De Verdad. Si a las personas no les gusta lo que haces o dices, pueden dejar de seguirte, pero lo más probable es que ganes muchos más seguidores de los que pierdes. \\
	    \textbf{Puede hacer crecer un negocio de dos maneras: horizontalmente, ampliando y creando diferentes productos para aplicar a diferentes personas, o verticalmente, profundizando y creando más niveles de compromiso con los clientes.}

    \chapter{Como obtener una franquicia}
	\section{Eres solo una persona o quizás dos}
	    Un camino para obtener una franquicia es asociarse con un socio de confianza. Esto no significa que fusione completamente su negocio con esa persona; de hecho, la forma más fácil y común de asociarse con alguien es crear una empresa conjunta. En este arreglo, dos o más personas unen fuerzas para colaborar en un solo proyecto nuevo. En otros acuerdos, se crea un negocio completamente nuevo que es propiedad conjunta de los socios.\\
	    Aquí hay una lista abreviada de decisiones que debe tomar al comienzo de cualquier empresa conjunta
	    \begin{enumerate}
		    \item ¿Cómo se dividirá el dinero? (Las divisiones comunes incluyen un 50-50, 60-40 y la parte más alta va para el socio que hace más trabajo, y 45-45 con un 10 por ciento reservado para costos administrativos
		    \item ¿Cuáles son las responsabilidades de cada socio?
		    \item ¿Qué tipo de información se comparte entre socios?
		    \item ¿Cómo se comercializará el proyecto conjuntamente?
		    \item ¿Cuánto tiempo estará vigente nuestro acuerdo?
		    \item ¿Con qué frecuencia nos pondremos en contacto para discutir la asociación?
	    \end{enumerate}

    \chapter{Yendo largo}

	    
\end{document}
