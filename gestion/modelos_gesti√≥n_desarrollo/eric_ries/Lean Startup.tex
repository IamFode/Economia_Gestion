\documentclass[10pt]{article}
\usepackage[text=17cm,left=2.5cm,right=2.5cm, headsep=20pt, top=2.5cm, bottom = 2cm,letterpaper,showframe = false]{geometry} 	
\usepackage{latexsym,amsmath,amssymb,amsfonts}	%(símbolos de la AMS).7
\parindent = 0cm 								%sangria
\usepackage{lmodern}							% tipos de letras
\usepackage[T1]{fontenc}						%acentos en español
\usepackage[spanish]{babel}
\usepackage{titlesec} %formato de títulos
\pagestyle{empty}								%elimina numeración de página
\usepackage{multicol}
\usepackage{color} %resaltar texto
\usepackage{enumerate}

\begin{document}
\begin{center}
\huge Lean Startup\\
\vspace*{0.5cm}
\large Eric Ries.\\
\vspace{1cm}
\Large Apuntes por Fode.
\vspace{1.5cm}
\end{center}

\section*{Introducción}
La cruda realidad es que la mayoría de las startups fracasan.\\
El éxito de una startup no es consecuencia de buenos genes o de estar en el lugar correcto en el momento adecuado. El éxito de una startup se puede diseñar siguiendo el proceso correcto y esto significa que se puede aprender y, por lo tanto, se puede enseñar.

\section*{Principios del método Lean Startup}
\begin{enumerate}
\item Los emprendedores están en todas partes.
\item El espíritu emprendedor es management.
\item Aprendizaje validado.
\item Crear-Medir-Aprender.
\item Contabilidad de la innovación.
\end{enumerate}
\subsection*{Por qué fracasan las startup?}
\begin{enumerate}
\item El primer problema es el atractivo de un buen plan, una estrategia sólida y una concienzuda investigación de mercado.
\item Adoptando las ideas de la escuela del "simplemente hazlo".
\end{enumerate}
\part*{\center Ver}
\begin{multicols}{2}
\section*{Comenzar}
\subsection*{Management emprendedor}
La creación de una startup es un ejercicio de creación de una institución; por lo tanto, requiere management.\\
el espíritu emprendedor requiere una disciplina de gestión para aprovechar la oportunidad empresarial que se le ha dado.
\subsection*{Las raíces del método Lean Startup}
El pensamiento Lean altera radicalmente la forma de organizar las cadenas de oferta y los sistemas de producción. Entre sus principios están:
\begin{itemize}
\item Diseño de conocimiento.
\item Creatividad de los trabajadores.
\item Derucción de las dimensiones de los lotes.
\item La pruducción just-in-time.
\item Control de inventarios.
\item Aceleración del tiempo del ciclo.
\end{itemize}
El método Lean Startup adapta estas ideas al contexto del espíritu emprendedor, proponiendo a los emprendedores que juzguen su progreso de una forma diferente a como lo hacen otro tipo de empresas.\\
El método Lean Startup pide a la gente que empiece a medir su productividad de otra forma.\\
{\color{blue}El objetivo de una startup es averiguar qué debe producirse, aquello que los consumidores quieren y por lo que pagarán, tan rápidamente como sea posible. En otras palabras, el método Lean Startup es una nueva forma de ver el desarrollo de productos innovadores que enfatiza la rápida iteración y la comprensión de los consumidores, una enorme visión y una gran ambición, todo al mismo tiempo.}\\
El motor de crecimiento de las Startup son parecidas al motor de un automóvil\\
Cada nueva versión de un producto, cada nueva característica y cada nuevo programa de marketing es un intento de mejorar este motor de crecimiento. Gran parte del tiempo en la vida de una startup transcurre poniendo a punto el motor a través de mejoras en los productos, el marketing o las operaciones.\\
El segundo feedback es el de conducir como un conductor conduce un automovil y no así como un cohete predeterminado para seguir una ruta.\\
{\color{blue}En lugar de hacer planes complejos basados en muchas asunciones, se pueden hacer ajustes constantes con un volante llamado circuito de feedback de Crear-Medir-Aprender.}\\
A través de este proceso de dirección, podemos aprender cómo saber cuándo y si ha llegado el momento de hacer un giro drástico llamado pivote o si debemos «perseverar» en nuestra trayectoria actual. Cuando tenemos el motor revolucionado, el método Lean Startup ofrece mecanismos para que el negocio se amplíe y crezca a la máxima velocidad.\\
Los productos cambian constantemente a través del proceso de optimización, lo que yo llamo girar el motor.\\
Lo único que no cambia es la misión de la empresa\\
\begin{center}
\begin{tabular}{r c l}
Producto&$\rightarrow$&Optimización\\
Estrategia&$\rightarrow$&Pivote\\
Visión&$\rightarrow$&Lo único que no cambia\\
\end{tabular}
\end{center}
En la vida real, una startup es una cartera de actividades. Pasan muchas cosas simultáneamente:
\begin{itemize}
\item El motor funciona.
\item Se crean nuevos clientes.
\item Se sirve a los existentes.
\item Giramos.
\item Intentamos mejorar nuestro producto.
\item El marketing.
\item Las operaciones.
\item Conducimos.
\item Decidimos y pivotar y cuando hacerlo.
\end{itemize}
El reto es equilibrar todas éstas actividades\\
{\color{blue}El motivo es que, en el management general, un fracaso a la hora de dar resultados se debe a la incapacidad de planificar adecuadamente o a la incapacidad para ejecutar el plan.}
\section*{Definir}
El método Lean Startup es un conjunto de prácticas que ayuda a los emprendedores a incrementar las probabilidades de crear una startup con éxito. Para dejar las cosas claras, es importante definir qué es una startup:\\
{\color{blue}Una startup es una institución humana diseñada para crear un nuevo producto o servicio bajo condiciones de incertidumbre extrema.}\\
La palabra institución connota burocracia, proceso e incluso letargo. Aun así, las startups con éxito incluyen muchas actividades asociadas a la construcción de una institución:
\begin{itemize}
\item Contratar empleados creativos.
\item Coordinar sus actividades.
\item Crear una cultura corporativa que ofrezca resultados.
\end{itemize}
Las startups usan muchos tipos de innovaciones:
\begin{itemize}
\item Nuevos descubrimientos científicos.
\item Reutilización de la tecnología existente para usos nuevos.
\item Idear un nuevo modelo de negocio que libere el valor que estaba escondido.
\item Llevar un nuevo producto o servicio a un sitio nuevo o a un grupo de consumidores previamente desatendido.
\end{itemize}
{\color{blue}La innovación es algo descentralizado, impredecible y que va de abajo arriba, pero esto no significa que no pueda gestionarse. Se puede, pero para hacerlo se requiere una nueva disciplina de management que necesita aplicarse no sólo a través de emprendedores que buscan crear el siguiente gran producto, sino también a través de la gente que les apoya, les nutre y les evalúa.}
De hecho, creo que, si quiere crecer a largo plazo, el único camino sostenible para una empresa es crear una fábrica de innovación que use las técnicas del método Lean Startup para crear innovaciones disruptivas de forma continua.
\section*{Aprender}
Los emprendedores, bajo presión para tener éxito, son extremadamente creativos cuando llega el momento de demostrar lo que han aprendido. Todos podemos explicar una buena historia cuando nuestro trabajo, carrera o reputación dependen de ello.\\
Debemos saber qué elementos de nuestra estrategia están funcionando para llevar a cabo nuestra visión y cuáles son disparatados. Debemos saber qué es lo que los consumidores quieren, no lo que ellos dicen que quieren o lo que nosotros creemos que deberían querer.\\
En el modelo del método Lean Startup, estamos rehabilitando el aprendizaje con un concepto al que yo llamo {\color{blue}aprendizaje validado.}\\
Es un método riguroso para mostrar hacia dónde seguir cuando uno está metido en la tierra de la extrema incertidumbre en la que crecen las startups.\\
El aprendizaje validado es el proceso para demostrar empíricamente que un equipo ha descubierto información valiosa sobre las posibilidades presentes y futuras del negocio. Es más concreto, riguroso y rápido que la previsión de mercado o la planificación clásica. Es el principal antídoto para el problema letal de obtener un fracaso: la ejecución con éxito de un plan que nos lleva a ninguna parte.
\subsection*{El aprendizaje validado}
Nos hacemos estas preguntas
\begin{itemize}
\item ¿Qué deberíamos producir y para quién?.
\item ¿En qué mercados podríamos entrar y dominar?.
\item ¿Cómo podríamos crear un valor duradero que no estuviera sujeto a la erosión por parte de la competencia?.
\end{itemize}
\subsubsection*{Valor vs. desplifarro}
¿cuáles de nuestros esfuerzos estaban creando valor y cuáles eran un despilfarro? Esta pregunta es el centro de la revolución del Lean manufacturing.\\
El aprendizaje es la unidad esencial para medir el progreso de una startup. Aquel esfuerzo que no es necesario para saber qué quieren los consumidores puede eliminarse. Yo llamo a eso aprendizaje validado porque siempre se puede demostrar a través de mejoras en los principales indicadores de la startup.
\subsubsection*{¿Donde encontramos validación?}
Cuando empezamos a entender a nuestros consumidores fuimos capaces de mejorar nuestros productos.
\subsubsection*{Lecciones Lean Startup}
El método Lean Startup no es una colección de tácticas individuales. Son unos principios para abordar la cuestión del desarrollo de un nuevo producto. La única manera de encontrar el sentido a sus recomendaciones es entender los principios subyacentes con los que trabaja.\\
{\color{blue}el único camino que seguir es aprender a ver cada startup, en cualquier sector, como un gran experimento.}\\
Las preguntas más pertinentes son ¿Debería crearse este producto? y ¿Podemos crear un modelo de negocio sostenible partiendo de este conjunto de productos y servicios?. Para responder a estas preguntas, debemos encontrar un método para desglosar el plan de negocio en sus componentes y probar cada parte empíricamente.
\section*{Experimentar}
\subsection*{De la alquimia a la ciencia}
El sistema del método Lean Startup redefine los esfuerzos de una startup como experimentos que prueban sus estrategias para ver qué partes son brillantes y cuáles son descabelladas.\\
{\color{blue}Un experimento de verdad sigue el método científico.}
{\color{blue}Empieza con una hipótesis que hace predicciones sobre lo que supuestamente pasa. Entonces prueba empíricamente estas predicciones. Del mismo modo que la experimentación científica se basa en la teoría, la experimentación de la startup se guía por su visión. El objetivo de cada experimento de l a startup es descubrir cómo crear un negocio sostenible a partir de esa visión.}\\
Por ejemplo podríamos poner como hipótesis ¿Hay demanda suficiente para una experiencia superior en la compra de zapatos online?
\subsubsection*{Desglósalo}
Las dos asunciones más importantes que hacen los emprendedores son:
\begin{itemize}
\item Hipótesis del valor.
\item Hipótesis del crecimiento.
\end{itemize}
la hipótesis del valor prueba si un producto o servicio proporciona valor a los clientes cuando lo usan podríamos hacernos preguntas como:
\begin{itemize}
\item ¿Cuantos vuelven a apuntarse?, ya que lo encuentra valioso
\end{itemize}
{\color{blue}{La hipótesis del crecimiento} que prueba cómo los nuevos clientes descubren un producto o servicio, haciendo preguntas como:
\begin{itemize}
\item ¿Como se expande el servicio o producto?.
\item ¿los primeros participantes difundirán activamente el programa.
\end{itemize}}
Los primeros usarios tienden a perdonar los errores y están dispuestos a proporcionar feedback.
\subsection*{Un experimento es un producto}
En el modelo del método Lean Startup, un experimento es más que una simple investigación teórica; también es un primer producto.\\
intento presionar a mi equipo para que primero respondan a cuatro preguntas:
\begin{enumerate}
\item ¿Los consumidores reconocen que tienen el problema que intentáis solucionar?.
\item Si hubiera una solución, ¿la comprarían?
\item ¿Nos la comprarían a nosotros?.
\item ¿Podemos crear una solución para este problema?.
\end{enumerate}
El éxito no es entregar algo; el éxito es aprender cómo solucionar los problemas del cliente.
\end{multicols}
\part*{\center Dirigir}
\begin{multicols}{2}
{\color{blue}En su interior, una startup es un catalizador que
transforma las ideas en productos.}\\
Para las startups, la información es más importante que el dinero.\\
El modelo del método Lean Startup es:
\begin{enumerate}
\item Crear.
\item Medir.
\item Aprender.
\end{enumerate}
{\color{blue}Necesitamos centrar nuestras energías en minimizar el tiempo total a lo largo de este circuito de feedback de información.}
{\color{blue} \subsection*{Producto Mínimo Viable (PMV)} es aquella versión del producto que permite dar una vuelta entera al circuito de Crear-Medir-Aprender con un mínimo esfuerzo y el mínimo tiempo de desarrollo.} Para tal efecto debemos:
\begin{itemize}
\item Ser capaces de medir su impacto.
\item Necesitamos ponerlo delante de consumidores potenciales para evaluar sus reacciones.
\item También podemos necesitar venderles el prototipo.
\end{itemize}

\subsection*{Contabilidad de la innovación}
Es un enfoque cuantitativo que nos permite ver si nuestros esfuerzos de ajuste del motor están dando frutos.También nos permite crear hitos de aprendizaje.
\subsection*{Pivote}
Nos enfrentaremos a la coyuntura más difícil que tiene que superar un emprendedor: si pivotar de la estrategia inicial o perseverar. Si hemos descubierto que una de nuestras hipótesis es falsa, ha llegado el momento de hacer un gran cambio hacia otra hipótesis estratégica.\\El método Lean Startup crea empresas eficientes en el uso del capital porque permite a las startups reconocer pronto que es el momento de pivotar, consiguiendo un menor despilfarro de tiempo y dinero.
\section*{Saltar}
{\color{blue} Según todas las fuentes, lo que más impresionó a los inversores fueron dos de los hechos del crecimiento inicial de Facebook. El primero era la cantidad de tiempo que los usuarios activos de Facebook pasaban en la web. Más de la mitad de los usuarios consultaba la página cada día. Éste es un ejemplo de cómo una empresa puede validar su hipótesis de que los consumidores encuentran valioso el producto.\\
 El otro aspecto impresionante de los primeros años de Facebook es la tasa a la cual habían conseguido implantarse en sus primeros campus universitarios. La tasa de crecimiento era asombrosa: Facebook fue lanzado el 4 de febrero de 2004 y al final del mes casi tres cuartas partes de los estudiantes de Harvard lo estaban usando, sin que hubieran gastado ni un dólar en marketing o publicidad.}\\
En otras palabras facebook había validado su hipótesis de crecimiento. Estas dos hipótesis representan las dos cuestiones de acto de fe más importantes a las que se enfrenta cualquier nueva startup.\\
Cada startup necesita realizar experimentos que le ayuden a determinar qué técnicas funcionarán en sus circunstancias únicas. Para las startups, el papel de la estrategia es contribuir a descubrir qué preguntas hacerse.
\subsection*{La estrategia se basa en asunciones}
Las Startups comienzas con asunciones que deben probarlas tan rápido como se posible.\\
La mayoría de actos de fe toman la forma de argumentos por analogía.
{\color{blue}asegurémonos de que hay consumidores hambrientos ahí fuera que desean adoptar nuestra nueva tecnología.}
\subsubsection*{Analogías y antilogías}
No hay nada intrínsecamente incorrecto en basar la estrategia en comparaciones con otras empresas e industrias. \\
Una analogía es saber que otro producto similar tuvo éxito. y una antilogía es que otro producto tuvo éxito pero sin ganancias. Y ahí es donde los actos de fe entran en acción para una startup. 
\subsection*{Mas allá del sitio justo en el momento adecuado} 
Lo que diferencia una historia de éxito de una de fracaso es que los emprendedores con éxito tuvieron la previsión, la habilidad y las herramientas necesarias para descubrir qué partes de sus planes estaban funcionando de forma brillante y cuáles estaban desencaminadas, y adaptaron su estrategia en función de esto.
\paragraph*{Valor y crecimiento}
Hay dos actos de fe que están por encima de los demás:
\begin{itemize}
\item La hipótesis de creación de valor.
\item La hipótesis de crecimiento.
\end{itemize}
El primer paso para entender un producto o servicio es descubrir si es creador o destructor de valor. Un ejemplo de destructor de valor sería una empresa que crece continuamente a través de la obtención de capital de los inversores y que gasta mucho en publicidad, pero que no desarrolla un producto creador de valor. Este tipo de negocios está llevando a cabo una apariencia de crecimiento para que parezca que tiene éxito. \\
{\color{blue} Sal de tu casa y comienza a aprender.}
\subsection*{Genchi gembutsu (ir al lugar del problema y verlo por nosotros mismos)}
La importancia de basar las decisiones estratégicas en el conocimiento de primera mano de los clientes es uno de los principios básicos subyacentes en el sistema de producción de Toyota. \\
El primer contacto de una startup con sus primeros clientes potenciales revela qué asunciones han de ser probadas con más urgencia.
\subsubsection*{Salir del edificio}
los hechos que necesitamos deducir sobre los consumidores, mercados, suministradores y canales sólo se producen fuera del edificio.\\
{\color{blue} El primer paso de este proceso es confirmar que tus preguntas de acto de fe se basan en la realidad, que el consumidor tiene un problema significativo que vale la pena solucionar}
\subsection*{Diseño y el consumidor arquetipo}
El objetivo de este primer contacto con los consumidores no es obtener respuestas definitivas. En lugar de eso, lo que se pretende es aclarar a un nivel básico, burdo, si entendemos a nuestro cliente potencial y los problemas que tiene. Para crear un consumidor arquetipo. Este arquetipo es una guía esencial para el desarrollo del producto, donde reconocen que el arquetipo de cliente es una hipótesis, no un hecho.
\subsubsection*{Parálisis del análisis}
Algunos emprendedores realizan su acto de fe apresuradamente luego de hablar con algunos clientes. Y otros son víctimas del parálisis del análisis refinando constantemente sus planes.\\
{\color{blue}¿cómo saben los emprendedores cuándo dejar de analizar y empezar a crear? La respuesta es un concepto llamado producto mínimo viable.}
\section*{Probar}
un \textbf{Producto mínimo viable} ayuda a los emprendedores a empezar con el proceso de aprendizaje lo más rápidamente posible. es la forma más rápida de entrar en el circuito de feedback de Crear-Medir-Aprender con el mínimo esfuerzo.\\
{\color{blue} El Objetivo del PMV es empezar el proceso de aprendizaje, no acabarlo, para probar kas hipótesis fundamentales del negocio.}
\subsection*{Por qué no se busca que los primeros productos sean perfectos}
El producto mínimo viable varía en cuanto a su complejidad, desde pruebas de humo extremadamente simples a prototipos completos con problemas y pocos elementos. \\
La lección del PMV es que cualquier trabajo adicional más allá del que se requiere para empezar a aprender es un despilfarro, independientemente de lo importante que pareciera en ese momento.\\
\subsubsection*{El producto mínimo viable en vídeo}
Los consumidores suelen no saber lo que quieren. Por eso una forma de explicarles el funcionamiento de algún producto es a través de un vídeo y ver si las personas aceptan el producto, así se tiene un PMV.
\subsubsection*{El producto mínimo viable conserje}
Se trata de darle un servicio personal al primer cliente.
\subsubsection*{No prestar atención a las ocho personas que hay detrás de la cortina}
Otro tipo de PMV es crear varios prototipos para ver cual responde de manera positivo.
\subsection*{El papel de la calidad y el diseño de un PMV}
Permitir un trabajo descuidado en nuestro proceso inevitablemente conduce a una variabilidad excesiva.\\
Principio de calidad: Si no sabemos quién es el consumidor, no sabemos qué es la calidad. Por ende algún producto de baja calidad podría ser de alta calidad para el consumidor.\\
{\color{blue} A los consumidores no les importa cuánto tiempo se tarde en crear algo. Sólo les interesa si es útil para sus necesidades.}\\
{\color{blue}Si los consumidores reaccionan como esperamos que lo hagan, podemos tomar esto como una confirmación de que nuestras asunciones eran correctas. Si lanzamos un producto diseñado modestamente y los consumidores (incluso los primeros usuarios) no pueden descubrir cómo usarlo, esto confirmará que necesitamos invertir en un diseño mejor. Pero siempre hay que preguntarse: ¿qué pasa si no les importa el diseño de la misma manera que a nosotros?}\\
{\color{blue} Cuando vaya a crear su propio producto mínimo viable, siga esta simple regla: elimine cualquier elemento, proceso o esfuerzo que no contribuya directamente al aprendizaje que está buscando.}
\subsection*{Obstáculos en la construcción de un PMV}
Los obstáculos más frecuentes son:
\begin{itemize}
\item  las cuestiones legales, 
\item el miedo a los competidores,
\item los riesgos como marca y 
\item el impacto en la moral del equipo.
\end{itemize}
Parte del reto especial de ser una startup es la casi imposibilidad de que alguien se entere de tu idea, de tu empresa o de tu producto, y mucho menos un competidor.\\
Si fracasa un PMV, es probable que los equipos pierdan la esperanza y abandonen el producto. Pero es un problema que se puede remediar.
\subsection*{Del PMV a la contabilidad de la innovación}
La solución a este dilema es comprometerse a iterar. {\color{blue} Antes de empezar, debe alcanzarse un acuerdo cerrado en el que se afirme que no importa el resultado del PMV: no se perderá la esperanza.} Antes de empezar, debe alcanzarse un acuerdo cerrado en el que se afirme que no importa el resultado del PMV: no se perderá la esperanza. \\
Necesitamos un enfoque disciplinado y sistemático para saber si estamos progresando y descubrir si estamos obteniendo aprendizaje validado. Yo llamo a este sistema\textbf{ la contabilidad de la innovación}, una alternativa a la contabilidad tradicional diseñada especialmente para las startups. 
\section*{Medir}
El trabajo de una startup es:
\begin{enumerate}
\item medir rigurosamente dónde está en el momento actual, afrontando la dura verdad que revele esa evaluación.
\item Y diseñar experimentos para descubrir cómo hacer avanzar las cifras reales hacia el ideal reflejado en su plan de negocios.
\end{enumerate}
\subsection*{¿Por qué algo aparentemente tan aburrido como la contabilidad cambiará tu vida?}
Responderemos con otras preguntas: ¿Cómo sabemos que los cambios que se han hecho están relacionados con los resultados que vemos? Y lo más importante, ¿cómo sabemos que estamos extrayendo las lecciones correctas a partir de estos cambios?
\subsubsection*{Un esquema de valoración que funciona en todas las industrias}
{\color{blue} La contabilidad de la innovación permite a las startups demostrar objetivamente que están aprendiendo a construir un negocio sostenible.}\\
La contabilidad de la innovación empieza convirtiendo las asunciones de acto de fe que se han discutido en el capítulo 5 en un modelo financiero cuantitativo. \\
Este modelo proporciona asunciones sobre qué aspecto tendrá el negocio cuando llegue al éxito en el futuro.\\
{\color{blue} La tasa de crecimiento depende fundamentalmente de tres aspectos:
\begin{itemize}
\item La rentabilidad que se obtiene por cliente.
\item El coste de adquirir nuevos clientes.
\item La tasa de repetición en la compra por parte de los clientes existentes.
\end{itemize}}
Cuanto mayores sean la rentabilidad por cliente y la tasa de repetición y menor sea el coste de adquirir clientes, más rápidamente crecerá la empresa y más rentable será. Éstos son los motores del crecimiento de la empresa.
\subsection*{Como funciona la contabilidad de la innovación: tres hitos de aprendizaje}
{\color{blue}La contabilidad del crecimiento funciona en tres etapas:
\begin{enumerate}
\item Usar un producto mínimo viable para recopilar datos reales sobre en qué punto se encuentra la empresa en el momento actual.
\item Las startups deben intentar poner a punto el motor para ir desde el punto de partida hasta el ideal.
\item Pivotar o perseverar.
\end{enumerate}}
Cuando una empresa pivota, empieza todo el proceso otra vez, restableciendo un nuevo punto de partida y preparando el motor desde allí.
\subsubsection*{Establecer el punto de partida}
{\color{blue}Antes de crear el prototipo, la empresa debería realizar una prueba de humo con sus materiales de marketing.Es una vieja técnica de marketing directo en la que se da a los consumidores la oportunidad de encargar un producto que todavía no se ha creado.}\\
\textbf{Una prueba de humo sólo mide un aspecto: si los consumidores están interesados en probar un producto.}\\
Sin embargo, puede ser muy útil para obtener feedback para esta asunción antes de gastar más dinero y recursos en el producto.\\
Un PMV permite a una startup proporcionar datos reales sobre su punto de partida para su modelo de crecimiento.\\
Cuando uno escoge entre las muchas asunciones de un plan de negocio, tiene sentido probar primero las asunciones más arriesgadas. Si no puede mitigar estos riesgos para llegar al ideal que se requiere para un negocio sostenible, no tiene sentido probar lo demás.
\subsubsection*{Poner el motor a punto}
Cada startup debería tener el objetivo de mejorar uno de los factores clave del modelo de crecimiento, {\color{blue}para a demostrar el aprendizaje validado, los cambios en el diseño deberían mejorar la tasa de activación de los nuevos clientes.  un buen diseño es el que hace cambiar el comportamiento del consumidor para mejor.}
\subsubsection*{Pivotar o perseverar}
Si no se cambian los factores clave de crecimiento del modelo de negocio, no se progresará. Es un signo de que ha llegado el momento de pivotar.
\subsubsection*{Mejorando un producto por 5 dolares al día}
Utilizamos indicadores embudo para nuestro motor de crecimiento:
\begin{itemize}
\item Registros de clientes, 
\item descargas de nuestra aplicación, 
\item prueba, 
\item uso repetido y 
\item compra. 
\end{itemize}
\subsubsection*{Análisis de cohortes}
A pesar de que suena complejo, se basa en una premisa muy simple. En lugar de mirar los totales acumulados o las cifras brutas como ingresos totales o número total de clientes, debe analizar el comportamiento de cada grupo de consumidores que entra en contacto con el producto de forma independiente.\\
Resultados cuantitativos pobres nos obligan a asumir el fracaso y crear la motivación, el contexto y el espacio para llevar a cabo más investigación cualitativa. Esta investigación produce nuevas ideas, nuevas hipótesis que hay que probar, llevando a un posible pivote. Cada pivote desbloquea nuevas oportunidades para más experimentación y el ciclo se repite. Cada vez repetimos este simple ritmo: establecer el punto de partida, poner a punto el motor y tomar la decisión de pivotar o perseverar.
\subsection*{Optimización vs. aprendizaje}
Si está construyendo algo equivocado, optimizar el producto o su comercialización no obtendrá resultados significativos. Una startup debe medir su progreso con un listón más alto: la evidencia de que se puede crear un negocio sostenible a partir de sus productos o servicios. \\
{\color{blue} SE DEBE CREAR UN NEGOCIO SOSTENIBLE}
\subsection*{Los indicadores vanidosos: una precaución}
No debemos caer en indicadores vanidosos como los resultados acumulativos.
\subsection*{Indicadores accionables vs indicadores vanidosos}
 El sistema ágil es un sistema de desarrollo eficiente desde el punto de vista de quienes desarrollan el producto. Les permite centrarse en crear elementos y en el diseño técnico. \\
{\color{blue}Un equipo disciplinado puede aplicar una metodología errónea, pero puede cambiar de rumbo rápidamente cuando descubre su error. Y lo más importante, un equipo disciplinado puede experimentar con un estilo de trabajo propio y extraer conclusiones importantes.
} 
\subsubsection*{Cohortes y Split-tests}
{\color{blue}Un experimento split-test es en el que se ofrecen diferentes versiones de un producto al mismo tiempo. \\
 Si quiere probar un diseño de catálogo, puede mandar la nueva versión al 50 \% de los consumidores y mandar el viejo catálogo al otro 50 \%. Para asegurarse de que va a obtener un resultado científico, ambos catálogos deberían contener productos idénticos; la única diferencia sería el cambio en el diseño. Para descubrir si el nuevo diseño es efectivo, todo lo que debería hacer es seguir las cifras de ventas para cada grupo de consumidores. Esta técnica se llama A/B testing} este método ahorra una gran cantidad de tiempo a largo plazo.\\
El split-testing también ayuda a los equipos a refinar su comprensión de lo que quieren y no quieren los consumidores.
\subsubsection*{Kanban}
Las historias de usuarios no se consideraban completas hasta que no habían conducido al aprendizaje validado. Así las historias podrían catalogarse en cuatro estadios de desarrollo:
\begin{itemize}
\item Productos pendientes.
\item Productos activamente creados.
\item Productos acabados.
\end{itemize}
\paragraph*{Diagrama kanban de trabajo a medida que se progresa de etapa en etapa}

\begin{center}
\begin{tabular}{c c c c}
\textbf{Pendientes}&\textbf{En curso}&\textbf{Acabados}&\textbf{Validados}\\
\hline
A&D&F&\\
B&E&&\\
C&&&\\
\end{tabular}
\end{center}
\subsection*{El valor de las tres A}
\begin{itemize}
\item Accionable.
\item Accesible
\item Auditable.
\end{itemize}
\subsubsection*{Accionable}
Para que un informe se considere accionable, debe demostrar una clara relación de causa-efecto. De lo contrario, es un indicador vanidoso.\\
Los humanos tienen un talento innato para aprender cuando reciben una evaluación clara y objetiva.
\subsubsection*{Accesible}
Se debe  hacer los informes tan simples como sea posible para que todo el mundo los entienda.\\
Cada análisis de cohorte dice: Entre la gente que usó nuestro producto durante este período, ésta es la proporción que mostró cada uno de los comportamientos que nos interesan. {\color{blue}el informe trata de la gente y de sus acciones, algo mucho más útil que montones de datos.}
\subsubsection*{Auditable}
Necesitamos asegurarnos de que los datos son creíbles para nuestros empleados.\\
Recordar que los indicadores también son personas». Debemos ser capaces de probar los datos a mano, en el caos del mundo real, hablando con los consumidores. Es la única manera de ser capaces de comprobar si los informes contienen datos reales. Los directivos necesitan tener la capacidad de hacer una comprobación in situ de los datos con los consumidores reales. Esto también tiene un segundo beneficio: los sistemas que proporcionan este nivel de auditabilidad dan a los directivos y a los emprendedores la oportunidad de obtener ideas sobre por qué los consumidores se comportan como indican los datos.
{\color{blue}En segundo lugar, la elaboración de informes debería asegurar que el mecanismo que genera los informes no sea demasiado complejo.}
\section*{Pivotar (o perseverar)}
¿Estamos haciendo progresos suficientes como para creer que nuestra hipótesis estratégica inicial es correcta o debemos hacer un cambio importante? Este cambio se llama {\color{blue}pivote: una corrección estructurada diseñada para probar una nueva hipótesis básica sobre el producto, la estrategia y el motor de crecimiento.} 
\subsection*{La contabilidad de la innovación lleva a pivotes más rápidos}
Quedarse atascado en la tierra de los muertos vivientes. Sucede cuando una empresa ha alcanzado un éxito modesto, suficiente para seguir viva, pero que no está a la altura de las expectativas de sus fundadores e inversores.\\
{\color{blue}El fracaso es un prerrequisito del aprendizaje\\
Un pivote requiere que se mantenga un pie anclado en lo que se ha aprendido hasta el momento, mientras se hace un cambio fundamental en la estrategia para buscar un mayor aprendizaje validado.}\\
{\color{red}Pivote de acercamiento: Simplificar a un producto mas especifico.\\
 Pivote de segmento de mercado: manteniendo la funcionalidad del producto pero cambiando el foco de audiencia.\\
Pivote de plataforma: Crear una plataforma de ventas de autoservicio donde cualquiera pudiera convertirse en consumidor usando la tarjeta de crédito.}
\subsection*{La pista de una startup es el nombre de pivotes que todavía puede dar}
{\color{blue}Una startup con un millón de dólares en el banco que gasta 100.000 dólares al mes tiene una pista proyectada de diez meses\\
Cuando en las startups empieza a escasear el efectivo, pueden extender la pista de dos maneras:
\begin{itemize}
\item Reduciendo costes o,
\item consiguiendo fondos adicionales.
\end{itemize}}
Pero cuando los emprendedores recortan los costes de forma indiscriminada, tanto pueden recortar los que permiten a la empresa entrar en su circuito de feedback de Crear-MedirAprender como recortar el despilfarro. Si los recortes tienen como resultado una ralentización del circuito de feedback, todo lo que han conseguido es contribuir a que la empresa deba abandonar el negocio más lentamente.
{\color{blue}, la startup debe averiguar cómo alcanzar la misma cantidad de aprendizaje validado con un coste menor y en menos tiempo.}
\subsection*{Los pivotes requieren coraje}
A un emprendedor le puede ser difícil pivotar, esto ocurre por tres razones:
\begin{enumerate}[\bfseries 1.]
\item  los indicadores vanidosos pueden hacer que los emprendedores lleguen a falsas conclusiones y vivan en su mundo.
\item cuando un emprendedor no tiene una hipótesis clara es casi imposible experimentar un fracaso total, y sin fracaso no suele haber el impulso necesario para embarcarse en el cambio radical que requiere un pivote.
\item muchos emprendedores tienen miedo. Reconocer el fracaso puede llevar a una moral peligrosamente baja. El mayor miedo de los emprendedores no es que su visión no sea correcta. Más aterradora es la idea de que la visión pueda considerarse incorrecta sin haber tenido una oportunidad real para demostrarse.
\end{enumerate}
\subsection*{La reunión sobre pivotar o perseverar}
La decisión de pivotar requiere ver las cosas claras y tener un objetivo en mente. \\
Los signos reveladores de la necesidad de pivotar son:
\begin{itemize}
\item La efectividad decreciente de los experimentos con el producto.
\item La sensación
generalizada de que el proceso de desarrollo del producto debería ser más productivo.
\end{itemize}
La decisión de pivotar debe dirigirse de forma estructurada. Por ello se debe tener reuniones regulares, para tomar dicha decisión.\\
{\color{blue}El equipo de desarrollo de producto debía llevar un informe completo con los resultados de sus esfuerzos de optimización de producto a lo largo del tiempo (no sólo del período previo), así como una comparación de cómo estos resultados cuadraban con las expectativas. Los directivos deberían llevar los registros de sus conversaciones con los clientes actuales y potenciales.}\\
Cuando se pivotara no es necesario empezar de nuevo, si no que trata de replantearse lo que se ha creado hasta el momento y lo que se ha aprendido para encontrar una dirección más positiva.
\subsection*{La incapacidad de pivotar}
Cuando se tiene tanto éxito con los primeros esfuerzos que se ignora qué había detrás. Como resultado, no nos damos cuenta de la necesidad de pivotar incluso cuando ésta nos estaba mirando a la cara.\\
 Cuando se ha alcanzado el éxito con los primeros usuarios, el objetivo debe ser vender a los consumidores mayoritarios. Los consumidores mayoritarios tienen requisitos diferentes y son mucho más exigentes.
{\color{red}Pivote de segmento de consumidores: En este pivote, la empresa se da cuenta de que el producto que está creando soluciona un problema real de los consumidores pero que estos consumidores no son los que había planeado inicialmente.}
\subsection*{Un catálogo de pivotes}
{\color{blue}Un pivote es un tipo especial de cambio, diseñado para probar una nueva hipótesis fundamental sobre el producto, el modelo de negocio y el motor del crecimiento.}
\subsubsection*{Pivote de acercamiento (zoom-in)}
En este caso, lo que antes se consideraba una característica del producto se convierte en el producto. 
\subsubsection*{Pivote de alejamiento (zoom-out)}
Es la situación inversa. A veces una única característica es insuficiente para sostener todo el producto. En este tipo de pivote, lo que se consideraba el producto entero se convierte en una simple característica de un producto mucho mayor.
\subsubsection*{Pivote de segmento de consumidor}
En este pivote, la empresa se da cuenta de que el producto que está creando resuelve un problema real para consumidores reales, pero que éstos no son el tipo de consumidores que inicialmente había planeado atender.
\subsubsection*{Pivote de necesidad del consumidor}
Como resultado de alcanzar un conocimiento del consumidor extremadamente bueno, a veces está claro que el problema que se intenta solucionar no es demasiado importante para ellos. Sin embargo, debido a esta gran intimidad con el consumidor, descubrimos otros problemas que son importantes y que pueden ser solucionados por nuestro equipo.
\subsubsection*{Pivote de plataforma}
Un pivote de plataforma se refiere a un cambio de aplicación en una plataforma o viceversa.
\subsubsection*{Pivote de arquitectura de negocio}
 En un pivote
de arquitectura del negocio, la empresa cambia de arquitectura. Algunas empresas cambian de alto margen y bajo volumen pasándose al mercado de masas (por ejemplo, el dispositivo de búsqueda de Google); otros, originariamente diseñados para el mercado de masas, se transforman y pasan a un modelo que requiere ciclos de ventas largos y costosos.
\subsubsection*{Pivote de captura de valor}
Hay muchas formas de capturar el valor que crea una empresa. Estos métodos suelen denominarse monetización o modelos de ingresos.
\subsubsection*{Pivote de motor de crecimiento}
Como veremos en el capítulo 10, hay tres motores de crecimiento que propulsan a la startup: el crecimiento viral, el crecimiento pegajoso y el crecimiento remunerado. En este tipo de pivote, una empresa cambia su estrategia de crecimiento para buscar un crecimiento más rápido o más rentable.
\subsubsection{Pivote de canal}
Un pivote de canal es el reconocimiento de que la misma solución básica puede ser suministrada a través de un canal diferente con mayor efectividad. Siempre que una empresa abandona un complejo de venta anterior para vender directamente a sus consumidores, se produce un pivote de canal.
\subsubsection*{Pivote de tecnología}
Una empresa puede descubrir una forma diferente para alcanzar una misma solución usando una tecnología completamente distinta.
\subsection*{Un pivote en una hipótesis estratégica}
La capacidad de pivotar no sustituye al pensamiento estratégico. \\
{\color{blue}Un pivote se entiende mejor como una nueva hipótesis estratégica que requiere un producto mínimo viable para probarla}\\
Un pivote no sólo es una exhortación a cambiar. Recuerde, es un tipo especial de cambio estructurado diseñado para probar una nueva hipótesis fundamental sobre el producto, el modelo de negocio y el motor de crecimiento. \\
{\color{red}En la parte II hemos analizado la idea de la startup a partir de sus actos de fe iniciales, hemos probado esta idea con un producto mínimo viable, hemos usado la contabilidad de la innovación y los indicadores accionables para evaluar los resultados y hemos tomado la decisión de pivotar o perseverar.}
\end{multicols}
\part*{\center Acelerar}
\begin{multicols}{2}
La mayoría de las decisiones a las que se enfrentan las startups no son claras. 
\begin{itemize}
\item ¿Con qué frecuencia se debería lanzar un producto?
\item ¿Hay
alguna razón para lanzarlo de forma semanal en vez de hacerlo diaria, trimestral o anualmente?
\end{itemize}
Desde el punto de vista de la eficiencia, el lanzamiento suele conllevar que se disponga de menos tiempo para dedicarlo a la creación del  producto. Sin embargo, esperar demasiado para el lanzamiento puede conducir al peor despilfarro posible: crear algo que nadie quiere.\\
Gastar demasiado puede hacernos despilfarrar un tiempo precioso que se podría haber dedicado a aprender. Gastar poco puede llevar a la incapacidad para aprovechar un éxito temprano y a ceder el liderazgo del mercado a un seguidor más rápido.\\
La primera pregunta crucial para cualquier transformación Lean es: {\color{red}¿Qué actividades crean valor y cuáles son una forma de despilfarro?} para ello respondemos a preguntas como:
{\color{red}\begin{itemize}
\item ¿Qué productos quieren los consumidores?
\item ¿Cómo crecerá nuestro negocio?
\item ¿Quiénes son nuestros clientes?
\item ¿A qué clientes deberíamos escuchar y a cuáles deberíamos ignorar?
\end{itemize}}
\section*{Formar lotes}
El enfoque de ir uno a uno es la manera más rápida de terminar el trabajo, a pesar de que parezca ineficiente. en LEAN se llama {\color{blue}Flujo de una sola pieza}.\\
La mayor ventaja de trabajar con pequeños lotes es que los problemas de calidad se pueden identificar mucho antes. 
\subsection*{Los lotes pequemos en el contexto del espíritu emprendedor}
{\color{blue}¿Qué pasa si el cliente no quiere el producto que se está creando? A pesar de que esto nunca es una buena noticia para un emprendedor, descubrirlo cuanto antes es mucho mejor que descubrirlo más tarde. Trabajar con lotes pequeños asegura que la startup puede minimizar el gasto de tiempo, dinero y esfuerzo que finalmente ha sido un despilfarro.}
\subsubsection*{Lotes pequeños}
Los pasos a seguir para desarrollar un nuevo producto tiene lugar a una cadena de montaje virtual. 
\begin{enumerate}[\bfseries P1.]
\item Los directores de producto se plantean qué características pueden gustar a los consumidores
\item los diseñadores del producto piensan entonces qué aspecto deberían tener estas características.
\item Los diseños se pasan a los ingenieros, que crean algo nuevo o modifican el producto existente
\item Una vez hecho, lo entregan al responsable de verificar que el nuevo producto funciona como querían los directores de producto
y los diseñadores.
\end{enumerate}
{\color{blue}la clave para ser capaz de operar así de rápido es la búsqueda de errores inmediatamente}
\subsubsection*{El desarrollo continuo más allá del software}
La lección esencial no es que todo el mundo debería lanzar su producto cincuenta veces al día, sino que, reduciendo el tamaño del lote, se puede entrar en el circuito de feedback de Crear-MedirAprender mucho más rápidamente que nuestros competidores. La capacidad para aprender más rápidamente cosas sobre nuestros clientes es una ventaja competitiva esencial que las startups deben poseer.
\subsection*{Tire no empuje}
La producción LEAN soluciona el problema de desabastecimiento con una técnica que se llama \textbf{tirar}. \\
El objetivo ideal es alcanzar lotes pequeños a lo largo del flujo de una única pieza en toda la cadena de producción. Cada paso en la línea tira las partes que necesita de la etapa anterior. Es el famoso método de producción just-in-time.\\
{\color{blue}Casi todas las técnicas del método Lean Startup que se han discutido hasta el momento funcionan mágicamente de dos maneras:
\begin{enumerate}
\item Convirtiendo los métodos de empujar en tirar y
\item reduciendo el tamaño de los lotes.
\end{enumerate}}
los consumidores no suelen saber qué es lo que quieren. 
\subsubsection*{La función de tirar de la hipótesis en las tecnologías limpias}
{\color{red}El método Lean Startup sólo funciona si somos capaces de crear una organización adaptable y rápida ante los retos a los que se enfrenta.}
\section*{Crecer}
\subsection*{De dónde viene el crecimiento}
{\color{blue} El motor de crecimiento es el mecanismo que usan las startups para alcanzar el crecimiento sostenible.El crecimiento sostenible se caracteriza por una simple norma: Los nuevos consumidores provienen de las acciones de los consumidores pasados}
Los consumidores pasados llevan al crecimiento sostenible de cuatro formas:
\begin{enumerate}[\bfseries 1.]
\item \textbf{El boca a boca}
\item \textbf{Efecto secundario del uso del producto}.- Cuando un consumidor manda dinero a un amigo usando PayPal, el amigo se ve expuesto automáticamente al producto de PayPal.
\item \textbf{A través de la publicidad financiada.} Cuanto mayor sea el ingreso marginal, más rápido será el crecimiento.Siempre que el coste de adquirir un nuevo cliente (el coste marginal) sea menor que el ingreso que genera este cliente (el ingreso marginal), el excedente (el beneficio marginal) puede usarse para conseguir más clientes. 
\item \textbf{a través de la compra o uso repetido}
\end{enumerate}
\subsection*{Los tres motores del crecimiento}
 una de las formas más caras de despilfarro potencial para una startup es perder el tiempo discutiendo sobre cómo priorizar los nuevos desarrollos cuando ya tiene un producto en el mercado.
\subsubsection*{El motor de crecimiento pegajoso}
las empresas que usan el motor de crecimiento pegajoso hacen un seguimiento muy cuidadoso de su tasa de abandono o de deserción. {\color{blue}si la tasa de adquisición de nuevos clientes supera la tasa de deserción, el producto crecerá. La velocidad de crecimiento se determina por lo que se llama tasa de capitalización, que es el resultado de restar la tasa de deserción a la tasa de crecimiento natural.} 
\subsubsection*{El motor de crecimiento viral}
El conocimiento del producto se expande rápidamente de persona a persona de forma similar a como un virus se convierte en una epidemia.\\
Igual que con los otros motores de crecimiento, el motor viral está impulsado por un circuito de feedback que puede cuantificarse. Se llama circuito viral, y su velocidad está determinada por un único término matemático llamado coeficiente viral. Cuanto mayor es este coeficiente, más rápido se expande el producto. El coeficiente viral mide cuántos nuevos consumidores usarán el producto como consecuencia de que se registre un nuevo consumidor. En otras palabras, ¿cuántos amigos traerá cada consumidor? Como cada amigo es también un nuevo consumidor, él o ella tendrá la oportunidad de reclutar a más amigos.
\subsubsection*{El motor de crecimiento remunerado}
{\color{blue}Para predecir qué empresa crecerá más rápidamente sólo necesita saber otra cosa: cuánto cuesta conseguir que un cliente se registre.}
\subsubsection*{Una advertencia técnica}
Técnicamente, más de un motor de crecimiento puede operar a la vez en un mismo negocio pero las startups con éxito suelen centrarse en un único motor de crecimiento así centrándose en un solo motor a la vez.
\subsection*{Los motores de crecimiento determinan el producto/encaje en el mercado}
{\color{blue}Producto/encaje es el momento en que una startup finalmente encuentra un amplo conjunto de consumidores que adquiere su producto}
\subsection*{Cuando se apaga el motor}
Al final todos los motores se quedan sin gasolina. Y para desbloquear tal efecto nos queda crear un producto mínimo viable. Así si el producto empieza a crecer podemos detener en teoría el proceso de desarrollo de producto. Cuando el motor de crecimiento  se ralentiza de repente, esto provoca una crisis.
\section*{Adaptar}
\subsection*{Crear una organización adaptativa}
Cada nuevo ingeniero debía ser asignado a un mentor, quien ayudaba al nuevo empleado a trabajar con este currículum de sistemas, conceptos y técnicas que él o ella necesitaba para ser productivo.
\subsubsection*{¿Se puede crecer demasiado rápido?}
No se puede cambiar calidad por tiempo. Si se están provocando (o ignorando) problemas de calidad ahora, los defectos resultantes producirán una ralentización más adelante. \\
El circuito de feedback de Crear-MedirAprender es un proceso continuo. No nos detenemos cuando ya se ha lanzado un producto mínimo viable, sino que usamos el aprendizaje que hemos obtenido para trabajar inmediatamente en la siguiente iteración. 
\subsection*{La sabiduría de los cinco porqués}
{\color{blue}El sistema toma el nombre del método de investigación de formular la pregunta ¿por qué? cinco veces para entender qué ha pasado (la causa fundamental).}
Por ejemplo:
\begin{enumerate}[\bfseries 1.]
\item ¿Por qué se ha parado la máquina? (Había una sobrecarga y se han fundido los fusibles.)
\item ¿Por qué había una sobrecarga? (El cojinete no estabasuficientemente lubricado.)
\item ¿Por qué no estaba suficientemente lubricado? (La bomba de lubricación no bombeabalo suficiente.)
\item  ¿Por qué no bombeaba lo suficiente? (El eje de la bombaestaba desgastado y suelto.)
\item  ¿Por qué estaba el eje desgastado? (No había filtro y entraron virutas de metal.)
\end{enumerate}
Imagine que, de repente, empezamos a recibir quejas de nuestros consumidores  sobre una nueva versión del producto que acabamos de lanzar.
\begin{enumerate}[\bfseries 1.]
\item  En el nuevo producto hay una característica que no funciona. ¿Por qué? Porque ha fallado un servidor.
\item  ¿Por qué ha fallado el servidor? Porque un subsistema oculto se utilizó de forma inadecuada.
\item  ¿Por qué se usó de forma inadecuada? Porque el ingeniero que lo usaba no sabía cómo usarlo adecuadamente.
\item  ¿Por qué no sabía usarlo adecuadamente? Porque nunca le enseñaron.
\item  ¿Por qué no le enseñaron? Porque su director no creía en enseñar a los nuevos ingenieros y porque él y su equipo estaban demasiado ocupados.
\end{enumerate}
\subsubsection*{Hacer una inversión proporcional}
Para utilizar de una manera optima los cinco porqués a la hora de invertir debería ser menor cuando el síntoma es leve y mayor cuando el síntoma es más doloroso. 
\subsection*{La maldición de las cinco culpas}
Se recomienda diversas tácticas para huir de las cinco culpas. Una de ellas es asegurarse de que todos los afectados por el problema están en la sala cuando se analiza la causa de fondo. 

\subsubsection*{Los primeros pasos}
Para asegurarse que los 5 porqués funcionen hay algunas reglas que hay que seguir.
\begin{enumerate}[\bfseries 1.]
\item Requiere de una confianza mutua.\\
\item  Ser tolerantes con los errores la primera vez.
\item No permitir que se cometa dos veces el mismo error.
\end{enumerate}
\subsubsection*{Enfrentándose a verdades desagradables}
Deberá estar preparado para el hecho de que la técnica de los cinco porqués le lleve a descubrir hechos desagradables sobre su organización. Le reclamará inversiones en prevención que tendrán que realizarse a expensas de tiempo y dinero que podrían invertirse en nuevos productos o características.
\subsubsection*{Empiece por algo pequeño, sea específico}
Cuanto más específicos sean los síntomas, más fácil será para todo el mundo reconocer cuándo es el momento de programar una reunión de los cinco porqués. Si el problema es grande se escogerá un subconjunto en el que se quiera centrar.
\subsubsection*{Elija un jefe de los cinco porqués}
Para facilitar el aprendizaje, he descubierto que es útil elegir a un jefe de los cinco porqués en cada área en que se use el método. \\
 Muchas veces un problema puede distanciar a la gente; los cinco porqués hacen justamente lo contrario.
\\
una buena sesión de los cinco porqués tiene dos resultados, el aprendizaje y la acción.  
\subsection*{Adaptándose a lotes más pequeños}
Alcanzar el fracaso se refiere a ejecutar con éxito un plan defectuoso.\\
Las organizaciones tienen memoria muscular» y es difícil para la gente desprenderse de antiguos habitos.\\
Se debe tener equipos de aproximadamente 6 personas con el objetivo de iterar con las personas, tan rápido como se pueda.
\section*{Innovar}
los emprendedores
{\color{blue}pueden crear organizaciones que aprendan a equilibrar las necesidades de los consumidores
 existentes con el reto de encontrar nuevos consumidores a quienes atender gestionando las líneas de negocio existentes y explorando nuevos modelos de negocio al mismo tiempo.}\\
\subsection*{Cómo nutrir la innovación disruptiva}
Los equipos de innovación que pretenden alcanzar el éxito deben estructurarse correctamente para obtenerlo. \\
las startups necesitan tres requisitos estructurales:
\begin{enumerate}[\bfseries 1.]
\item Recursos escasos pero seguros.
\item independencia para desarrollar su negocio.
\item Participación en los beneficios.
\end{enumerate}
\subsubsection*{Recursos escasos pero seguros}
Un presupuesto demasiado elevado es tan perjudicial como un presupuesto demasiado bajo.
\subsubsection*{Autoridad independiente en el desarrollo}
{\color{blue}Deben ser capaces de concebir y ejecutar experimentos sin obtener un número excesivo de permisos para hacerlos.}\\
Recomiendo firmemente que los equipos de una startup sean multifuncionales, es decir, que estén representados en todos los departamentos funcionales de la empresa involucrados en la creación o en el lanzamiento de sus primeros productos. 
\subsubsection*{Una participación personal en el resultado}
{\color{blue}los mejores incentivos son los vinculados a los resultados a largo plazo de la nueva innovación.}
\subsection*{Crear una plataforma para la experimentación}
\subsubsection*{Proteger la organización matriz}
Usted debería ser capaz de identificar los problemas de una situación de conflicto: el uso de indicadores vanidosos en lugar de indicadores accionables, un ciclo de tiempo demasiado largo, el uso de lotes grandes, una hipótesis de crecimiento que no está clara, un diseño del experimento débil, la falta de participación del equipo en los resultados y, por lo tanto, muy poco aprendizaje.
\subsubsection*{Miedos racionales}
Esta empresa no es una startup minúscula y aleatoria que no tiene nada que perder. Es una empresa consolidada que puede perder mucho. Si disminuyen los ingresos que provienen del negocio central, ruedan cabezas. Y esto no puede tomarse a la ligera.
\subsubsection*{Los peligros de esconder la innovación dentro de la caja negra}
El imperativo de innovar es inexorable. Sin la capacidad para experimentar de la manera más ágil, la empresa sufrirá el destino descrito en El dilema de los innovadores: beneficios y márgenes crecientes año tras año hasta que el negocio se colapse de repente.\\
Hacer las cosas a escondidas de la empresa matriz puede tener consecuencias negativas a largo plazo
\subsubsection*{Crear una caja de arena para la innovación}
El reto es crear un mecanismo que impulse a los equipos de innovación de forma abierta. Éste es el camino hacia una cultura sostenible de la innovación a lo largo del tiempo, a medida que las empresas se enfrentan a amenazas existenciales, que funcionen de las siguiente manera:
\begin{enumerate}[\bfseries 1.]
\item Cualquier equipo puede crear un experimento de split-test que sólo afecte a las partes del producto o servicio que estén dentro de la caja de arena (un producto que tenga diferentes partes) o sólo para un determinado segmento de consumidores o área (un nuevo producto).
\item Un equipo debe realizar el experimento completo, de principio a fin.
\item Ningún experimento puede durar más de una cantidad específica de tiempo (normalmente unas semanas para los experimentos sobre simples elementos concretos, unos meses para una innovación más disruptiva).
\item Ningún experimento puede afectar a más de un número de consumidores específico (normalmente expresado en forma de porcentaje de la base total de consumidores de la empresa).
\item Todos los experimentos deben evaluarse mediante un informe estándar único que use de cinco a diez (no más) indicadores accionables.
\item Todos los equipos que trabajen dentro de la caja de arena y todos los productos que se creen deben usar los mismos indicadores para evaluar el éxito.
\item Cualquier equipo que cree un experimento debe hacer un seguimiento de los indicadores y de las reacciones de los consumidores (llamadas de apoyo, reacción en los medios sociales, foros, etc.) mientras el experimento está en proceso, y abortarlo si ocurre algo catastrófico
\end{enumerate}
Las empresas que
Por ejemplo se quiere llevar al mercado un producto totalmente nuevo pueden crear la restricción alrededor de los consumidores de un segmento determinado.
{\color{blue}Siempre que se pueda, el equipo de innovación debería ser multifuncional y tener un jefe de equipo claro, como el shusa de Toyota. Éste debería ser capaz de crear, comercializar y desarrollar productos o características en la caja de arena sin autorización previa.}
\subsubsection*{Evaluar a los equipos internos}
Operando dentro de este marco de referencia, los equipos internos actúan como startups. A medida que demuestran el éxito, necesitan integrarse en la cartera general de productos y servicios de la empresa.
\subsection*{Cultivar la cartera de management}
Hay cuatro grandes tipos de trabajo que deben gestionar las empresas. A medida que crece una startup interna.\\ 
Todos los productos o las características de éxito empiezan en la etapa de investigación y desarrollo ( +D), y acaban formando parte de la estrategia de la empresa. 
\subsubsection*{Emprendedor como nombre de un oficio}
{\color{red}Ser emprendedor debería considerarse una carrera viable para los innovadores dentro de las grandes empresas}
Cuando un emprendedor haya incubado un producto en la caja de arena de la innovación, tiene que ser reintegrado en la organización matriz. Al final necesitará un equipo más grande para hacerlo crecer, comercializarlo y extenderlo.\\
La caja de arena se diseñó para protegerles a ellos y a la sede de la empresa, y cualquier expansión necesita tener esto en cuenta.\\
Trabajar en la caja de arena de la innovación es como desarrollar los músculos de la startup. Al principio, el equipo sólo será capaz de hacer experimentos modestos. Los primeros experimentos quizá no consigan producir demasiado aprendizaje y puede que no lleven a un éxito que pueda ampliarse. A lo largo del tiempo, estos equipos tienen mayores garantías de mejorar, siempre que obtengan el feedback constante del desarrollo de lotes pequeños y los indicadores accionables, y se evalúen a través de los hitos de aprendizaje.
\subsubsection*{Convertirse en el statu quo}
{\color{red}Aquellos que ven la adopción del método Lean Startup como un conjunto definido de pasos no tendrán éxito}.\\
Queremos forzar a los equipos a trabajar de forma multifuncional para alcanzar el aprendizaje validado. Muchas de las técnicas para hacer esto, los indicadores accionables, el despliegue continuo y el circuito general de feedback de Crear-Medir-Aprender hacen que los equipos trabajen a nivel subóptimo en sus funciones individuales. No importa lo rápido que podamos crear. {\color{red}No importa lo rápido que podamos medir. Lo que importa es lo rápido que podemos pasar por el circuito entero.\\
el método Lean Startup es un marco de referencia, no un proyecto de pasos a seguir diseñado para adaptarse a las condiciones específicas de cada empresa}.
\section*{Epílogo. No despilfarrar}
En 1911, Taylor escribió: En el pasado, el hombre iba por delante; en el futuro, el sistema irá por delante.\\
{\color{blue}Este despilfarro proviene no de la ineficiencia en la organización del trabajo, sino del hecho de que se trabaja en las cosas erróneas}\\
Creemos que la mayoría de formas de despilfarro en la innovación se pueden prevenir si se entienden sus causas. Pero para ello debemos cambiar nuestra mentalidad colectiva respecto a cómo debe hacerse el trabajo. \\
{\color{red}lo que importa no es establecer objetivos cuantitativos sino arreglar el método a través del cual se alcanzan estos objetivos.} Una de las preguntas del movimiento del método Lean Startup es \textbf{¿Cómo podemos crear una organización sostenible alrededor de un nuevo conjunto de productos o servicios?}
\subsection*{Superpoderes organizativos}
\subsubsection*{Poner el sistemas por delante: algunos peligros}
Nuestra sociedad necesita más que nunca la creatividad y la visión de los emprendedores. De hecho, como estos recursos son muy valiosos, no nos podemos permitir despilfarrarlos.
\subsubsection*{La seudociencia del desarrollo del producto}
\textbf{Toda la innovación empieza con la visión.} La parte crucial se produce a continuación. Como hemos visto, demasiados equipos de innovación se entregan al teatro del éxito, buscando de forma selectiva datos que apoyen su visión en lugar de exponer los elementos de la visión a experimentos.\\
\subsubsection*{Una nueva investigación científica}
Se debe crear nuevos métodos eficientes para poder llegar a objetivos establecidos.


\end{multicols}
\end{document}