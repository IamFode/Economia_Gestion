\documentclass[10pt]{book}
\usepackage[text=17cm,left=2.5cm,right=2.5cm, headsep=20pt, top=2.5cm, bottom = 2cm,letterpaper,showframe = false]{geometry} %configuración página
\usepackage{latexsym,amsmath,amssymb,amsfonts} %(símbolos de la AMS).7
\parindent = 0cm  %sangria
\usepackage[T1]{fontenc} %acentos en español
\usepackage[spanish]{babel} %español capitulos y secciones
\usepackage{graphicx} %gráficos y figuras.

%---------------FORMATO de letra--------------------%

\usepackage{lmodern} % tipos de letras
\usepackage{titlesec} %formato de títulos
\usepackage[backref=page]{hyperref} %hipervinculos
\usepackage{multicol} %columnas
\usepackage{tcolorbox, empheq} %cajas
\usepackage{enumerate} %indice enumerado
\usepackage{marginnote}%notas en el margen
\tcbuselibrary{skins,breakable,listings,theorems}
\usepackage[Bjornstrup]{fncychap}%diseño de portada de capitulos
\usepackage[all]{xy}%flechas
\counterwithout{footnote}{chapter}
\usepackage{xcolor}

%--------------------GRÀFICOS--------------------------

\usepackage{tkz-fct}

%---------------------------------

\titleformat*{\section}{\LARGE\bfseries\rmfamily}
\titleformat*{\subsection}{\Large\bfseries\rmfamily}
\titleformat*{\subsubsection}{\large\bfseries\rmfamily}
\titleformat*{\paragraph}{\normalsize\bfseries\rmfamily}
\titleformat*{\subparagraph}{\small\bfseries\rmfamily}

\renewcommand{\thechapter}{\Roman{chapter}}
\renewcommand{\thesection}{\arabic{chapter}.\arabic{section}}
%------------------------------------------

\renewcommand{\labelenumi}{\Roman{enumi}.}%primer piso II) enumerate
\renewcommand{\labelenumii}{\arabic{enumii}$)$}%segundo piso 2)
\renewcommand{\labelenumiii}{\alph{enumiii}$)$}%tercer piso a)
\renewcommand{\labelenumiv}{$\bullet$}%cuarto piso (punto)

%----------Formato título de capítulos-------------

\usepackage{titlesec}
\renewcommand{\thechapter}{\arabic{chapter}}
\titleformat{\chapter}[display]
{\titlerule[2pt]
\vspace{4ex}\bfseries\sffamily\huge}
{\filleft\Huge\thechapter}
{2ex}
{\filleft}

\usepackage[htt]{hyphenat}

\begin{document}
\normalfont
\input xy
\xyoption{all}
\author{\Large Apuntes por FODE}
\title{Project Management}
\date{}
\pagestyle{empty}
\maketitle
\thispagestyle{empty}
\let\cleardoublepage\clearpage
\tableofcontents								%indice


%------------------------------------------
 
\let\cleardoublepage\clearpage

\part{Inicio del proyecto}

\chapter{Definición de los objetivos, el alcance y los criterios de éxito del proyecto.}

    \section{Identificar los objetivos del proyecto}
    Se necesita una imagen clara de lo que está tratando de lograr, como lo va a lograr y como sabe cuándo se a logrado.

    \subsection{Pasos}
    \begin{enumerate}[\bfseries 1.]
	\item \textbf{Objetivo del proyecto.-} Es el resultado deseado del proyecto. Es lo que le pidieron que haga y lo que está tratando de lograr.
	\item \textbf{Se debe tener muy claro el objetivo del proyecto.-} Los objetivos bien definidos son tanto específicos como medibles. 
	\item Una vez que tenga los objetivos definidos, es hora de examinar el proyecto. \textbf{Entregables.-} Se refiere a los resultados tangibles del proyecto. En otras palabras, es lo que se produce o presenta al final de una tarea, evento o proceso. 
	\item \textbf{Hay dos tipos de de entragables.-} Un informe, cuando se alcanza una meta puede ver visiblemente los resultados documentados en un cuadro, gráfico o presentación. 
    \end{enumerate}

    \subsection{Cómo establecer objetivos SMART}
    El método SMART agrega tres consideraciones para el éxito:
    \begin{enumerate}[\bfseries 1.]
	\item \textbf{Ser específico.}
	\item \textbf{Ser medible.}
	\item \textbf{Ser alcanzables.-} Deben ser objetivos desafiantes.
	\item \textbf{Ser relevantes.-} ¿Tiene sentido alcanzar este objetivo?.
	\item \textbf{Estar sujetos a plazos.}\\\\
    \end{enumerate}

    \begin{enumerate}[\bfseries 1.]

	\item Los objetivos \textbf{específicos} deben responder al menos dos de las preguntas siguientes.
	\begin{itemize}
	    \item ¿Qué quiero lograr?.
	    \item ¿Por qué es esto un objetivo?.
	    \item ¿Tiene una razón, propósito o beneficio específico?.
	    \item ¿quién esta involucrado y quién es el destinatario: empleados, clientes, al comunidad en general?
	    \item ¿Dónde debería cumplirse el objetivo?
	    \item ¿En qué medida? ¿Qué son los requisitos y limitaciones?.\\
	\end{itemize}
	
	\item Puedo saber si una meta es \textbf{medible} preguntando:
	\begin{itemize}
	    \item ¿Cuántos y cómo sabré cuando se cumpla?.
	    \item ¿Aprendiste a tocar guitarra? Si o no.
	    \item Ponga un punto de referencia para el inicio de la métrica.\\
	\end{itemize}
	
	\item Una pista para ayudar a descubrir si tu objetivo es \textbf{alcanzable} es preguntar
	    \begin{itemize}
		\item ¿Cómo se puede lograr?
		\item Divida el objetivo en partes pequeñas y ver si tiene sentido.
	    \end{itemize}

	\item Es \textbf{relevante}
	    \begin{itemize}
		\item ¿Tiene sentido intentar alcanzar este objetivo?.
		\item Piense en como se alinea con otras metas, prioridades y valores.
		\item ¿La meta parece valiosa?.
		\item ¿El esfuerzo involucrado equilibra el beneficio?.
		\item ¿Coincide con el de su organización?.
		\item Hay una audiencia que continuará utilizar el producto o servicio una vez entragado?
		\item La empresa ¿podrá sostener el proyecto en el tiempo?.
	    \end{itemize}

	\item \textbf{Límite de Tiempo} 
	    \begin{itemize}
		\item El tiempo y las métricas van de la mano.
	    \end{itemize}
    \end{enumerate}

    \subsection{Creación de OKR para su proyecto}
    Se definirá y creará objetivos y entregables de proyectos medibles. OKR -> Objetivos y resultados clave. Combinan un objetivo y una métrica para determinar un resultado medible. 
	\subsubsection{Creación de OKR para su proyecto}
	    \begin{enumerate}[\bfseries 1.]
		\item \textbf{Fija tus objetivos.-} deben cumplir los siguientes criterios
		    \begin{itemize}
			\item Aspiracionales.
			\item Alineado con los objetivos de la organización.
			\item Acción orientada.
			\item Concretos.
			\item Significativos.
		    \end{itemize}

		\item Para dar forma a cada objetivo pregúntese.
		\begin{itemize}
		    \item ¿El objetivo ayuda a lograr las metas generales del proyecto?
		    \item ¿El objetivo se alinea con los OKR de la empresa y del departamento?
		    \item ¿Es el objetivo inspirador y motivador?
		    \item ¿Conseguir el objetivo tendrá un impacto significativo?
		\end{itemize}
	    \end{enumerate}

	    \paragraph{Ejemplos}
	    \begin{itemize}
		\item Cree el software de seguridad de datos más seguro.
		\item Mejorar continuamente la analítica web y las conversiones.
		\item Brindar un servicio de alto rendimiento.
		\item Crea una aplicación disponible universalmente.
		\item Incrementar el alcance del mercado.
		\item Logre las mejores ventas entre los competidores de la región.
	    \end{itemize}

	\subsubsection{Desarrollar resultados clave}
	A continuación agregue 2-3 resultados clave para cada objetivo. Los resultados clave deben estar sujetos a plazos.
	\begin{enumerate}[\bfseries 1.]
	\item Los resultados clave sólidos cumplen los siguientes criterios.
	\begin{itemize}
	    \item Orientado a resultados, no una tarea.
	    \item Medible y verificable.
	    \item Específico y de duración determinada.
	    \item Agresivo pero realista.
	\end{itemize}
	
	\item Para ayudar a dar forma a sus resultados clave, pregúntese:
	\begin{itemize}
	    \item ¿Qué significa el éxito?
	    \item ¿Qué métricas probarían que hemos logrado el objetivo con éxito?
	\end{itemize}
	\end{enumerate}

	    \paragraph{Ejemplos}
	    \begin{itemize}
		\item $X\%$ de nuevos registros en el primer trimestre posterior al lanzamiento
		\item Aumentar la inversión de los anunciantes en un $X\%$.
		\item La adopción de nuevas funciones es de al menos un $X\%$. 
		\item Los clientes informan un máximo de 2 errores críticos por Sprint.
		\item Mantener la tasa de cancelación de suscripción al boletín en $X\%$.
	    \end{itemize}

	    \textbf{El objetivo describe lo que quiere hacer y el resultado clave describe cómo sabrá que lo hizo.}
    
    \section{Apuntes}
    \begin{itemize}
	\item Utilizamos los OKR para ayudar a las personas a mantenerse enfocadas en los objetivos más importantes y evitar que se distraigan con objetivos urgentes pero menos importantes.
	\item Los OKR tienen dos variantes y es importante diferenciarlas:
	    \begin{enumerate}
		\item Los compromisos son OKR que acordamos que se cumplirán, y estaremos dispuestos a ajustar los cronogramas y los recursos para asegurarnos de que se cumplan.
		\item Por el contrario, los OKR aspiracionales expresan cómo nos gustaría que se viera el mundo, aunque no tenemos una idea clara de cómo llegar allí y / o los recursos necesarios para entregar el OKR.
	    \end{enumerate}
    \end{itemize}

\section{Alcance del proyecto}
\begin{itemize}
    \item Un alcance claramente definido describe todos los detalles de un proyecto y regula lo que puede ser agregado o eliminado a medida que avanza.
    \item Es un entendimiento acordado sobre lo que es incluido o excluido de un proyecto.
    \item El alcance ayuda a garantizar que su proyecto está claramente definido y trazado.
    \item Incluye el \textbf{cronograma del proyecto, presupuestos y recursos.}\\\\
\end{itemize}

\begin{enumerate}[\bfseries 1.]
    \item Preguntas para determinar el alcance del proyecto.
    \begin{itemize}
	\item ¿De dónde surgió el proyecto?
	\item ¿Por qué es necesario?
	\item ¿Qué se espera lograr con el proyecto?
	\item ¿Qué tiene en mente el patrocinador del proyecto?
	\item ¿Quién aprueba los resultados finales?\\\\
    \end{itemize}
\end{enumerate}
    Tomarse el tiempo para hacer preguntas y asegurarse de que comprende el alcance del proyecto ayudará a reducir los gastos, el reproceso, la frustración y la confusión. Asegúrese de comprender quién , qué , cuándo , dónde , por qué y cómo se aplica al alcance.\\

    \subsection{Seguimiento y mantenimiento del alcance del proyecto}
    \begin{enumerate}[\bfseries 1.]
	\item Se debe identificar los cambios de alcance y ser proactivo.
	\item Para combatir el deslizamiento del alcance (cambiar el proyecto después que comienza), es bueno saber que hay dos fuentes principales de las cuales vienen.
	\begin{itemize}
	    \item Externo.- Se debe tener los objetivos bien específicos para que las partes interesadas no cambien el rumbo del proyecto.
	    \item Interno.
	\end{itemize}
    \end{enumerate}

    \subsection{Gestionar cambios en el alcance de un proyecto}
    \begin{itemize}
	\item La gestión del alcance va de la mano con el establecimiento de objetivos.\\\\
    \end{itemize}
    \begin{enumerate}[\bfseries 1.]
	\item Para decidir si un cambio de alcance es aceptable y qué impacto tendrá, se deber aplicar el modelo de triple restricción.
	\begin{enumerate}
	    \item \textbf{Alcance}.
	    \item \textbf{Tiempo.} Se refiere al cronograma y los plazos del proyecto.
	    \item \textbf{Costo.} Incluye el presupuesto y también cubre los recursos y las personas que trabajarán en el proyecto.
	\end{enumerate}
	\item Comprender cómo el cambio de una restricción afecta a las otras dos limitaciones es clave.
	\item Necesita una comprensión clara de las prioridades del proyecto.
	\item A pesar de todo si hay una buena justificación se pueden realizar cambios.
    \end{enumerate}
    
    \subsection{La importancia de mantenerse dentro del alcance}
    \begin{itemize}
	\item El deslizamiento del alcance es cuando el alcance cambia después de haber comenzado el proyecto.
    \end{itemize}

    \subsection{Estrategias para controlar el deslizamiento del alcance}
    \begin{itemize}
	\item \textbf{Defina los requisitos de su proyecto.} Comuníquese con sus partes interesadas o clientes para averiguar exactamente qué quieren del proyecto y documente esos requisitos durante la fase de inicio. 
	\item \textbf{Establezca un cronograma de proyecto claro.} La gestión del tiempo y las tareas son esenciales para ceñirse al alcance de su proyecto. Su cronograma debe describir todos los requisitos de su proyecto y las tareas que son necesarias para lograrlos.
	\item \textbf{Determine qué está fuera de alcance.} Asegúrese de que las partes interesadas, los clientes y el equipo del proyecto comprendan cuándo los cambios propuestos están fuera de alcance. Llegue a un acuerdo claro sobre los impactos potenciales al proyecto y documente su acuerdo. 
	\item \textbf{Brindar  alternativas.} Sugiera soluciones alternativas a su cliente o accionista. También puede ayudarlos a considerar cómo los cambios propuestos podrían crear riesgos adicionales. Realice un análisis de costo-beneficio, si es necesario.
	\item \textbf{Configure un proceso de control de cambios.} Durante el transcurso de su proyecto, algunos cambios son inevitables. Determine el proceso de cómo se definirá, revisará y aprobará (o rechazará) cada cambio antes de agregarlo a su plan de proyecto. Asegúrese de que su equipo de proyecto esté al tanto de este proceso.
	\item \textbf{Aprenda a decir que no.} A veces tendrá que decir que no a los cambios propuestos. Decir no a una parte interesada o cliente clave puede resultar incómodo, pero puede ser necesario para proteger el alcance de su proyecto y su calidad general. Si se le pide que asuma tareas adicionales, explique cómo interferirán con el presupuesto, el cronograma y / o los recursos definidos en los requisitos iniciales de su proyecto. 
	\item \textbf{Recaude los  costos del trabajo fuera de su alcance.} Si se requiere trabajo fuera del alcance, asegúrese de  documentar todos los costos incurridos. Eso incluye los costos de trabajo afectados indirectamente por el aumento del alcance. Asegúrese de indicar para qué son los cargos. 
    \end{itemize}

	\subsection{Apuntes}
	\begin{itemize}
	    \item Se tiene que lidiar con mas de tres restricciones de proyecto:
	    \begin{enumerate}[1.]
		\item \textbf{Alcance.}
		\item \textbf{Calidad.-} Características de un entregable.
		\item \textbf{Calendario.-} Valor que espera que brinde a la organización.
		\item \textbf{Presupuesto.-} El riesgo se refiere a la probabilidad de que ocurra un evento que afecta el proyecto y su impacto potencial. Esta restricción tiene que ver con el nivel de riesgo que las partes interesadas o el equipo del proyecto están dispuestos a tolerar.
		\item \textbf{Beneficio.}
		\item \textbf{Riesgo.}
	    \end{enumerate}
	    \item Si el costo la mayor prioridad.
	    \begin{enumerate}[1.]
		\item entonces se deberá ajustar los plazos del proyecto.
		\item reducir el alcance.
		\item acordar sobre la calidad reducida de ciertos entrgables.
	    \end{enumerate}
	    \item Si el tiempo la mayor prioridad.
	    \begin{enumerate}[1.]
		\item entonces se deberá poner mas recursos,
		\item recortar el alcance y/o la calidad del producto final.
	    \end{enumerate}
	    \item Si el alcance la mayor prioridad.
	    \begin{enumerate}[1.]
		\item entonces se deberá flexibilizar el tiempo a medida que el equipo se adapta a los cambios del alcance.
		\item aumentar el costo de los entregables agradados al alcance.
	    \end{enumerate}
	\end{itemize}

    \section{Medir el éxito de un proyecto}
	\subsection{Lanzamiento de un proyecto}
	\begin{itemize}
	    \item Se debe saber cuando entregar un proyecto y llamarlo un éxito.
	    \item Entregando el resultado final de su proyecto al cliente o usuario es lo que se llama lanzamiento de proyecto.
	    \item Su éxito debe continuar más allá del punto de entregar el proyecto final. Debe poder medir si el proyecto funciona según lo previsto una vez que se pone en práctica.
	    \item \textbf{Criterio de éxito.-} Forma de medir y ayuda a garantizar el éxito de su proyecto.
	\end{itemize}

	\subsection{Midiendo el éxito del proyecto}
	    \begin{itemize}
		\item Los proyectos no solo se lanzan también se aterrizan.
		\item Los aterrizajes ocurren una vez que su proyecto alcanza cierto grado de éxito.
		\item Su trabajo en un proyecto no estará completo hasta que lo aterrice  midiendo minuciosamente los resultados. Aquí es cuando los criterios de éxito y las métricas que definió inicialmente al establecer los objetivos SMART serán útiles.  
		\item ¿Su proyecto aumentará la retención?
		\item ¿Su proyecto acelerará la función de un producto?
		\item Un error común de los proyecto es lanzar y olvidar los resultados.
	    \end{itemize}

	\subsection{Definición de criterios de éxito}
	    \begin{enumerate}
		\item ¿Como saber que su proyecto es un éxito?
		    \begin{itemize}
			\item Al principio, definió metas y entregables que son medibles, lo que significa que puede determinar si se cumplieron.
		    \end{itemize}
		\item Para medir los criterios de éxito tendremos que preguntarnos:
		    \begin{itemize}
			\item ¿Cómo sabré cuando se haya logrado con éxito?
			\item puede medir el éxito de su proyecto similar a la medición de una meta.
			\item Por lo tanto revise los alcances y los aspectos medibles del proyecto.
			\item Otra cosa que deberá hacer es obtener claridad de las partes interesadas sobre el proyecto. Requisitos y expectativas. 
			\item ¿Quién dice en última instancia si un proyecto es éxito o no?
			\item ¿Qué criterios se medirá para determinar el éxito?
			\item ¿En que se basa el éxito de este proyecto?
		    \end{itemize}

		\item Se puede medir el criterio de éxito  por:
		    \begin{itemize}
			\item Métricas de felicidad, a través de encuestas.
			\item Métricas de adopción que se refiere a cómo el cliente usa y adopta un producto o servicio sin problemas. 
			\item Métricas de participación incluyen el aumento del uso diario de una característica de diseño o aumentar los pedidos y las interacciones con los clientes.
		    \end{itemize}
		\item Una vez definido las métricas se debe pensar en como se hará el seguimiento. Evaluar que herramientas pueden ayudarlo a recopilar datos.
		    \begin{itemize}
			\item Si estas midiendo métricas comerciales como los ingresos, considere un seguimiento en una hoja de calculo.
			\item Si esta midiendo la satisfacción del cliente puede pensar e una forma de incentivar a los clientes a participar en encuestas periódicas por correo.
			\item Realizar una revisión del proyecto una vez al mes.
		    \end{itemize}

		\item E una buena idea que junto con cada criterio de éxito en su lista también incluya:
		    \begin{itemize}
			\item Métodos de cómo medirá el éxito.
			\item con qué frecuencia se medirá.
			\item y quien será responsable de medirlo.
		    \end{itemize}
	    \end{enumerate}

    \section{Seguimiento y comunicación de criterios de éxito}
    También podemos determinar el éxito de un proyecto por la calidad del producto, la capacidad de satisfacer las necesidades de sus clientes y la necesidad de satisfacer las expectativas de sus partes interesadas. 
	\begin{enumerate}
	    \item \textbf{Calidad del producto} Los atributos del producto incluyen:
		\begin{itemize}
		    \item La completitud en las caracter:wística.
		    \item la calidad de las características.
		    \item el costo unitario.
		    \item la facilidad de uso.\\\\
		\end{itemize}
	    Para medir el éxito de un producto, considere incluir estas métricas en su lista de verificación: 
		\begin{itemize}
		    \item Realice un seguimiento de si implementó los requisitos de prioridad del producto
		    \item Realice un seguimiento y evalúe la cantidad de problemas técnicos o defectos del producto.
		    \item Mida el porcentaje de funciones que entregó o lanzó al final del proyecto
		\end{itemize}

	    \item \textbf{Qué es importante para los clientes o las partes interesadas.} A menudo, puede medir el cumplimiento de los objetivos estratégicos a través de métricas de usuarios o clientes.
		\begin{itemize}
		    \item Evaluar la participación del usuario con el producto.
		    \item Medición de la satisfacción de las partes interesadas y los clientes a través de encuestas
		    \item Seguimiento de la adopción del producto por parte del usuario mediante el uso de datos de ventas.
		\end{itemize}

	    \item \textbf{Documentar, alinear y comunicar el éxito}
	    
	\end{enumerate}

	\subsection{Uso de OKR para evaluar el prgreso}
	\begin{enumerate}[1.]
	\item Debemos comunicar y seguir el OKR  a través de:
	    \begin{itemize}
		\item Comparta sus OKR con su equipo.
		\item Asigne responsables.
	    \end{itemize}

	\item Midiendo el progreso
	    \begin{itemize}
		\item Determina cómo puntarás tus OKR.-
		    \begin{itemize}
			\item Por un porcentaje del objetivo.
			\item por escalas del 1 al 10.
			\item Por colores (semáforo).
			\item Método si o no.
			\item Escalas.
		    \end{itemize}
	    \end{itemize}

	\item Establecer las expectativas de puntación. Mínima puntación de 0.6.
	\item Programar puntos de control. Mensuales 
	\end{enumerate}
	

\chapter{Trabajar eficazmente con los stakeholders}

    \section{Explotar los roles y responsabilidades con los stakeholders}

	\subsection{Accesibilidad para jefes de proyecto}
	    
	    \begin{itemize}
		\item La accesibilidad se refiere a eliminar activamente cualquier barrera que pueda evitar que las personas con discapacidad puedan acceder a la tecnología.
	    \end{itemize}

	\subsection{Elegir un equipo de proyecto}
	    
	    \begin{itemize}
		\item Para decidir quien hará que debemos considerar y delinear nuestras necesidades.
		    \begin{enumerate}
			\item Un gerente de proyecto hará una lista de roles que necesitarán en su equipo.
			\item Decidir quien hace que.
			\item La motivación es un ingrediente clave para cumplir tareas.
		    \end{enumerate}
		\item Hagámonos estas preguntas clave.
		    \begin{itemize}
			\item ¿Cuántas personas necesito para cada paso del camino?
			\item ¿Qué miembros del equipo necesito y cuándo?
			\item ¿Están esos expertos ya ocupados en otros proyectos?
			\item ¿Quién toma las decisiones finales sobre los recursos del proyecto?
		    \end{itemize}
	    \end{itemize}

	\subsection{Los componentes básicos de un proyecto de equipo de ensueño}
	    \begin{enumerate}
		\item El tamaño del equipo.
		\item Las habilidades.
		    \begin{itemize}
			\item Habilidades técnicas necesarias.
			\item Habilidades interpersonales.
			\item Habilidades resolución de problemas. Aseguremos de tomar nota del nivel de motivación de los miembros de su equipo y el impacto que pueda tener en su proyecto.
			\item Habilidades de liderazgo.
		    \end{itemize}
		\item La disponibilidad.
		\item La motivación.
	    \end{enumerate}

	\subsection{Definición de roles de proyecto}
	\begin{itemize}
	    \item Para tener confianza en su equipo, debe conocer el rol de cada persona desde el principio. Y así se sabremos de que roles serán responsables.
	\end{itemize}

	\subsection{Definición de roles de proyecto}
	\begin{itemize}
	    \item Los stakeholders se dividen en dos
	    \begin{enumerate}[1.]
		\item Stakeholders Principales.- Incluyen:
		\begin{itemize}
		    \item Miembros del equipo.
		    \item Líderes.
		    \item Clientes.
		    \item Director ejecutivo.
		\end{itemize}
		\item Stakeholders Secundarios. Incluyen:
		\begin{itemize}
		    \item Punto de contacto en asuntos legales.
		\end{itemize}
	    \end{enumerate}

	\end{itemize}

    \section{Evaluación de los stakeholders}

	\subsection{Evaluación de las partes interesadas}

	\begin{itemize}

	    \item Hay tres pasos clave para iniciar un análisis de los stakeholders.

	    \begin{enumerate}[1.]
		\item Hacer una lista de todos los grupos de interés sobre los que impacta el proyecto.
		\item Determinar el nivel de interés e influencia de cada actor.
		\item Evaluar su capacidad para participar y encontrar formas de involucrarlos.
	    \end{enumerate}

	    \item Cuanto mayor sea el interés y la influencia, cuanto más importante es el stakeholders para el éxito del proyecto.
	    \item Se tiene 4 técnicas para administrar a los stakeholders.
	    \item Para gestionar mejor a las partes interesadas clave, querrás asociarte estrechamente con ellos para alcanzar los resultados deseados.\\

	\end{itemize}

	\subsection{Priorizar a las partes y generar su aceptación}
	Los consejos para lograr la aceptación de las partes interesadas clave incluyen: 

	\begin{itemize}
	    \item  Mapear claramente el trabajo del proyecto con los objetivos de las partes interesadas.
	    \item Describir cómo el proyecto se alinea con los objetivos del departamento o equipo de la parte interesada.
	    \item Escuchar los comentarios de las partes interesadas y encontrar formas de incorporar sus comentarios en el estatuto del proyecto cuando sea apropiado.
	\end{itemize}

    \section{Asignar roles y responsabilidades al equipo del proyecto}

	\subsection{Elementos de un gráfico RACI}

	\begin{itemize}
	    
	    \item Veremos una herramienta útil llamada gráfico $RACI$. Un gráfico $RACI$ ayuda a definir roles y responsabilidades para individuos o equipos para Asegúrese de que el trabajo se haga de manera eficiente.

	    \item Hay cuatro tipos de participación incluida en un cuadro $RACI$

	    \begin{enumerate}[1.]
		\item R. Responsable.- Se refiere a aquellos haciendo el trabajo para completar la tarea.
		\item A. Explicable.-  Se refiere a aquellos asegurándose de que le trabajo se haga.
		\item C. Consultado.- Incluye a aquellos que dan retroalimentación, como expertos en la materia o tomadores de decisiones.
		\item I. Informado.- Incluye a aquellos que solo necesitan saber las decisiones finales o que una tarea está completa.
	    \end{enumerate}

	    \item Al crear una gráfica $RACI$ necesitas escribir cada tarea o entregable para su proyecto y luego asignarle el rol apropiado para cada stakeholder. Los pasos son:

	    \begin{enumerate}[1.]
		\item Piense en quién está involucrado en el proyecto.
		\item Escriba los roles o los nombres de las personas en una fila en la parte superior del gráfico.
		\item En una columna escriba las tareas o entragables.
		\item Preguntese:

		\begin{itemize}
		    \item ¿Quién es responsable de hacer esto?
		    \item ¿Quién es responsable si no se hace?
		    \item ¿Quién tendrá opiniones fuertes para agregar, y por lo tanto, debe ser consultado acerca de cómo se hace esto?
		    \item ¿Y quién necesita estar informado del progreso o las decisiones tomadas al respecto? 
		\end{itemize}
		\item Asigna las letras R, A, C y I basándose en tus respuestas.
	    \end{enumerate}
	\end{itemize}

	\subsection{Construyendo un gráfico RACI}

	    \subsubsection{Responsable}

	    \begin{itemize}
		\item Cada tarea necesita por lo menos un parte responsable.
		\item Es buena práctica tratar de limitar la cantidad de miembros del equipo asignados al rol responsable de una tarea. Debemos preguntarlos 
		\begin{itemize}
		    \item ¿A qué departamento pertenece el trabajo?
		    \item ¿Quién realizará el trabajo?
		\end{itemize}
	    \end{itemize}
	
	    \subsubsection{Explicable}
	    \begin{itemize}
		\item Es responsable de asegurarse de que se realice la tarea.
		\item Es importante que tener solo una persona responsable de cada tarea.
		\item Para determinar quién debe ser el miembro explicable del equipo, considere:
		\begin{itemize}
		    \item ¿Quién delegará la tareas a completar?
		    \item ¿Quién revisará el trabajo para determinar si la tarea está completa?
		\end{itemize}
		\item La parte responsable también podría ser la parte explicable.
	    \end{itemize}

	    \subsubsection{Consultando}
	    \begin{itemize}
		\item Los miembros del equipo o los stakeholders que se colocan en el rol de consultados tiene información útil para ayudar a completar la tarea.
		\item No hay un número máximo para asignar esta tarea. Pero cada persona debe tener una razón para estar ahí.
		\item Formas en las que se puede ayudar a identificar quién es apropiado para el puesto.
		\begin{itemize}
		    \item ¿A Quién afectará la tarea?
		    \item ¿Quién tendrá aportes o comentarios de la persona responsable para ayudar a completar el trabajo?
		    \item ¿Quiénes son los expertos en la materia (PYME) para la tarea?
		\end{itemize}
		\item los consultandos deben ser personas adecuadas que estén en el rol para ayudar a realizar la tarea de manera eficiente y correcta.
	    \end{itemize}

	    \subsubsection{Informado}
	    \begin{itemize}
		\item Las personas que necesitan ser informadas necesitan saber las decisiones finales que se tomaron y cuándo se completó la tarea.
		\item Es común tener muchas personas asignadas a esta categoría.
		\item Las preguntas clave debe hacerse para asegurarse de haber capturado adecuadamente a las personas en el rol de informado:
		\begin{itemize}
		    \item ¿A quién le importa la finalización de esta tarea?
		    \item ¿Quién se verá afectado por el resultado?
		    \item ¿Quién utilizará o aplicará la tarea finalizada?
		\end{itemize}
	    \end{itemize}

	\subsection{Aprovechar al máximo un gráfico RACI}
	\begin{itemize}
	    \item Cuando complete su cuadro, es una buena idea revisar y contar el número de R asignadas a cada parte interesada. Esto puede ayudarlo a identificar la posible sobrecarga de trabajo de un miembro del equipo. 
	\end{itemize}



\chapter{Utilizando recursos y herramientas para el éxito del proyecto}

    \section{Comprender las necesidades de recursos del proyecto}
	
	\subsection{Recursos esenciales del proyecto}
	    
	    \begin{itemize}
		\item Comprender sus necesidades de recursos es fundamental para lograr sus objetivos.
		\item Los recursos del proyecto incluyen 
		    \begin{enumerate}
			\item Presupuesto.
			    \begin{itemize}
				\item Estimación de la cantidad de dinero que costará completar el proyecto.
				\item Se podrá hacer preguntar a los stakeholders:
				    \begin{itemize}
					\item ¿Hay algún impuesto sobre los productos que deba tener encuesta?
					\item ¿Qué pasa con las tarifas adicionales?
				    \end{itemize}
			    \end{itemize}
			\item Personal.
			    \begin{itemize}
				\item Cada persona que tenga parte en el proyecto es un recurso.
			    \end{itemize}
			\item Materiales.
			    \begin{itemize}
				\item Puede ser la madera para realizar trabajos de construcción.
			    \end{itemize}
		    \end{enumerate}
		\item Es importante determinar sus recursos antes de que el proyecto se ponga en marcha.
		\item Para poder organizar estos recursos debemos utilizar herramientas.
	    \end{itemize}





    
    

\end{document}
