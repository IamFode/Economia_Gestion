\documentclass[10pt]{book} 
\usepackage[text=17cm,left=2.5cm,right=2.5cm, headsep=20pt, top=2.5cm, bottom = 2cm,letterpaper,showframe = false]{geometry} %configuración página
\usepackage{latexsym,amsmath,amssymb,amsfonts} %(símbolos de la AMS).7
\parindent = 0cm  %sangria
\usepackage{lmodern} % tipos de letras
\usepackage[T1]{fontenc} %acentos en español
\usepackage[spanish]{babel} %español capitulos y secciones
\usepackage{graphicx} %gráficos y figuras.
\pagestyle{empty}%elimina numeración de página

%-----------------------------------------%

\usepackage{titlesec} %formato de títulos
\usepackage[backref=page]{hyperref} %hipervinculos
\usepackage{multicol} %columnas
\usepackage{wrapfig} %Figuras al lado de texto
\usepackage{tikz}\usetikzlibrary{shapes.misc}
\usepackage{tikz,tkz-tab} % diseño de cajas
\usetikzlibrary{matrix,arrows, positioning,shadows,shadings,backgrounds,
calc, shapes, tikzmark}
\usepackage{tcolorbox, empheq} %cajas
\tcbuselibrary{skins,breakable,listings,theorems}
\usepackage{xparse} % cajas y entornos para teoremas etc
\usepackage{pstricks} %cambiar color de letra
\usepackage[Bjornstrup]{fncychap}%diseño de portada de capitulos
\usepackage{rotating}
\usepackage{enumerate}
\usepackage{booktabs}
\usepackage{synttree} 
\usepackage{chngcntr}
\usepackage{venndiagram}
\usepackage[all]{xy}%flechas
\counterwithout{footnote}{chapter}
\usepackage{xcolor}
\usetikzlibrary{datavisualization.formats.functions}
\usepackage{marginnote}%notas en el margen

%------------------------------------------

\titleformat*{\section}{\LARGE\bfseries\sffamily}
\titleformat*{\subsection}{\Large\bfseries\sffamily}
\titleformat*{\subsubsection}{\large\bfseries\sffamily}
\titleformat*{\paragraph}{\normalsize\bfseries\sffamily}
\titleformat*{\subparagraph}{\small\bfseries\sffamily}

%------------------------------------------

\renewcommand{\labelenumi}{\Roman{enumi}.}%primer piso II) enumerate
\renewcommand{\labelenumii}{\arabic{enumii}$)$}%segundo piso 2)
\renewcommand{\labelenumiii}{\alph{enumiii}$)$}%tercer piso a)
\renewcommand{\labelenumiv}{$\bullet$}%cuarto piso (punto)

%----------Formato título de capítulos-------------

\usepackage{titlesec}
\renewcommand{\thechapter}{\arabic{chapter}}
\titleformat{\chapter}[display]
{\titlerule[2pt]
\vspace{4ex}\bfseries\sffamily\huge}
{\filleft\Huge\thechapter}
{2ex}
{\filleft}

\setcounter{secnumdepth}{0}

%---------------------------------------------------------
\begin{document}
\normalfont
\input xy
\xyoption{all}
\author{\Large por FODE}
\title{Gerencia de Personas}
\date{}
\pagestyle{empty}
\maketitle
\thispagestyle{empty}
\let\cleardoublepage\clearpage
\tableofcontents 								%indice

%------------------------------------------
 
\let\cleardoublepage\clearpage

\chapter{Sobre ser un gerente de personas}
\section{Conceptos básicos de la gestión de personas}
En esta sección, cubriremos:
\begin{itemize}
\item Desafíos que enfrenta un gerente por primera vez.
\item Gestión de personas vs. Gestión de recursos humanos.
\item Importancia de la gestión de personas.
\item Rol de un Gerente de Personas.
\end{itemize}
\subsection{Retos que enfrentan los gerentes por primera vez}
- Hello Mr. Joe.\\
- Hello\\
- La jefa quiere conocerte.\\
- Menciono por que ?\\
- No.\\
- Buenas tardes Jefa, quería conocerme ?\\
- Buenas tardes Joe. Al ver tu buen trabajo, hemos decidido promocionarte como Gerente de Personas. Serás responsable de un equipo de 15 personas. Felicidades. Puedes hacerte cargo a partir de mañana.\\
- Eso es genial. Gracias jefa.
- Ahora seras gerente de personas, con un equipo de 15 personas. \\
- Hola.\\
- Hola señor. Esta es Neeta de RRHH. Hay una capacitación organizada para los miembros de su equipo sobre la creación efectiva de equipos a las 10.00 a.m. Y, sin embargo, solo tengo la confirmación de dos miembros de su equipo.\\
- Estoy tan desorientado. Ni siquiera conozco a los miembros de mi equipo. No tengo ni idea de qué hacer. Debería llegar a la oficina lo antes posible y corregirlo.
- Cien correos en un día.\\
- Sr. Joe, tiene una reunión con un cliente muy valioso en media hora sobre el problema urgente de la entrega.\\
- No tengo idea de cuál es el problema. Todo está en mal estado. Debo hablar con la Jefa.\\
- Hola, me gustaría reunirme con la jefa. ¿Ella está disponible?.\\
- Lo siento, señor Joe. Ella está en una reunión importante. Debería estar disponible a las 4 de la tarde. Estaba a punto de llamarte yo mismo. La señorita Boss me había ordenado que le informara sobre la ceremonia de apertura de nuestra oficina de Churchgate y debe asistir en su nombre a la 1.00 PM.\\
- ¿Dónde debería comenzar mi día?.\\\\
Acaban de ver el primer día de Joe como gerente, y estoy bastante seguro de que varios de ustedes podrían relacionarse con él.  Les pido que recuerden los tres incidentes más críticos que experimentaron, que moldearon fundamentalmente su identidad como gerentes. Invariablemente, la transición a un rol de gerente por primera vez, se presenta como parte de ese incidente crítico que cambia la vida y que resultó en el crecimiento profesional.\\
De hecho, Lynda Hill en su libro, On Becoming Manager, capta este viaje de convertirse en una gerente por primera vez, de manera muy efectiva. Sus encuestados en el libro hablan de las sorpresas que experimentaron después de convertirse en gerentes por primera vez. Su visión inicial del trabajo de gerente era que eran jefes y que ahora tenían la autoridad para influir en los demás. Pero lo que realmente experimentaron fue que la gestión significaba que ahora eran responsables de los demás y, por lo tanto, tenían que resolver problemas y ayudar a otros a cumplir sus objetivos. Y, de hecho, muchas veces, la autoridad se interpuso en el camino para hacerlo de manera efectiva.\\
La segunda sorpresa fue que la experiencia técnica, que define ser un buen contribuyente individual, ya no era el determinante del éxito.\\
La tercera sorpresa fue que la mayoría de ellos no estaban preparados para el tipo de desafíos que las personas necesitaban para ser efectivos en sus trabajos.\\
, finalmente, muchos de ellos no estaban preparados para los tipos de demandas que se hicieron en su tiempo de diferentes partes interesadas.\\
Es bastante evidente que si tiene que ser un gerente exitoso y eficaz por primera vez, la Gestión de personas se convierte en un aspecto crítico de la gestión.\\
\subsection{Diferencia entre gestión de personas y gestión de recursos humanos}
Definitivamente, hay una diferencia entre la gestión de personas y la gestión de recursos humanos.\\
La gestión de recursos humanos o gestión de recursos humanos como lo llamamos en gran medida se refiere a aquellas actividades, programas, políticas, procedimientos que son desarrollados por el departamento de recursos humanos de las organizaciones, mientras que; La gestión de personas se refiere a las actividades que realizan los gerentes. Día tras día, como gerentes de línea, implementen las políticas y prácticas de recursos humanos desarrolladas por la organización y, de hecho, las adapta de manera única para satisfacer los requisitos de su equipo. En Startups, entonces, podría hacer las preguntas, diciendo: "¿es posible que los gerentes de personas sean gerentes de recursos humanos"? Sí, definitivamente posible. Por lo tanto, si observa un Startup, el fundador o el propietario: 
\begin{itemize}
\item Contrata a los nuevos miembros del equipo,
\item desarrolla el equipo,
\item se asegura de que se brinde la capacitación,
\item realiza la evaluación y
\item evaluación del desempeño.
\end{itemize}    
Y siempre y cuando el tamaño del equipo sea pequeño, los propietarios y fundadores desempeñan el papel del departamento de Recursos Humanos y también de los Gerentes de personas. A medida que las organizaciones crecen en tamaño y escala, existe la necesidad de centralizar y consolidar las actividades de Recursos Humanos. Y esta consolidación de las Actividades de Recursos Humanos recae en el departamento de Recursos Humanos. El rol de Gerente de línea, principalmente, es demasiado capaz de implementar los Procesos y Procedimientos y Políticas que el Departamento de Recursos Humanos ha establecido dentro de su propio contexto.
\subsection{Rol de un Gerente de Personas}
Entonces, como gerentes de personas, ¿qué es lo que realmente hacemos? Adaptamos las políticas de la organización y realmente las personalizamos cuando cumplimos con los requisitos de nuestros empleados. Como gerentes de línea, usted comprende mejor a su equipo. Entiende la motivación, su aspiración, sus expectativas, sus desafíos, su competencia, sus capacidades y qué es lo que pueden aportar a la mesa en el lugar de trabajo.\\
Esta comprensión que tiene del individuo, también le permite personalizar la entrega de las políticas de recursos humanos. Déjame darte un ejemplo. Si realmente mira a una organización, especialmente a las grandes organizaciones, tienen una política sobre la gestión del rendimiento. Dentro de eso, tienen una política relacionada con dar retroalimentación a los empleados. Las políticas se ven muy similares en todas las organizaciones. Sin embargo, como gerente cuando está implementando esta política y tiene que dar retroalimentación a los miembros de su equipo, estoy seguro de que cada uno de ustedes lo personalizará para adaptarse al temperamento de ese miembro individual del equipo, su comprensión previa de ellos, pero lo más importante es que contexto en el que están operando. Entonces, cuando hablamos de ser un Gerente de Personas efectivo, su rol es ser capaz de comprender la política de la organización, internalizarla y encontrar formas de poder implementar esas políticas de una manera que motive y mejore el compromiso de los empleados con la organización.
\subsection{¿Por qué es importante la gestión de personas?}
A menudo los nuevos gerentes se hacen la pregunta: $"$¿por qué la gestión de personas es importante para nosotros?$"$ Las organizaciones tratan sobre personas y cualquier estrategia de una organización, por maravillosa que sea, no se puede ejecutar si no hay personas. Por lo tanto, creo que es bastante obvio que la gestión de personas es necesaria para cualquier entrega de productos y servicios.\\
Veamos ahora la razón más importante de por qué la gestión de personas es importante. Las personas son la fuente de ventaja competitiva en una economía globalizada. Particularmente cuando el mercado y los consumidores son tan diversos, los empleados están dentro de la organización y su comprensión de los matices de los mercados y los clientes puede ser un factor diferenciador clave para las organizaciones. Entonces, cuando decimos que las personas son la fuente de la ventaja competitiva, \textbf{realmente estamos hablando de dos tipos de ventajas, a corto y largo plazo}. En particular, esto se vuelve muy importante cuando busca servicios. Todos sabemos que en el contexto de las organizaciones de servicios, los clientes experimentan momentos de verdad cada vez que interactúan con la organización. Qué significa eso? Cuando entro en un hotel y me registro, no importa si soy el quinto ocupante que ingresa al hotel o la quinta persona que ingresa al hotel, espero que me traten exactamente de la misma manera. Y te puedes imaginar si estás hablando de una gran cadena de hoteles. Luego, en cada ciudad en la que entro y entro en el hotel, espero la misma calidad y nivel de experiencia cada vez, siempre. Entonces, si una gran cadena de hoteles tiene que ser rentable y sostenible a largo plazo, deben asegurarse de que los clientes experimenten una experiencia confiable y similar en cada uno de sus hoteles. Por lo tanto, esto significaría que los empleados están capacitados, los empleados reciben capacitación y los empleados cuentan con herramientas, técnicas y capacidades para continuar brindando el servicio de clase mundial. Y, sobre todo, deben estar motivados y comprometidos para poder hacer que eso suceda. Desde una perspectiva de ventaja competitiva a largo plazo, cuando la organización hace estrategias de expansión o estrategias de crecimiento o adquieren compañías en cada una de estas estrategias, es una suposición latente que las organizaciones tienen los recursos necesarios: personas para poder traducir esas estrategias en acción. Entonces, lo que significa que la Gestión de personas se trata realmente de desarrollar la capacidad administrativa y técnica a largo plazo de las personas en las organizaciones. Una razón muy personal para que pueda mirar, invertir en Gestión de personas es solo interés personal.\\
Como cada vez hay más demandas sobre usted como gerentes, es muy importante que tenga un equipo de personas competentes a quienes pueda delegar con confianza. Y este equipo de personas competentes y seguras a quienes puede delegar liberará el ancho de banda para usted en términos de su tiempo, energía y esfuerzo para poder invertir en actividades de creación de valor. Entonces, en cierto sentido, cuanto más pueda invertir en su gente, mayor será la posibilidad de que crezca personalmente. Si la gestión de personas es tan importante para usted, entonces podría darse la vuelta y preguntarme $"$¿es algo que se pueda pensar?$"$ o $"$¿es un sentido común simple?$"$ o $"$¿es algo que intuitivamente le llega a algunas personas y no tan fácilmente a otras?$"$
\section{Compresión del comportamiento humano}
En esta sección, cubriremos:
\begin{itemize}
\item Las dos facetas de la gestión de personas.
\item Fundamentos de la percepción.
\item Conceptos básicos de la motivación.
\end{itemize}
\subsection{Dos facetas de la gestión de personas - Parte 1}
Debería ser evidente para usted que la Gestión de personas tiene dos partes. \textbf{La primera parte se trata de comprender los comportamientos de los empleados en el lugar de trabajo}.\textbf{ La segunda parte trata sobre la comprensión de las estructuras de la organización, la estrategia y los sistemas y procesos relacionados con las personas.} Su trabajo, como Gerente de Personas, está realmente en la intersección. Debe tener una buena comprensión de los sistemas de la organización, los procesos y, al mismo tiempo, también debe tener una comprensión profunda del equipo que le informa. En la siguiente sección, exploraremos estas dos partes con mayor detalle, y hasta entonces pensaremos en qué se puede aprender en Gestión de personas y qué se puede enseñar en Gestión de personas.
\subsection{Dos facetas de la gestión de personas - Parte 2}
Solo para recapitular, en la última sesión hablamos sobre la gestión de personas que consta de dos aspectos. El primer aspecto fue realmente comprender el comportamiento humano en el lugar de trabajo. Y el segundo aspecto fue sobre la comprensión de las organizaciones. ¿Por qué es importante para ustedes, como gerentes de personas, comprender estas dos dimensiones? Es extremadamente importante entenderlo porque muchos de ustedes correrán, participarán y administrarán en este espacio. Tendrá que trabajar a través de las políticas de la organización y al mismo tiempo deberá comprender el comportamiento humano. Entonces, comencemos por comprender el comportamiento humano en el lugar de trabajo. Las personas son los componentes básicos de una organización. De hecho, es imposible concebir una organización sin personas. Y como Gerentes, por lo tanto, uno de los requisitos clave de su trabajo es realmente comprender y predecir el comportamiento humano dentro de su equipo y en otras partes de la organización. Se ha realizado una gran cantidad de investigación sobre el comportamiento humano y el comportamiento humano en el lugar de trabajo. Entonces, lo que haremos, en esta sesión, es enfocarnos en dos aspectos del comportamiento humano. Uno, relacionado con las percepciones y el segundo, relacionado con la motivación.
\subsubsection{Comprensión de la percepción}
Veamos el papel de la percepción en nuestras vidas. En un momento u otro, hemos dicho esto sobre un amigo: $"$Acabo de mirarlo y sabía que podríamos conectarnos$"$. Tenemos muy poca información sobre esa persona y lo que hicimos fue realmente, basado en nuestro primeras impresiones, llegamos a un juicio. Entonces, si esto puede suceder en nuestras vidas personales, dediquemos un momento a analizar el impacto de la percepción en el juicio gerencial. No importa si es una entrevista de selección o si es una entrevista de salida o si es una entrevista de evaluación. En cada contexto de entrevista, la percepción impacta el juicio y las decisiones que se toman. ¿Con qué frecuencia te has sentado en una entrevista y tal vez en los primeros dos o tres minutos llegaste a la decisión de contratar o no al candidato? Hay evidencia que respalda que muchas veces llegamos a una decisión muy rápidamente y usamos el resto del tiempo para apoyar la decisión o el juicio al que hemos llegado. Entonces, veamos el contexto de un gerente y el equipo. Día tras día, interactuamos con los miembros del equipo en varios aspectos relacionados con tareas, planes de trabajo, presupuestos, recursos, etc. Tenemos nuestras propias percepciones e interpretaciones sobre el comportamiento de nuestros informes directos. Al mismo tiempo, además de nuestra observación, ¿hay otros que nos brindan información sobre su desempeño? A veces, un tercero puede acercarse a usted y brindarle comentarios sobre su informe directo. Recibimos mucha información sobre los miembros de nuestro equipo. Y lo que hacemos es imponer nuestra interpretación sobre esta información que recibimos y llegar a un juicio. Tomemos un ejemplo. Le aconsejaste a un miembro del equipo que trabajara sola en un proyecto. Mientras trabajaba en él, varios colegas se detienen y ella tiene largas conversaciones con ellos. Observas su interacción con varios colegas. Hay dos formas de imponer su interpretación en su observación. La primera es que $"$está buscando ayuda de colegas cuando en realidad le dije que lo hiciera sola$"$ o podría interpretar que tiene una buena relación de trabajo con varios colegas que simplemente pasan por su escritorio para poder intercambiar notas. Dependiendo de la lente con la que esté interpretando esta situación, el mismo contexto en el que puede llegar a decisiones y juicios completamente diferentes y que, a su vez, determinará cómo interactuará con ese empleado en el futuro. Por ahora, está claro que cada uno de nosotros puede percibir la misma situación con diferentes lentes perceptuales. Una conclusión clave que me gustaría que recordaras en este momento es que la lente perceptiva que utilizamos da forma a la realidad de lo que vemos. Y por lo tanto, ¿podría haber prejuicios que entran en esto sin nuestro propio conocimiento? Y como gerentes, es posible que existan sesgos inconscientes que afecten las decisiones que tomamos.
\subsection{Comprender la motivación}
Entonces, un resumen rápido. Hasta ahora, hemos analizado qué es la gestión de personas y por qué es importante la gestión de personas y luego comenzamos a hacer esta pregunta qué significa realmente la gestión de personas. Y acordamos que la Gestión de personas se trata realmente de comprender el comportamiento humano en el lugar de trabajo y también se trata de comprender el sistema organizacional. Hasta ahora hemos dedicado nuestro tiempo a comprender el comportamiento humano en el lugar de trabajo y dentro de eso nos hemos centrado en la parte superior de la percepción. Ahora vamos a pasar a mirar la motivación. Como gerentes, la motivación es extremadamente importante para que comprendamos a las personas que trabajan con nosotros. \textbf{¿Qué sabemos sobre la motivación? Toda la literatura parece sugerir que hay factores intrínsecos y factores extrínsecos que motivan a los individuos.} Los factores intrínsecos que motivan son:
\begin{itemize}
\item Pasión,
\item impulso,
\item logro,
\item reconocimiento y
\item tareas desafiantes.
\end{itemize} 
Y los factores externos parecen ser:
\begin{itemize}
\item El pago, 
\item las políticas de recursos humanos,
\item el apoyo a la gestión,
\item las condiciones de trabajo y
\item el apoyo de los compañeros.
\end{itemize} 
Permítanos retroceder rápidamente en su contexto organizacional y ver uno o dos ejemplos. Tome el ejemplo de los incentivos. La mayoría de las organizaciones tienen incentivos. Y estos incentivos están vinculados a ciertos tipos de comportamientos. Por ejemplo, los incentivos de ventas se centran en lograr objetivos de ventas. Los incentivos de producción se centran en mejorar la eficiencia o la productividad. Todos estos incentivos funcionan bajo el supuesto de que son factores extrínsecos y si se configuran de manera efectiva pueden inducir la motivación en las personas. Entonces, veamos ahora los objetivos de estiramiento. En la mayoría de los procesos de gestión del desempeño, \textbf{se les pide a los gerentes que establezcan objetivos flexibles para sus empleados}. Estos objetivos están destinados a hacer que los empleados demuestren diferentes tipos de comportamientos que generalmente reflejan el potencial del empleado. De nuevo, déjame darte un ejemplo. Recuerdo que en una de las organizaciones, el Objetivo de Estiramiento para un empleado era poder desarrollar un nuevo sistema de información del cliente. La persona realmente pertenecía al Departamento de Finanzas. Ahora, preguntaría qué tiene que ver la persona de Finanzas con un sistema de información del cliente. Este objetivo de estiramiento se le otorgó principalmente para permitirle adquirir una perspectiva de ventas para que en el futuro si fuera promovido a un puesto de unidad de negocio sería más efectivo. Entonces, como puede ver, los motivadores extrínsecos, como los incentivos, los objetivos de estiramiento realmente evocan comportamientos entre los empleados que están alineados con los objetivos de la organización. Por lo tanto, para resumir, la motivación es un aspecto clave de la gestión del rendimiento. \textbf{Y como gerentes, identificar los motivadores intrínsecos y los motivadores extrínsecos para sus empleados es un aspecto clave del desarrollo y también de su propio crecimiento personal.}\\\\
Nos centraremos en la motivación intrínseca. La motivación intrínseca es algo interno de un individuo. La motivación intrínseca actúa como un fuerte impulso para el comportamiento individual.\\
Entonces, ¿cómo aprovechan las organizaciones esta motivación intrínseca? 
\begin{itemize}
\item Cada vez que una organización promueve a un empleado.
\item cada vez que una organización brinda oportunidades de desarrollo para el crecimiento personal.
\end{itemize}
Las organizaciones asignan tareas responsables a los empleados. Y todo esto se basa en el reconocimiento de que hay algo interno, que es personal para un empleado que lo impulsa a desempeñarse y entregar en el trabajo.\\
Por lo tanto, \textbf{se sabe que la motivación intrínseca evoca pasión}. \textbf{La motivación intrínseca también evoca la responsabilidad.}\textbf{ Por lo tanto, los gerentes efectivos aprovechan tanto los factores extrínsecos de motivación como los factores intrínsecos de motivación para ser líderes efectivos.}
\section{Impacto de los factores organizacionales}
En esta sección, nos centraremos en los factores organizativos que afectan la gestión de las personas, tales como:
\begin{itemize}
\item Estructura organizativa.
\item Cultura organizacional. 
\end{itemize}
\subsection{Entendiendo la Estructura Organizacional}
Pasemos ahora a analizar los factores organizacionales y cómo impactan el comportamiento humano. Entonces, todos nosotros pertenecemos a un departamento. La forma en que se estructura el departamento es que hay un grupo de tareas especializadas, especializadas similares, que se unen. Por lo tanto, cada departamento está estructurado en torno a la especialización. Del mismo modo, hay diferentes posiciones y niveles dentro del departamento. Y cada nivel tiene una cierta responsabilidad en la toma de decisiones. Del mismo modo, dentro del departamento, cada nivel tiene personas que reportan y tienen un rango de control. Y finalmente, cada departamento tiene la autoridad para actuar y tomar ciertas decisiones y tampoco tiene la autoridad para tomar otras decisiones. Entonces, esta agrupación de tareas especializadas con ciertas características que he mencionado juntas constituye el departamento. Todas las características que mencioné juntas a menudo se denominan estructura organizativa. Retroceda un momento y pregunte: $"$¿Estas características de la estructura organizacional impactan el comportamiento individual?$"$ Por supuesto que sí. Estoy seguro de que todos nos quejamos en nuestras organizaciones sobre cuánto retraso hay en la forma en que se toman las decisiones. Y a veces es frustrante en diferentes niveles cuando hay un retraso en las decisiones, particularmente cuando impactan su trabajo de manera crítica. Es muy posible que en esos contextos, la satisfacción de los empleados se vea afectada negativamente. Del mismo modo, cuando se mira el rango de control , ¿le importa a usted, como gerente, si tiene cuatro denunciados directos o diez denunciados directos? Por supuesto, es importante. La cantidad de tiempo que necesita pasar, la cantidad de revisiones que necesita hacer, la cantidad de supervisión que debe hacer, todos se verán afectados y la gestión del tiempo será un desafío para usted. Es bastante evidente que las características de la estructura organizativa tienen un impacto en el comportamiento de los empleados. diferentes estructuras dependiendo de su estrategia, tamaño, tecnología y el tipo de entornos que operan. Incluso dentro de la misma industria, las organizaciones individuales pueden tener su propia estructura organizativa. Y, por lo tanto, como gerentes debemos ser conscientes del hecho de que las estructuras de la organización impactan el comportamiento de los empleados, lo que a su vez impacta el desempeño. Nuevamente, hay una gran cantidad de investigación realizada sobre las estructuras de la organización, el diseño de la organización.
\subsection{Cultura organizacional}
Un aspecto importante de la organización que impacta el comportamiento es la Cultura de la Organización. \textbf{La cultura es exclusiva de una organización. Y no habrá dos organizaciones que tengan la misma Cultura Organizacional.} Has escuchado a personas describir su cultura organizacional. \textbf{La cultura se refiere a la comprensión compartida que los empleados tienen sobre los diferentes aspectos de su organización.} La cultura es intangible. Solo puede ser experimentado; Es muy difícil ser descrito. ¿Cómo se crea la cultura entonces? La cultura se crea a través de:
\begin{itemize}
\item La forma en que seleccionan a las personas,
\item a través de la forma en que inducimos a las personas a nuestra organización,
\item a través del tipo de comportamientos que son reconocidos y recompensados en la organización a los modelos a seguir que las personas pueden emular y 
\item mediante las señales La alta dirección envía sobre lo que es importante para la organización.
\end{itemize}
Como puede ver, \textbf{la cultura se trata de comportamientos}. En resumen, la Estructura organizacional y la Cultura organizacional juntas tienen un impacto en términos del comportamiento de los empleados.
\section{Preguntas y respuestas}
\begin{enumerate}[\bfseries 1.]
\item Aditya es promovida como Gerente por primera vez. ¿Quiere saber cuál de las siguientes actividades debe realizar en su nuevo rol?\\
Como Gerente por primera vez, Aditya necesita comprender las aspiraciones, la motivación y las expectativas de su equipo. Este es su papel clave, que a su vez lo ayudará a dar su opinión. La creación de políticas de salud y la organización de la capacitación recaen en el ámbito del papel del Departamento de Recursos Humanos.
\item Tracking Ink, una agencia de publicidad global, está creando una estrategia de crecimiento para 2020. ¿Cree que debe analizar la estrategia de las personas? ¿Si es así por qué?\\
No se puede planificar una estrategia de crecimiento sin personas, ya que las personas son una fuente de ventaja competitiva. Traducen la estrategia en acción y sin ellos no se puede realizar la entrega de productos y servicios.
\item Como Gerente de Personas, Mary necesita comprender cuál de las siguientes opciones.\\
Como Gerente de Personas, uno necesita tener una comprensión profunda de los sistemas y procesos organizacionales, por un lado, y los equipos que informan, por otro lado.
\item La percepción tiene un papel importante que desempeñar en:\\
La percepción juega un papel clave en la creación de las primeras impresiones, mientras se sienta para una entrevista para seleccionar un candidato e incluso mientras realiza una capacitación.
\item ¿Cuál de los siguientes es / son factores extrínsecos para la motivación?\\
La pasión y la tarea desafiante son factores intrínsecos de motivación.
\item Rita dirige su unidad de fabricación. Ella tiene 6 personas que le reportan. La compañía está viendo un rápido crecimiento. Rita tiene que contratar a unos 12 empleados. Ella está pensando en crear un nuevo departamento. ¿Cuál de las siguientes opciones debería considerar?
\item Rita dirige su unidad de fabricación. Ella tiene 6 personas que le reportan. La compañía está viendo un rápido crecimiento. Rita tiene que contratar a unos 12 empleados. Ella está pensando en crear un nuevo departamento. ¿Cuál de las siguientes opciones debería considerar?\\
La especialización se refiere a la manera en que las actividades y tareas se subdividen en grupos comunes. La agrupación resulta en departamentos. Todos los demás, a saber, responsabilidad, alcance de control y autoridad, se producen dentro de un departamento.
\item ¿Cuál es la función clave de los objetivos de estiramiento?\\
Los incentivos financieros a menudo están vinculados al cumplimiento de objetivos. Los incentivos difieren en función de los productos y objetivos asignados a una meta. Sin embargo, los objetivos elásticos en una organización se tratan de asumir objetivos que no son incrementales, pero que en realidad le permiten demostrar un comportamiento que no es lo de siempre. Esto a su vez da como resultado el desarrollo de la excelencia humana y la demostración de potencial.
\item La agrupación de tareas especializadas con la toma de decisiones, el rango de control y la autoridad para actuar a menudo se conoce como:\\
Estructura organizativa. El comportamiento organizacional se refiere al comportamiento individual y de equipo en el contexto de una organización. La cultura organizacional se refiere al significado compartido que los empleados tienen sobre su organización. El diseño de la organización se refiere a la forma en que la estructura, los procesos, los sistemas y la tecnología se alinean dentro de la organización. La estructura de la organización es un componente del diseño organizacional y se refiere a la agrupación de tareas especializadas dentro de una jerarquía, con claros intervalos de control y delegación de autoridad para la toma de decisiones.
\item ¿Cuál de los siguientes puede afectar la estructura de una organización?\\
Se escribe mucho sobre cómo el tamaño de la organización impacta la estructura. Las organizaciones más pequeñas tienden a tener estructuras más simples en contraste con las grandes organizaciones globales.
\item ¿Cuál de las siguientes frases define mejor los elementos de la cultura organizacional?
\\
La mayoría de los estudiosos de la organización mencionan que la cultura solo se puede experimentar. Cualquier cantidad de descripción no puede capturar sus matices porque es un entendimiento compartido que los miembros de una comunidad desarrollan a lo largo de un período de tiempo a través de historias, mitos, rituales, eventos e incidentes críticos.
\end{enumerate}
\section{Resumen}
Como gerente de personas, es importante que comprenda el comportamiento humano en las organizaciones y también es importante que comprenda cómo funcionan las organizaciones.\\
En este capítulo la atención se ha centrado principalmente en tu informe directo. Si bien continuaremos enfocándonos en los informes directos, también es importante que hagamos esta pregunta: ¿hay otras partes interesadas para usted como gerentes dentro de sus propias organizaciones?\\
Algunos de los puntos clave que discutimos son:
\begin{itemize}
\item El rol del gerente de personas es principalmente implementar el proceso, los procedimientos y las políticas establecidas por el Departamento de Recursos Humanos.
\item Cuando las organizaciones elaboran o implementan cualquier estrategia, requieren que las personas traduzcan los recursos en una estrategia.
\item La gestión de personas tiene dos partes: comprender el comportamiento de los empleados en el lugar de trabajo y comprender la estructura organizativa y los sistemas y procesos relacionados con las personas.
\item La lente perceptiva que usa para interpretar la situación afectará su decisión y juicio . Esto a su vez determinará su compromiso con sus empleados.
\item Un aspecto clave del desarrollo de los empleados y su propio crecimiento personal es identificar los motivadores extrínsecos e intrínsecos para su empleado.
\item Cada departamento está estructurado en torno a la especialización .
\item La cultura organizacional es intangible y única para cada organización.
\end{itemize}
\section{Recursos adicionales}
\subsection{Motivación de empleados}
La motivación de los empleados es el nivel de energía, compromiso y creatividad que los trabajadores de una empresa aplican a sus trabajos. En el entorno empresarial cada vez más competitivo de los últimos años, encontrar formas de motivar a los empleados se ha convertido en una preocupación apremiante para muchos gerentes. De hecho, han surgido varias teorías y métodos diferentes de motivación de los empleados, que van desde incentivos monetarios hasta una mayor participación y empoderamiento. La motivación de los empleados a veces puede ser particularmente problemática para las pequeñas empresas, donde el propietario a menudo ha pasado tantos años construyendo una empresa que le resulta difícil delegar responsabilidades significativas a otros. Pero los empresarios deben ser conscientes de tales trampas, Los efectos de la baja motivación de los empleados en las pequeñas empresas pueden ser devastadores. Algunos de los problemas asociados con los trabajadores desmotivados incluyen la complacencia, la disminución de la moral y el desánimo generalizado. Si se les permite continuar, estos problemas pueden reducir la productividad, las ganancias y la competitividad en una pequeña empresa.\\
Por otro lado, las pequeñas empresas también pueden proporcionar una atmósfera ideal para fomentar la motivación de los empleados, porque los empleados pueden ver los resultados de sus contribuciones de una manera más inmediata que en las grandes empresas. Además de aumentar la productividad y la competitividad, una fuerza laboral altamente motivada puede permitirle al propietario de una pequeña empresa renunciar al control operativo diario y concentrarse en estrategias a largo plazo para hacer crecer el negocio. "Los trabajadores realmente quieren inspirarse en su trabajo, y cuando lo hacen, trabajan mejor, de manera más inteligente y más difícil", dijo el entrenador de negocios Don Maruska a Entrepreneur.\\
demás, una empresa que instituye formas efectivas, ya sea tangibles (como un bono financiero) o intangibles (por ejemplo, una asignación de ciruela para un próximo proyecto), de recompensar a los empleados por un buen trabajo puede ser una herramienta invaluable en la retención de empleados. "Las personas disfrutan trabajar y tienden a prosperar en organizaciones que crean entornos de trabajo positivos", dijo un investigador de negocios a HR Focus. "[Prosperan] en entornos en los que pueden marcar la diferencia, y donde la mayoría de las personas de la organización son competentes y se unen para impulsar la empresa. Los programas de recompensa y reconocimiento adecuadamente estructurados son componentes importantes, pero no exclusivos, en esta mezcla. "
\subsubsection{¿Qué Motiva?}
Un enfoque para la motivación de los empleados ha sido ver los "complementos" en el trabajo de un individuo como los factores principales para mejorar el rendimiento. Las empresas han utilizado infinitas combinaciones de beneficios para empleados, como atención médica, seguro de vida, participación en las ganancias, planes de propiedad de acciones para empleados, instalaciones de ejercicio, planes de comidas subsidiados, disponibilidad de cuidado infantil, automóviles de la empresa y más, en sus esfuerzos por mantenerse felices. empleados en la creencia de que los empleados felices son empleados motivados.\\
Sin embargo, muchos teóricos modernos proponen que la motivación que siente un empleado hacia su trabajo tiene menos que ver con recompensas materiales que con el diseño del trabajo en sí. Estudios que se remontan a 1950 han demostrado que los trabajos altamente segmentados y simplificados dieron como resultado una menor moral y producción de los empleados. Otras consecuencias de la baja motivación de los empleados incluyen el absentismo y la alta rotación, los cuales son muy costosos para cualquier empresa. Como resultado, las iniciativas de "ampliación de empleo" comenzaron a surgir en las principales empresas en la década de 1950.\\
En el frente académico, Turner y Lawrence sugirieron que hay tres características básicas de un trabajo "motivador":\\
\begin{enumerate}[\bfseries 1.]
\item Debe permitir que un trabajador se sienta personalmente responsable de una parte significativa del trabajo realizado. Un empleado debe sentir la propiedad y la conexión con el trabajo que realiza. Incluso en situaciones de equipo, un esfuerzo exitoso fomentará la conciencia en un individuo de que sus contribuciones fueron importantes para lograr las tareas del grupo.
\item Debe proporcionar resultados que tengan un significado intrínseco para el individuo. El trabajo efectivo que no lleva a un trabajador a sentir que sus esfuerzos importan no se mantendrá. El resultado del trabajo de un empleado debe tener valor para sí mismo y para los demás en la organización.
\item Debe proporcionar al empleado comentarios sobre sus logros. Una crítica constructiva y creíble del trabajo realizado es crucial para la motivación de un trabajador para mejorar.
\end{enumerate}
Si bien la terminología cambia, los principios de la motivación de los empleados permanecen relativamente sin cambios con respecto a los hallazgos de hace más de medio siglo. Las palabras de moda de hoy incluyen "empoderamiento", "círculos de calidad" y "trabajo en equipo". Todos estos términos demuestran las tres características de los trabajos motivadores establecidos en la teoría de Turner y Lawrence. El empoderamiento le da autonomía y le permite a un empleado tener la propiedad de ideas y logros, ya sea actuando solo o en equipo. Los círculos de calidad y la creciente presencia de equipos en los entornos laborales actuales brindan a los empleados oportunidades para reforzar la importancia del trabajo realizado por los miembros, así como recibir comentarios sobre la eficacia de ese trabajo.\\
En las pequeñas empresas, que pueden carecer de los recursos para implementar programas formales de motivación de los empleados, los gerentes pueden cumplir los mismos principios básicos. Para ayudar a los empleados a sentir que sus trabajos son significativos y que sus contribuciones son valiosas para la empresa, el propietario de la pequeña empresa debe comunicar el propósito de la empresa a los empleados. Esta comunicación debe tomar la forma de palabras y acciones. Además, el propietario de una pequeña empresa debe establecer altos estándares para los empleados, pero también debe apoyar sus esfuerzos cuando no se pueden alcanzar los objetivos. También puede ser útil permitir a los empleados tanta autonomía y flexibilidad como sea posible en el desempeño de sus trabajos. Se fomentará la creatividad si los errores honestos se corrigen pero no se castigan. Finalmente, el propietario de una pequeña empresa debe tomar medidas para incorporar la visión de los empleados de la empresa con su propia visión. Esto motivará a los empleados a contribuir a los objetivos de la pequeña empresa, y ayudará a evitar el estancamiento en su dirección y propósito.
\subsubsection{Métodos de motivación}
Existen tantos métodos diferentes de motivar a los empleados hoy como empresas que operan en el entorno empresarial global. Aún así, algunas estrategias prevalecen en todas las organizaciones que luchan por mejorar la motivación de los empleados. Los mejores esfuerzos de motivación de los empleados se centrarán en lo que los empleados consideran importante. Puede ser que los empleados dentro del mismo departamento de la misma organización tengan motivadores diferentes. En la actualidad, muchas organizaciones consideran que la flexibilidad en el diseño del trabajo y los sistemas de recompensa ha resultado en una mayor longevidad de los empleados con la empresa, una mayor productividad y una mejor moral.\\
\paragraph{Empoderamiento} Dar a los empleados más responsabilidad y autoridad para tomar decisiones aumenta su dominio de control sobre las tareas de las que son responsables y los equipa mejor para llevarlas a cabo. Como resultado, disminuyen los sentimientos de frustración que surgen de ser responsable de algo que uno no tiene los recursos para llevar a cabo. La energía se desvía de la autoconservación a la mejora de la realización de tareas.\\
\paragraph{Creatividad e innovación} En muchas empresas, los empleados con ideas creativas no las expresan a la gerencia por temor a que sus comentarios sean ignorados o ridiculizados. La aprobación de la compañía y el seguimiento de la línea de la compañía se han arraigado tanto en algunos entornos de trabajo que tanto el empleado como la organización sufren. Cuando el poder de crear en la organización se reduce desde el personal superior hasta el personal de línea, los empleados que conocen mejor un trabajo, producto o servicio tienen la oportunidad de usar sus ideas para mejorarlo. El poder de crear motiva a los empleados y beneficia a la organización al tener una fuerza de trabajo más flexible, usar más sabiamente la experiencia de sus empleados y aumentar el intercambio de ideas e información entre empleados y departamentos.\\
\paragraph{Aprendizaje}Si los empleados reciben las herramientas y las oportunidades para lograr más, la mayoría asumirá el desafío. Las empresas pueden motivar a los empleados a lograr más al comprometerse con la mejora perpetua de las habilidades de los empleados. Los programas de acreditación y licencia para empleados son una forma cada vez más popular y efectiva de generar crecimiento en el conocimiento y la motivación de los empleados. A menudo, estos programas mejoran las actitudes de los empleados hacia el cliente y la empresa, al tiempo que refuerzan la confianza en sí mismos. Para respaldar esta afirmación, un análisis de los factores que influyen en la motivación para aprender descubrió que está directamente relacionado con la medida en que los participantes en la capacitación creen que dicha participación afectará su utilidad laboral o profesional.\\
\paragraph{Calidad de vida} El número de horas trabajadas cada semana por trabajadores estadounidenses está en aumento, y muchas familias tienen dos adultos trabajando esas horas aumentadas. En estas circunstancias, muchos trabajadores se preguntan cómo satisfacer las demandas de sus vidas más allá del lugar de trabajo. A menudo, esta preocupación ocurre mientras trabaja y puede reducir la productividad y la moral de un empleado. Las empresas que han instituido acuerdos de empleados flexibles han ganado empleados motivados cuya productividad ha aumentado. Los programas que incorporan horarios flexibles, semanas de trabajo condensadas o trabajo compartido, por ejemplo, han tenido éxito en enfocar a los empleados abrumados hacia el trabajo a realizar y lejos de las demandas de sus vidas privadas.\\
\paragraph{Incentivo monetario}Para todos los defensores de motivadores alternativos, el dinero todavía ocupa un lugar importante en la combinación de motivadores. Compartir las ganancias de una empresa incentiva a los empleados a producir un producto de calidad, realizar un servicio de calidad o mejorar la calidad de un proceso dentro de la empresa. Lo que beneficia a la empresa beneficia directamente al empleado. Se están otorgando recompensas monetarias y de otro tipo a los empleados por generar ahorros de costos o ideas de mejora de procesos, para aumentar la productividad y reducir el absentismo. El dinero es efectivo cuando está directamente vinculado a las ideas o logros de un empleado. Sin embargo, si no se combina con otros motivadores no monetarios, sus efectos motivadores son de corta duración. \\
\paragraph{Otros incentivos} Estudio tras estudio ha encontrado que los motivadores más efectivos de los trabajadores son no monetarios. Los sistemas monetarios son motivadores insuficientes, en parte porque las expectativas a menudo exceden los resultados y porque la disparidad entre los individuos asalariados puede dividir en lugar de unir a los empleados. Los motivadores positivos no monetarios comprobados fomentan el espíritu de equipo e incluyen reconocimiento, responsabilidad y avance. Los gerentes que reconocen las "pequeñas ganancias" de los empleados, promueven entornos participativos y tratan a los empleados con justicia y respeto encontrarán que sus empleados están más motivados. Los gerentes de una compañía hicieron una lluvia de ideas para obtener 30 recompensas poderosas que cuestan poco o nada de implementar. Las recompensas más efectivas, tales como cartas de recomendación y tiempo libre en el trabajo, mayor satisfacción personal y autoestima. A largo plazo, los elogios sinceros y los gestos personales son mucho más efectivos y más económicos que las recompensas de dinero solo. Al final, un programa que combina sistemas de recompensa monetaria y satisface necesidades intrínsecas y autorrealizadas puede ser el motivador más poderoso para los empleados.
\subsection{Estructura Organizativa}
La estructura organizacional se refiere a la forma en que una organización organiza a las personas y los trabajos para que su trabajo se pueda realizar y se puedan cumplir sus objetivos. Cuando un grupo de trabajo es muy pequeño y la comunicación cara a cara es frecuente, la estructura formal puede ser innecesaria, pero en una organización más grande se deben tomar decisiones sobre la delegación de varias tareas. Por lo tanto, se establecen procedimientos que asignan responsabilidades para diversas funciones. Son estas decisiones las que determinan la estructura organizacional.\\
En una organización de cualquier tamaño o complejidad, las responsabilidades de los empleados generalmente se definen por lo que hacen, a quién le informan y a los gerentes, quién les informa a ellos. Con el tiempo, estas definiciones se asignan a puestos en la organización en lugar de a individuos específicos. Las relaciones entre estas posiciones se ilustran gráficamente en un organigrama (ver Figuras 1a y 1b). La mejor estructura organizativa para cualquier organización depende de muchos factores, incluido el trabajo que realiza; su tamaño en términos de empleados, ingresos y la dispersión geográfica de sus instalaciones; y la gama de sus negocios (el grado en que se diversifica en todos los mercados).\\
Existen múltiples variaciones estructurales que las organizaciones pueden asumir, pero hay algunos principios básicos que se aplican y una pequeña cantidad de patrones comunes. Las siguientes secciones explican estos patrones y proporcionan el contexto histórico del que surgieron algunos de ellos. La primera sección aborda la estructura organizativa en el siglo XX. La segunda sección proporciona detalles adicionales de estructuras organizativas tradicionales dispuestas verticalmente. Esto es seguido por descripciones de varias estructuras organizativas alternativas, incluidas las ordenadas por producto, función y mercados geográficos o de productos. Lo siguiente es una discusión de estructuras combinadas u organizaciones matriciales. La discusión concluye abordando las estructuras organizacionales futuras emergentes y potenciales.
\subsubsection{Estructura organizativa durante el siglo XX}
Comprender el contexto histórico a partir del cual se han desarrollado algunas de las estructuras organizativas actuales ayuda a explicar por qué algunas estructuras son como son. Por ejemplo, ¿por qué las fábricas de acero antiguas, pero aún operativas, como US Steel y Bethlehem Steel están estructuradas utilizando jerarquías verticales? ¿Por qué las miniacerías de acero más nuevas, como Chaparral Steel, están estructuradas de manera más horizontal, aprovechando la innovación de sus empleados? Parte de la razón, como se analiza en esta sección, es que la estructura organizativa tiene una cierta inercia: la idea tomada de la física y la química de que algo en movimiento tiende a continuar en el mismo camino. Cambiar la estructura de una organización es una tarea administrativa desalentadora.\\
A principios del siglo XX, el sector empresarial de los Estados Unidos estaba prosperando. La industria estaba pasando de la fabricación en el taller a la producción en masa, y pensadores como Frederick Taylor en los Estados Unidos y Henri Fayol en Francia estudiaron los nuevos sistemas y desarrollaron principios para determinar cómo estructurar las organizaciones para la mayor eficiencia y productividad, lo que en su opinión Era muy parecido a una máquina. Incluso antes de esto, el sociólogo e ingeniero alemán Max Weber había concluido que cuando las sociedades adoptan el capitalismo, la burocracia es el resultado inevitable. Sin embargo, debido a que sus escritos no fueron traducidos al inglés hasta 1949, el trabajo de Weber tuvo poca influencia en la práctica administrativa estadounidense hasta mediados del siglo XX.\\
El pensamiento de la gerencia durante este período fue influenciado por las ideas de burocracia de Weber, donde el poder se atribuye a los puestos en lugar de a las personas que ocupan esos cargos. También fue influenciado por la gestión científica de Taylor, o la "mejor manera" de lograr una tarea utilizando estudios de tiempo y movimiento científicamente determinados. También influyeron las ideas de Fayol de invocar la unidad dentro de la cadena de mando, la autoridad, la disciplina, la especialización de tareas y otros aspectos del poder organizacional y la separación laboral. Esto creó el contexto para organizaciones estructuradas verticalmente caracterizadas por distintas clasificaciones de trabajo y estructuras de autoridad de arriba hacia abajo, o lo que se conoció como la estructura organizativa tradicional o clásica.\\
La especialización laboral, una estructura jerárquica de informes a través de una cadena de mando estrechamente unida, y la subordinación de los intereses individuales a los objetivos superiores de la organización combinados para dar como resultado organizaciones organizadas por departamentos funcionales con orden y disciplina mantenidos por las reglas, regulaciones, y procedimientos operativos estándar. Esta visión clásica, o estructura burocrática, de las organizaciones fue el patrón dominante a medida que las organizaciones pequeñas crecieron cada vez más durante el auge económico que ocurrió desde el siglo XX hasta la Gran Depresión de los años treinta. Las plantas de Henry Ford eran típicas de este crecimiento, a medida que la emergente Ford Motor Company se convirtió en el mayor fabricante de automóviles de EE. UU. en la década de 1920.\\
La Gran Depresión sofocó temporalmente el crecimiento económico de EE. UU., Pero las organizaciones que sobrevivieron surgieron con sus estructuras burocráticas orientadas verticalmente intactas cuando la atención pública se dirigió a la Segunda Guerra Mundial. La reconstrucción de posguerra reavivó el crecimiento económico, impulsando a las organizaciones que sobrevivieron a la Gran Depresión a aumentar su tamaño en términos de ingresos por ventas, empleados y dispersión geográfica. Junto con el crecimiento creciente, sin embargo, vino una complejidad creciente. Los problemas en las estructuras comerciales de los Estados Unidos se hicieron aparentes y comenzaron a aparecer nuevas ideas. Los estudios sobre la motivación de los empleados plantearon preguntas sobre el modelo tradicional. La "mejor manera" de hacer un trabajo desapareció gradualmente como la lógica dominante.
\subsubsection{Estructura organizacional tradicional}
Mientras que la sección anterior explicaba el surgimiento de la estructura organizacional tradicional, esta sección proporciona detalles adicionales sobre cómo esto afectó la práctica de la gestión. La estructura de cada organización es única en algunos aspectos, pero todas las estructuras organizacionales se desarrollan o están diseñadas conscientemente para permitir que la organización realice su trabajo. Por lo general, la estructura de una organización evoluciona a medida que la organización crece y cambia con el tiempo.\\
Los investigadores generalmente identifican cuatro decisiones básicas que los gerentes deben tomar a medida que desarrollan una estructura organizativa, aunque pueden no ser explícitamente conscientes de estas decisiones. Primero, el trabajo de la organización debe dividirse en trabajos específicos. Esto se conoce como la división del trabajo. En segundo lugar, a menos que la organización sea muy pequeña, los trabajos deben agruparse de alguna manera, lo que se denomina departamentalización. En tercer lugar, debe decidirse el número de personas y trabajos que se agruparán. Esto está relacionado con la cantidad de personas que debe administrar una persona, o el rango de control: la cantidad de empleados que se reportan a un solo gerente. Cuarto, se debe determinar la forma en que se debe distribuir la autoridad para tomar decisiones.\\
Al tomar cada una de estas decisiones de diseño, es posible una variedad de opciones. En un extremo del espectro, los trabajos están altamente especializados con empleados que realizan una gama limitada de actividades, mientras que en el otro extremo del espectro los empleados realizan una variedad de tareas. En En las estructuras burocráticas tradicionales, existe una tendencia a aumentar la especialización de tareas a medida que la organización crece. Al agrupar trabajos en departamentos, el gerente debe decidir la base sobre la cual agruparlos. La base más común, al menos hasta las últimas décadas, fue por función. Por ejemplo, todos los trabajos de contabilidad en la organización se pueden agrupar en un departamento de contabilidad, todos los ingenieros se pueden agrupar en un departamento de ingeniería, etc. El tamaño de las agrupaciones también puede variar de pequeño a grande, dependiendo de la cantidad de personas que supervisen los gerentes. El grado en que se distribuye la autoridad en toda la organización también puede variar, pero las organizaciones tradicionalmente estructuradas generalmente otorgan la autoridad final para la toma de decisiones por parte de los más altos en la jerarquía estructurada verticalmente. Aun cuando las presiones para incluir a los empleados en la toma de decisiones aumentaron durante las décadas de 1950 y 1960, las decisiones finales generalmente fueron tomadas por la alta gerencia. El modelo tradicional de estructura organizativa se caracteriza por una alta especialización laboral, departamentos funcionales, estrechos tramos de control y autoridad centralizada. Dicha estructura se ha denominado tradicional, clásica, burocrática, formal, mecanicista o de comando y control. Una estructura formada por elecciones en el extremo opuesto del espectro para cada decisión de diseño se denomina no estructurada, informal u orgánica. Aun cuando las presiones para incluir a los empleados en la toma de decisiones aumentaron durante las décadas de 1950 y 1960, las decisiones finales generalmente fueron tomadas por la alta gerencia. El modelo tradicional de estructura organizativa se caracteriza por una alta especialización laboral, departamentos funcionales, estrechos tramos de control y autoridad centralizada. Dicha estructura se ha denominado tradicional, clásica, burocrática, formal, mecanicista o de comando y control. Una estructura formada por elecciones en el extremo opuesto del espectro para cada decisión de diseño se denomina no estructurada, informal u orgánica. Aun cuando las presiones para incluir a los empleados en la toma de decisiones aumentaron durante las décadas de 1950 y 1960, las decisiones finales generalmente fueron tomadas por la alta gerencia. El modelo tradicional de estructura organizativa se caracteriza por una alta especialización laboral, departamentos funcionales, estrechos tramos de control y autoridad centralizada. Dicha estructura se ha denominado tradicional, clásica, burocrática, formal, mecanicista o de comando y control. Una estructura formada por elecciones en el extremo opuesto del espectro para cada decisión de diseño se denomina no estructurada, informal u orgánica. El modelo tradicional de estructura organizativa se caracteriza por una alta especialización laboral, departamentos funcionales, estrechos tramos de control y autoridad centralizada. Dicha estructura se ha denominado tradicional, clásica, burocrática, formal, mecanicista o de comando y control. Una estructura formada por elecciones en el extremo opuesto del espectro para cada decisión de diseño se denomina no estructurada, informal u orgánica. El modelo tradicional de estructura organizativa se caracteriza por una alta especialización laboral, departamentos funcionales, estrechos tramos de control y autoridad centralizada. Dicha estructura se ha denominado tradicional, clásica, burocrática, formal, mecanicista o de comando y control. Una estructura formada por elecciones en el extremo opuesto del espectro para cada decisión de diseño se denomina no estructurada, informal u orgánica.\\
El modelo tradicional de estructura organizativa se representa fácilmente en forma gráfica mediante un organigrama. Es una estructura jerárquica o piramidal con un presidente u otro ejecutivo en la parte superior, un pequeño número de vicepresidentes o gerentes superiores bajo el presidente, y varias capas de administración debajo de esto, con la mayoría de los empleados en la parte inferior de la pirámide. El número de capas de gestión depende en gran medida del tamaño de la organización. Los trabajos en la estructura organizacional tradicional generalmente se agrupan por función en departamentos como contabilidad, ventas, recursos humanos, etc. Las Figuras 1a y 1b ilustran una organización de este tipo agrupada por áreas funcionales de operaciones, marketing y finanzas.
\subsubsection{Bases para la departamentalización}
Como se señaló en la sección anterior, muchas organizaciones agrupan trabajos de diversas maneras en diferentes partes de la organización, pero la base que se utiliza al más alto nivel desempeña un papel fundamental en la configuración de la organización. Hay cuatro bases de uso común.
\paragraph{Departamentalización funcional}
Cada organización de un tipo dado debe realizar ciertos trabajos para hacer su trabajo. Por ejemplo, las funciones clave de una empresa manufacturera incluyen producción, compras, marketing, contabilidad y personal. Las funciones de un hospital incluyen cirugía, psiquiatría, enfermería, limpieza y facturación. El uso de tales funciones como base para estructurar la organización puede, en algunos casos, tener la ventaja de la eficiencia. Agrupar trabajos que requieren los mismos conocimientos, habilidades y recursos les permite hacerse de manera eficiente y promueve el desarrollo de una mayor experiencia. Una desventaja de los grupos funcionales es que las personas con las mismas habilidades y conocimientos pueden desarrollar un enfoque departamental estrecho y tener dificultades para apreciar cualquier otra visión de lo que es importante para la organización; en este caso, los objetivos organizacionales pueden sacrificarse en favor de los objetivos departamentales. Además, la coordinación del trabajo a través de límites funcionales puede convertirse en un desafío de gestión difícil, especialmente a medida que la organización crece en tamaño y se extiende a múltiples ubicaciones geográficas.
\paragraph{Departamentalización geográfica}
Las organizaciones que se extienden en un área amplia pueden encontrar ventajas en organizarse a lo largo de líneas geográficas para que todas las actividades realizadas en una región se gestionen juntas. En una organización grande, la separación física simple dificulta la coordinación centralizada. Además, las características importantes de una región pueden hacer que sea ventajoso promover un enfoque local. Por ejemplo, comercializar un producto en Europa occidental puede tener requisitos diferentes que comercializar el mismo producto en el sudeste asiático. Las empresas que comercializan productos a nivel mundial a veces adoptan una estructura geográfica. Además, la experiencia adquirida en una división regional es a menudo una excelente capacitación para la gestión en niveles superiores.
\paragraph{Departamentalización del producto}
Las grandes empresas diversificadas a menudo se organizan según el producto. Todas las actividades necesarias para producir y comercializar un producto o grupo de productos similares se agrupan. En tal disposición, el gerente superior del grupo de productos generalmente tiene una considerable autonomía sobre la operación. La ventaja de este tipo de estructura es que el personal del grupo puede concentrarse en las necesidades particulares de su línea de productos y convertirse en expertos en su desarrollo, producción y distribución. Una desventaja, al menos en términos de organizaciones más grandes, es la duplicación de recursos. Cada grupo de productos requiere la mayoría de las áreas funcionales, como finanzas, marketing, producción y otras funciones.
\paragraph{Departamentalización cliente / mercado}
Una organización puede encontrar ventajoso organizarse según los tipos de clientes a los que sirve. Por ejemplo, una empresa de distribución que vende a consumidores, clientes gubernamentales, grandes empresas y pequeñas empresas puede decidir basar sus divisiones principales en estos mercados diferentes. Su personal puede volverse competente para satisfacer las necesidades de estos diferentes clientes. Del mismo modo, una organización que proporciona servicios como contabilidad o consultoría puede agrupar a su personal de acuerdo con este tipo de clientes. La Figura 2 muestra una organización agrupada por clientes y mercados.
\subsubsection{Estructura organizacional matriz}
Algunas organizaciones encuentran que ninguna de las estructuras mencionadas satisface sus necesidades. Un enfoque que intenta superar las deficiencias es la estructura matricial, que es la combinación de dos o más estructuras diferentes. La departamentalización funcional se combina comúnmente con grupos de productos por proyecto. Por ejemplo, un grupo de productos quiere desarrollar una nueva adición a su línea; Para este proyecto, obtiene personal de departamentos funcionales como investigación, ingeniería, producción y comercialización. Luego, este personal trabaja bajo el gerente del grupo de productos durante la duración del proyecto, que puede variar mucho. Este personal es responsable ante dos gerentes \\
Una ventaja de una estructura matricial es que facilita el uso de personal y equipos altamente especializados. En lugar de duplicar funciones como se haría en una estructura de departamento de producto simple, los recursos se comparten según sea necesario. En algunos casos, el personal altamente especializado puede dividir su tiempo entre más de un proyecto. Además, mantener departamentos funcionales promueve la experiencia funcional, mientras que al mismo tiempo trabajar en grupos de proyectos con expertos de otras funciones fomenta la fertilización cruzada de ideas.\\
Las desventajas de una organización matricial surgen de la estructura dual de informes. La alta dirección de la organización debe tener especial cuidado en establecer procedimientos adecuados para el desarrollo de proyectos y mantener claros los canales de comunicación para que no surjan conflictos potenciales y obstaculicen el funcionamiento de la organización. Al menos en teoría, la alta gerencia es responsable de arbitrar tales conflictos, pero en la práctica las luchas de poder entre el gerente funcional y el gerente de producto pueden evitar la implementación exitosa de arreglos estructurales de matriz. Además de la matriz de producto / función, otras bases pueden estar relacionadas en una matriz. Las grandes corporaciones multinacionales que usan una estructura matricial combinan comúnmente grupos de productos con unidades geográficas.
\subsubsection{Unidades estratégicas de negocio}
A medida que las corporaciones se hacen muy grandes, a menudo se reestructuran como un medio para revitalizar la organización. El crecimiento de un negocio a menudo va acompañado de un crecimiento en la burocracia, ya que se crean puestos para facilitar el desarrollo de necesidades u oportunidades. Los cambios continuos en la organización o en el entorno empresarial externo pueden hacer de esta burocracia un obstáculo en lugar de una ayuda, no solo por el tamaño o la complejidad de la organización, sino también por una forma de pensar burocrática lenta. Un enfoque para fomentar nuevas formas de pensar y actuar es reorganizar partes de la empresa en grupos mayormente autónomos, llamadas unidades de negocio estratégicas (SBU). Dichas unidades generalmente se configuran como compañías separadas, con la responsabilidad total de pérdidas y ganancias invertida en la alta dirección de la unidad, a menudo el presidente de la unidad y / o un vicepresidente senior de la corporación más grande. Este gerente es responsable ante la alta gerencia de la corporación. Se puede ver que esta disposición lleva a cualquiera de los esquemas departamentalización mencionados un paso más allá. Las SBU pueden basarse en líneas de productos, mercados geográficos u otros factores diferenciadores. La Figura 4 muestra las SBU organizadas por área geográfica.
\subsubsection{Tendencias emergentes en la estructura organizativa}
A excepción de la organización matricial, todas las estructuras descritas anteriormente se centran en la organización vertical; es decir, quién informa a quién, quién tiene la responsabilidad y la autoridad de qué partes de la organización, etc. Tal integración vertical es a veces necesaria, pero puede ser un obstáculo en entornos que cambian rápidamente. Un organigrama detallado de una gran corporación estructurada según el modelo tradicional mostraría muchas capas de gerentes; la toma de decisiones fluye verticalmente hacia arriba y hacia abajo de las capas, pero principalmente hacia abajo. En términos generales, este es un tema de interdependencia.\\
En cualquier organización, las diferentes personas y funciones no operan de manera completamente independiente. En mayor o menor grado, todas las partes de la organización se necesitan mutuamente. Los desarrollos importantes en el diseño organizacional en las últimas décadas del siglo XX y principios del siglo XXI han sido intentos de comprender la naturaleza de la interdependencia y mejorar el funcionamiento de las organizaciones con respecto a este factor. Un enfoque es aplanar la organización, desarrollar las conexiones horizontales y restar importancia a las relaciones de informes verticales. A veces, esto implica simplemente eliminar las capas de la gerencia media. Por ejemplo, algunas empresas japonesas, incluso empresas manufactureras muy grandes, tienen solo cuatro niveles de gestión: alta dirección, gestión de planta, gestión de departamentos y gestión de secciones. Algunas compañías estadounidenses también han reducido drásticamente el número de gerentes como parte de una estrategia de reducción de personal; no solo para reducir los gastos salariales, sino también para racionalizar la organización a fin de mejorar la comunicación y la toma de decisiones.\\
En un sentido virtual, la tecnología es otro medio para aplanar la organización. El uso de redes informáticas y software diseñados para facilitar el trabajo grupal dentro de una organización puede acelerar las comunicaciones y la toma de decisiones. Aún más efectivo es el uso de intranets para hacer que la información de la empresa sea fácilmente accesible en toda la organización. El rápido aumento de dicha tecnología ha hecho posible que organizaciones virtuales y organizaciones ilimitadas sean posibles, donde los gerentes, técnicos, proveedores, distribuidores y clientes se conectan digitalmente en lugar de físicamente.\\
Se puede ver una perspectiva diferente sobre el tema de la interdependencia al comparar el modelo orgánico de organización con el modelo mecanicista. La estructura mecanicista tradicional se caracteriza por ser altamente compleja debido a su énfasis en la especialización laboral, énfasis altamente formalizado en procedimientos y protocolos definidos, y autoridad y responsabilidad centralizadas. Sin embargo, a pesar de las ventajas de coordinación que presentan estas estructuras, pueden obstaculizar tareas que son interdependientes. En contraste, el modelo orgánico de organización es relativamente simple porque desestima la especialización laboral, es relativamente informal y descentraliza la autoridad. Los procesos de toma de decisiones y establecimiento de objetivos se comparten en todos los niveles\\
Una forma común en que las organizaciones empresariales modernas avanzan hacia el modelo orgánico es mediante la implementación de varios tipos de equipos. Algunas organizaciones establecen equipos de trabajo autodirigidos como el grupo básico de producción. Los ejemplos incluyen celdas de producción en una empresa de fabricación o equipos de servicio al cliente en una compañía de seguros. En otros niveles organizacionales, se pueden establecer equipos interfuncionales, ya sea de manera ad hoc (por ejemplo, para la resolución de problemas) o de manera permanente como el medio habitual para llevar a cabo el trabajo de la organización. La Asociación de Ayuda para Luteranos es una gran organización de seguros que ha adoptado el enfoque de equipo de trabajo autodirigido. Parte del ímpetu hacia el modelo orgánico es la creencia de que este tipo de estructura es más efectiva para la motivación de los empleados. Varios estudios han sugerido que pasos tales como expandir el alcance de los trabajos, involucrar a los trabajadores en la resolución y planificación de problemas y fomentar las comunicaciones abiertas brindan una mayor satisfacción laboral y un mejor rendimiento.\\
Saturn Corporation, una subsidiaria de General Motors (GM), enfatiza la organización horizontal. Comenzó con una "hoja de papel limpia", con la intención de aprender e incorporar las mejores prácticas comerciales para ser un fabricante de automóviles exitoso en los Estados Unidos. La estructura organizativa que adoptó se describe como un conjunto de círculos anidados, en lugar de una pirámide. En el centro está la célula de producción autodirigida, llamada Unidad de trabajo. Estos equipos toman la mayoría, si no todas, las decisiones que afectan solo a los miembros del equipo. Varios de estos equipos forman un círculo más amplio llamado Módulo de Unidad de Trabajo. Los representantes de cada equipo forman el círculo de decisión del módulo, que toma decisiones que afectan a más de un equipo u otros módulos. Varios módulos forman un equipo de negocios, de los cuales hay tres en fabricación. Los líderes de los módulos forman el círculo de decisión del Equipo de Negocios. Los representantes de cada equipo de negocios forman el Consejo de Acción de Fabricación, que supervisa la fabricación. En todos los niveles, la toma de decisiones se realiza por consenso, al menos en teoría. El presidente de Saturno, finalmente, se reporta a la sede de GM.
\subsubsection{El futuro}
La consolidación de la industria, la creación de grandes corporaciones globales a través de empresas conjuntas, fusiones, alianzas y otros tipos de esfuerzos cooperativos interorganizacionales, se ha vuelto cada vez más importante en el siglo XXI. Entre las organizaciones de todos los tamaños, conceptos como la fabricación ágil, la administración de inventario justo a tiempo y las organizaciones ambidiestras están impactando el pensamiento de los gerentes sobre su estructura organizacional. De hecho, es probable que pocos líderes implementen ciegamente la estructura jerárquica tradicional común en la primera mitad del siglo. La primera mitad del siglo XX estuvo dominada por la estructura tradicional de talla única. A principios del siglo XXI ha estado dominado por el pensamiento de que las estructuras organizativas cambiantes.
\subsection{Cultura organizacional}
A medida que las personas trabajan juntas para lograr los objetivos, los grupos se convierten en organizaciones. A medida que los objetivos se vuelven más específicos y a más largo plazo, y el trabajo se especializa más, las organizaciones se vuelven más formales e institucionalizadas. Las organizaciones tienden a adquirir una vida propia y se desarrollan creencias, valores y prácticas ampliamente reconocidas, diferenciando una organización de otra y, a menudo, afectando el éxito o el fracaso de la organización. A principios de la década de 1980, los académicos de gestión comenzaron a intentar describir estos sistemas de creencias, a los que se referían como culturas organizacionales o corporativas.\\
El interés en las culturas organizacionales fue creado por el éxito de ventas de William Ouchi en 1981, Theory Z: How American Business Can Meet the Japanese Challenge. Ouchi consideró que la cultura organizacional es un determinante clave de la efectividad organizacional. En 1982, otros dos best-sellers, Terrance Deal y Allan Kennedy Corporate Cultures: The Rites and Rituals of Corporate Life y Thomas Peters y Robert Waterman en busca de la excelencia, respaldaron la idea de que las empresas excelentes tendían a tener culturas fuertes.\\
Una cultura organizacional se define como los supuestos, valores y creencias compartidos que guían las acciones de sus miembros. La cultura organizacional tiende a ser moldeada por los valores de los fundadores, la industria y el entorno empresarial, la cultura nacional y la visión y el comportamiento de los líderes principales. Hay muchas dimensiones o características de la cultura organizacional que se han definido. Por ejemplo, un estudio de investigación realizado por JA Chatman y KA Jehn en 1994, identificó siete características principales que definen la cultura de una organización: innovación, estabilidad (mantener el status quo versus crecimiento), orientación de las personas, orientación a los resultados, tranquilidad, orientación detallada y Orientación del equipo.\\
Las grandes organizaciones generalmente tienen una cultura dominante que es compartida por la mayoría de la organización y subculturas representadas por grupos de individuos con valores o creencias únicos que pueden o no ser consistentes con la cultura dominante. Las subculturas que rechazan la cultura dominante se llaman contraculturas. Las culturas organizacionales fuertes son aquellas en las que la gran mayoría de los miembros organizacionales creen firmemente en los valores centrales de la cultura dominante. Una cultura fuerte tiende a aumentar la consistencia del comportamiento y reducir la rotación. Sin embargo, las culturas fuertes pueden ser menos adaptables al cambio, crear barreras a la diversidad y crear barreras para adquisiciones y fusiones exitosas.
\subsubsection{Ajuste cultural entre la organización y los miembros}
Hay muchas prácticas dentro de una organización que tienden a mantener viva una cultura y medir el ajuste cultural entre la organización y sus empleados. Muchas de las prácticas de recursos humanos como la selección, la evaluación del desempeño, la capacitación y el desarrollo profesional refuerzan la cultura de la organización. Las creencias organizacionales también tienden a influir en las normas laborales, las prácticas de comunicación y las posturas filosóficas de los empleados. Las organizaciones utilizan un proceso llamado socialización para adaptar a los nuevos empleados a la cultura de la organización. Si los empleados no se adaptan bien, sienten una presión creciente de los supervisores y de los compañeros de trabajo que están mejor aculturados. Pueden quedarse y pelear, quedarse y aislarse, o abandonar la organización, voluntaria o involuntariamente.\\
En contraste, los empleados que entienden y comparten los valores de la organización tienen una mejor base para tomar decisiones que coincidan con los objetivos de la empresa. Muchas organizaciones compiten a través de la innovación. Cuando la mayoría de los empleados comprenden y respaldan las expectativas de la organización, se dedica menos tiempo a explicar, instruir y crear consenso antes de intentar algo innovador. Además, el nivel de error será más bajo en la mayoría de los casos. Los empleados que están bien aculturados también encuentran que su trabajo es más significativo: son parte de algo más grande que ellos y contribuyen a él. Por lo tanto, un buen ajuste cultural entre los empleados y la organización contribuye a la retención de los empleados, la productividad organizacional y las ganancias.
\subsubsection{Medios de transportar cultura}
Las organizaciones a menudo transmiten valores culturales explícitamente mediante declaraciones de misión o credos corporativos, o en menor medida a través de lemas, logotipos o campañas publicitarias. Los líderes y gerentes también muestran lo que la organización valora por lo que dicen y hacen, qué recompensan, a quién hacen aliados y cómo motivan el cumplimiento. Otros elementos de la cultura aparecen tácitamente en los símbolos y el comportamiento simbólico: por ejemplo, los protocolos de reunión, el comportamiento de saludo, la asignación y el uso del espacio, y los símbolos de estado son algunas de las áreas donde a menudo se desarrollan las normas organizacionales. La cultura puede regular las normas sociales, así como las normas de trabajo o tarea.\\
La orientación para los nuevos empleados que suelen ofrecer las organizaciones transmite elementos culturales seleccionados de los que la gerencia es consciente y orgullosa. Sin embargo, algunos elementos culturales pueden ser inicialmente desagradables, y otros pueden ser difíciles de expresar con palabras. Por ejemplo, una orientación rara vez diría abiertamente que la cultura recompensa el abandono de la vida personal y exige una semana laboral de 60 horas, aunque estas expectativas no son desconocidas en la vida corporativa. Los nuevos empleados perspicaces aprenden sobre elementos culturales tácitos a través de la observación y al interrogar a empleados o mentores de confianza. Este no es un aprendizaje de una sola vez; Los empleados deben seguir atentos a las señales de que las reglas están cambiando.\\
Estas reglas organizacionales incluyen declaraciones de política explícitas, pero también un conjunto mucho más grande y menos evidente de expectativas organizacionales no escritas. Los empleados atentos los resuelven antes que otros. Escuchan las metáforas, imágenes y dichos que son comunes en la organización. Observan, por ejemplo, las consecuencias de los errores de los demás para llegar a conclusiones sobre el comportamiento apropiado.\\
Las organizaciones también comunican valores y reglas a través de artefactos mostrados. Por ejemplo, en algunas organizaciones, la oficina del CEO muestra muchos símbolos de riqueza, como arte original costoso o antigüedades. En otros, el espacio de trabajo del CEO es muy espartano y difiere poco del de otros ejecutivos y gerentes de alto nivel. En el primer caso, un gerente con otras fuentes de ingresos podría permitirse símbolos de estatus similares, pero no sería prudente mostrarlos, ya que esto podría percibirse como una competencia con el CEO. En el último caso, la exhibición de riqueza personal por parte de las personas en general probablemente sería contraria a los valores organizacionales.\\
Incluso la forma en que se presenta una planta física comunica mensajes culturales: ¿Es un área abierta donde todos pueden ver a todos? ¿Hay cubículos? ¿Hay oficinas privadas? ¿Es fácil o difícil moverse y comunicarse entre áreas funcionales? ¿Se han considerado o ignorado la ergonomía y la conveniencia? ¿Existen espacios neutrales adecuados para que las personas se reúnan para tomar decisiones y resolver problemas? ¿Las salas de descanso y almuerzos invitan o desalientan el uso?
\subsubsection{Algunos componentes de la cultura}
La idea de que las organizaciones tienen culturas surgió originalmente de la etnografía, el estudio y la descripción de las culturas sociales humanas. Los investigadores en cultura organizacional han tomado prestado parte de ese lenguaje. Los individuos en las sociedades asumieron "roles" específicos, como gobernante, sacerdote, historiador o maestro. En las organizaciones, surgen roles similares. El historiador o narrador, por ejemplo, suele ser un empleado de toda la vida que narra historias inspiradoras sobre los primeros años de la compañía o su evolución. En las historias se incluyen muchos de los valores centrales que impregnan la organización. Este "folklore organizacional" incluye historias repetidas a menudo sobre el fundador, un CEO a largo plazo, un despido dramático, o un individuo que subió de rango muy rápidamente debido a algún atributo altamente valorado por la empresa. Las estrellas de una organización son comparables a los héroes de una cultura social. Las historias de éxito de una organización producen "modelos a seguir" para los ambiciosos.\\
Las organizaciones desarrollan "ritos y rituales" comparables a las actividades tradicionales dentro de una cultura étnica. Mientras que algunas organizaciones pueden enfatizar las ceremonias de premiación, otras pueden desestimar el reconocimiento explícito y los comportamientos de afiliación. Incluso otros podrían fomentar la "gestión caminando", en la cual los gerentes pasan tiempo uno a uno lejos de sus escritorios para alabar o criticar a las personas. Como otro ejemplo, el almuerzo con el presidente podría ser una larga tradición, aunque la cantidad de comunicación real variará de una organización a otra según las reglas no escritas sobre quién habla con quién.\\
Aunque todas las organizaciones tienen redes de comunicación tanto formales como informales, la cultura organizacional afecta fuertemente el contenido, la confiabilidad y la influencia de la red informal o "vid". Cuando la información a través de canales formales es escasa, la vid transporta más tráfico. Los líderes conscientes de la importancia de la cultura intentan encontrar formas de aprovechar y monitorear la vid y, a veces, usarla agregando información.
\subsubsection{Cambio de cultura}
La cultura de una organización se compone de características relativamente estables que se basan en valores profundamente arraigados y reforzados por muchas prácticas organizacionales. Sin embargo, una cultura organizacional puede ser cambiada. Es más probable que ocurran cambios culturales cuando hay un revés dramático, como una crisis financiera o cuando hay una rotación en el liderazgo superior. Además, las organizaciones más jóvenes y más pequeñas y las organizaciones con una cultura débil son más susceptibles de cambio.\\
El cambio cultural deliberado e importante ocurre por mandato ejecutivo, por la implementación de un plan o una combinación de estos medios. Cuando el liderazgo cambia o cuando el liderazgo existente se compromete a cambiar, los empleados aprenden que los viejos supuestos con los que se sentían cómodos ya no son seguros. Después de una fusión o adquisición, por ejemplo, "cómo hacemos las cosas aquí" cambiará, a veces rápida y radicalmente. Un equipo de liderazgo inteligente implementa un proceso planificado de cambio cultural. El proceso generalmente consiste en una serie de comunicaciones bidireccionales que generan las suposiciones predominantes, aseguran a los empleados que los cambios pueden beneficiarlos, introducen (a veces gradualmente) la nueva visión y trabajan para obtener el compromiso y el apoyo de los empleados. Los líderes también deben modelar la nueva cultura para los demás y cambiar la estructura y las prácticas de gestión de la organización para apoyar la nueva cultura. Si los líderes se saltan el proceso o hacen un trabajo inadecuado, los empleados de todos los niveles experimentan estrés, confusión y enojo. Sin embargo, cuando se introduce un cambio para no despertar miedo y resentimiento, la transición puede ser relativamente suave.\\
Un estudio de investigación de 1992 realizado por JP Kotter y JL Heskett mostró que el desempeño financiero a largo plazo era el más alto para las organizaciones con una cultura adaptativa. Un ejemplo de cuándo las organizaciones deben adaptar su cultura es cuando las organizaciones se vuelven multinacionales. Con el aumento de las organizaciones globales, ha quedado claro que las culturas nacionales afectan a las culturas organizacionales. Además de las diferencias de idioma, los empleados aportan al trabajo muchos supuestos radicalmente diferentes sobre aspectos tales como la dignidad del trabajo, la relación adecuada entre el empleado y el supervisor, el valor de la iniciativa, el tratamiento de la información no deseada y la presentación de quejas. Organizaciones con clientes internacionales, y aún más, aquellos con operaciones globales han necesitado aprender cómo adaptarse a un entorno multicultural. La falta de adaptación pone en peligro las posibilidades de éxito de una organización en el extranjero.\\
Para resumir, la cultura organizacional son las suposiciones, creencias y valores compartidos que poseen la mayoría de los miembros de una organización. La cultura se transmite de manera explícita e implícita. Los recién llegados a una organización deben asimilar rápidamente una gran parte de la cultura. Los empleados veteranos también deben estar conscientes del cambio cultural, especialmente cuando cambia el liderazgo. Una cultura fuerte que esté alineada con el contexto estratégico de la organización y que se adapte a los cambios ambientales puede mejorar el desempeño financiero a largo plazo de una organización.
\subsection{Predicamento del gerente por primera vez}
omo gerentes por primera vez, ¿cuáles son nuestros sentimientos iniciales? El primer sentimiento es de euforia al ser promovido o contratado como gerente. Pero lentamente las cosas cambian. Conocemos la realidad más dura, los desafíos de la gestión de personas. El buen humor se reemplaza con aprensiones y ansiedad sobre cómo hacer el trabajo a través de otros. Este artículo nos presenta los desafíos que enfrentamos cuando nos convertimos en gerentes. Luego, analiza las formas en que podemos abordar algunas de estas preocupaciones iniciales y comenzar nuestro viaje de gestión con gracia.\\
¿Cuándo nos da una organización el título de ser gerente? No se trata simplemente de responsabilidades adicionales o trabajo interfuncional o interacciones con los clientes. Este título está reservado exclusivamente para ejecutivos con responsabilidades de personas. Un gerente en virtud de su título, en primer lugar, "administra a las personas". Es responsable no solo de sus propias contribuciones individuales, sino también del desempeño de su equipo. Los asesora, establece sus objetivos en alineación con los objetivos de la organización, los guía sobre cómo lograr esos objetivos, mejora sus habilidades, evalúa su desempeño y les brinda retroalimentación para la improvisación. Parece una tarea gigantesca, ¿no es así? Sin embargo, si tenemos algunos consejos pertinentes antes de embarcarnos en este duro viaje, puede volverse más simple y atractivo.\\
¿Qué cambia cuando nos convertimos en gerente? Anteriormente fuimos responsables de nuestros entregables y logros individuales. Ahora somos responsables de todo el equipo: su establecimiento de objetivos, evaluación del desempeño, retroalimentación, etc. Por un lado, nos complace alcanzar el puesto de gerente y, por otro lado, tenemos la ansiedad y la inquietud de cómo manejaremos a los demás. nuestro equipo. Los miembros de nuestro equipo tienen diferentes aspiraciones, antecedentes, aptitudes y habilidades. A veces son mayores para nosotros en edad y experiencia. Tienen sus propias nociones de lo que funciona y lo que no funciona. Incluso si les decimos qué hacer, es posible que no lo hagan de esa manera o que solo hagan exactamente lo que se les dice en lugar de aplicarle la cabeza; ambos son situaciones difíciles para estar.\\
En su artículo de HBR, Linda Hill señala que los gerentes son responsables de crear agendas para todo el equipo, pero su antigua carrera no los ha preparado para tal trabajo. Como gerentes por primera vez, las personas se dan cuenta de que existe una brecha sustancial entre sus capacidades actuales y los nuevos requisitos laborales. Ella menciona que cuando nos convertimos en gerentes, creemos que ahora tendremos la autonomía y la libertad para implementar nuestras ideas, y poco nos damos cuenta de la responsabilidad y responsabilidad adicionales que el rol del gerente trae consigo.\\
Hay una variedad de desafíos que enfrentamos cuando nos convertimos en gerentes por primera vez. Algunos de ellos son externos, como no poder comprender las aspiraciones de los miembros del equipo o sus fortalezas y luego asignar tareas apropiadas o no poder darles retroalimentación oportuna, etc. Esto se puede aprender con el tiempo con experiencia. Incluso su organización sabe que es posible que no lo haga bien la primera vez. Lo que es más preocupante es que estos desafíos externos son a menudo el resultado de aprehensiones internas como nuestra propia inseguridad para enfrentar la situación. Por ejemplo, nos preocupa si tendremos éxito como gerente. Si la tarea no está sucediendo bien, lo haremos nosotros mismos en lugar de confiar en el equipo para completarla. A menudo nos angustiamos por la actitud del equipo hacia el trabajo. A veces no tenemos respuestas apropiadas para lo que creemos que son excusas del lado del equipo de por qué se retrasó el trabajo. Y creemos que nos están dando por sentado. Nos volvemos más agresivos en nuestros tratos y mezquinos en pasar tiempo con el equipo en lugar de llegar a la raíz del problema.
\subsubsection{Facilita la transición}
Como gerentes por primera vez, comenzamos con una nota muy alta, pero lentamente la fascinación se reemplaza por la frustración. Para las personas que pueden dominar el arte de administrar personas, el trabajo se vuelve más atractivo y lleno de acontecimientos. La pregunta es qué hará que el equipo nos respete y nos vea como su líder. A menos que traigamos a la mesa algo distintivo y valioso, es difícil establecer una relación con el equipo. Aquí hay algunos consejos para los gerentes por primera vez, que los ayudarán a mantener a raya parte de su aprehensión y harán de la gestión de las personas un esfuerzo constructivo.\\
\paragraph{Tomesé el tiempo}
La mayoría de las veces queremos lograr demasiado demasiado pronto. Ponemos nuestro mejor pie adelante, pasamos largas horas y trabajamos muy duro. Pero cuando se trata de administrar personas, lleva tiempo: tiempo para comprender la situación, tiempo para comprender las habilidades y debilidades de las personas, tiempo para alinearse con el equipo. Por lo tanto, tómese un tiempo para instalarse antes de saltar a conclusiones y agendas.
\paragraph{Simplifique}
Las organizaciones a menudo son culpables de complicar las cosas. Podemos dejar una marca simplificando las cosas. Cualquiera sea el problema, haga las preguntas básicas. Si hay problemas con la entrega o las compras o las ventas, piense en el problema fundamental que enfrenta el equipo y podemos hacer algo para resolver el problema. No asuma que el método prescrito es la mejor manera de hacer las cosas. Pregúntele al equipo si tienen una mejor idea. Esto aumentará su participación en la toma de decisiones y les permitirá asumir la responsabilidad.
\paragraph{Desarrollo del equipo}
Ya no somos responsables de nuestra propia progresión profesional. Hay un equipo que nos admira por su desarrollo. Tener una agenda de desarrollo para el equipo. Vea cómo podemos ayudarlos a pasar al siguiente nivel. Sé sincero al respecto. Dígales explícitamente lo que les falta, cómo pueden superarlo. Brinde retroalimentación periódica al equipo sobre su desempeño y hágales saber las oportunidades de crecimiento en la empresa y lo que pueden hacer para lograrlas.
\paragraph{Tener y dar confianza}
cree en ti mismo. Fuiste elegido para liderar porque tienes algo en ti. Incluso si comete errores, tenga el coraje de reconocer. El equipo te mirará con mucho más respeto. Cuando hable con la alta gerencia, represente y proteja al equipo. Hable no solo de sus logros individuales, sino también de las contribuciones del equipo. Esto le dará confianza a su equipo de que, bajo su liderazgo, estarán seguros y la alta gerencia se sentirá tan segura de que el equipo puede lograr el éxito con usted como el gerente.
\paragraph{Tomar decisiones}
A veces dudamos en tomar una decisión porque tememos estar equivocados. En la gestión, nada viene en absoluto. Ponemos nuestra debida diligencia y tomamos la decisión más adecuada según las circunstancias. A la luz de los nuevos hechos, esta decisión puede no parecer plausible. Pero eso está bien. El equipo, y también la organización, sufre más debido a las indecisiones que a las decisiones que tuvieron que cambiarse más tarde en función de los nuevos desarrollos.
\paragraph{No te estreses}
Está bien si cometes algunos errores al principio. Acéptelos y avance. Es difícil ser un gerente de personas. No se asuste si las cosas no van como usted quiere que sean. Dale tiempo. Intervenir cuando sea necesario. No se atasque con las sacudidas iniciales y, por el amor de Dios, nunca haga microgestiones.
\paragraph{Solicite orientación}
No piense que estamos solos en esto. La alta gerencia está allí para guiarnos y apoyarnos. A veces, cuando somos vagos en nuestra representación del problema, es posible que no recibamos respuestas adecuadas, pero a menudo recibimos perlas de sabiduría de la alta gerencia, lo que nos ayuda a manejar mejor a largo plazo. A menudo pensamos que al preguntar expondremos nuestra debilidad. En realidad, es al revés, al no preguntarnos, estamos aumentando las posibilidades de fracaso.
\subsubsection{Conclusión}
El desafío frente a un gerente es el logro de objetivos a través de otros. Es diferente cuando eres un contribuyente individual. Con un equipo subordinado, debe supervisarlos, guiarlos, evaluarlos y darles los comentarios necesarios. Con un gerente incompatible, no solo hay un costo emocional personal soportado por los empleados, hay un enorme costo financiero que es soportado por la organización debido a la pérdida de motivación y productividad. Este artículo nos presenta los desafíos de convertirse en gerente y las formas de resolver algunas de estas preocupaciones. Estas ideas nos brindan consejos sobre cómo ser un gerente efectivo por primera vez y llevar a nuestro equipo por el camino del éxito.


\chapter{Hacer el trabajo a través de otros}
\section{Cumpliendo múltiples expectativas}
En esta sección, cubriremos:
\begin{itemize}
\item Importancia de cumplir con múltiples expectativas.
\item Importancia de los miembros del equipo directivo. 
\item Importancia de la gestión de las partes interesadas.
\item Importancia de la priorización.
\end{itemize}
\subsection{Múltiples stakeholders}
Hay varios interesados en la vida de un Gerente por primera vez. Cada una de estas partes interesadas tiene su propia importancia y cumplir con sus expectativas es un aspecto crucial de ser un Gerente por primera vez.\\
Algunas de las partes interesadas son: \textbf{sus clientes, jefes, proveedores, informes directos, gerente del gerente y gobierno}.\\
Dediquemos un tiempo a comprender $"$quiénes son las partes interesadas para un gerente$"$ por qué y es el término $"$stakeholders$"$ tan importante para ser un administrador eficaz. Las partes interesadas son personas que tienen interés en lo que hacemos. Por ejemplo, como gerente tiene entregas que tendrá que entregar a la organización. Entonces la organización es una de las partes interesadas. Su gerente inmediato cuyas metas están alineadas con sus metas es otra parte interesada. Supongamos que usted está en la función de ventas y que está tratando con clientes. Son otra parte interesada. O podría estar en el departamento de compras y tratar con vendedores y proveedores. Ellos son sus partes interesadas. Su equipo que le informa directamente es otra parte interesada. Hay compañeros en otros departamentos con los que interactúa. Digamos que usted está en Producción y que interactúa con las personas desde el mantenimiento, desde la compra, desde la calidad, por lo que todos son sus partes interesadas. Y, finalmente, la alta dirección tiene una expectativa de usted. Y ellos son sus partes interesadas. Como puede ver, como gerente, muchas de las actividades que realiza tocarán a uno o más de estos interesados, todo el tiempo. No todos los intereses y expectativas de las partes interesadas son legítimos. Muchas veces, el interés de una parte interesada puede chocar con el interés de otra parte interesada. Por ejemplo, se espera que entregue un producto. Durante la misma semana hay una capacitación, un importante programa de capacitación que se está llevando a cabo en la organización para el cual debe nominar a uno de los miembros de su equipo. Como puede ver, aquí hay un cliente como parte interesada a quien debe entregar el producto por un lado y aquí está el empleado que requiere una capacitación periódica que está por el otro. Y como gerente, debe equilibrar ambos intereses para ser efectivo a largo plazo. Por ahora, creo que es bastante evidente que la gestión de las partes interesadas es uno de los aspectos clave de ser un administrador eficaz.
\subsection{Desafíos de priorización}
- hola
- ¡Hola señor! Esta es Neeta de RRHH. Hay una capacitación organizada para los miembros de su equipo sobre el desarrollo efectivo del equipo a las 10 a.m. de hoy, y sin embargo, solo tengo la confirmación de dos miembros de su equipo.
- Estoy tan despistado Cien correos en un día?
- Sr. Joe, tiene una reunión con un cliente muy valioso en media hora sobre el problema urgente de la entrega.
- No tengo idea de cuál es el problema.
- ¡Hola! Me gustaría reunirme con la Sra. Ross. ¿Ella está disponible?.
- Lo siento, señor Joe. Ella está en una reunión importante. Debería estar disponible a las cuatro de la tarde. De todos modos, estaba a punto de llamarte yo mismo. Ella me había ordenado que le informara sobre la ceremonia de apertura de nuestra oficina de Church Gate. Necesitas asistir a esto en su nombre a la 1 PM, hoy.
- ¿Dónde debería comenzar mi día?
- No debería entrar en pánico. Déjame priorizar mi trabajo. Permítanme enumerar las tareas que necesito abordar de inmediato y las que puedo completar en una semana. Entonces, estas son las cosas que necesito abordar de inmediato: envíe a los miembros del equipo para la capacitación, asista a la ceremonia de apertura de Church Gate y responda correos electrónicos urgentes. Lo que puedo completar en esta semana es conocer a mi gerente y comprender sus prioridades y expectativas. Conozca a cada miembro del equipo individualmente para comprender sus objetivos y tareas, comprender los desafíos y obstáculos que enfrenta mi gerente, y también conocer a personas de otros departamentos. Oh! Esto se ve bien ahora.
\subsubsection{Priorización-Parte 1}
$"$La clave no es priorizar lo que está en su horario, sino programar sus prioridades$"$ : Stephen Covey.\\
El ejemplo de Joe, el gerente, es algo a lo que todos nos podemos conectar fácilmente. Solo tenemos un tiempo limitado en el día para cumplir con una serie de tareas. Entonces, en cierto sentido, todos nosotros como gerentes necesitamos tener la capacidad de poder priorizar nuestro tiempo, nuestra energía y nuestro esfuerzo en actividades que creen el impacto deseado para la organización.\\
Entonces, ahora nos estamos moviendo para analizar el segundo aspecto clave del trabajo gerencial, que es la priorización. La priorización es una habilidad. Ahora, ¿qué es una habilidad? Una habilidad es algo que se puede aprender si se hace con disciplina durante un período de tiempo.\\
¿Qué es la priorización? \textbf{La priorización es la capacidad de poder mirar la lista de tareas e identificar cuáles de las tareas en la lista de tareas son aquellas en las que quiero invertir mi tiempo, esfuerzo y energía.} Estoy seguro de que todos nosotros, como gerentes, hemos visto que comenzamos con una lista, y al comienzo del día y al final del día cuando volvemos y verificamos, no parece que la lista se haya completado . Sí, hemos eliminado algunos de esos artículos. Pero, ha habido muchas tareas, que se han agregado a esa lista. Entonces, administrar esa lista, bueno, requiere la capacidad de priorizar.
\subsubsection{Priorización-Parte 2}
Para los gerentes priorizar hay cuatro cosas clave que son importantes. La priorización está en la cabeza. \textbf{Primero, necesito poder compilar una lista de tareas $"$pendientes$"$}. Entonces, lo que significa que cuando usted es un gerente, debe pasar esa media hora por la mañana, o puede ser que al final del día anterior simplemente enumere las cosas que deben hacerse. Puede significar escribir correos electrónicos a las personas o puede significar que debe procesar los pagos de sus proveedores rápidamente o puede significar una reunión con un miembro de otro departamento o puede significar pasar tiempo con uno de sus empleados que está teniendo un problema personal; podría ser cualquiera de ellos, pero todos estos deben estar en su lista de $"$cosas por hacer$"$.\\
\textbf{El segundo aspecto de la priorización se trata de determinar qué es urgente y qué es importante.} De hecho, Stephen Covey, uno de los principales expertos, habla sobre los gerentes y todos nosotros, incluso como individuos que adquieren la capacidad de poder discriminar entre urgente e importante. Solo quiero ver lo que sucedió en la última semana de tu vida. Encontrará que todas las tareas urgentes se realizarán porque alguien más lo está presionando para que las haga, pero las tareas realmente importantes nunca se realizan porque son demasiado importantes. Y, de hecho, si lo miras, \textbf{las tareas importantes son las que tienen un alto impacto en el futuro}. Entonces, como gerente, es importante para nosotros determinar y hacer esta distinción entre urgente e importante.\\
\textbf{El tercer aspecto de la priorización es comprender cuál de las tareas tiene valor a largo plazo para usted, su departamento, su gerente y su organización.} Cuando hablemos sobre la autogestión, volveré a reflexionar sobre este punto. Las tareas más importantes que están en su lista también pueden ser las tareas que puedan generar un alto impacto y valor para usted. Y por lo tanto, evaluar periódicamente cuáles de las tareas son de alto impacto y de alto valor para usted personalmente y para la organización es otro aspecto de la priorización.\\
\textbf{El cuarto aspecto es pensar en el esfuerzo que requiere cada una de esas tareas que ha priorizado.} A menudo he descubierto que algunas tareas requieren muy poco esfuerzo y algunas tareas requieren mucho esfuerzo. Y mi propia experiencia ha sido que muchas veces cometo un error al evaluar el esfuerzo que se requiere para completar una tarea. Estoy seguro de que todos tenemos ejemplos. Y recuerdo a uno de mis participantes en el programa que me estaba dando un ejemplo de cómo hacer una presentación para un cliente puede llevar unas cuatro horas. Pero una vez que comenzó a trabajar en él, se dio cuenta de que necesitaba una comprensión profunda del negocio del cliente, los otros competidores en el campo que estaban proporcionando productos similares al cliente y luego podían discutir qué era el producto de su organización. Eso hizo la diferencia. Y me estaba diciendo que se dio cuenta de que había pasado cerca de 12 horas durante un período de cinco días haciendo esa presentación. Y en su evaluación, realmente dijo con sus propias palabras lo que dijo: $"$¿Cómo es posible que pueda ser tan poco realista con respecto a mi capacidad?$"$ Estoy seguro de que mientras hablo esto, muchos de ustedes están recordando experiencias que han tenido de este tipo en las que han ido y le han dicho a su gerente: $“$Oh, debería ser posible. Estoy seguro de que puedo hacerlo mañana y luego volví solo para descubrir que mañana era tan poco realista $"$. Creo que lo más importante en la priorización es también obtener una evaluación realista y precisa de cuánto tiempo nos llevará a mí o a los miembros de mi equipo poder realizar una tarea.\\
\textbf{Y finalmente, sé flexible.} Sí, es importante priorizar, es importante delinear sus tareas, es importante poder ver la diferencia entre urgente e importante, pero ser flexible en su mente. Y esa flexibilidad le permite volver a priorizar muy rápidamente una vez que surge una crisis o surge un problema inesperado.
\section{Comprensión de los miembros del equipo}
En esta sección, nos centraremos en:
\begin{itemize}
\item Comprender a los miembros de tu equipo.
\item Reconociendo la diferencia individual.
\item Beneficios de la diversidad del equipo.
\end{itemize}
\subsection{Entendiendo a tu equipo}
Muy a menudo, al ser un Gerente por primera vez,  no tiene el lujo de elegir a los miembros de su equipo. Sin embargo, su capacidad para comprender bien a los miembros de su equipo es extremadamente crítica para ser un gerente efectivo.\\
El tercer aspecto de ser un gerente efectivo es la capacidad de comprender a los miembros de su equipo. Estoy cien por ciento seguro de que ninguno de ustedes ha tenido el lujo de poder elegir a los miembros de su equipo. En la mayoría de los casos, en realidad lo que sucede es que somos promovidos a un equipo donde la mayoría de los miembros del equipo ya han estado en ese equipo y en esa organización durante un período de tiempo. Hay situaciones en las que podría haber sido un miembro del equipo en el equipo y podría haber sido promovido como gerente de ese equipo. Obviamente, los desafíos en tales contextos son mayores porque aquellos miembros que fueron sus pares hasta la semana pasada ahora se convierten en sus informes directos. Y obviamente, hay dinámicas asociadas con esto.\\
Cuando se convierte en gerente y tiene reportes directos en el equipo que han sido parte del equipo o una división o unidad durante mucho más tiempo que usted, comprenden la organización y también han trabajado con diferentes gerentes antes tú. Ahora, cuando entras como gerente, hay dos desafíos. Primero, usted es nuevo y, por lo tanto, es muy necesario que pueda comprender las motivaciones, las necesidades, las aspiraciones, las habilidades, el conocimiento y la experiencia que poseen estos miembros del equipo.\\
El segundo aspecto es que eres un extraño. Debido a que ya han estado allí y han establecido formas de hacer el trabajo, para que entres y entiendas si están actuando o no, es realmente un gran desafío. Aún más interesante. Cuando ingresa a este nuevo equipo, hay otros miembros de la organización que podrían haber sido gerentes de este equipo o que tienen opiniones y percepciones sobre los miembros del equipo, y también es probable que le informen sobre los juicios que tienen.\\
Déjame darte un ejemplo.  Alguien se esta mudando a un nuevo departamento como gerente. Su compañero, que antes era gerente en ese grupo, vino y le dijo que había una persona en particular y, digamos, Steve. Peer le dio un comentario mencionando que Steve es una persona realmente bien informada, pero de alguna manera no puede traducir eso en el trabajo y, por lo tanto, su calidad de trabajo es simplemente promedio. Curiosamente, casi el mismo Steve, su gerente, vino y le dijo que es una persona muy trabajadora pero que no tiene un buen desempeño. Ahora, aquí hay dos personas que dan dos percepciones sobre el mismo individuo. En la última semana, cuando hicimos la percepción, todos hubiéramos reconocido que la percepción juega un papel crítico en la forma en que vemos a los demás. Ahora imagínese, usted es el gerente y ha recibido estos dos comentarios, pero personalmente no ha tenido la oportunidad de interactuar con Steve. En tu cabeza, ya has desarrollado percepciones sobre Steve, incluso sin conocer a Steve. Y cuando interactúas con Steve, ¿cuál de las percepciones hay en tu mente? ¿O tus compañeros? Ambas percepciones afectarán la forma en que interactúas con Steve, pero lo más importante es qué juicios haces sobre Steve como miembro de tu equipo después de la reunión. Luego el gerente seis semanas después se dio cuenta de lo equivocadas que estaban ambas evaluaciones sobre Steve. Y en su evaluación, Steve era trabajador, sabía cuál era el problema y los problemas, pero era reservado y reservado para sí mismo. Y uno de los descubrimientos que hizo  fue que Steve fue capaz de entregar a tiempo, pero nunca pensó que era importante mantener informado al gerente o a los compañeros sobre lo que estaba haciendo.\\
A estas alturas, puede ver que muchas dinámicas interpersonales relacionadas con su equipo y sus informes directos, una gran parte de eso está conformado por las percepciones que otros tienen sobre los miembros de su equipo y, por lo tanto, su capacidad para comprender bien a los miembros de su equipo es extremadamente crítico para ser efectivo y para obtener el respeto de los miembros de su equipo.\\
Al final del día, las personas trabajan para usted porque usted es creíble, transparente y merece respeto.
\subsection{Beneficios de la diversidad del equipo}
No hay dos personas iguales . Como gerente, es importante comprender que los miembros de su equipo pueden no tener el mismo nivel de experiencia. Es importante que reconozca la diversidad para dirigirla, guiarla, apoyarla y capacitarla de manera efectiva.\\
\subsubsection{Entendiendo la diversidad}
- $"$El único problema que tengo ahora es que de alguna manera los miembros de mi equipo no piensan como yo$"$.  
- $"$Si los miembros de tu equipo pensaron como tú, entonces quizás no te necesiten$"$.\\
Quiero que reflexione sobre esto, porque, como gerentes cuando somos promovidos, nuestras organizaciones creen que tenemos algo de potencial. Y ese potencial se refleja en la forma en que cumplimos con nuestras tareas y actividades. Es posible que los miembros de nuestro equipo no tengan la experiencia necesaria para poder apreciar o ver el punto de vista que usted puede ver. Y es por eso que es necesario que usted dirija, guíe, apoye y capacite a los miembros de su equipo. Pero, a menudo, lo que usan los gerentes son sus propios criterios de cómo se evalúan a sí mismos. Usan esos criterios en sus informes directos.\\
Y déjame darte un ejemplo. Uno de los gerentes dijo que sus reportados directos no toman la iniciativa de venir a hablar con él. Y. por lo tanto, y dijo, cuando era un contribuyente individual, acudía con mayor frecuencia a sus gerentes para preguntar, buscar y formular preguntas. Creo que lo importante para todos ustedes es reconocer que no hay dos personas iguales. Y sus reportes directos son diferentes. Cada uno de ellos tiene sus propias personalidades únicas. Y, dada su experiencia, su comprensión, su forma de pensar, todo esto afectará su forma de comportarse. Como gerente, una de las reglas más importantes es que no utilices tu criterio para evaluar a los demás. Ellos no son tú. Son personas diferentes. Y, como gerente, también es importante reconocer que si todo el mundo pensaba como usted, entonces tal vez no sea necesario.
\subsubsection{Significado de la diversidad}
Su papel como gerente se trata realmente de alinear los pensamientos y expectativas de diferentes personas. \textbf{En este momento hay mucha investigación que dice que la diversidad en los equipos y la diversidad significa diversidad de pensamientos, diversidad de antecedentes, diversidad de experiencias}; todo esto contribuye significativamente a la resolución de problemas porque diferentes personas verán el mismo problema desde diferentes perspectivas.\\
Por lo tanto, como gerente, si realmente desea contribuir de manera significativa a su propio equipo y a la organización, es muy importante que pueda reconocer y participar con diversidad de pensamientos. Pasemos un momento para entender. Cuando hay diferentes personas que le dan diferentes soluciones al mismo problema, ¿cuál es la ventaja que tiene como gerente? Que ya ha anticipado debido a los miembros de su equipo las diferentes formas en que se puede abordar ese problema. Segundo, para cada uno de esos enfoques, los miembros de su equipo también le han proporcionado alternativas. El tercer punto es que, cuando las personas presentan diferentes puntos de vista, habrá desacuerdos. Estos desacuerdos entre los miembros conducirán a conversaciones más significativas. Cuando todo esto sucede, como gerente, te educas no solo en tu forma de ver el mundo, sino también en la forma de ver el mundo de las diferentes personas. Y que las múltiples perspectivas , que es su gestión de múltiples stakeholders, significa que la calidad de las decisiones que tome, la calidad de las soluciones que brinde será automáticamente beneficiosa para el equipo así como para la organización.\\
Por lo tanto, un gerente verdaderamente efectivo es aquel que puede manejar reportes directos que tienen diferentes perspectivas porque, como gerente, su función es sacar esas perspectivas, integrarlas y proporcionar algún tipo de solución significativa para los miembros de su equipo.\\
Regrese y mire cuál es su tolerancia a la diversidad. Mira a tus amigos, mira a las personas con las que trabajas. ¿Son exactamente como tú? Quiero que te tomes un minuto para mirar el mundo que nos rodea: la naturaleza. Hay tanta diversidad. Mira tu jardín. El jardín se ve hermoso principalmente por la diversidad de si son las flores, si son las plantas, si es solo el color verde. Por un minuto, aprecie y mire cuánto contribuye esa diversidad a la belleza.\\
Entonces, como líder efectivo, ¿realmente está aprovechando la diversidad dentro de los miembros de su equipo? Siempre y cuando evalúes a los demás según tus estándares y mientras creas que eres un gerente y que tienes la mejor solución, me temo que solo seguirás siendo mediocre.
\section{Definición y asignación de tareas}
En esta secciín, cubriremos
\begin{itemize}
\item ¿Qué es la delegación?
\item ¿Cómo definir y asignar tareas de manera efectiva a nuestro equipo?
\end{itemize}
\paragraph{Definiendo tareas efectivamente}
Déjame darte un ejemplo, estás planeando celebrar un evento. Hay varias cosas que deben hacerse. Tienes que mirar el lugar, tienes que mirar a los proveedores, tienes que mirar el sistema de sonido si tienes algún tipo de entretenimiento. Debes mirar las finanzas y el presupuesto, debes mirar las invitaciones y cada una de ellas debe definirse explícitamente si quieres un buen resultado positivo, que es un evento al que muchos asisten y la calidad de ese evento es realmente algo que la gente lleva al salir del evento.\\
En tal contexto, definir las tareas es extremadamente importante para poder gestionarlo de manera efectiva. Más cerca de casa cuando vienes a tus propios equipos y personas que trabajan contigo, la historia es muy similar. Por lo tanto, como gerente no puede escapar al hecho de que tiene que definir tareas para usted mismo, para los miembros de su equipo y, a veces, para otras partes interesadas como sus pares o sus gerentes.\\
\subsection{Conceptos básicos de la delegación}
Por último, pero no menos importante, el cuarto y el aspecto más importante de ser gerente es definir y asignar trabajo para otros. En la mayoría de las clases que hago, le pregunto a los gerentes que vienen a mis programas qué quieren decir con la palabra delegación. Estoy seguro de que esta palabra se ha utilizado en sus organizaciones y a menudo se le pide que delegue. Tómese un minuto y reflexione sobre qué es la delegación y por qué es tan difícil hacerlo como gerentes. Entonces, pregúntese $"$¿Qué es la delegación?$"$ Hay dos o tres respuestas estándar que obtendré. El primero y el más importante es \textbf{$"$asignar trabajo a las personas$"$}. La segunda cosa que la gente dice es que la delegación es \textbf{$"$distribuir tareas y actividades a los miembros de su equipo$"$}. El tercer aspecto de la delegación generalmente se refiere a \textbf{$"$la autoridad que tengo para poder pedirle a un empleado que haga algo$"$}. Si en realidad solo se trataba de asignar responsabilidades a las personas, definir tareas para los individuos y garantizar que asigne autoridad a las personas, entonces ¿por qué la delegación es tan compleja? Porque debería ser muy simple. Todas estas son cosas que cualquiera puede hacer. Encontrará que la delegación es difícil porque requiere que usted, como gerente, confíe. Y todos tenemos diferentes preferencias de confianza: algunos de nosotros confiamos inherentemente en otros más, algunos no confiamos y algunos confiamos en personas que están muy cerca de nosotros, pero no confiamos en otros. Cuando asigne responsabilidad a otros, recuerde que usted, como gerente, cree que $"$cuando asigno responsabilidad, sé que esa persona cumplirá lo mejor que pueda$"$. Y como gerente, si he realizado bien mi evaluación de la capacidad del individuo, entonces la tarea que le asigno al individuo y el resultado que obtengo de ese individuo en función de mi evaluación de la capacidad de esa persona debe coincidir. Sin embargo, si hemos realizado una evaluación incorrecta de la capacidad del individuo y le he asignado tareas a ese individuo en base a este error, ¿cómo puedo esperar que el individuo se desempeñe?\\
Entonces, la próxima vez como gerente, podría ser más difícil confiar en esa misma persona. Entonces, ¿qué sucede cuando no confiamos en las personas? Cuando no confiamos en las personas, las monitoreamos más de cerca, las revisamos más de cerca y solicitamos periódicamente información y estado del proyecto. Ahora es interesante ver eso porque hacemos que los empleados piensen que mi gerente no confía en mí y de todos modos mi gerente va a revisar, monitorear y evaluar qué trabajo hago. Y durante un período de tiempo, esta creencia de que los empleados en los que no confías significa que estás haciendo más y más del trabajo que se supone que debe hacer el empleado, porque estás revisando más, estás monitoreando más y estás rastreando aún más.\\
Del mismo modo, ¿se han mirado para hacer esta pregunta: $"$¿Qué les sucede a las personas en las que confío?$"$ Cuando confío en las personas, les doy la responsabilidad y claramente no las controlo tanto como a las personas en las que no confío. Si es un gerente eficaz, la cantidad de personas en las que confía a las que puede delegar debe ser mucho mayor que la cantidad de personas en las que no confía y, por lo tanto, no desea delegar. ¿Por qué es tan importante la delegación? Porque se trata de usted, el gerente, no de ellos, el individuo. Cuando pueda delegar, tendrá tiempo libre para usted. Y con el tiempo que tiene con usted, regresa y mira las cosas importantes que deben hacerse para su equipo, su gerente, su departamento y su organización. Te enfocas en cosas que agregan valor a ti mismo y a la organización. Por lo tanto, al delegar más, podrá hacer cosas que son importantes, que agregan valor a la organización y, finalmente, se agregan valor a usted mismo.\\
Una forma rápida de saber si ha delegado o no. Cuando despega durante una semana y regresa y todo ha ido bien en la organización.\\
$"$Entonces significa que soy redundante$"$. Sí, por supuesto, eres redundante para hacer cosas mejores, cosas más importantes que contribuirán al progreso y crecimiento de tu carrera. Por lo tanto, la delegación no se trata solo de garantizar que los objetivos de la organización se ejecuten de manera efectiva a través de los miembros de su equipo, sino que también se trata de la oportunidad que tengo de poder crecer en mi carrera y responsabilidad. Delegar no es algo que le haces a los demás. Delegar es algo que te haces a ti mismo para poder inspirar a otros.
\section{Preguntas y respuestas}
\begin{enumerate}[\bfseries 1.]
\item ¿Por qué es importante la gestión de las partes interesadas para un gerente?\\
Está claro por su experiencia que consultar con un gerente y un compañero le brinda perspectivas muy diferentes. Del mismo modo, si una parte interesada como un cliente no está contenta, se necesita mucho tiempo y esfuerzo para hacerla feliz. No solo existe su tiempo y esfuerzo, sino que generar confianza nuevamente requiere una mayor inversión de energía. Todos sabemos que un solo proveedor que no cumple con nuestra calidad puede causar interrupciones en nuestra marca / cliente y, por lo tanto, puede influir significativamente en la decisión final tomada por una organización.
\item  ¿Cuáles de los siguientes son los aspectos clave de la priorización?\\
Comprender la tarea que tiene valor a largo plazo para la organización, compilar una lista de tareas pendientes, obtener una evaluación realista y precisa de la tarea.
\item Tener un equipo diverso resulta en:\\
Todos nosotros hemos visto que en equipos dispersos, a través de fronteras, nacionalidades e idiomas, existe una mayor necesidad de garantizar que se mantengan diferentes tipos de canales de comunicación para reducir la falta de claridad en la comprensión. Debe utilizar el rastreador de proyectos, correos electrónicos, llamadas en conferencia y reuniones cara a cara para asegurarse de que todos entiendan la estrategia de manera similar. Esto también significa que existe una propensión a que surjan conflictos. Sin embargo, el resultado es invariablemente que se tienen en cuenta diferentes perspectivas cuando se toman decisiones.
\item ¿Por qué las empresas requieren diversidad?
Sabemos que cada vez que trabajamos con alguien que es muy diferente a usted, surgen diferentes perspectivas que a menudo resultan en una lluvia de ideas con nuevas ideas. También hay una mayor cantidad de respeto y tolerancia por los puntos de vista de los demás. Hay mucha investigación en la literatura de grupos y equipos que respalda que la diversidad permite la resolución de problemas.
\item ¿Cuál es el aspecto más difícil de la delegación desde una perspectiva gerencial?\\
Debe confiar en alguien para delegar y la confianza es una cualidad personal y esto no se puede enseñar. Por lo tanto, la confianza es el aspecto más crítico de la delegación.
\item Kumar es gerente por primera vez y ha estado en este cargo durante un año. Está de vacaciones durante una semana y cuando regresa descubre que todo ha ido bien en la organización. ¿Qué significa esto en términos de efectividad de Kumar como gerente?\\
Kumar ha delegado efectivamente
\item ¿Cuáles son los efectos nocivos de no confiar en los miembros del equipo?\\
Todos sabemos que cuando no confiamos en alguien, tendemos a revisar y monitorear el trabajo más de cerca; También tendemos a pedir más información en tales casos y todo este seguimiento, revisión y monitoreo adicional es un trabajo adicional para nosotros.
\item Anisha tiene un alto desempeño en su equipo, y lo que sea que se le asigne se ejecuta muy bien. Usted cree que ella tiene el potencial de jugar un papel más activo en las relaciones con los clientes, especialmente en el manejo de clientes difíciles. ¿Cuál de las siguientes habilidades necesitará en su nuevo rol?\\
Esta es una pregunta con mayor nivel de dificultad. Dado que el nuevo rol requiere que ella maneje clientes difíciles, las habilidades necesarias son comprender los problemas de los clientes, descubrir qué está causando el problema y responder a las preocupaciones de los clientes. La planificación es necesaria para cada gerente en cada función y, por lo tanto, no es la más crítica para la nueva función.
\item Le gustaría mover a Anisha a un nuevo rol en el próximo año. Como parte de prepararla para el nuevo rol, le ha pedido que redacte un correo para un cliente. Como gerente por primera vez, quién está asignando una nueva tarea o actividad a un empleado, ¿qué es lo que usted como gerente debe cuidar cuando le asigna esta responsabilidad?\\
Proporcionarle un fondo detallado para el correo, resuma los puntos específicos que deben ser , pídale que desarrolle un primer borrador y luego puede darle su opinión.
\item ¿Cuáles de los siguientes son los aspectos críticos de ser un gerente por primera vez que se discutieron en esta semana?\\
Capacidad para comprender a los miembros de su equipo, definiendo y asignando tarea.
\end{enumerate}
\section{Resumen}
 Los aspectos más destacados son la definición de tareas para las personas, la delegación de las tareas y asegurarnos de que estamos asignando el conjunto correcto de tareas al conjunto correcto de personas para realizar.\\
Algunos de los puntos clave que discutimos esta semana son:
\begin{itemize}
\item La gestión de las partes interesadas es uno de los aspectos clave de ser un gerente efectivo.
\item La priorización es la capacidad de identificar la tarea en la lista de tareas donde un gerente desea invertir tiempo, energía y esfuerzo.
Como gerente, es importante delinear tareas y poder ver la diferencia entre urgente e importante .
\item La dinámica interpersonal relacionada con el equipo y los informes directos está determinada en gran medida por las percepciones que otros tienen sobre sus equipos. Por lo tanto, la capacidad de comprender a los miembros del equipo es extremadamente crítica para ser un gerente efectivo.
Un aspecto clave para entender a su equipo es: no use sus propios criterios para evaluar a los miembros de su equipo .
\item Es muy importante que uno pueda reconocer y comprometerse con la diversidad de pensamientos .
\item La delegación es difícil porque requiere que usted, como gerente, confíe . 
\end{itemize}   
\section{Recursos adicionales}
\subsection{8 consejos para la transición del compañero de trabajo al gerente}
$“$Solía ser muy buen amigo de todos en el departamento, pero luego obtuve la promoción de supervisor. Ahora siento que todos me odian ", gimió mi cliente. “Parece que no pueden aceptar que soy su supervisor ahora y que estoy pasando un momento especialmente difícil con una persona que cree que puede salirse con la suya con casi cualquier cosa. ¿Qué tengo que hacer?$"$\\
Una de las situaciones profesionales más difíciles en las que he entrenado a personas, y he pasado por mí mismo, es ser promovida desde un departamento para convertirme en el gerente de ese grupo. Como mi cliente descubrió, al pasar de un compañero que habló anteriormente “basura” por el gerente con otros colegas a la persona a otros hablar acerca puede hacer para una transición estresante.\\
Cambiar de compañero de trabajo a gerente de departamento puede ser una transición difícil porque, como nuevo gerente, usted es responsable de la productividad y los resultados de su departamento. A menudo, los antiguos compañeros de trabajo, ya sea por celos o por costumbre, no querrán tratarte como el jefe; pueden querer seguir tratándote como uno del grupo.\\
Aquí hay 8 consejos para facilitar la transición:
\paragraph{Consejo 1:}
Tenga en cuenta que sus relaciones personales anteriores con sus compañeros de trabajo deberán trasladarse a un nivel diferente porque ya no es un compañero: ahora es la persona que asigna trabajo, analiza la productividad y proporciona evaluaciones de desempeño.
\paragraph{Consejo 2:}  Consulte a su representante de Recursos Humanos para averiguar qué capacitación y apoyo está disponible a medida que asume su nuevo rol de liderazgo.
\paragraph{Consejo 3:}Siéntese uno a uno con cada persona en el departamento para discutir sus sentimientos sobre su transición al gerente. Hablen sobre las expectativas que tienen unos de otros y expongan los posibles problemas sobre la mesa para que puedan abordarlos.
\paragraph{Consejo 4:} Permanezca profesional en todo momento. Y trate a todos y cada uno de los empleados de manera justa y con respeto.
\paragraph{Consejo 5:} Elimine cualquier sesión de chismes ¿¿ de los enfriadores de agua o de la sala de descanso con los empleados.
\paragraph{Consejo 6:} No permita que el trabajo previo y / o las amistades con sus ex pares influyan en sus nuevas responsabilidades de gestión.
\paragraph{Consejo 7:} Asegúrese de que todos en el equipo entiendan su nuevo rol como su gerente y las responsabilidades que se esperan de usted, así como el rol que cada uno de ellos desempeña en el éxito (o fracaso) del departamento.
\paragraph{Consejo 8:} Elabore un plan de juego sobre cómo usted y su equipo pueden trabajar mejor juntos para lograr las metas y objetivos del departamento. (Asegúrese de comunicar de manera clara y concisa esas metas y objetivos).\\
Mi cliente siguió estos consejos y, con el tiempo, mientras su equipo observaba sus habilidades de liderazgo y profesionalismo, dejaron de verla como su compañera y llegaron a aceptarla como su gerente.\\
Recuerde, su posición como nuevo gerente no se trata de tratar de ser popular; se trata de guiar a otros para lograr resultados. No puede ganarse a todos en el departamento; especialmente si uno o dos más también solicitaron el puesto que finalmente recibió. Pase lo que pase, manténgase enfocado en el trabajo a realizar, haga su mejor esfuerzo todos los días, trate a todos de manera justa ... y el respeto debe seguir.
\subsection{Matriz de gestión del tiempo de Stephen Covey explicada}
Aunque el tiempo parece pasar volando, todos tenemos lo mismo las 24 horas del día. Entonces, ¿por qué algunas personas pueden lograr mucho más que la mayoría de la población? Una posible explicación se puede encontrar en su habilidad para administrar el tiempo de manera más eficiente que otros. Pero, ¿cómo es posible hacer frente a la avalancha de tareas que requieren nuestra atención inmediata? En un momento en que los plazos faltantes no son una opción, la cuadrícula de gestión del tiempo de Covey puede ayudarlo a administrar su tiempo disponible de manera más eficiente. La matriz de Covey le permite organizar sus prioridades mucho mejor que antes. La idea de usar cuatro cuadrantes para determinar la prioridad de una tarea fue presentada por el orador principal estadounidense Stephen Covey , autor de Los siete hábitos de las personas altamente efectivas. El sistema de Covey utiliza cuatro cuadrantes diferentes que le permiten priorizar tareas en relación con su importancia y urgencia, ayudándole a decidir si necesita abordar una tarea de inmediato o si puede posponerla.\\
Como puede ver en el gráfico a continuación, la matriz de gestión del tiempo se divide en cuatro cuadrantes que están organizados por importancia y urgencia.\\
La matriz, también conocida como Principio Urgente-Importante de Eisenhower , distingue entre importancia y urgencia:
\begin{itemize}
\item Las responsabilidades importantes contribuyen al logro de sus objetivos.
\item Las responsabilidades urgentes requieren atención inmediata. Estas actividades a menudo están estrechamente vinculadas al logro del objetivo de otra persona. No lidiar con estos problemas causará consecuencias inmediatas.
\end{itemize}
Aquí hay un resumen del significado de cada cuadrante:
\begin{itemize}
\item \textbf{Cuadrante I:} Plazos importantes con alta urgencia. \\
 El primer cuadrante contiene tareas y responsabilidades que requieren atención inmediata.
\item \textbf{Cuadrante II:} desarrollo y estrategias a largo plazo\\
El segundo cuadrante es para elementos que son importantes sin requerir una acción inmediata. Covey señala que este cuadrante debe usarse para la estrategia a largo plazo.
\item \textbf{Cuadrante III:} Distracciones con alta urgencia\\
El tercer cuadrante está reservado para tareas que son urgentes, sin ser importantes. Covey recomienda minimizar o incluso eliminar estas tareas, ya que no contribuyen a su producción. La delegación también es una opción aquí.
\item \textbf{Cuadrante IV:} Actividades con poco o ningún valor\\
El cuarto y último cuadrante se enfoca en tareas y responsabilidades que no producen ningún valor, elementos que no son importantes y no son urgentes. Estas pérdidas de tiempo deben eliminarse a toda costa.
\end{itemize}
Si aplica la matriz de gestión del tiempo Covey a su propia vida profesional y privada, notará que la mayoría de sus actividades se pueden encontrar dentro de los cuadrantes I y III. La experiencia muestra que la mayoría de las personas descuida el cuadrante II, especialmente en el área de su propio desarrollo personal.\\
Sin embargo, no se debe subestimar la importancia del segundo cuadrante. Si nota una gran brecha en este cuadrante, significa que su atención se centra demasiado en el aspecto operativo, mientras que la perspectiva estratégica se queda atrás. Por esta razón, Covey aborda el cuadrante II como una parte excepcionalmente importante de la matriz. Sin este cuadrante, la gestión eficiente del tiempo no sería posible, ya que también requiere elementos estratégicos.
\subsection{Explicación de la matriz de gestión del tiempo de Covey}
A continuación, puede encontrar una explicación detallada de los cuatro cuadrantes que se pueden encontrar en la matriz de gestión del tiempo de Covey.
\subsubsection{Los cuatro cuadrantes de gestión del tiempo}
\paragraph{Cuadrante 1: urgente e importante}
Las actividades en el cuadrante 1 se pueden diferenciar en elementos que no se pudieron haber previsto y aquellos elementos que sí se pudieron haber previsto. Esto último puede evitarse desarrollando planes y prestando mucha atención a su ejecución.\\
El primer cuadrante solo debe contener aquellas actividades y responsabilidades que requieren su atención inmediata. El espacio está reservado para emergencias y plazos extremadamente importantes. Si surge una crisis importante, tendrá que posponer otras tareas.
\begin{center}
\begin{itemize}
\item Crisis.
\item Problemas de presión.
\item Proyectos que están sujetos a plazos.
\item Emergencias.
\item Preparaciones de última hora.
\end{itemize}
\end{center}

\paragraph{Cuadrante 2: no urgente pero importante}
Los elementos que se encuentran en el cuadrante 2 no tienen una gran urgencia, pero pueden desempeñar un papel importante en el futuro. Este cuadrante no solo está reservado para la planificación estratégica, sino también para temas relacionados con la salud, la educación, el ejercicio y la carrera. Invertir tiempo en estas áreas puede no ser urgente en la actualidad, pero a largo plazo será de gran importancia.\\
Preste mucha atención a que ha programado suficiente tiempo para las actividades del cuadrante 2, para evitar que se conviertan en elementos del cuadrante 1. Durante este proceso, le permitirá aumentar su capacidad de terminar sus tareas a tiempo.
\begin{center}
\begin{itemize}
\item Planificación.
\item Preparando.
\item Formación.
\item Ejercicio, salud y recreación.
\end{itemize}
\end{center}
\paragraph{Cuadrante 3: urgente pero no importante}
El tercer cuadrante resume los elementos que parecen tener una alta urgencia, pero que no son en absoluto importantes. Algunas de estas actividades pueden estar completamente impulsadas por el ego, sin aportar ningún valor. De hecho, estas actividades son obstáculos que se interponen entre usted y sus objetivos. Si es posible, intente delegar estos elementos o considere reprogramarlos.\\
Si otra persona le está causando tareas en el cuadrante 3, podría ser apropiado rechazar su solicitud cortésmente . Si esto no es una opción, trate de evitar ser interrumpido constantemente designando intervalos de tiempo para aquellos que a menudo necesitan su ayuda. De esta manera, puede abordar todos sus problemas a la vez, sin interrumpir regularmente su concentración.
\begin{center}
\begin{itemize}
\item Interrupciones.
\item Reuniones.
\item Charla.
\end{itemize}
\end{center}
\paragraph{Cuadrante 4: no es urgente ni importante}
El cuarto y último cuadrante contiene todas aquellas actividades que no aportan ningún valor, las pérdidas de tiempo obvias. Todas las actividades que contiene no son más que distracciones; evítalos tanto como puedas. También debe intentar eliminar todos los elementos de esta lista, sin importar cuán entretenido sea.
\begin{center}
\begin{itemize}
\item Trivialidades.
\item Perdedores de tiempo.
\item Navegando en internet sin propósito.
\item Viendo televisión por horas.
\end{itemize}
\end{center}
\subsubsection{¿Cómo aplicar la matriz de tiempo?}
Al usar la matriz Importante-Urgente, se recomienda tratar de maximizar el tiempo dedicado a las actividades del cuadrante II. Esto le permitirá (a largo plazo) reducir las actividades del cuadrante I, ya que muchas de ellas podrían haber sido actividades del cuadrante II, si se hubiera implementado una mejor planificación.\\
El objetivo de utilizar la matriz de gestión del tiempo es cuestionar si una determinada actividad lo acerca o no a sus objetivos. Si este es el caso, estas responsabilidades deben priorizarse sobre aquellas tareas que pueden exigir su tiempo pero que no contribuyen a sus objetivos. Retrasar las actividades que no aportan ningún resultado significativo hasta que se terminen las tareas más importantes.\\
La red de gestión del tiempo de Covey tiene muchas aplicaciones posibles, dos de las cuales se explicarán a continuación.
\paragraph{Cambiar la prioridad de su lista actual de tareas pendientes}
La matriz de tiempo se puede aplicar como una herramienta que le permite priorizar la importancia y la urgencia de sus tareas actuales y futuras. Al clasificar las tareas y responsabilidades en la cuadrícula adecuada, podrá identificar rápidamente las actividades que necesitan su atención inmediata.
\paragraph{Evaluaciones de una semana}
El segundo enfoque de utilizar la matriz de gestión del tiempo requiere una evaluación semanal. Necesitará seis copias en blanco de la matriz, cinco para cada día de trabajo y una para su evaluación semanal. Al final de cada día laboral, enumera todas las tareas y responsabilidades y la cantidad de tiempo dedicado. Al final de la semana, resume los cinco días de su semana en una matriz. Asegúrese de resumir la cantidad de tiempo dedicado a una tarea determinada.


\chapter{Valoración y evaluación}
\section{Gestión de rendimiento}
\section{Proceso de planificación y establecimiento de objetivos}
En esta subsección cubriremos:\\
\begin{itemize}
\item Importancia de la planificación y la fijación de objetivos.
\item Cómo ser más efectivo en la fijación de objetivos 
\end{itemize}
\subsection{Planificación y fijación de objetivos}
Pasemos ahora un tiempo mirando el proceso de planificación y establecimiento de objetivos. Estos planes luego se conectan en cascada como: planes de negocios unitarios, planes de negocios de departamento, planes de equipo y planes individuales. Por lo tanto, cada área u objetivo clave de resultados que cada uno de ustedes lleva de una forma u otra impacta la estrategia general de la empresa. Por lo tanto, la planificación y el establecimiento de objetivos a nivel de un gerente es muy importante porque se trata de establecer la dirección para los miembros de su equipo. \\
Los cuatro objetivos mencionados anteriormente en realidad evocan comportamientos muy diferentes. Los dos primeros objetivos: contactar a 50 clientes por trimestre y lograr 1 millón de dólares por trimestre, ambos requieren que un vendedor sea eficiente, recurra a sus experiencias pasadas y se centre en "hacer más de lo mismo" de lo que han hecho en el pasado . Sin embargo, el objetivo de convertir al menos el 5\% de los clientes existentes mediante ventas adicionales requiere una comprensión profunda de los requisitos del cliente, las necesidades, los desembolsos presupuestarios que el cliente ha planeado y también una propuesta de valor en términos de cómo va la inversión adicional que el cliente va hacer tendrá un impacto en el negocio del cliente en el futuro. Esto requiere una comprensión profunda del negocio del cliente, las limitaciones, los desafíos que enfrenta el cliente y también requiere que el vendedor tenga un enfoque proactivo, de resolución de problemas y un enfoque orientado a la solución para cumplir con este objetivo. Es evidente por ahora que diferentes objetivos requieren diferentes conjuntos de comportamiento. Finalmente, asistir a una capacitación avanzada es un objetivo profesional de autodesarrollo. Este objetivo contribuye a los tres objetivos tanto a corto como a largo plazo. Contribuye contactando a 50 clientes por trimestre y logrando 1 millón de dólares por trimestre al hacer que el vendedor reflexione sobre diferentes formas de hacer lo mismo, mientras que esta capacitación avanzada podría ayudar al vendedor a convertir al menos el 5\% de los clientes existentes al vender para pensar en soluciones absolutamente nuevas y nuevas formas de acercamiento para los clientes. La combinación de objetivos brinda al empleado la oportunidad de invertir en comportamientos existentes que pueden contribuir a aumentar la eficiencia y también a adquirir nuevos comportamientos que contribuyan al desarrollo profesional y personal del empleado en el futuro.
\section{Asesoramiento y retroalimentación}
En esta subsección, cubriremos:
\begin{itemize}
\item Dos aspectos clave del buen coaching.
\item Cuatro dimensiones clave antes de entrar en una sesión de entrenamiento.
\item Restricciones organizacionales y personales para un empleado.
\end{itemize}
\subsection{Asesoramiento  y retroalimentación parte-1}
El entrenamiento y la retroalimentación son el aspecto más crítico del desarrollo del liderazgo. Como gerentes por primera vez, deben poder delegar el trabajo a personas que puedan asumir la responsabilidad. Por lo tanto, es importante entrenar bien a nuestros empleados.\\
La siguiente etapa del proceso de gestión del desempeño es el asesoramiento y la retroalimentación.  Encuentro que el coaching y la retroalimentación son el aspecto más crítico del desarrollo del liderazgo. En este momento, como gerentes, ha reconocido que no puede hacer todo solo. Sabes que tienes que depender de los demás y también sabes que si aspiras a crecer en tu carrera cada vez más, es importante que te hayas preparado, personas en las que puedas confiar, a quienes puedas delegar y a quién luego asuma la responsabilidad de usted, lo que a su vez significa que le libera más tiempo para hacer lo que es importante tanto desde una perspectiva organizacional como desde su propio crecimiento personal y su perspectiva profesional. Todo esto puede suceder solo cuando entrena bien a sus empleados. Dos ingredientes críticos son necesarios para hacer un buen entrenamiento. El primero es la revisión, el segundo es la retroalimentación. Sin hacer un buen trabajo de revisión y retroalimentación, no hay forma de que pueda ser un entrenador eficaz. 
\subsubsection{Cuáles son las cuatro dimensiones clave que debe abordar antes de comenzar una sesión de coaching con sus empleados}
\begin{enumerate}[\bfseries 1.]
\item El establecimiento de metas. 
\item Evaluación del desempeño. 
\item La planificación del desarrollo. 
\item Retroalimentación constructiva.
\end{enumerate}
\subsection{Asesoramiento  y retroalimentación parte-2}
Cuando entras en un contexto de coaching como gerente, debes estar siempre preparado con cuatro preguntas. 
\begin{enumerate}[\bfseries 1.]
\item ¿tengo claro los objetivos de rendimiento y las mediciones para cada miembro del equipo?
\item ¿tengo un buen sentido de cuáles son las restricciones organizacionales que podrían estar obstruyendo el desempeño desde el punto de vista de un empleado?
\item ¿sé cuáles son las limitaciones personales que permiten que un empleado no se desempeñe?
\item ¿he pensado cómo voy a dar retroalimentación a ese empleado?
\end{enumerate}
Sin preparación en estos cuatro frentes y  sin ser claro en estas cuatro dimensiones,  no entre en una entrevista de entrenamiento o una sesión de entrenamiento con su empleado porque lo atraparán con el pie equivocado.
\subsection{Asesoramiento  y retroalimentación parte-3}
Cuando entrena a sus empleados, es extremadamente importante tener en cuenta las limitaciones de la organización. También debe ser consciente de las limitaciones personales que pueden afectar su desempeño.\\
El segundo aspecto del coaching es saber cuáles son las restricciones organizativas en las que se espera que este empleado cumpla con los objetivos. Creo que la mayoría de los gerentes no dedican suficiente tiempo a este aspecto. Recuerdo en un caso cuando estaba entrevistando a un gerente. Este gerente tenía dos jefes a los que estaba informando y ambos jefes asignaban prioridades muy diferentes a un conjunto de actividades diferentes. Y, de hecho, recuerdo que este empleado me estaba diciendo que si llegaba por la mañana, tendría que averiguarlo, primero hablar con ambos jefes para elaborar un cronograma de actividades para el día. ¿Ahora puede imaginar lo frustrante que debe ser para un empleado que no puede planificar su día? Puedes imaginar lo frustrante que puede ser para un empleado cuando entro por la mañana y descubro que no puedo continuar con mis tareas porque tengo que obtener las prioridades para el día de dos jefes. E imagínense si ambos jefes estuvieran en una reunión, de manera tan efectiva, durante medio día, que no saben lo que están haciendo y lo que están haciendo es lo que creen que es importante para los jefes. Entonces, en última instancia, esto resultará en pérdida de productividad y desmotivación para el empleado. Entonces, ¿es extremadamente importante entender cuáles son las restricciones organizacionales? ¿Cuáles son las otras cosas que, desde una perspectiva organizacional que podrían actuar como restricciones para el desempeño de los empleados? Primero, ¿este empleado está sobrecargado porque alguien más en el departamento ha dejado el trabajo o ha sido trasladado a otro puesto? ¿El empleado no obtiene los recursos adecuados para poder hacer su trabajo de manera efectiva? Por ejemplo, muchas veces, cosas simples como no tener un software en particular para trabajar, tener que depender de una misma máquina que usan dos o tres departamentos y no poder acceder a esa máquina a tiempo. Todo esto está más allá del control del empleado. Entonces, una de las cosas que los gerentes deben hacer cuando están evaluando es también comprender cuáles son las restricciones que se imponen a los empleados debido a los recursos de la organización. El tercer aspecto gira en torno al análisis de restricciones personales. Como gerente, una de las preguntas que debe hacer es cuáles son las limitaciones personales que impiden el desempeño, ¿de acuerdo? En mi experiencia, nuevamente, podría ser falta de conocimiento, podría ser falta de habilidad, podría ser falta de motivación. Ahora, cada uno de estos impacta el rendimiento y la productividad de maneras muy diferentes. Y requieren que usted, como gerente, se concentre en formas muy diferentes de ayudar a los empleados a ser capaces de administrar de manera más efectiva y, por último, pero no menos importante, la parte más importante es cómo me voy a comunicar. el empleado mi evaluación de su desempeño basado en mi comprensión de los objetivos de desempeño, de la medición y estándares de desempeño, de la restricción organizacional, de la restricción personal de los empleados. Entonces, lo que significa que como gerente en su cabeza, tendrá que conciliar las cuatro entradas de datos diferentes que ha recibido para hacer un juicio subjetivo. Quiero pasar un minuto aquí porque esta es la parte más crítica de ser un buen gerente. Al final del día, usted es responsable de tomar decisiones. Todas las decisiones que tomes serán subjetivas. Utilizará datos objetivos para respaldar una evaluación subjetiva que está haciendo. Quiero que reflexiones sobre esta declaración porque solo puedes ser un gerente efectivo si estás dispuesto a vivir con subjetividad. Entonces, solo reflexione sobre la subjetividad porque encuentro que muchos gerentes simplemente temen ser subjetivos. Si eres gerente, serás subjetivo. 
\section{Evaluación y medición}
En esta subsección, cubriremos:
\begin{itemize}
\item Evaluación y medida
\item Aspectos de la gestión eficaz del desempeño.
\end{itemize}
\subsection{Evaluación y Medición Parte-1}
La última parte del proceso de gestión del rendimiento es la evaluación y la medición. Esta es la parte más complicada cuando eres gerente. Muchos de ustedes que provienen de grandes organizaciones, encontrarán que su organización tiene un formato de evaluación estándar. Como gerente, cómo presenta su evaluación de los objetivos de desempeño, las expectativas que tenía del empleado y sus mediciones. Cómo presentar estos tres en el formato de su organización es algo que requiere práctica. Muchos gerentes pueden haber hecho un buen trabajo al evaluar a los empleados, pero les resulta muy difícil traducir eso al formato. Por lo tanto, una de las cosas en las que es posible que desee reflexionar es a principios de año cuando está estableciendo objetivos para mantener en realidad el formato en el que va a presentar al final del año cuando Estás juntando los objetivos. Un consejo que les daría a los nuevos gerentes es que echen un buen vistazo a su formato de evaluación del desempeño. Debe ser minucioso y comprender ese documento al revés. Solo si comprende el documento podrá calibrar lo que ha hecho como establecimiento de objetivos de rendimiento, lo que tiene como análisis de restricciones organizativas, lo que tiene como análisis de restricciones personales y los comentarios que tiene la intención de dar al empleado en términos de mejora, todos estos deben ser reunidos y presentados de manera coherente y completa en el documento. La presentación en el formato es igualmente crítica.
\subsection{Evaluación y Medición Parte-2}
El segundo aspecto de la evaluación y la medición es la convicción con la que yo, como gerente, puedo presentar los comentarios, la evaluación y la valoración que he hecho del empleado al empleado. Permítanme repetir, como gerente, mi convicción de que, de hecho, he sido justo y, de hecho, he considerado las limitaciones bajo las cuales se desempeña el empleado, ambas influirán en mi convicción con la que presento al empleado. Si no lo he hecho bien, no hay forma de que sea convincente. De hecho, lo que puede ser más desmotivador para un empleado es cuando un gerente se da vuelta y dice: "Lo siento, la política de la organización me obliga a hacerle esta evaluación. Si no hubiera sido por la política de la organización, le hubiera dado una calificación diferente. Estoy seguro de que en ese momento, cada miembro del equipo piensa mentalmente y dice: "Trabajé para usted durante un año y sabe de lo que soy capaz". ¿Cómo es que hoy, en el momento de la evaluación, estás pasando esta culpa a la organización? "¿Por qué es esto importante? Esto es importante porque, como gerente, necesitas ser visto para ser creíble.
\subsection{Aspectos de la gestión efectiva del desempeño}
Por ahora, está claro que la gestión del rendimiento es muy central para lo que hacen los gerentes. Si tenemos que manejar bien este proceso, creo que necesitamos una comprensión profunda de ciertas facetas. En términos generales, creo que si quiere ser eficaz en la gestión del rendimiento, debe centrarse en cuatro aspectos. 
\begin{enumerate}[\bfseries 1.]
\item Debe tener claridad y comprensión del contexto en el que el empleado realiza el trabajo cuando establece objetivos para los miembros del equipo.
\item Debe tener la capacidad de observar y monitorear el desempeño de los empleados de manera continua y revisar periódicamente su desempeño.
\item Debe ser capaz de emitir juicios sobre el rendimiento individual para poder dar retroalimentación a los empleados de manera significativa y también centrarse en la mejora de los empleados.
\item Debe ser capaz de crear un plan de desarrollo personal para el empleado que se centre en el desarrollo profesional futuro.
\end{enumerate}
\section{Preguntas y respuestas}
\begin{enumerate}[\bfseries 1.]
\item El propósito principal de proporcionar retroalimentación a los empleados durante una evaluación de desempeño es motivar a los empleados a:\\
Eliminar cualquier deficiencia de rendimiento. El propósito de proporcionar retroalimentación al empleado es motivarlo a eliminar las deficiencias de desempeño o continuar desempeñándose por encima del par.
\item En la mayoría de las organizaciones, ¿ quién es el principal responsable de evaluar el desempeño de un empleado? \\
Supervisor directo del empleado. El supervisor, no el departamento de recursos humanos, generalmente realiza la evaluación real, y un supervisor que califica a sus empleados como demasiado alto o demasiado bajo (o todo el promedio) les está perjudicando a ellos y a la empresa. Los subordinados evalúan a los supervisores en algunas organizaciones, pero la retroalimentación ascendente no es la evaluación principal del supervisor. Suele ser una intervención de desarrollo.
\item ¿Por qué es muy importante la planificación y el establecimiento de objetivos a nivel de un gerente? \\
Se trata de establecer la dirección para los miembros de tu equipo. La planificación y el establecimiento de objetivos le permiten establecer objetivos en el contexto de la estrategia organizacional general. Al establecer objetivos y establecer expectativas, está estableciendo instrucciones para que los miembros de su equipo las sigan. Si la dirección es clara y las expectativas son claras, los empleados terminarán con un desempeño superior.
\item ¿Cuál de las siguientes opciones es MENOS probable que provoque que la evaluación del desempeño de un subordinado sea sesgada? \\ 
Ubicación de la valoración. Se ha demostrado que el sesgo de la evaluación es causado por el propósito de la evaluación pero no por la ubicación de la evaluación. La personalidad del supervisor, las características personales del subordinado y la relación entre las dos partes tienden a conducir a sesgos.
\item El proceso continuo de identificación, medición y desarrollo del desempeño de individuos y equipos y alinear su desempeño con los objetivos de la organización se conoce como:\\
Gestión del desempeño. La gestión del desempeño es el proceso continuo de identificación, medición y desarrollo del desempeño de individuos y equipos y alinear su desempeño con los objetivos de la organización.
\item El sistema de gestión del rendimiento (PMS) desempeña un papel fundamental en la implementación de la estrategia de una organización, ya que garantiza un comportamiento coherente de la estrategia. ¿Qué significa esta declaración?\\
Alinear las metas organizacionales con las metas individuales y de equipo es importante para cambiar los comportamientos. Primero, es falso porque los individuos son parte de un equipo y los equipos son parte de unidades / funciones o departamento. Por lo tanto, alinear las metas individuales con las metas del equipo y las unidades no es factible y aconsejable. El tercero es falso porque la alineación de incentivos sin una comprensión de los objetivos de la organización no dará como resultado un comportamiento consistente de la Estrategia. Por lo tanto, alinear las metas organizacionales con las metas individuales y de equipo es importante para cambiar los comportamientos.
\item ¿Cuál de las siguientes es la secuencia correcta de tres etapas del proceso de gestión del rendimiento? 
\begin{itemize}
\item Planificación y establecimiento de objetivos.
\item Coaching y retroalimentación.
\item Evaluación y medición.
\end{itemize}
El requisito clave para un proceso efectivo de gestión del desempeño es un proceso sólido de establecimiento de objetivos y planificación. Cada año, las organizaciones aumentan sus objetivos. En tal situación, los empleados deben ser entrenados y recibir retroalimentación para que puedan mejorar su desempeño. Finalmente, la tercera y última fase es la evaluación y medición. Esta etapa está vinculada a incentivos, bonificaciones y crecimiento profesional. Desafortunadamente, hay demasiado énfasis en la evaluación y la medición.
\item ¿Cuál de las siguientes son las restricciones organizacionales para un empleado?\\
Informar a dos jefes, la sobrecarga de trabajo y los recursos inadecuados son la restricción organizativa, mientras que la falta de habilidad del empleado es una restricción personal.
\end{enumerate}
\section{Resumen}
Puntos clave para recordar:
\begin{itemize}
\item Performance Management es un proceso mediante el cual los gerentes, supervisores y el personal trabajan para obtener una comprensión compartida de los objetivos de la organización y también alinear la acción de los empleados con los objetivos de la empresa.
\item El proceso de gestión del rendimiento consta de tres etapas : 
\begin{itemize}
\item planificación y establecimiento de objetivos,
\item asesoramiento y retroalimentación y
\item evaluación y medición.
\end{itemize}
\item La planificación y el establecimiento de objetivos a nivel de un gerente es muy importante porque se trata de establecer la dirección para los miembros de su equipo.
\item  En el entrenamiento yace su propio crecimiento personal.
\item Tener un conocimiento profundo del formato de evaluación del desempeño.
\item Como gerente, sea eficaz en el proceso de gestión del rendimiento, ya que esta es la parte más complicada cuando es gerente por primera vez .
\end{itemize}
\section{Recursos adicionales}
\subsection{Cómo establecer objetivos para los empleados}
\subsubsection{Consejos}
\begin{itemize}
\item Los objetivos deberían darle a su equipo algo a lo que llegar.
\item Escriba los objetivos para su equipo y cada uno de sus informes directos.
\item Las metas deben ser aceptadas y reconocidas como importantes.
\item Los objetivos deben alinearse con la misión y estrategia de la organización.
\item Revisar los objetivos de forma regular.
\end{itemize}
\textbf{¿Qué debemos hacer?} Esa es la primera pregunta que el gerente debe responder. \textbf{¿Cuál es la misión de la organización que estoy administrando?} \textbf{¿Cuál es la estrategia para lograr esa misión?} \textbf{¿Cuáles son mis objetivos para el futuro, de acuerdo con la estrategia y la misión?} \textbf{¿Cuáles son los objetivos generales para mi equipo y para cada miembro del equipo?}\\
Esto puede sonar obvio. Pero es notable cuántos gerentes nunca llegan a esta pregunta básica. Toman su misión, estrategia y objetivos como dados, algo que otros han establecido. Tal vez sea su jefe, o el jefe de su jefe, o tal vez solo esté integrado en la situación u organización en la que trabajan. Estos gerentes pasarán toda su vida laboral reaccionando: reaccionando a las órdenes desde arriba, reaccionando a las presiones y problemas desde abajo, o simplemente reaccionando a las demandas insistentes de un lugar de trabajo ocupado.\\
Si todo lo que haces es reaccionar, fracasarás como gerente. Puede ser bueno para resolver los problemas que surjan. Puede ser hábil para responder a las necesidades y solicitudes de aquellos para quienes trabaja o de las personas de su equipo. Puede trabajar largas horas, ser amado y respetado por sus empleados y ser el modelo de eficiencia organizacional. Pero no será un gerente efectivo.\\
Uno de los libros de gestión más populares jamás escritos es "Los 7 hábitos de las personas altamente efectivas" de Stephen Covey, un libro que describe siete prácticas que conducirán al éxito en su vida personal o profesional. El hábito número uno es simplemente esto: ser proactivo . La esencia misma de ser humano, escribe Covey, es la autoconciencia: la capacidad de pensar y, en última instancia, tomar decisiones independientes sobre su vida. Sus acciones no están determinadas simplemente por la "naturaleza" - su composición genética, o por la "crianza" - su educación, o por el entorno en el que vive y trabaja. Por el contrario, deberían reflejar su capacidad para elegir su propio curso. Las personas efectivas se centran en ser proactivas, no reactivas, y buscar las cosas que pueden hacer, en lugar de detenerse en las que no pueden.\\
El gerente efectivo es el mismo. Cualquier lugar de trabajo está lleno de limitaciones. Puede tener un jefe dominante, puede tener un presupuesto extremadamente limitado, puede que no tenga suficientes personas para hacer el trabajo que le han pedido que haga, puede ser parte de una cultura que es muy resistente al cambio. Hay una tendencia comprensible para que las personas caigan en un patrón familiar, y simplemente hagan las cosas como siempre se ha hecho, o hagan lo que otros les piden que hagan. Su trabajo como gerente efectivo es hacerlo mejor que eso. Debe comprender todas las restricciones y saber cuáles se pueden cambiar y cuáles no. Y luego debe decidir un curso positivo hacia adelante.\\
Eso lleva al hábito número dos del Sr. Covey: comenzar con el fin en mente. No es suficiente simplemente elegir un curso; debes tener una idea clara de dónde tomar ese curso. Debe decidir qué va a hacer y hacia dónde lo tomará. En resumen, debes tener una estrategia.\\
Como gerente, es su responsabilidad decidir los objetivos para usted y su grupo. Pero no es algo que debas hacer de forma aislada. Debe cumplir con los objetivos que establezca para su equipo alineado con los de la organización en general. Y debe cumplir con su equipo, aceptar y se comprometer con esos objetivos. Cuanto más pueda involucrar a sus empleados en el establecimiento de objetivos para ellos y para el grupo, más compromisos relacionados con esos objetivos.\\
La mayoría de los gerentes recomiendan que anote las metas para su equipo y cada uno de sus informes directos, y luego revise esas metas específicas, tal vez cada seis meses o una vez al año. Al escribir objetivos, es útil tener en cuenta los siguientes consejos:
\begin{itemize}
\item Los objetivos deben alinearse con la misión y estrategia de la organización.
\item Deben ser claros y fáciles de entender.
\item Deben ser aceptados y reconocidos como importantes por todos los que requieren implementarlos.
\item El progreso hacia los objetivos debe ser medible.
\item Los objetivos deben estar enmarcados en el tiempo, con puntos claros de inicio y finalización.
\item Deben estar respaldados por recompensas.
\item Deberían ser desafiantes, pero alcanzables.
\end{itemize}
El último punto es particularmente importante. Los objetivos deberían darle a su equipo algo a lo que llegar. Pero no deben ser inalcanzables, y su logro o falta de logro no debe depender de una serie de circunstancias más allá del control de la persona.
\subsection{¿Qué es la estrategia?}
\subsubsection{Consejos}
\begin{itemize}
\item Si mantiene sus costos más bajos que los de cualquier otra persona, puede mantener las ganancias.
\item Si puede crear algo que se valore como único, puede tener éxito en ganar más dinero que otros en la industria.
\item Al enfocarse en las necesidades únicas de un grupo particular de compradores, una región geográfica particular o un segmento particular de la línea de productos, puede obtener ganancias superiores al promedio.
\end{itemize}
El estudio serio de la estrategia generalmente se atribuye a "Estrategia competitiva", un libro histórico publicado en 1980 por el profesor de la Escuela de Negocios de Harvard Michael Porter. El libro trata de la estrategia a un alto nivel. Se trata de trazar un rumbo claro para su empresa, en lugar de unidades individuales de la empresa, con el CEO desempeñando el papel de comandante en jefe. Pero las herramientas analíticas que desarrolla el Sr. Porter en el libro son útiles para cualquier persona involucrada en el intento de llevar un producto o servicio al mercado.\\
En la visión del mundo del economista clásico, numerosos jugadores en un mercado compiten entre sí, reduciendo los precios y la calidad, y manteniendo las ganancias modestas. Desde el punto de vista del Sr. Porter, la estrategia consiste en escapar de ese modelo de "competencia perfecta" y, en cambio, en crear una posición sólida para su producto o servicio que le permita obtener ganancias descomunales.\\
Cita cinco "fuerzas" competitivas clave que determinarán la capacidad de su producto o servicio para lograr una posición estratégica sólida:
\begin{enumerate}[\bfseries 1.]
\item  \textbf{Entrada}. ¿Qué tan fácil es para otros ingresar a su mercado? ¿Los recién llegados enfrentan barreras importantes o esperan represalias de los competidores existentes? Las barreras de entrada pueden incluir economías de escala, un producto altamente diferenciado, grandes requisitos de capital para los nuevos participantes, grandes costos para que los clientes cambien, acceso limitado para los recién llegados a los canales de distribución y regulaciones o subsidios gubernamentales.
\item  \textbf{Amenaza de sustitución}. ¿Hay otros productos y servicios que puedan sustituirse fácilmente por los suyos? Considere, por ejemplo, lo que el aumento del jarabe de maíz le hizo a la industria azucarera, o lo que le hizo el iPod al negocio de los CD.\\
\item \textbf{Poder de negociación de los compradores.} ¿Un pequeño número de compradores es responsable de una gran parte de sus ventas? ¿Sus compras de usted representan una gran parte de sus costos? ¿Pueden cambiar fácilmente de proveedores o entrar en su negocio ellos mismos? ¿Su producto es relativamente poco importante para la calidad de su producto o servicio? Si la respuesta a estas preguntas es "sí", el comprador tiene una influencia significativa sobre usted y sus precios.
\item \textbf{Poder de negociación de los proveedores}. ¿Tienes múltiples proveedores? ¿Hay sustitutos que puedas usar? ¿Es fácil cambiar de proveedor? ¿Eres un cliente relativamente importante? ¿Es su producto una entrada relativamente poco importante para usted? En este caso, una respuesta de "sí" significa que usted tiene un poder de negociación significativo sobre ellos.
\item \textbf{Rivalidad entre los competidores actuales}. ¿Qué tan intensa es la rivalidad entre las empresas con las que compites? Esto también afectará su capacidad de mantener ganancias.
\end{enumerate}
Al hacer frente a estas cinco fuerzas, el Sr. Porter argumenta que hay tres estrategias genéricas que una empresa puede tomar para crear ganancias superiores:
\begin{itemize}
\item \textbf{Liderazgo global de costos}. Si mantiene sus costos más bajos que los de cualquier otra persona, puede mantener las ganancias. Esta fue la estrategia del fabricante de computadoras Dell Inc., por ejemplo, o de Wal-Mart Stores Inc.
\item \textbf{Diferenciación}. Si puede crear algo que se valore como único (piense en los automóviles Mercedes o en las computadoras Apple), puede tener éxito en ganar más dinero que otros en la industria.
\item \textbf{Atención}. Al enfocarse en las necesidades únicas de un grupo particular de compradores, una región geográfica particular o un segmento particular de la línea de productos, puede obtener ganancias superiores al promedio.
\end{itemize}
Porter argumenta que es crítico que las compañías tomen decisiones estratégicas claras sobre su enfoque. La peor posición, argumenta, es estar "atrapado en el medio", sin un claro liderazgo en precios, un producto claramente diferenciado o un enfoque distinto.  
\subsection{¿Qué estrategia de gestión debo utilizar en una recesión económica?}
\subsubsection{consejos}
\begin{itemize}
\item Concéntrese en lo que la empresa hace mejor y gaste para ganar participación.
\item Realice adquisiciones a precio de ganga para desarrollar el núcleo de la empresa, incluso cuando esto implica tomar riesgos financieros calculados.
\item Acérquese a la recesión económica como un conductor que se dirige hacia una curva cerrada: entrada lenta, salida rápida.
\end{itemize}
Al igual que las curvas peligrosas en una pista de carreras, las recesiones económicas crean más oportunidades para que las empresas pasen de la mitad del grupo a posiciones de liderazgo que en cualquier otro momento en los negocios.\\
A diferencia de los lugares donde los líderes pueden prosperar solo con el poder bruto, las curvas empinadas requieren delicadeza estratégica. Eso a menudo resulta en diferencias dramáticas en el desempeño a medida que los líderes se salen de la curva.\\
Considere cómo Southwest Airlines Co. surgió durante la recesión de 2001. Con un balance limpio, una clara ventaja de costos y costos de combustible adecuadamente cubiertos, la compañía de descuentos creció a expensas de sus rivales. A medida que otros eliminaron la capacidad y los empleos, Southwest bajó las tarifas para ganar cuota de mercado. Impulsó la publicidad para promover su ventaja de precio y construyó relaciones sólidas con la mano de obra al evitar despidos.\\
El suroeste no es único. Alrededor de un 24 por ciento más de empresas pasaron de la parte posterior de la manada al frente en la recesión de 2001 en comparación con el período posterior de calma económica, según un estudio de ocho años realizado por la consultora Bain \& Co. que analizó los márgenes de beneficio neto y las ventas Crecimiento de más de 2.500 empresas. Mientras tanto, aproximadamente una quinta parte de todas las compañías de liderazgo, aquellas en el cuartil superior de desempeño financiero en su industria, cayeron al cuartil inferior. En comparación, solo las tres cuartas partes de muchas compañías obtuvieron ganancias o pérdidas tan dramáticas después de la recesión.\\
Las recesiones afectan más a algunas industrias que a otras, por lo que mantenerse alerta es importante. Las variaciones se amplifican en una economía globalizada e interdependiente. Eso agrega oportunidad y complejidad. La oportunidad es cambiar el enfoque hacia regiones económicamente más saludables. La complejidad surge de tener que hacer inversiones a largo plazo en operaciones globales con menos certeza que nunca sobre dónde estará expuesto cuando llegue la próxima recesión.\\
Muchos líderes de la industria caen desde la cima durante las recesiones porque suponen que una posición sólida en el mercado es una póliza de seguro contra problemas. Ese enfoque genera exceso de confianza. Los ejecutivos posponen tomar precauciones o tomar las mismas palancas que tiraron en el pasado, como cubrir sus apuestas diversificando. Cuando la recesión golpea con fuerza, generalmente reaccionan de forma exagerada. Reducen los costos y el personal indiscriminadamente, reducen los gastos de capital, exprimen a los proveedores y evitan adquisiciones estratégicas. Luego, cuando las condiciones mejoran, deben gastar mucho para recuperar el impulso.\\
El mejor enfoque: lento, rápido, como un buen conductor que se dirige hacia una curva cerrada. Los ganadores en las recesiones tienden a frenar rápidamente hacia una recesión al administrar los costos de manera cuidadosa y consistente. Es como cambiar a una marcha más baja para disminuir el impulso y aumentar la capacidad de respuesta. Se centran en lo que la empresa hace mejor, reforzando el negocio principal y el gasto para ganar participación. Supervisan agresivamente la competencia para asegurarse de que tienen la mejor línea posible a través de la curva. Eso los prepara para acelerar en el vértice de la curva, cuando la economía comienza a mejorar. Cuanto más lejos puedas ver y más rápido puedas girar, más rápido podrás arrinconar.\\
Otra característica de las empresas expertas en una recesión: hacen adquisiciones de negociación para desarrollar su núcleo, incluso cuando eso significa tomar riesgos financieros calculados. A medida que los mercados mejoran, están bien posicionados para acelerar.
\subsection{Cómo crear una cultura de acción en el lugar de trabajo}
\subsubsection{Consejos}
\begin{itemize}
\item Fomentar la experimentación implacable.
\item Elogie y recompense a las personas que hacen que las cosas sucedan.
\item A menos que las personas se sientan libres de cometer errores, no se sentirán libres de tomar medidas audaces. El fracaso no puede ser castigado indebidamente.
\end{itemize}
La inercia es una gran fuerza en las organizaciones, así como en la naturaleza. Las cosas en reposo tienden a permanecer en reposo. Un corolario de esa propiedad física es esta: por lo general, es más fácil evitar que sucedan cosas que hacer que sucedan.

Es por eso que el primer paso para crear una cultura exitosa de ejecución es crear un sesgo hacia la acción. Las personas que hacen que las cosas sucedan deben ser elogiadas y recompensadas. Las personas que no lo hacen deben ser entrenadas para cambiar, o eliminadas. El fracaso no puede ser castigado indebidamente. A menos que las personas se sientan libres de cometer errores, no se sentirán libres de tomar medidas audaces.

En su libro $"$En busca de la excelencia$"$, Tom Peters y Robert Waterman enumeran un $"$sesgo para la acción$"$ como el primero de los ocho atributos que distinguen a las compañías excelentes e innovadoras. Muchas de las compañías que estudiaron eran muy $"$analíticas en su enfoque para la toma de decisiones, pero no están paralizadas por ese hecho (como muchas otras parecen estar). En muchas de estas compañías, el procedimiento operativo estándar es:" Hazlo , arréglalo, pruébalo $"$.\\
Los Sres. Peters y Waterman critican a la mayoría de las empresas por su "dependencia excesiva en el análisis de las torres de marfil corporativas y la dependencia excesiva en el juego financiero de la mano, las herramientas que parecen eliminar el riesgo pero también, desafortunadamente, eliminar la acción". Hablan de "parálisis a través del análisis", impulsado en parte por una cultura de la escuela de negocios que brinda a los gerentes las herramientas para estudiar, definir y analizar un problema, pero a menudo no las habilidades de liderazgo necesarias para convertir la comprensión en acción. Escriben:\\
$“$Las grandes empresas parecen fomentar grandes operaciones de laboratorio que producen papeles y patentes por toneladas, pero rara vez nuevos productos. Estas compañías están asediadas por vastos conjuntos de comités y grupos de trabajo entrelazados que expulsan la creatividad y bloquean la acción. El trabajo se rige por una ausencia de realismo, generado por personal de personas que no han fabricado o vendido, probado, probado o incluso visto el producto, sino que lo han aprendido al leer informes secos producidos por otros empleados $"$.\\
¿Cómo se crea un sesgo para la acción? En las grandes organizaciones, dicen los Sres. Peters y Waterman, puede requerir algo que ellos llaman "adhocracia". En lugar de relegar nuevas ideas de productos al personal de planificación corporativa, o asignarlas a gerentes de línea que ya están sobrecargados, las mejores empresas crean equipos de proyectos ad-hoc, grupos de trabajo, centros de proyectos o trabajos de mofeta, diseñados para hacer el trabajo.\\
Las fuerzas de tarea, por supuesto, pueden volverse tan burocráticas y amortiguadoras como cualquier otro proceso organizacional. Para evitar eso, Peters y Waterman sugieren algunas reglas:
\begin{itemize}
\item Los grupos de trabajo deben ser pequeños, generalmente diez o menos. No deben doblegarse al deseo burocrático habitual de involucrar a todos los que puedan estar interesados en el proyecto.
\item La antigüedad de los miembros del grupo de trabajo debe ser proporcional a la importancia del problema. Si el problema es grave, las personas que lo atienden deben ser personas mayores, sin importar cuán ocupados puedan estar.
\item La duración del grupo de trabajo debe ser limitada. Las organizaciones tienden a perpetuarse a sí mismas; Los grupos de trabajo orientados a la acción deben estructurarse para resolver un problema y luego disolverse.
\item La membresía debe ser voluntaria. No hay nada más probable que mate un proyecto que tener un grupo de trabajo lleno de personas que piensan que es una pérdida de tiempo.
\item El grupo de trabajo debe unirse rápidamente, sin un proceso formal de fletamento. Los estatutos formales son una señal segura de que la burocracia se está
instalando . El seguimiento debe ser rápido. Si un grupo de trabajo no puede lograr sus objetivos en un período de tiempo razonable, debe disolverse.
\item Las fuerzas de tarea no deben tener personal asignado. El personal permanente es otra señal de los comienzos de la esclerosis burocrática.
\item La documentación debe ser informal y escasa. Las fuerzas de trabajo no deben formarse con el objetivo de producir documentos analíticos largos. Por el contrario, deberían hacer resúmenes breves, tal vez incluso de una página. Los informes de cincuenta páginas tienden a escribirse con el objetivo de mostrar cuánto trabajo se ha hecho; Los resúmenes de una página obligan al grupo a centrarse en las conclusiones.
\end{itemize}
Finalmente, los Sres. Peters y Waterman argumentan que una clave para crear un sesgo para la acción es alentar la experimentación implacable. En lugar de analizar un nuevo producto o servicio hasta la muerte, busque formas de probarlo a un costo relativamente bajo. $"$Listo. Fuego. Objetivo$"$, escriben. $“$Aprende de tus intentos. Eso es suficiente.$"$
\subsection{¿Cuáles son las claves para una buena ejecución?}
\subsubsection{Consejos}
\begin{itemize}
\item Establecer objetivos claros.
\item Mida el progreso hacia esos objetivos.
\item Establecer una responsabilidad clara para el progreso hacia las metas.
\end{itemize}
Si la estrategia es decidir qué hacer, la ejecución se trata de hacer que suceda. Es el seguimiento.\\
Los requisitos principales para una ejecución exitosa son:
\begin{enumerate}
\item Objetivos claros para todos en la organización, que apoyen la estrategia general,
\item  un medio para medir el progreso hacia esos objetivos de manera regular; y 
\item una clara responsabilidad por ese progreso. Esos son los conceptos básicos.
\end{enumerate}
Más allá de eso, una buena ejecución requiere tener una "forma sistemática de exponer la realidad y actuar en consecuencia", argumentan Larry Bossidy y Ram Charan en el libro "Ejecución". La mayoría de las organizaciones, dicen, no se enfrentan muy bien a la realidad. Es el trabajo del gerente obligar a su organización a enfrentar la realidad y luego lidiar con ella.\\
No es necesario ser un experto en gestión para diagnosticar si una organización tiene una fuerte cultura de ejecución. Suele ser obvio. Simplemente siéntese en un par de reuniones de la alta gerencia y rápidamente tendrá la idea.\\
Si la reunión consiste en una larga presentación en Power Point, llena de diapositivas que pretenden mostrar todas las cosas maravillosas que ha hecho el grupo presentador; si otros en la reunión se sientan en silencio durante todo el tiempo, dispuestos a hacer preguntas o hacer agujeros, sabiendo que sus propias presentaciones pronto seguirán; si todos abandonan la reunión sin tener un sentido claro de lo que sucede después; y si el gerente principal se sienta en silencio todo el tiempo, entonces tiene todas las razones para preocuparse. Esta no es una cultura de ejecución.\\
Por otro lado, si la presentación es corta y va al grano; si el presentador destaca claramente tanto los éxitos como los fracasos; si otros se sienten libres de cuestionar y debatir la presentación; Si hay una comprensión común entre todos en la sala sobre los objetivos y los plazos, y si todos salen de la sala con un claro sentido de lo que debe suceder a continuación y quién debe hacerlo, es probable que esté presenciando una fuerte cultura de ejecución.\\
Curiosamente, no siempre son las acciones del gerente principal en la sala de reuniones lo que indicará la naturaleza de la cultura. Si un gerente se sienta en silencio a través de una presentación larga, poco crítica e incuestionable, es probable que no esté haciendo el trabajo. Lo mismo para un gerente que plantea preguntas o sugiere objetivos que parecen una sorpresa total para los demás en la sala.\\
Pero si un gerente se sienta en silencio mientras el presentador hace una crítica obstinada; como otros pesan libremente; y cuando todos se van con un claro sentido de objetivos, plazos y próximos pasos, el gerente está haciendo el trabajo. Él o ella ha creado una cultura exitosa de ejecución que puede gobernarse a sí misma.
\subsection{Delegación exitosa}
\subsubsection{Por qué la gente no delega}
Para descubrir cómo delegar adecuadamente, es importante entender por qué las personas lo evitan. En pocas palabras, la gente no delega porque requiere mucho esfuerzo por adelantado.\\
Después de todo, ¿qué es más fácil: diseñar y escribir contenido para un folleto que promueve un nuevo servicio que ayudó a encabezar, o hacer que otros miembros de su equipo lo hagan? Conoces el contenido por dentro y por fuera. Puede arrojar declaraciones de beneficios mientras duerme. Sería relativamente sencillo sentarse y escribirlo. ¡Incluso sería divertido! La pregunta es: $"$¿Sería un buen uso de tu tiempo?$"$\\
Si bien en la superficie es más fácil hacerlo usted mismo que explicar la estrategia detrás del folleto a otra persona, hay dos razones clave que significan que probablemente sea mejor delegar la tarea a otra persona:
\begin{itemize}
\item Primero, si tiene la capacidad de encabezar una nueva campaña, lo más probable es que sus habilidades se utilicen mejor para desarrollar aún más la estrategia y tal vez proponer otras ideas nuevas. Al hacer el trabajo usted mismo, no está haciendo el mejor uso de su tiempo.
\item En segundo lugar, al involucrar significativamente a otras personas en el proyecto, usted desarrolla las habilidades y capacidades de esas personas. Esto significa que la próxima vez que se presente un proyecto similar, puede delegar la tarea con un alto grado de confianza de que se hará bien, con mucha menos participación de su parte.
\end{itemize}
La delegación le permite hacer el mejor uso de su tiempo y habilidades, y ayuda a otras personas en el equipo a crecer y desarrollarse para alcanzar su máximo potencial en la organización.

\subsubsection{Cuando delegar}
La delegación es beneficiosa para todos cuando se realiza correctamente, sin embargo, eso no significa que pueda delegar cualquier cosa. Para determinar cuándo la delegación es más apropiada, hay cinco preguntas clave que debe hacerse:
\begin{itemize}
\item ¿Hay alguien más que tenga (o se le pueda dar) la información o experiencia necesaria para completar la tarea? Esencialmente, ¿es esta una tarea que alguien más puede hacer, o es crítico que lo haga usted mismo?
\item ¿La tarea brinda la oportunidad de crecer y desarrollar las habilidades de otra persona?
\item ¿Es esta una tarea que se repetirá, de forma similar, en el futuro?
\item Tiene suficiente tiempo para delegar el trabajo de manera efectiva? Debe haber tiempo disponible para una capacitación adecuada, para preguntas y respuestas, para oportunidades de verificar el progreso y para volver a trabajar si es necesario.
\item ¿Es esta una tarea que debo delegar? Las tareas críticas para el éxito a largo plazo (por ejemplo, reclutar a las personas adecuadas para su equipo) realmente necesitan su atención.
\end{itemize}
Si puede responder "sí" al menos a algunas de las preguntas anteriores, entonces podría valer la pena delegar este trabajo.
\subsubsection{El quién y cómo delegar}
Habiendo decidido delegar una tarea, hay otros factores a considerar también. A medida que piense en esto, puede usar nuestra hoja de trabajo de registro de delegación gratuita para mantener un registro de las tareas que elige delegar y a quién desea delegarlas.
\paragraph{¿A quién debe delegar?} Los factores a considerar aquí incluyen:
\begin{enumerate}[\bfseries 1.]
\item La experiencia, el conocimiento y las habilidades del individuo a medida que se aplican a la tarea delegada.
\begin{itemize}
\item ¿Qué conocimiento, habilidades y actitud tiene la persona?
\item ¿Tiene tiempo y recursos para proporcionar la capacitación necesaria?     
\end{itemize}
\item El estilo de trabajo preferido del individuo.
\begin{itemize}
\item ¿Qué tan independiente es la persona?
\item ¿Qué quiere él o ella de su trabajo?
\item ¿Cuáles son sus objetivos e intereses a largo plazo y cómo se alinean con el trabajo propuesto?
\end{itemize}
\item La carga de trabajo actual de esta persona.
\begin{itemize}
\item ¿La persona tiene tiempo para trabajar más?
\item ¿Delegar esta tarea requerirá reorganizar otras responsabilidades y cargas de trabajo?
\end{itemize}
\end{enumerate}
\subsection{¿Cómo debe delegar?}
Use los siguientes principios para delegar con éxito:
\begin{enumerate}[\bfseries 1.]
\item Articular claramente el resultado deseado. Comience con el final en mente y especifique los resultados deseados.
\item Identifique claramente las restricciones y los límites. ¿Dónde están las líneas de autoridad, responsabilidad y rendición de cuentas? ¿Debería la persona:
\begin{itemize}
\item ¿Esperar a que te digan qué hacer?
\item Pregunta qué hacer?
\item ¿Recomendar lo que se debe hacer y luego actuar?
\item ¿Actuar y luego informar los resultados de inmediato?
\item ¿Iniciar acciones y luego informar periódicamente?
\end{itemize}
\item empre que sea posible, incluya personas en el proceso de delegación. Capacítelos para decidir qué tareas se les delegarán y cuándo.
\item Haga coincidir la cantidad de responsabilidad con la cantidad de autoridad. Comprenda que puede delegar cierta responsabilidad, sin embargo, no puede delegar la responsabilidad final. ¡El dinero se detiene contigo!
\item Delegar al nivel organizacional más bajo posible. Las personas más cercanas al trabajo son las más adecuadas para la tarea, ya que tienen el conocimiento más íntimo de los detalles del trabajo diario. Esto también aumenta la eficiencia en el lugar de trabajo y ayuda a desarrollar personas.
\item Brinde el apoyo adecuado y esté disponible para responder preguntas. Asegure el éxito del proyecto a través de la comunicación y el monitoreo continuos, así como la provisión de recursos y crédito.
\item Centrarse en los resultados. Preocúpese por lo que se logra, en lugar de detallar cómo se debe hacer el trabajo: ¡Su camino no es necesariamente el único o incluso el mejor! Permita que la persona controle sus propios métodos y procesos. Esto facilita el éxito y la confianza.
\item Evite la "delegación ascendente". Si hay un problema, no permita que la persona le devuelva la responsabilidad de la tarea: solicite las soluciones recomendadas; y no simplemente proporcione una respuesta.
\item nerar motivación y compromiso. Discuta cómo el éxito afectará las recompensas financieras, las oportunidades futuras, el reconocimiento informal y otras consecuencias deseables. Brinde reconocimiento donde lo merezca.
\item Establecer y mantener el control.
\begin{itemize}
\item Discuta los plazos y plazos.
\item Acuerde un cronograma de puntos de control en el que revisará el progreso del proyecto.
\item Haga los ajustes necesarios.
\item Tómese el tiempo para revisar todo el trabajo enviado.
\end{itemize}
\end{enumerate}
Al considerar detenidamente estos puntos clave antes y durante el proceso de delegación, encontrará que delega con más éxito.
\subsection{Manteniendo el control}
Ahora, una vez que haya realizado los pasos anteriores, asegúrese de informar adecuadamente a su miembro del equipo. Tómese el tiempo para explicar por qué fueron elegidos para el trabajo, qué se espera de ellos durante el proyecto, los objetivos que tiene para el proyecto, todos los plazos y plazos y los recursos en los que pueden recurrir. Y acuerde un cronograma para registrarse con actualizaciones de progreso.\\
Por último, asegúrese de que el miembro del equipo sepa que desea saber si se produce algún problema y que está disponible para cualquier pregunta u orientación necesaria a medida que avanza el trabajo.\\
Todos sabemos que, como gerentes, no debemos microgestión. Sin embargo, esto no significa que debamos abdicar por completo del control: al delegar de manera efectiva, tenemos que encontrar el equilibrio a veces difícil entre dar suficiente espacio para que las personas usen sus habilidades para obtener el mejor efecto, mientras seguimos monitoreando y apoyando lo suficientemente cerca para asegurar que el trabajo se realiza de manera correcta y efectiva.
\subsection{La importancia de la aceptación total}
Cuando se le devuelva el trabajo delegado, reserve tiempo suficiente para revisarlo a fondo. Si es posible, solo acepte trabajos de buena calidad y totalmente completos. Si acepta un trabajo con el que no está satisfecho, el miembro de su equipo no aprende a hacer el trabajo correctamente. Peor que esto, acepta un nuevo tramo de trabajo que probablemente necesitará completar usted mismo. Esto no solo lo sobrecarga, significa que no tiene el tiempo para hacer su propio trabajo correctamente.\\
Por supuesto, cuando se le devuelva un buen trabajo, asegúrese de reconocer y recompensar el esfuerzo. Como líder, debes practicar elogiar a los miembros de tu equipo cada vez que te impresiona lo que han hecho. Este esfuerzo de su parte contribuirá en gran medida a desarrollar la autoconfianza y la eficiencia de los miembros del equipo, que mejorarán en la próxima tarea delegada; por lo tanto, ambos ganan.


\chapter{Construyendo  Redes de compañeros}
\section{Compañeros como partes interesadas significativas}
En esta subsección, cubriremos:
\begin{itemize}
\item Los comapañeros como partes interesadas importantes.
\item Dimensiones de las interdependencias.
\end{itemize}
\subsection{Compañeros como partes interesadas: personal y profesionalmente}
Los compañeros son una parte interesada importante para cualquier gerente. Hasta ahora, si observa el camino, los módulos se han estructurado; nuestro enfoque se ha centrado principalmente en los informes directos. Pero en el Módulo 2, si recuerdan, hablamos sobre las partes interesadas para un gerente. Y una de las partes interesadas clave para cualquier gerente son los compañeros: compañeros dentro de su departamento, compañeros fuera de su departamento. Y en este módulo, nuestro enfoque es realmente obtener algunas ideas sobre cómo gestionar a los compañeros. Podríamos preguntar: $"$¿Por qué son tan importantes los compañeros?$"$ Los compañeros son importantes tanto profesional como personalmente. Profesionalmente, los compañeros, tanto dentro de su departamento como en otros departamentos, desempeñan un papel integral en el cumplimiento de los objetivos de la organización. Los compañeros se convierten en sus partes interesadas y son importantes para usted, principalmente, porque pueden afectar sus entregas. ¿Por qué los compañeros son tan importantes personalmente? Los compañeros son tus iguales. Tienen capacidades similares a las tuyas, habilidades de conocimiento y competencia muy similares. En tal caso, ¿por qué alguien debería respetarte? Si no tiene algo que sea único, si no tiene algo que complemente a los compañeros, entonces realmente no les aporta ningún valor. Por lo tanto, uno de los desafíos de la gestión de compañeros es construir asociaciones y la mentalidad cuando se relaciona con un compañero es construir asociaciones. Si tenemos que cumplir con los objetivos de la organización, los compañeros son un grupo importante para cada uno de nosotros como gerentes.
\section{Tres dimensiones de las interdependencias}
Entonces, cuando hablamos de interdependencias, hay tres dimensiones para estas interdependencias a nivel de compañeros. 
\begin{enumerate}[\bfseries 1.]
\item El primero es el tipo de expectativas que tenemos unos de otros. Tanto en términos de las tareas a realizar, como de los objetivos a cumplir.
\item El segundo aspecto de la interdependencia es que nuestros entregables están relacionados. Entonces, lo que significa que si un compañero se resbala en una entrega, entonces es probable que me impacte directamente; o si me resbalo en una entrega, es probable que afecte a mi compañero, lo que a su vez tiene consecuencias organizativas.
\item Y, tercero, debido a estas expectativas, interdependencias y entregas, se dedica mucho tiempo a la gestión de conflictos y también a garantizar la coordinación entre compañeros.  
\end{enumerate}
Por lo tanto, está bastante claro que, profesionalmente, si tenemos que cumplir con los objetivos de la organización, que son entregados por diferentes departamentos de la organización, y necesitamos lograr la coordinación entre estos diferentes departamentos, es importante para cada uno de nosotros que reconozcamos el papel que nuestro Los compañeros juegan.
\section{Influir en los compañeros}
En esta subsección, cubriremos:
\begin{itemize}
\item Capacidad para influir en los compañeros.
\item Elementos para influir en los compañeros
\end{itemize}
\subsection{Capacidad para influir en los compañeros}
$"$¿Cómo responsabilizo a alguien que no me informa?$"$ Bien, entonces, realmente lo que la gente dice es: $"$No tengo influencia ni poder sobre mis compañeros$"$. De acuerdo, entonces, es una pregunta interesante porque si se tratara de sus reportes directos, usted tenía poder de varias maneras. Usted tiene, la toma decisiones relacionadas con sus incrementos, sus promociones, los programas de capacitación en los que son nominados. hay un mecanismo de recompensa disponible para usted, además de que su organización le otorga el poder legítimo, lo que significa que puede pedirle a sus reporteros directos que hagan lo que quieren que hagan. Pero con sus pares, qué ¿Cuál es el tipo de poder que tienes? Quiero que te tomes unos minutos para pensar: $"$Si tengo que influir en mi compañero, ¿cuáles son los tipos de poder que necesito demostrar?$"$ hacer un ejercicio. Tómese unos minutos y piense en las veces que ha podido influir positivamente en sus compañeros. ¿Por qué han sido influenciados? ¿Y qué es lo que trajo a la mesa que los hizo influir en ellos?
\subsection{Elementos para influir en los compañeros}
En la mayoría de los casos, cada vez que se hace este ejercicio en grupos grandes, generalmente hay tres cosas que saldrán a la luz y espero que las tengan allí. El primero generalmente se debe a mi relación, tan interesante si lo piensas un poco cada relación es recíproca, lo que significa que hay algo que has dado que la persona está volviendo a ti recíprocamente. Entonces, si pudiste influenciar a alguien y pudiste influir debido a la relación, eso significa que hay una reciprocidad de toma y daca. La segunda razón por la que pudiste influenciar a alguien en el segundo elemento que aparece en esa lista suele ser la experiencia. Es porque entiendo un problema. Tengo experiencia en un dominio que mi par no tiene ni necesita. Entonces, la segunda base del poder es realmente la experiencia. Y la tercera y, sorprendentemente, una base de poder muy interesante en lo que 
respecta a los pares se trata realmente de apoyo.
\subsection{Notas}
\paragraph{Cómo influir en los compañeros en el lugar de trabajo} La influencia de los compañeros es una parte crítica de ser un profesional efectivo. Las buenas relaciones de trabajo nos brindan otros beneficios: nuestro trabajo se vuelve más agradable cuando tenemos buenas relaciones con quienes nos rodean. Es mucho más fácil obtener comentarios sobre nuestras ideas y también encontrar formas más innovadoras y creativas de ver los problemas. Las relaciones sólidas con los compañeros nos permiten estar más involucrados en el lugar de trabajo.\\
Hace varios años, como consultor, participé en Una auditoria de cultura del departamento de Agricultura del Estado en India. Este departamento había realizado un trabajo sobresaliente en prácticas agrícolas y había ganado un premio. Este fue un hecho bastante inusual en el Estado. Cuando analizamos los datos de la encuesta de empleados, descubrimos que la mayoría de los empleados atribuyen su sobresaliente éxito al grupo de trabajo de compañeros. Uno de los científicos mencionó en el documento de la encuesta que su motivación para venir a trabajar eran sus amigos. Muchas décadas después, Gallup, la firma consultora que realiza encuestas de compromiso de los empleados, mencionó que los empleados que tenían un amigo en el lugar de trabajo, informaron un mayor nivel de compromiso. Todos somos seres sociales y muchos de nosotros pasamos una parte significativa de nuestro día en el trabajo. Por lo tanto, las relaciones entre compañeros juegan un papel importante en nuestra satisfacción en el lugar de trabajo.\\
La influencia de los compañeros se produce de tres maneras: 
\begin{itemize}
\item A través de las relaciones,
\item la experiencia y 
\item el apoyo
\end{itemize} 
Las relaciones entre compañeros se basan en la confianza, el respeto mutuo, la reciprocidad y la comunicación abierta. Requieren inversiones en términos de tiempo, recursos, esfuerzo y energía. Como pares, cuando realizamos estas inversiones, a menudo encontramos que se nos abren oportunidades y redes, que hasta ahora no estaban disponibles. Hace varios años, cuando una organización estaba pasando por una adquisición, mi antiguo compañero sugirió mi nombre para una auditoría de debida diligencia. Cuando la empresa se me acercó, tenía curiosidad por saber cómo obtuvieron mi referencia. El CEO de la firma mencionó que mi compañero les había dado mi referencia. Debemos tener en cuenta que a medida que avanzamos en la organización, nuestros compañeros también progresan y construyen sus propias redes. Muchas veces, estas redes no se superponen. Las relaciones entre compañeros brindan oportunidades para acceder a dichas redes no superpuestas.\\
Una segunda forma, influir en los compañeros es a través de la experiencia. La experiencia es la capacidad de influir en el comportamiento de otras personas a través del conocimiento o las habilidades de uno. La experiencia deriva de la posesión de cierta comprensión única, tecnología, información o algunas ideas. Esta experiencia es personal y, dado que está centrada en el individuo, es difícil de replicar. Es probable que los compañeros que no poseen la experiencia valoren lo mismo en el logro de sus objetivos. Por lo tanto, la experiencia tiene un papel clave que desempeñar para influir en los compañeros.\\
Finalmente, el apoyo de compañeros también juega un papel crítico. En un artículo de Langford, Bowsher, Maloney y Lillis, los autores observaron que el apoyo podría estar en cuatro áreas:
\begin{itemize}
\item Apoyo emocional (cuidado, empatía, confianza).
\item Apoyo instrumental (proporcionar ayuda o bienes tangibles). 
\item Apoyo informativo (asistencia en la resolución de problemas ).
\item Apoyo de evaluación (afirmación o comunicación de autoevaluación realista).
\end{itemize} 
También se sabe que el apoyo social amortigua el estrés (Bakker, Schaufeli, et al., 2006), el agotamiento emocional y el agotamiento. Las interacciones positivas de los compañeros de trabajo están relacionadas con sentimientos de bienestar y satisfacción laboral. Todo esto a su vez impacta el bienestar de los empleados.\\
En conclusión, ser un compañero efectivo requiere que invirtamos en hacer el esfuerzo y gastar el tiempo, construir relaciones, desarrollar experiencia y, finalmente, brindar apoyo a los demás. Este viaje de inversión en las relaciones entre compañeros dará resultados en el desarrollo profesional a largo plazo.
\section{Constuyendo una relaciones efectiva con sus compañeros}
En esta subsección, cubriremos:
\begin{itemize}
\item Comprender la necesidad de los compañeros y aportar valor
\item Conocer las fortalezas y debilidades de los compañeros.
\item Gestión de conflictos y co-creación.
\end{itemize}
\subsection{Comprender las necesidades de los compañeros y aportar valor}
Si tiene que construir relaciones efectivas con sus compañeros, ¿cuáles son los requisitos clave?
\begin{enumerate}[\bfseries I.]
\item Debe comprender las necesidades de sus compañeros y ver cómo aportarles valor.
\item Necesita conocer las fortalezas y debilidades y ver cómo complementarlas.
\item , por último, debe ser capaz de gestionar los conflictos de manera efectiva y cocrear nuevas ideas con ellos.
\end{enumerate}
Entonces, comencemos ahora con la comprensión de las necesidades de sus compañeros. Si tiene que trabajar eficazmente con un compañero, debe comprender las necesidades del mismo.\\
¿Qué significa esto?\\
\begin{itemize}
\item Significa que tienes que entender el contexto comercial en el que trabaja el compañero.  
\item Debe comprender el papel y las funciones del departamento en el que se encuentra. 
\item También debe comprender la naturaleza de la interdependencia entre su función o su departamento y el departamento de pares. Y todo esto dentro del contexto más amplio de la organización.
\end{itemize}
Por lo tanto, está bastante claro que si desea establecer una colaboración con sus pares, debe comprender muy bien el contexto de los compañeros y su propio contexto. Si ha entendido este contexto, entonces necesita encontrar espacios para la colaboración. Ahora las necesidades pueden ser de diferentes tipos.\\
\begin{enumerate}[\bfseries I.]
\item A veces podría ser un ejemplo muy simple centrado en el problema, donde uno de mis estudiantes mencionaba, diciendo que necesitaba una aprobación y que su jefe no estaba en la ciudad y luego el jefe de su jefe estaba en una reunión y esta aprobación se necesitaba con urgencia y curiosamente, uno de los compañeros con los que no tuvo una relación muy buena anteriormente, que solo vio el problema en el que se encontraba, intervino y dijo si podía lograr que su jefe lo aprobara. Por lo tanto, este tipo de apoyo del centro de problemas por parte de un compañero puede fortalecer las relaciones entre iguales de manera muy rápida y profunda.
\item El otro es la colaboración centrada en la idea. Otra vez, otro estudiante mío me mencionó cómo su organización estaba pasando por un recurso limitado y tenían muy pocos presupuestos, y los presupuestos se estaban recortando en toda la organización y ella bien, estaba mencionando que uno de sus compañeros y  durante un descanso para tomar café hablaban sobre cómo ambos iban a perder de esta reducción de costos y de esa conversación surgió un idea de cómo podrían juntar sus recursos y dentro del mismo presupuesto poder crear una nueva característica para un producto.
\end{enumerate}
Por un lado se trata de comprender las necesidades y los problemas, y por otro se trata de poder crear un valor para el compañero. Lo que ves en estos dos ejemplos son tanto un problema o una necesidad como un valor o una idea que forja una asociación fuerte y saludable.\\
¿Cuándo puede esa asociación no convertirse en realidad?\\
Es posible que las asociaciones no se conviertan en realidad si, como compañero, está amenazado.\\
Entonces, ¿qué puede hacer usted al respecto?\\
Es posible que desee reflexionar sobre por qué está amenazado y toda la idea de autoconciencia y autogestión es muy crítica. Por otro lado, el par podría verse amenazada por usted, incluso entonces, ¿qué es lo que está haciendo que parece estar amenazando al compañero y hay alguna forma de forjar una relación y construir colaboración y fortalecer la relación entre pares? En ambos sentidos, se trata de reflexionar sobre sus comportamientos y lo que podría hacer de manera diferente.
\subsection{Conocer las fortalezas y debilidades de los compañeros}
Ahora que ha visto que el vocabulario que he estado usando en el contexto de los compañeros es completamente diferente. He estado hablando de apoyo, de colaboración y de asociaciones.\\
Es extremadamente importante entender que en este vocabulario de apoyo, colaboración y asociación, lo más importante es ser consciente de sí mismo. Consciente de sus fortalezas y debilidades, consciente de las fortalezas y debilidades de la pareja.\\
Nos hacemos dos preguntas importantes,
\begin{enumerate}[\bfseries I.]
\item ¿Eres consciente de cuáles son tus fortalezas y debilidades, eres consciente de cuáles son las fortalezas y debilidades de tus compañeros?
\item ¿Tus fortalezas complementan la debilidad de tu compañero, las fortalezas de tu compañero complementan tus debilidades?
\end{enumerate}
Esta comprensión y autoconciencia es fundamental para fortalecer las asociaciones.\\
En cualquier asociación, hay dos aspectos importantes a tener en cuenta.
\begin{enumerate}[\bfseries I.]
\item El primero es jugar con la fortaleza de la pareja.
\item Y la segunda y más importante es buscar lo positivo en la pareja.
\end{enumerate}
Cada uno de nosotros tiene nuestros defectos, si está trabajando en una asociación, lo importante es centrarse en las fortalezas y los aspectos positivos del socio, de todos modos las diferencias en la asociación seguirán apareciendo y esas deben ser manejadas de manera saludable.
\subsection{Manejo de conflictos y co-creación}
Si tenemos que gestionar las relaciones entre compañeros de manera efectiva, existe la necesidad de una revisión periódica y retroalimentación, y también es necesario poder manejar los conflictos de manera efectiva. Si ambas cosas se hacen de manera regular, a menudo he descubierto que las relaciones se fortalecen. Entonces, déjame pasar un poco de tiempo en esto.\\
Informalmente, he descubierto que es muy, muy valioso recibir comentarios de sus compañeros. Y uno de los momentos más llamativos fue cuando, hace varios años, un compañero mío, estábamos trabajando juntos en un proyecto, estábamos revisando la entrega del proyecto, e hizo un comentario muy perspicaz y dijo que $"$ Si bien entregamos al cliente a tiempo, no creo que, personalmente, los dos, nos hayamos beneficiado significativamente, juntos, de esta asignación $"$. Y eso, más o menos, me impactó profundamente, porque aunque profesionalmente lo habíamos entregado, personalmente no estamos muy seguros de si le quitamos valor. Y, varios años después, los comentarios que recibí de ese compañero continúan influyéndome en varias de mis tareas, incluso ahora, porque vuelvo a hacer esta pregunta: $"$¿Ganamos profesional y personalmente de esta asociación?$"$\\
Gestión de conflictos: no todos los conflictos son malos. De hecho, los conflictos relacionados con el trabajo son necesarios y esenciales para pensar de manera diferente, para traer diferentes perspectivas. A menudo, cuando hay un conflicto, tendemos a etiquetar esos conflictos como conflictos de personalidad, es por eso que cuando se trata con sus compañeros y se relaciona con ellos, debe retroceder y pensar en los conflictos en gran medida por la tarea, el trabajo y los resultados relacionado. Entonces, enfóquese en los aspectos estructurales del conflicto. No convierta cada conflicto en una personalidad o un conflicto personal.\\
Veo las relaciones entre iguales como co-creación de valor. Estoy usando un término muy grande aquí. ¿Qué significa esto realmente? Quiero decir, en mi experiencia, a medida que el entorno cambia y se vuelve más complejo, y a medida que las organizaciones exigen cada vez más de los empleados, y que la naturaleza de las soluciones que los clientes esperan y los clientes cambian constantemente, y que las interrupciones tecnológicas comienzan a ocurrir; Realmente creo que los compañeros son los únicos interesados que pueden ayudarlo a estar alerta todo el tiempo.\\
Existen enormes oportunidades en las organizaciones para romper los silos interdepartamentales que se construyen. Las relaciones entre pares le permiten romper esos silos y co-crear valor de una manera que sería difícil de imitar o copiar para otros. Entonces, cuando su organización habla de innovación, y están hablando de interrupciones, y están hablando de cómo puede crear valor para la organización; recuerde que la forma segura de crear valor para la organización es a través de sólidas redes y colaboración entre pares.
\section{Preguntas y Respuestas}
\begin{enumerate}[\bfseries 1.]
\item ¿Por qué es importante entender las interdependencias que tenemos con nuestros compañeros?\\
Si uno de los compañeros se resbala, afecta nuestros entregables\\
No trabajamos en silos; nuestros entregables están interconectados\\
La fuerza de la cadena está determinada por el eslabón más débil
\item Ajit es gerente de marketing de una empresa de bienes de consumo. Ha diseñado una campaña publicitaria para uno de los productos. La agencia externa le ha dado una buena calificación al anuncio, sin embargo, el equipo de ventas ha dado su opinión de que el anuncio no puede atraer clientes. El equipo de finanzas también está descontento con los gastos. ¿Cómo tomará Ajit la decisión de ejecutar o no el anuncio?\\
Él comprenderá las preocupaciones de los compañeros. Debería tener una discusión sincera con ellos sobre cómo hacerlo efectivo para una mayor tracción del cliente, y también económicamente viable. La influencia de los compañeros es crítica para un gerente.Debe abordar las preocupaciones de los compañeros y garantizar la alineación con el objetivo de la organización.
\item ¿Por qué los compañeros son partes interesadas importantes en la organización?\\
Los compañeros juegan un papel integral en el cumplimiento de los objetivos de la organización. Como miembro de la organización, debemos mantener relaciones cordiales no solo con los compañeros sino también con otras partes interesadas.Habrá decisiones que tendremos que tomar que sean de interés para nuestro equipo y departamento y que no estén alineadas con nuestros compañeros.
\item Amy es gerente de tecnología en una empresa manufacturera. El departamento de fabricación que brinda la mayor cantidad de ingresos está mayormente retrasado. Impide los entregables de Amy y su equipo. ¿Cómo puede manejar esta situación?\\
Ella discutirá con su contraparte en el equipo de fabricación para comprender sus problemas y llegar a un horario compartido común.
\item Para establecer una colaboración con su compañero, ¿cuál de las siguientes acciones debe hacer usted como gerente?\\
Comprender el contexto empresarial en el que trabaja el compañero.
\item Radhika es cofundador de una empresa de diseño textil por valor de 100 millones. La firma comenzó hace 5 años con 3 fundadores: Radhika, Joy y Tara. Cada fundador ahora dirige un departamento separado. Radhika encabeza la innovación y el nuevo brazo de diseño. Joy y Tara encabezan los departamentos de manufactura y finanzas respectivamente. Recientemente, los departamentos están en desacuerdo debido a un gran pedido que perdieron. ¿Qué hará Radhika?\\
Discuta con Joy y Tara y comprenda por qué se perdió el orden. Si hay algo que se puede hacer para aprovechar la cooperación entre el departamento, se debe hacer correctamente. 
\item Si como gerente tiene que administrar la relación entre compañeros de manera efectiva, ¿cuál de los siguientes debe considerar?\\
Gestionar conflictos de manera efectiva. 
\item Suman, un gerente por primera vez, que también tomó este curso, comenzó a interactuar con sus compañeros con más frecuencia después de asistir a la semana 5. Ella observó que a mayor interacción, mayor es la diversidad de opiniones. Ahora siente que tiene que tomar más aspectos en consideración mientras toma una decisión. ¿Está malinterpretando lo que el profesor quiere que haga?\\
No, ella está bien. Los pares tendrán diferentes puntos de vista en virtud de estar en diferentes departamentos. Es posible que hasta ahora no haya escuchado atentamente los puntos de vista de los compañeros. Después del curso, ella se ha sintonizado más con sus conversaciones entre compañeros. Es bien sabido que una vez que está expuesto a una nueva forma de pensar, tiende a ser un alumno eficaz, notar más y observar más. Lo que Suman está experimentando es solo una curva de aprendizaje donde una mayor conciencia conduce a una mayor comprensión.
\item ¿Cómo veremos a nuestros compañeros?\\
Como socios de crecimiento. 
\item Ross tuvo que presentar un informe y necesitaba analizar los datos. Durante el fin de semana, Sam, su compañero del departamento de planificación lo ayudó. Completaron el informe juntos. La semana siguiente, el jefe de Sam le pidió que hiciera una presentación sobre el informe. Sam se sintió realmente bien con esta experiencia. Ahora cree que:\\
Si no hubiera apoyado a Ross, no habría tenido la oportunidad de presentar un nuevo tema a su jefe y equipo.
\end{enumerate}
\section{Resumen}
Algunos de los puntos clave que discutimos esta semana son:
\begin{itemize}
\item Los pares, tanto dentro de su departamento como en otros departamentos, desempeñan un papel integral  en el cumplimiento de los objetivos de la organización.
\item Los tres mecanismos de influencia que trabajan con los compañeros son la relación, la experiencia y el apoyo.
\item Para establecer una colaboración con sus pares, debe comprender muy bien el contexto de los pares y su contexto.
\item Tanto el apoyo centrado en el problema como la colaboración centrada en la idea forjan una asociación fuerte y saludable con sus compañeros.
\item Juegue con las fortalezas de la pareja y busque aspectos positivos en la pareja.
\item Para gestionar la relación entre pares de manera efectiva, es necesario realizar revisiones periódicas, comentarios y también es necesario gestionar los conflictos de manera efectiva .
\end{itemize}
\section{Recursos adicionales}
\subsection{Relaciones entre pares: cómo dejar de ser competitivo con sus pares y ser un mejor líder}
Cada año me convierto en un mejor gerente. A medida que pasa el tiempo me pongo cada vez mejor (como una gran botella de vino). Mientras reflexiono sobre mi crecimiento, una de las lecciones más impactantes fue aprender lo importante que era construir relaciones con mis compañeros.\\
En las grandes empresas, su desempeño a menudo se calibra y se clasifica en comparación con las personas del mismo nivel en su organización (que generalmente son sus pares). Por lo general, su calificación de desempeño, aumento y bonificación dependen de qué tan bien usted y su equipo se desempeñaron contra estos contrapartes. Trabajar en ese tipo de ambiente ayudó a cultivar la sensación de que mis compañeros eran mi competencia, no mis compañeros de equipo, y hasta hace un par de años dejé que mis interacciones con ellos mostraran ese sentimiento.\\
Una vez que me mudé a la gerencia, aprendí a crear relaciones e interacciones positivas con los miembros de mi equipo. Hice un buen trabajo diferiéndoles elogios y nunca retrocedí si obtuvieron la gloria por mis ideas o si tuvieron más éxito debido a mis comentarios e influencia. Era fácil operar de esta manera porque si se veían bien, me veía bien; nuestros objetivos alineados
Pero todavía no tenía la misma deferencia con mis compañeros. Estaría resentido si mi jefe les diera crédito a ellos oa su equipo por algo que realmente fue idea mía o del trabajo de mi equipo. Todavía sentía la necesidad de demostrar que era mejor que mis compañeros, incluso cuando hacerlo no ayudó a la empresa ni a mi equipo a avanzar. De hecho, me da vergüenza admitir esta revelación, ya que a menudo digo la importancia de las relaciones .\\
A través de los años, afortunadamente, mi visión de mis compañeros ha cambiado drásticamente. Me di cuenta (quizás más tarde que la mayoría) de que al hacer que otras personas fueran exitosas, mi propio trabajo duro y el éxito se mostrarían a largo plazo . No necesitaba competir por el crédito a cada paso; en cambio, si ayudara a otras personas a tener éxito (y les permitiera disfrutar de la gloria de ese éxito), entonces también me haría exitoso .\\
Y aprendí (cinco años más adelante) que esas relaciones entre pares que trabajé para dejar de destruir y comenzar a desarrollar serían las más beneficiosas para mí para mejorar mi red y oportunidades profesionales.\\
Desearía haber aprendido esta lección mucho antes que yo. Me llevó mucho tiempo madurar, crecer y llegar al punto de mi carrera en el que sabía que era excelente en lo que hago y que mi trabajo podía sostenerse por sí solo. Llegué a un punto en que ya no necesitaba que mi gerente me diera palmaditas en la espalda. Ahora puedo concentrar toda mi energía en ayudar a todos, tanto dentro como fuera de mi empresa; tanto en mi equipo, lateralmente como en la organización.\\
Porque en la vida, como en una startup, se trata realmente de hacer avanzar su negocio: no importa quién realmente esté haciendo el trabajo para que esto suceda.\\
\subsubsection{¿Por qué somos tan competitivos con nuestros compañeros?}
Muchos de nosotros que trabajamos y lideramos en entornos como las grandes corporaciones y las startups de rápido movimiento estamos en esos roles porque somos inteligentes, impulsado y tenemos habilidades valiosas para aprovechar. Pero esas cualidades que nos hacen tan deseables a menudo también son nuestra caída en las relaciones con nuestros compañeros.\\
La mayoría de nosotros no llegamos a nuestro lugar actual de la noche a la mañana. La mayoría de nosotros crecimos trabajando duro en la escuela, y muchos de nosotros introvertidos geek nos perdimos el plan de estudios sobre habilidades para construir relaciones. En cambio, los estudiantes con un rendimiento excesivo crecen siendo elogiados por ser la persona más inteligente en la sala, con poca prioridad dada a enseñarles a compartir sus conocimientos y habilidades con sus compañeros de clase. La naturaleza individual del éxito en las pruebas, los exámenes y la tarea refuerza la idea de que el éxito significa trabajar por su cuenta para obtener la mejor calificación en la clase, y la competencia se desarrolla entre los niños más brillantes para obtener la mayor parte de los elogios y atención del maestro.\\
¿Te suena familiar? En demasiados entornos de trabajo, este tipo de dinámica persiste mucho más allá del aula. Las personas no están motivadas porque quieren ayudar a impulsar la empresa, sino porque quieren ser notadas y apreciadas por su trabajo que impulsa a la empresa hacia adelante (ya sea que su trabajo realmente haya contribuido o no). \\
Piensa en tus sentimientos sobre tu trabajo, tu empresa y tu equipo. ¿Te sientes competitivo con las personas que sabes que se supone que deben estar de tu lado? ¿Busca oportunidades para promover su propio trabajo, tomar crédito por ideas o incluso hablar sobre las contribuciones de otro compañero? Es posible que ni siquiera se dé cuenta de que lo está haciendo; pero si puede reconocer sus sentimientos y comportamiento, puede tomar medidas para mejorar sus relaciones y su posición con sus compañeros.\\
\subsubsection{Cómo mejorar las relaciones con tus compañeros}
Esto puede ser un gran desafío para las personas que se han pasado la vida esforzándose por ser la número uno todo el tiempo. Requiere un cambio no solo en cómo tratar a las personas, sino también en cómo piensas en otras personas y en ti mismo.\\
Sin embargo, si puede implementar algunas prioridades en su vida laboral diaria, puede comenzar a ajustar sus prioridades y comenzar a cosechar los beneficios a largo plazo de ser un colaborador en lugar de un competidor.
\paragraph{Renunciar al control}
Intenta no manejar un aspecto de tu trabajo por tu cuenta. Ser un buen gerente se trata de delegar, incluso cuando trabajas con otros gerentes. Encuentre maneras de permitir que otras personas lo ayuden, y hágales saber cuánto aprecia su ayuda. Deja que te impresionen con lo que pueden hacer.
\paragraph{Comienza a entenderlos}
Muchos desacuerdos surgen porque ambas personas quieren que la otra persona se acerque a su punto de vista. Intenta ser la persona que viene de vez en cuando. Hágase preguntas para desarrollar la comprensión, como: ¿qué impulsa la perspectiva de esta persona? ¿Cuáles son sus motivaciones y objetivos? ¿Cuál es su método de comunicación favorito? ¿Cuáles son sus no negociables?\\
Manténgase enfocado en lo que esta persona realmente necesita de usted y por qué. Si puede entender a sus compañeros, puede llegar a acuerdos con ellos más rápido. Sabrá cómo comunicarse con ellos y qué los hace tomar medidas (o no), para que ambos obtengan lo que desean.
\paragraph{Piense como "nosotros", no como "yo"}
Puede ser difícil ver que un compañero tenga éxito, especialmente cuando parece que te hace parecer inadecuado en comparación con ellos. Pero si estás haciendo tu mejor trabajo (y sabes cuándo lo estás haciendo), entonces tener buenos compañeros de equipo que brillan es algo bueno . Recuerde, usted es parte de una empresa, por lo tanto, cuanto más fuertes sean todos en la organización, mejor será para todos ustedes. Silencie esa voz que quiere que otras personas fallen para que pueda tener éxito.
\paragraph{Reconocer buen trabajo}
Elogie a alguien, sinceramente y de manera apropiada, cuando haya hecho algo con éxito. Ya sea que esto signifique señalar públicamente todo su arduo trabajo en su próxima reunión ejecutiva, o simplemente pasar por su escritorio para agradecerles por ayudarlo a perfeccionar su presentación, generará confianza y consideración por parte de sus colegas con un simple reconocimiento.\\
Ninguno de nosotros escucha "gracias" lo suficiente en el trabajo, así que sé la persona que dice "gracias". Te sorprenderá cómo puede cambiar el tono de una conversación o incluso una relación. De hecho, trata de hacerle un cumplido genuino a tu jefe; quizás algo que aprendiste de ellos o que te impresionó. Parece que más arriba vas, menos elogios recibes.
\paragraph{Evita estar a la defensiva}
Este es el otro lado de la sugerencia anterior. Cuando un compañero critica tu trabajo o no está de acuerdo contigo, no luches fuego con fuego. Haga preguntas y sea receptivo a los comentarios. Tomar en serio las opiniones de las personas también genera confianza, y también genera mejores gerentes.
\paragraph{Abordar rencores}
¿Un compañero lo hace a propósito más difícil para usted hacer su trabajo? En lugar de tratar de mantenerse fuera de su camino, asegúrese de conocerlos (y que ellos también lo conozcan a usted). Es más difícil expulsar a alguien que está haciendo un esfuerzo por entrar.
\paragraph{Que te diviertas}
Ver a tus compañeros fuera del trabajo los humaniza. Salga de la mentalidad de la competencia invitando a todos a la hora feliz o reuniéndose el fin de semana para una caminata. Haga algo que no esté relacionado con el trabajo e interactúe con estas personas a un nivel más personal; Es posible que tenga una nueva perspectiva sobre sus ideas el lunes.
\paragraph{Concéntrese en las relaciones reales, no en las políticas}
Puede saber cuándo alguien lo chatea porque quiere algo de usted o cree que puede ayudarlo a ser publicado, promocionado o publicitado. Y no se siente muy bien ser usado. Así que recuerde eso antes de hablar con personas en su grupo de pares.\\
Incluso si una persona tiene la capacidad de mejorar enormemente su carrera, debe priorizar ser auténtico con ellos, no autopromocional, para construir una conexión personal. Busque los vínculos existentes que tenga con sus pares para establecer contactos iniciales más significativos, luego (¡y esto es importante!) Fomente esas relaciones volviendo a ellas regularmente. Consulte este artículo para obtener toneladas de excelentes consejos sobre cómo mantener y aprovechar las relaciones auténticas.\\
Las relaciones que trate de esta manera durarán la prueba del tiempo, y son las que lo sorprenderán pagando una y otra vez a lo largo de su carrera.
\paragraph{Sea activo sobre compartir}
No solo "tome" en sus relaciones con otros gerentes; priorizar "dar" en su lugar. Desarrolle confianza y favor al ser proactivo sobre dar (apoyo, ideas, lo que sea necesario) antes de que necesite que alguien le devuelva algo. Cuanto más practique pensar en formas en que puede ser útil para sus compañeros, más fácil se volverá, y más pensarán en formas en que pueden devolverle el favor.\\
Esto es especialmente útil si muchos de sus compañeros también son competitivos con usted. La mayoría de las personas responderán positivamente a sus esfuerzos activos para hacerles la vida más fácil, especialmente cuando su asistencia se brinda sin condiciones. Puedes ayudar a construir una cultura de colaboración.
\paragraph{Vea a sus compañeros como sus defensores más poderosos}
El hecho de que alguien tenga el mismo nivel profesional que usted no significa que no pueda tener un gran impacto en su éxito futuro. Solía enfocar toda mi energía de redes en personas que creía que eran poderosas e influyentes, suponiendo que pudieran tener el mayor impacto general en mi futuro.\\
¿Pero quién terminó convirtiéndose en la red en la que confié para obtener oportunidades, apoyo y grandes ideas? Fueron las personas con las que me hice amigo las que estaban en la misma posición que yo cuando todos lo estábamos descubriendo.\\
Nunca se sabe quién a su alrededor tiene el potencial de cambiar completamente su vida. No descarte a sus compañeros por su aparente falta de influencia o poder; Involúcralos y combina fuerzas mientras todos mejoran en lo que haces.
\paragraph{Toma la vista larga}
Si no obtiene crédito por este proyecto, lo supervisó porque su jefe asumió que uno de sus compañeros lo manejó: ¿es el fin del mundo? Probablemente no. Haga que su enfoque principal sea hacer las cosas que realmente hacen un buen gerente: construir un equipo fuerte, hacer un trabajo que haga avanzar a la empresa y crear relaciones que sean beneficiosas para su equipo y su empresa.\\
Un proyecto no importa; ser visto como un visionario, un influyente y un colaborador fuerte. Cuando hagas tu trabajo, piensa en ti mismo en términos de tu rol, no en ti mismo. ¿Cómo manejaría esto un buen gerente? ¿Qué respuesta es más útil para mi equipo, no para mí? ¿Cómo puedo crear un entorno en el que prosperen los que me rodean?\\
Si toma esta perspectiva, su trabajo diario será mejor, y usted será (y será visto) como un mejor administrador. Piense a largo plazo sobre lo que realmente significa tener éxito. No insista en el crédito ahora, y concéntrese en construir una carrera y un legado que hablen por sí mismos.
\subsection{5 consejos para desarrollar una relación laboral efectiva con tus compañeros}
Además de sus subordinados y su jefe (gerente directo), tener una relación de trabajo efectiva y buena con sus compañeros también es crucial para que tenga éxito en su función de gestión.\\
Aquí hay 5 consejos para desarrollar una relación de trabajo efectiva con sus compañeros:
\begin{enumerate}[\bfseries 1.]
\item \textbf{busca objetivos comunes}\\
Proactivamente busque un denominador común. Piensa en cómo puedes agregar valor. Extienda una mano para apoyar a sus compañeros en áreas en las que pueda influir.\\
Si un servicio al cliente excelente es crítico para usted y sus pares, inicie una discusión con ellos sobre la mejor manera de lograrlo. Descubra las variables relevantes que afectarán el servicio al cliente. Por ejemplo, una de las variables es la entrega oportuna, en su totalidad, de sus productos dentro de los 30 días a partir de la fecha de un pedido.\\
Debe buscar de manera proactiva comprender todo el proceso de entrega de dicho producto y determinar qué controla directamente y qué puede influir indirectamente.\\
Digamos que todo el proceso de entrega implicará recibir un pedido oficial del cliente, ingresar los detalles del pedido en su sistema de cadena de suministro, revisar los detalles del pedido por parte del grupo de cumplimiento del producto, confirmar la disponibilidad de los productos, asignarlos para el envío, empacar los productos y envío / entregándolos a los clientes.\\
Según el proceso de entrega, usted sabrá exactamente sobre qué componentes tiene control directo. Y para esos componentes, se ejecuta rápidamente.\\
Del mismo modo, también sabría en qué componentes puede influir indirectamente. Y para estos componentes, ofrezca asistencia y apoyo a sus compañeros de todo corazón.
\item \textbf{Establecer confianza y respeto}\\
Cree un ambiente de confianza y respeto con sus compañeros. La forma efectiva de establecer eso es demostrar consistentemente estos comportamientos:
\begin{itemize}
\item No ponga a sus pares en el lugar: si hay algún problema relacionado con los sistemas de trabajo de sus pares, asegúrese de comprometerse y discutir con ellos directamente primero. Lo último que desearía que ocurriera es que tales problemas se mencionen en la reunión de administración o del personal o en la correspondencia por correo electrónico sin su conocimiento previo.
\item Sea un verdadero profesional: evite hablar sobre sus compañeros a sus espaldas o participar en chismes de oficina.
\item Mantenga sus compromisos: si dice que quiere hacer algo dentro de un tiempo determinado, hágalo y entregue a tiempo.
\item Resuelva conflictos con urgencia: trabaje para abordar los conflictos con sus compañeros lo antes posible. Los conflictos no resueltos obstaculizarán el progreso en la construcción de relaciones, el trabajo en equipo y los proyectos en curso.
\item Esté disponible: adopte una política de puertas abiertas. Esté disponible para escuchar, debatir, debatir y elaborar estrategias junto con sus compañeros. De vez en cuando, hagan tiempo para almorzar juntos.
\end{itemize}
\item \textbf{Buscar colaboración}
Siempre pregúntese: $"$¿Cómo puedo trabajar conjuntamente con mis compañeros para el beneficio de mi organización?$"$\\
Puede considerar los siguientes enfoques para colaborar con sus compañeros:
\begin{itemize}
\item Busque claridad sobre los objetivos, roles y responsabilidades compartidos: debe comenzar por el camino correcto. Interactúa con tus compañeros y discute sobre proyectos en los que trabajen juntos. Sea claro sobre los objetivos del proyecto. Hable con detalles sobre quién se supone que debe hacer qué y cuándo. Capture sus acuerdos mutuos en términos de objetivos, roles y responsabilidades compartidos para facilitar el proceso de monitoreo del progreso posterior.
\item Negocie de manera justa: cuando negocie iniciativas, presupuestos y recursos, recuerde siempre que las necesidades de su organización son lo primero, no los deseos. Toma el camino alto. Centrarse en el propósito mutuo. Identificar las necesidades respectivas. En caso de que termine con resultados desfavorables, conceda sin ser personal.
\item Sea responsablemente receptivo: responda a correos electrónicos y correos de voz de manera oportuna. Trate los problemas y preocupaciones de sus compañeros con tanta prioridad como la suya. Evite copiar una gran lista de personas en una larga cadena de correo electrónico con la intención de exponer fallas o ganar la mentalidad de otros. Cuando tenga un desacuerdo serio en una reunión, desconecte el problema, en privado.
\item Celebre el éxito: reconozca el éxito tal como viene, no importa cuán pequeño sea. Acostúmbrese a celebrar cada hito alcanzado. Tal celebración y reconocimiento mantienen su impulso y motivación para seguir adelante. De hecho, aproveche la oportunidad, en una reunión de personal, por ejemplo, para reconocer a sus compañeros y resaltar sus contribuciones significativas.
\end{itemize}
\item \textbf{Juego sin culpa}\\
$"$Cuando un hombre señala con el dedo a otra persona, debe recordar que cuatro de sus dedos se señalan a sí mismo$"$.\\
Deja de culpar a los demás. Si algo sale terriblemente mal, ya sea en proyectos o relaciones con sus compañeros, dé un paso atrás y revise las situaciones bajo estas lentes:
\begin{itemize}
\item Trata todo como una experiencia de aprendizaje. Cada día presenta una nueva oportunidad para que aprendas a ser una mejor persona. La mejor manera de aprender más es enfocándose en estos tres aspectos:
\begin{itemize}
\item Haga un balance de lo que realmente sucedió.
\item Capture lo que funciona y lo que no.
\item Acordar lo que se debe hacer de manera diferente para avanzar.
\end{itemize}
\item Tomar una decisión. Elija no etiquetarse como la "víctima". Culpar a otros por todo lo que va mal a tu alrededor les dará tu poder. Terminarás en el extremo perdedor, sintiéndote mal por ti mismo, continuando con el problema y sin poder dejarlo ir. Como resultado, no podrá avanzar. En cambio, sea un poco autocrítico. Reconozca su imperfección y el hecho de que podría haber contribuido en parte a toda la situación. Con este estado mental, podrá analizar objetivamente la situación y decidir por sí mismo las mejoras clave que desea seguir.
\end{itemize}
\item \textbf{Tómese el tiempo para vincularse entre sí}\\
Esto es tan obvio que muchos tienden a darlo por sentado.\\
Vale la pena conocer bien a tus compañeros. Tome un interés genuino en sus vidas. Pase tiempo con ellos en un ambiente relajado fuera de la oficina, como ir a almorzar / cenar, jugar al golf, hacer pasatiempos comunes, etc.\\
Ayuda a construir una relación que contribuirá en gran medida a establecer una situación de respeto mutuo y consideración mutua.
\end{enumerate}
\chapter{Fundamentos de la comunicación}
En este capítulo, usted:
\begin{itemize}
\item Obtenga una visión general  del proceso de comunicación.
\item Comprender  las dimensiones de las barreras a la comunicación.
\item Examine  las consecuencias de la comunicación ineficaz.
\item Reflexionar  sobre el conflicto y su gestión.
\item Contempla las formas de superar las barreras a la comunicación.
\end{itemize}

\section{Introducción}
La comunicación es tan fundamental para la existencia humana que muchos de nosotros no pasamos suficiente tiempo pensando o reflexionando sobre cómo nos comunicamos. ¿Cómo nos encontramos con los demás y, lo que es más importante, somos comunicadores efectivos? Quiero que retrocedas un poco y reflexiones sobre tu viaje en este curso hasta ahora. Cada módulo tiene subyacente un componente de comunicación. En el primer módulo, hablamos sobre las partes interesadas. En el momento en que hay múltiples partes interesadas, también mencionamos que los intereses de las partes interesadas pueden variar. Y si, como gerente, tiene que gestionar múltiples partes interesadas con sus propios intereses, es extremadamente importante que pueda comprender cuáles son los intereses y participar en conversaciones que ayuden a abordar sus desafíos y alinear sus intereses. En el segundo módulo, hemos hablado sobre cómo hacer las cosas a través de otros. Entonces, cuando miras a los miembros de tu equipo, hay mucha persuasión involucrada allí. También hay mucha dirección que debe proporcionarse. Habrá conflictos y deberá gestionarlos. Y, todo esto, y, por supuesto, todos estamos familiarizados con las reuniones a las que asistimos día tras día. Todas y cada una de estas formas de relacionarse con los miembros de nuestro equipo tienen latente e incorporado un proceso de comunicación. Nosotros, entonces, hablamos en el tercer módulo sobre evaluación y evaluación de personas. Y uno de los aspectos más importantes es dar y recibir comentarios. A estas alturas, debe haber registrado que uno de los aspectos fundamentales para dar retroalimentación se relaciona directamente con la comunicación. Y, en el último módulo, hablamos sobre pares y la creación de redes con pares. Hablamos sobre la necesidad de colaborar de manera efectiva. Por ahora, ni siquiera tengo que decir que la comunicación es absolutamente crítica para tener buenas redes de pares.

\section{El proceso de comunicación}
\subsection{Tipos de Comunicación}
Como gerentes, nos involucramos en varias formas de conversación. Por lo tanto, es extremadamente importante pensar en su día y cuáles son las diferentes formas en que se comunica. Entonces, \textbf{comencemos con la primera clasificación, que es la comunicación oral y la comunicación escrita.} Si es una reunión, hay mucha comunicación oral. Sin embargo, una gran cantidad de comunicación escrita ocurre todos los días, a través de nuestros correos. Entonces, esa es una forma de ver la comunicación.\\
\textbf{Otra forma de clasificar la comunicación es verbal y no verbal.} Verbal, como sugiere la palabra, es realmente sobre lo que dices y lo que oyes. Pero, no verbal es una taza de té separada. No verbal se trata de lo que no dices, sino de lo que haces. ¿Cómo gesticulo? ¿Cómo me inclino y me inclino hacia adelante cuando escucho? ¿Cuáles son los comportamientos y modales que adopto que realmente indican algo a las personas? Lo encuentro más fascinante cuando entro a las reuniones porque una reunión es un lugar maravilloso para ver señales no verbales. Encuentro una gran cantidad de personas que estarán allí físicamente en las reuniones, pero mentalmente ausentes. ¿Por qué? Porque envían mensajes de texto o están ocupados pensando en otras prioridades que tienen. Por lo tanto, es interesante que realmente pueda ver señales no verbales y recogerlas. Por supuesto, como maestros, estamos capacitados para hacerlo porque, en nuestro aula, nuestro desafío es interpretar realmente las señales no verbales de nuestros estudiantes y poder ver cómo podemos evocar un comportamiento positivo y comprometido de ellos. Entonces, esa es otra forma de ver la comunicación.\\
\textbf{Una tercera forma de poder clasificar la comunicación es realmente la comunicación formal y la comunicación informal.} Y como gerentes, participarán en una gran cantidad de ambos. La comunicación formal es la que realmente comunica la organización a los empleados. Tendrás tus planes estratégicos, tendrás la visión de la compañía, tendrás las noticias que se están publicando en los medios sobre la compañía, tendrás la casa abierta de tus líderes, habrá conversaciones de tus miembros de tu equipo directivo , tendrás una revista interna. Todos estos son canales formales de comunicación. Y luego tienes los canales informales de comunicación, y el más popular entre ellos lo que llamamos conversaciones de taza de café, a menudo de manera informal, en grupos pequeños. Pero, muchos de esos son lugares realmente poderosos, donde las personas se comunican y se unen.\\
\textbf{Una cuarta forma de ver la comunicación es realmente alrededor de la comunicación de arriba hacia abajo y de abajo hacia arriba.} De arriba hacia abajo es muy similar a la comunicación formal de la que hablé antes. Lo cual es realmente la comunicación de los principales líderes de su organización, sobre la organización. Pero también puede incluir políticas, procedimientos, manuales operativos estándar y muchos otros como parte de la comunicación de arriba hacia abajo. La comunicación de abajo hacia arriba es aquella en la que tiene una casa abierta, un ayuntamiento o tiene buzones de sugerencias o encuestas de comentarios de los empleados. Todos ellos, en realidad dando aportes sobre lo que piensan los empleados de la organización. Y ese es un canal para poder capturar sus ideas y sus perspectivas. Por lo tanto, es bastante fácil ver que hay varias formas diferentes en las que puede ver la comunicación.
\subsection{Proceso de comunicación - Parte 1}
Creo que lo que debemos hacer es ver la comunicación como un proceso. Entonces, hay un remitente y hay un receptor. Ahora, deconstruyamos esto un poco. Así que aquí hay un remitente, y si vuelvo al primer módulo, y espero que todos recuerden que hablamos sobre la percepción y la motivación de las personas. Entonces, cada uno de nosotros tenemos emisores de comunicación, tenemos nuestras propias percepciones, tenemos nuestras propias formas de ver el mundo, tenemos nuestras propias motivaciones para poder conversar o elegir no conversar.\\
Entonces, como remitente, cuando elijo comunicarme, he formulado algunas ideas en mi cabeza. Y luego uso el lenguaje para comunicar estas ideas. Cuando comunico estas ideas, estas son escuchadas por el receptor. La persona escucha y luego, según su experiencia, según sus percepciones del mundo, según sus formas de mirar el mundo, interpretan lo que han escuchado.\\
Entonces, déjenme resumir rápidamente; está el remitente, quién está pensando y quién se está comunicando. Está el receptor, quién escucha y quién interpreta. A estas alturas, debería ser evidente para todos ustedes que lo que creo que puede no ser lo que digo, lo que digo puede no ser lo que se escucha, lo que se escucha puede no ser lo que se interpreta. Y, por lo tanto, si retrocedemos un poco, es interesante ver que en todo este proceso, el emisor y el receptor, y las formas en que ven el mundo, tienen un papel fundamental en términos de hacer que la comunicación sea efectiva o ineficaz.
\subsection{Proceso de comunicación - Parte 2}
Entonces, la comunicación no se trata solo del emisor y el receptor, porque hay un mensaje que se está comunicando. El mensaje puede ser demasiado complejo; El mensaje puede ser demasiado simple.  Estar sentado en una reunión, donde la gente usa todo tipo de siglas; SAP, JIT, TRP y yo no tenemos idea de lo que significan, a pesar de que dentro de esa organización en particular, dentro de esa comunidad en particular, tienen mucho sentido. Pero para un extraño como yo, en realidad son un montón de jergas que no entiendo. Entonces, a pesar de que el mensaje puede ser importante, el uso de tales jergas puede hacer que sea muy difícil para mí, como receptor, entender el mensaje. Entonces, el mensaje en sí mismo tiene sus propias características peculiares. Entonces, vamos un paso más allá porque tenemos correos electrónicos y tenemos comunicación cara a cara. Y hay varias discusiones que las personas tienen sobre la tecnología y el uso de la tecnología. Uno de los mensajes que quiero comunicarle a alguien es que estoy absolutamente descontento con la forma en que han entregado un proyecto en particular, y que creo que ese trabajo es inaceptable. He intentado escribir y redactar correos muchas, muchas veces, para comunicar esto, y siempre me queda un sentimiento de insuficiencia, porque lo que quiero comunicar, a menudo me pregunto si algo de ese mensaje podría perderse mientras estoy realmente poniéndolos por escrito. Estoy seguro de que muchos de ustedes, como líderes, lucharían con esto periódicamente. Entonces, hay algo sobre el mensaje en sí, ese es otro aspecto del proceso de comunicación; Cuál es el mensaje. Y qué tipo de distorsiones podrían ocurrir en el mensaje mismo. Y uno de los mejores ejemplos, y estoy seguro que de niños, todos hemos jugado este juego, son los susurros chinos, ¿de acuerdo? Entonces, comienzas con una declaración y la susurras a los demás. Y en el momento en que llega a la última persona, lo que pretendía que fuera un mensaje y lo que recibió, puede ser completamente, tan completamente diferente, que realmente lo impacta, cómo puede ocurrir la distorsión de la comunicación. Entonces, hay algo sobre el mensaje. Y luego está el medio en el que te estás comunicando. ¿Te estás comunicando cara a cara, es el medio de comunicación más rico. ¿Por qué? Principalmente porque puedo escuchar tu verbal, pero también puedo captarte señales no verbales. Al instante, puedo hacer adaptaciones a mi comportamiento. Y por lo tanto, lo que da cara a cara como medio es muy diferente de un correo electrónico o carta, ¿de acuerdo? Entonces, creo que es importante que haya un medio, y los diferentes medios tienen diferentes formas de mejorar la efectividad de la comunicación o debilitar la efectividad de la comunicación. Y finalmente, en medio de todo esto, hay ruido. En el entorno, el emisor y el receptor participan; Hay ruido físico y ruido psicológico. El ruido físico podría ser solo que estamos discutiendo algo en un lugar equivocado. ¿Te imaginas discutir y estar muy en desacuerdo en una cafetería o en un restaurante de alta cocina? Muy dificil, verdad? El ambiente no le permite ser auténtico o el ruido en el ambiente en realidad no permitirá que se articule. Entonces, ese no es un ambiente para mantener una conversación de ese tipo. O tomemos el ruido psicológico; que estoy preocupado por algo, como receptor de información. Tengo una fecha límite que cumplir, y en la mañana he tenido un cliente enojado. Y usted, como remitente, me está presentando información en este momento, cuando mi espacio psicológico está abarrotado de ruido. ¿Qué posibilidades hay de que reciba la información que está compartiendo positivamente, o en una perspectiva o de una manera en la que deba escucharse? Entonces, en resumen, ¿cuáles son las implicaciones de ver la comunicación como un proceso para los gerentes? Las implicaciones son dos; primero, no se trata solo del emisor y el receptor. Entonces, ¿está reflexionando sobre si su efectividad de la comunicación se ve afectada por el receptor o por otros factores? No subestimes el papel del medio, el mensaje y el ruido, distorsionando o incluso interrumpiendo el proceso de comunicación.
\section{Barreras a la comunicación}
En esta subsección, cubriremos barreras a la comunicación tales como:
\begin{itemize}
\item Percepción selectiva.
\item Sobrecarga de información.
\item Idioma.
\item Falta de autoconciencia.
\end{itemize}
\subsection{Barreras a la comunicación - Parte 1}
En mi evaluación, hay cuatro barreras principales para la comunicación. La primera barrera es la percepción selectiva. La segunda barrera es la sobrecarga de información. La tercera barrera es el lenguaje. Y la cuarta y última barrera es la falta de autoconciencia. Entonces, echemos un vistazo a cada una de estas barreras. La primera barrera, que es la percepción selectiva; elegimos ver lo que queremos ver. La belleza de los seres humanos es realmente que nuestro filtro perceptivo nos impide bloquear ciertos tipos de información y recopilar y elegir ciertos tipos de información. Quiero que reflexionen, como gerentes, sobre tres aspectos de cómo la percepción selectiva los impacta. La mayoría de los gerentes tienen uno o dos de sus informes directos, que son sus favoritos. Esto no es parcial, pero es solo que son buenos trabajadores, escuchan lo que dices, lo ejecutan muy bien, no tienes que preocuparte por nada después de haberles asignado el trabajo. Ahora, muchas veces no podrás ver nada malo en ellos. Y cuando las personas resaltan o señalan algunos de los aspectos negativos de esas personas, no es probable que escuches. Segundo; si ha hecho algo muy bien, pero también hay margen de mejora y su jefe le brinda comentarios sobre las cosas en las que podría haber trabajado mejor, verá por sí mismo que se sentirá incómodo cuando reciba esos comentarios porque su filtro perceptivo busca más aspectos positivos de su jefe y rechaza los aspectos negativos que está escuchando. Y finalmente, ¿has notado que cuando estás en crisis o en una fecha límite, tiendes a perderte varias actividades principalmente porque estás tan concentrado en lo que tienes que entregar que todo lo que te rodea se sale de tu radar? Cada uno de estos se trata realmente de elegir y observar aquellos aspectos que encajan en el mundo en el que vives. Por lo tanto, hay innumerables ejemplos de prejuicios, debido a la percepción selectiva. Vaya a las lecturas adicionales, donde hemos proporcionado varias de ellas, porque la percepción selectiva influye en cómo interactúa con el mensaje y el receptor.
\subsection{Barreras a la comunicación - Parte 2}
La segunda barrera para la comunicación es, realmente, la sobrecarga de información. en cierto sentido, con ese tipo de sobrecarga de información a través de WhatsApp, a través de correos electrónicos, llamadas telefónicas, videoconferencias y con la cantidad de informes e información generada por Google que tienen, creo, como gerentes, nosotros, llegar muy rápido a los juicios sin realmente examinar o pensar en la cantidad de información que se nos presenta. Y, por supuesto, el papel de los medios en la presentación de información de una manera particular o distorsionar la forma en que se presenta la información. Entonces, como gerentes, creo que es extremadamente importante, en esta era de la revolución digital, que tengamos cuidado de que cuando nos estamos comunicando, estamos usando datos y hechos de múltiples fuentes para llegar a un juicio que nos informe a comunicarnos efectivamente . Además, con poca información, cuando llegamos a juicios profundos, automáticamente se convierte en una barrera para una comunicación efectiva.\\
El tercer aspecto de la comunicación es el papel del lenguaje. Porque, muchos de ustedes operan en un contexto verdaderamente global. Y es interesante porque tiene clientes que son de una nacionalidad diferente y hay vendedores de otra nacionalidad, y no todos los países del mundo hablan inglés con el mismo grado de fluidez y, por lo tanto, encuentro que una gran cantidad de Las barreras de comunicación surgen solo de la falta de dominio o fluidez en un idioma común, que ahora se ha convertido más o menos en inglés. Ahora, junto con esto está el hecho de que venimos de diferentes naciones y regiones del mundo que tienen sus propias culturas; sus propias formas de ver el mundo; sus propias perspectivas a la economía, a la política, a la geografía, a las zonas horarias. Por lo tanto, hay diferencias significativas entre regiones y naciones. ¿Cómo afecta esto a la comunicación? No todos somos tan transitados. Entonces, lo que significa que hacemos muchos de nuestros juicios basados en las videoconferencias que tenemos o sobre los estereotipos que llevamos en nuestras cabezas sobre países y personas de diferentes partes del mundo. Por lo tanto, es interesante que estos estereotipos nos influyan en la forma en que nos comunicamos con los demás.\\
Por ahora, está claro que estoy hablando de dos dimensiones. Uno es el idioma y el segundo son las diferencias culturales que existen entre las naciones que influyen en la forma en que nos comunicamos y en la forma en que nos encontramos con los demás. Si desea leer más sobre esto, hay mucha información disponible sobre comunicación intercultural. Entonces, si está buscando una palabra para Google, continúe y haga una comunicación intercultural, y estoy seguro de que encontrará una gran cantidad de artículos que hablan sobre la efectividad y la ineficacia de la comunicación en un contexto global. Y, ahora, superponga este proceso con tecnología y podrá ver cuánto más complicado se vuelve el proceso de comunicación.\\
Por último, pero no menos importante, está la dimensión de la autoconciencia. Voy a discutir más sobre esto el módulo. ¿Eres consciente de cómo estás distorsionando el proceso de comunicación? ¿Sabe cómo podría intervenir y habilitar el proceso? ¿Eres consciente de cómo se recibe tu mensaje? ¿Sabe si el medio que está utilizando es el medio que debería usarse? ¿Está tomando precauciones para poder manejar el ruido? Entonces, puede ver que cada uno de ustedes como gerentes individuales tiene un papel fundamental que desempeñar en términos de ser conscientes de cuáles son los prejuicios que traen debido a su falta de autoconciencia. Entonces, para mí, todas estas cuatro barreras requieren un poco de pensamiento, reflexión y acción para que seas más efectivo.
\section{Consecuencias de la comunicación ineficaz}
Creo que uno de los defectos en la comunicación es el vocabulario limitado que podría tener un gerente, ya que algunas palabras describen una situación mejor que otras, y por lo tanto, podrían describir mejor lo que uno quiere transmitir.\\
Cuando surgen las barreras para la comunicación, a menudo, hay malentendidos. El malentendido entre el remitente y el receptor requiere que una de las partes busque claridad. Tomemos un ejemplo. Hubo una reunión. Por lo general, uno de los miembros de su equipo hace las minutas de la reunión, pero ese miembro llegó tarde. Entonces, como gerente, le preguntaste a otro ingeniero si él / ella u otra persona, podrían hacer los minutas. Y, después de eso, procediste con la reunión. Esta persona llegó un poco tarde, la persona que generalmente toma las minutas, y comenzó a tomar las minutas de la reunión. Pero él / ella piensa que se suponía que no debían hacerlo. Existe la persona a quien pensó que le había asignado la responsabilidad de hacer las minutas de la reunión. Pero la persona no lo cree o no tiene muy claro si tiene que hacerlo, es Muy común que sucede cada vez. No solo en nuestra vida profesional sino también en nuestros hogares.\\
Ahora, aquí hay una situación interesante en la que dos miembros de su equipo piensan que el otro va a hacer las reuniones. Y, en lo que a usted respecta, cree que le ha dicho a la otra persona que lo haga. Ahora, este malentendido requerirá que los dos miembros hablen entre sí y solo aclaren diciendo que espero que estén haciendo las minutas de la reunión porque llegué tarde. O, la persona a la que se le asignó esto en la reunión solo va y comprueba diciendo: Haré las actas de la reunión. Eso es todo lo que requiere. Asumir la responsabilidad y asegurarse de que ambos estén conscientes. Ahora, supongamos que esto no sucede. Entonces, hay confusión. Hay confusión, crees que hay claridad en tu cabeza que se lo has asignado a alguien. Pero, en realidad, hay confusión. Cuando haya confusión, intervendrá como gerente. Y solicitará el acta de la reunión.\\
Ahora, cuando solicite las minutas de la reunión, encontrará algo muy interesante que la persona que regularmente toma las minutas y la persona que creía que había asignado para tomar las minutas, ambos se pondrán a la defensiva. Y, el ejemplo típico, me imagino, sería algo como esto. Él dice: $"$¿Qué pasó con el acta de la reunión?$"$ Y una persona dice: $"$Pero llegué tarde, así que pensé que sí y que estaba tomando. Revisé con algunas personas y me dijeron que le había asignado que lo hiciera$"$. Ahora, lo que podría suceder es que esta persona se da vuelta y dice: $"$Sí, pero no estaba muy seguro porque es la primera vez que lo hacía, y pensé que la otra persona no lo hizo$"$. entra, quiero decir, entró bastante rápido, tal vez él también haya tomado las actas de la reunión $"$. Entonces, lo hemos hecho rápidamente: ¿estás observando lo que está sucediendo aquí? Esto debe ser un hecho cotidiano en algún lugar u otro en su lugar de trabajo. Lo que comenzó como un malentendido resultó en una confusión, que rápidamente degeneró en actitud defensiva por parte de todos los interesados. A partir de ahí, rápidamente se convierte en desconfianza. Y, por lo general, su respuesta sería, quiero decir, pero pensé que le había asignado la responsabilidad muy claramente. Quiero decir, llegó tarde, así que no importa si llegó dos minutos tarde o 10 minutos tarde. El hecho es que tenías que hacerlo. Ahora, en tu mente, ya has comenzado a hacer esta pregunta, ¿es esto deliberado? ¿No está haciendo las reuniones porque realmente no entendió? Pero, pensé, estaba muy claro. Entonces, mira todo lo que está sucediendo en tu mente. Y, ahora, la próxima vez que mire a esta persona, lo que le viene a la mente es: ¿puedo confiar si le doy esta tarea en particular?\\
Y, finalmente, todo esto puede degenerar rápidamente en un conflicto. Por lo tanto, las barreras a la comunicación son tan importantes que deben abordarse porque tienen resultados y consecuencias, que son realmente de círculo vicioso. Comenzando con malentendidos, generando confusión, pasando a la defensiva, generando desconfianza y, finalmente, un conflicto total. Entonces, creo que ahora, como gerentes, es posible que puedan reflexionar sobre varias situaciones en las que muchas, muchas conversaciones han pasado por el círculo vicioso. ¿Es posible que seas efectivo en cada una de estas etapas haciendo algo de manera proactiva? Me gustaría que reflexionaras sobre esto y observaras cuándo has sido proactivo para evitar este círculo vicioso. 
\section{El conflicto y su manejo}
En esta subsección, cubriremos:
\begin{itemize}
\item Razones para el conflicto.
\item El conflicto y su manejo.
\end{itemize}
\subsection{El conflicto y su manejo - Parte 1}
A medida que avanza en este módulo, ya se habrá dado cuenta de que donde hay comunicación, habrá conflicto. Entonces, para mí, la comunicación y el conflicto son dos caras de la misma moneda. Cuando tenemos múltiples partes interesadas con diferentes perspectivas, es probable que haya desacuerdos. Los desacuerdos, las diferencias en las perspectivas y las visiones del mundo que conocemos conducen a conflictos. Sin embargo, no todos los conflictos son sobre personas.\\
Los conflictos también pueden surgir debido a factores organizativos. Podría suceder que exista una ambigüedad en los roles y responsabilidades de las personas que luego aparece como conflicto. También podría deberse a que tengo altas interdependencias con otra persona o departamento y ese departamento no cumple y, por lo tanto, no puedo hacerlo. En cuyo caso, podría tener conflictos funcionales o conflictos departamentales. También podría tener conflictos porque sus recompensas e incentivos en realidad promueven un comportamiento, pero como miembro del equipo podría estar demostrando otro comportamiento.\\
Así que tomemos por ejemplo: la organización puede tener ciertos tipos de niveles de servicio para entregar, pero una persona puede descubrir que el ingeniero de servicio o la persona de servicio pueden descubrir que el problema en realidad lleva mucho más tiempo. Entonces, ¿Qué haces? Debe cumplir un acuerdo de nivel de servicio, pero sabe que el problema es más complejo y tomará mucho más tiempo. Entonces, en una situación como esta, hay un conflicto intra-personal. Muchas veces también experimentamos conflictos relacionados con la personalidad debido a diferencias en los sistemas de valores, diferencias en la forma en que demuestran comportamientos. Por lo tanto, es muy posible que cualquiera de estos pueda contribuir a crear situaciones de conflicto.     
\subsection{Conflictos y su manejo - Parte 2}
¿Por qué es importante reconocer cuál es la base del conflicto? Es importante reconocer cuál es la base del conflicto porque, en mi experiencia, los conflictos se intensifican muy rápidamente. Y, una vez que los conflictos aumentan, es muy difícil para nosotros como individuos dar un paso atrás y pensar, por eso cuando hay un problema o se está gestando un conflicto, es mejor dar un paso atrás y obtener múltiples perspectivas de las personas. En algunas culturas, las personas evitan activamente los conflictos. Pero mi experiencia ha sido que cuando trabajamos en organizaciones, evitar conflictos es contraproducente para usted como gerente. Tendrá que manejar los conflictos de manera efectiva porque muchos conflictos en las organizaciones son oportunidades constructivas para poder encontrar soluciones y alternativas que sean mejores de lo que ambas partes pueden aportar individualmente.\\
Mi propia experiencia: los conflictos han sido algunos de los mejores espacios en los que hemos podido pensar fuera de la caja, hemos podido pensar de manera constructiva y, sobre todo, hemos podido agregar valor a la organización. Entonces, como gerentes, es extremadamente importante que participen en el conflicto de manera constructiva. De hecho, cuanto más se involucre en un conflicto de manera constructiva, es probable que sus compañeros lo respeten mejor porque ese compromiso constructivo, en conflicto, crea oportunidades para la colaboración. Es un poco inusual para mí dar consejos, y a estas alturas ya habrás notado que no es algo que yo haga mucho. Habiendo dicho eso, sin embargo, sobre los conflictos, quiero decir, he escuchado, una y otra vez, que la gente viene y me pregunta, ¿cómo manejo el conflicto, qué necesito hacer? Y, debido a que estamos conduciendo al siguiente módulo, que se trata de autogestión, me voy a tomar la libertad de darle 10 formas de cómo puede manejarse a sí mismo para poder manejar el conflicto de manera efectiva, ¿de acuerdo? Entonces, aquí va.\\
Primero, mantén la calma, comprende diferentes perspectivas. Escuche para entender, no para discutir o defender. Busque los espacios en blanco positivos que están disponibles en la situación de conflicto, que pueden permitirle construir conversaciones. Presente su caso con tacto y apropiadamente. Concentrarse en el futuro. No quites la dignidad del otro haciendo ataques personales. No pierdas tu respeto por gritos y abusos. Hacer preguntas; elige las batallas con las que quieres pelear. Sea creativo en la búsqueda de soluciones. Pero, lo más importante es celebrar acuerdos.\\
En mi opinión, si pudieras hacer un par de ellas, serías más efectivo de lo que eres hoy.
\section{Maneras de superar las barreras}
En esta subsección, cubriremos los dos ingredientes para superar las barreras a la comunicación:
\begin{itemize}
\item Escucha activa.
\item Ser de mente abierta.
\end{itemize}
\subsection{Formas de superar las barreras - Parte 1}
Entonces, antes analizamos las barreras a la comunicación y las consecuencias de la comunicación ineficaz. Me gustaría cerrar este módulo observando cuáles son las formas de superar las barreras. En mi experiencia, hay dos ingredientes críticos para superar las barreras a la comunicación. Y ambos han funcionado para mí, por eso puedo traerlo a esta sesión.\\
El primero es la escucha activa y el segundo es de mente abierta. Y permítanme compartir un poco sobre por qué creo que estos dos son críticos. Permítanme comenzar con una mente abierta y luego ir a la escucha activa. Solo mira las veces que has sido de mente abierta. Cuando he sido de mente abierta, estoy dispuesto a escuchar diferentes perspectivas. Estoy dispuesto a hacer preguntas. También estoy abierto a examinar problemas desde el punto de vista del receptor. Estoy buscando activamente soluciones que nos permitan resolver el problema. También estoy pensando en qué es lo que no he hecho para ser más eficaz en esta interacción particular. Y debido a que soy abierto, tengo menos juicios que he hecho y, por lo tanto, también estoy abierto a revisar mi decisión.\\
Por lo tanto, esta mentalidad abierta es un atributo muy crítico en situaciones de conflicto en aquellos espacios donde no eres efectivo en el proceso de influencia. Entonces, ¿soy de mente abierta? ¿Estoy dispuesto a recibir la perspectiva del receptor sin usar filtros que son míos? La mentalidad abierta, creo, es extremadamente importante para nosotros para encontrar soluciones que sean mejores que las que tenemos individualmente.
\subsection{Formas de superar las barreras - Parte 2}
Por último, pero no menos importante, la parte de este módulo es la escucha activa. No estoy seguro si alguien tiene guía porque escuchó activamente. Sin embargo; la víctima en el mundo gerencial de hoy realmente está escuchando. Creo que escuchamos más de lo que escuchamos. Cuando escuchemos, y me tomaré unos minutos para dedicarme a la escucha activa porque cuando escuchas y escuchas activamente, hay dos cosas que suceden. Primero no reaccionas. Tu respondes. Permítanme darles un ejemplo y dónde estarán familiarizados muchos de ustedes. Recibiste un correo y el correo ha sido escrito con ira. Mientras lee el correo, su percepción selectiva se ha estabilizado y está leyendo el correo y observando las palabras en el correo, no necesariamente las emociones detrás del correo. Estoy seguro de que cada uno de nosotros tuvo la situación de que tan pronto como hayamos recibido ese correo, si hubiéramos enviado otro correo, habrían lamentado haberlo enviado muchas veces. Porque en ese momento lo que has hecho reacciona a un correo.\\
Una reacción es emoción, no está bien pensada y una reacción a un correo emocional enviado por el remitente solo puede crear una cadena de correos que son viciosos y deshabilitables. Y lo he hecho muchas veces, y les he dicho a los gerentes que lo hagan e invariablemente la gente regresa y me dice que esta fue la gran comida para llevar que tenían. Siempre les digo que redacten el correo y lo dejen en la carpeta del borrador. Regrese después de una hora y mire ese correo, lo cambiará. Regrese después de seis horas y vea el correo, reescribirá el correo. ¿Por qué? Porque si escuchas activamente lo que había en el correo, darás un paso atrás y responderás. Porque habrá preocupaciones, miedos y no sabrás qué motivó a esa persona a escribir ese correo de esa manera.\\
Entonces, para mí, la escucha activa te permite ese momento entre la reacción y la respuesta. El segundo aspecto de por qué la escucha activa es tan crítica porque cada uno de ustedes como gerentes opera en entornos complejos donde hay varias incógnitas. También tiene múltiples partes interesadas para colaborar. En este tipo de contexto, la escucha activa es escuchar lo que no se dice, no lo que se dice. Entonces, ¿qué es lo que realmente quiere decir la parte interesada cuando dice algo, no necesariamente lo que dice? Entonces, comprender eso significa que debes estar escuchando activamente. Hacer las preguntas correctas, parafrasear, resumir, ayudar a llegar a una claridad. Todas estas capacidades son críticas si tienes que ser un administrador efectivo.\\
Después de eso, si tiene que ser efectivo no solo en su vida profesional, sino también en su vida personal porque la comunicación al final del día es una habilidad para la vida. Recoges en tus espacios públicos, lo llevas a tus espacios privados. Recoge las habilidades en sus hogares y las lleva al trabajo y, por lo tanto, la comunicación se trata de invertir en la vida personal y profesional y es por eso que es probablemente la habilidad más crítica para un gerente.
\section{Resumen}
Nuestro enfoque fue comprender los elementos esenciales de la comunicación . Si tiene que ser efectivo no solo en su vida profesional sino también en su vida personal, comience a invertir en sus habilidades de comunicación. Recoges estas habilidades en tus espacios públicos y las llevas a tus espacios privados; usted recoge las habilidades en sus hogares y las lleva al trabajo.\\
Algunos de los puntos clave que discutimos esta semana son:
\begin{itemize}
\item El remitente, el receptor y las formas en que ven el mundo tienen un papel fundamental en términos de hacer que la comunicación sea efectiva o ineficaz.
\item No subestimes el papel del medio, el mensaje y el ruido, distorsionando o incluso interrumpiendo el proceso de comunicación.
\item Cada uno de nosotros como individuos tener Un papel fundamental que desempeñar en términos de ser conscientes de los prejuicios que traemos debido a la falta de autoconciencia.
\item Las barreras a la comunicación son importantes para ser abordadas de manera oportuna ya que tienen resultados y consecuencias que son viciosas por naturaleza.
\item Evitar conflictos es contraproducente para cualquier gerente.
\end{itemize}
\section{Recursos adicionales}
\subsubsection{Comunicación intercultural}
\paragraph{La buena colaboración es imprescindible}
No es ningún secreto que el lugar de trabajo de hoy se está volviendo rápidamente vasto, ya que el entorno empresarial se expande para incluir varias ubicaciones geográficas y abarcar numerosas culturas. Sin embargo, lo que puede ser difícil es comprender cómo comunicarse de manera efectiva con las personas que hablan otro idioma o que confían en diferentes medios para alcanzar un objetivo común.
\paragraph{Comunicación intercultural: la nueva norma}
Internet y la tecnología moderna han abierto nuevos mercados que nos permiten promover nuestros negocios a nuevas ubicaciones geográficas y culturas. Y dado que ahora puede ser tan fácil trabajar con personas de forma remota como trabajar cara a cara, la comunicación intercultural es cada vez más la nueva norma.\\
Después de todo, si la comunicación es electrónica, es tan fácil trabajar con alguien en otro país como trabajar con alguien en la siguiente ciudad.\\
¿Y por qué limitarse a trabajar con personas dentro de una distancia de conducción conveniente cuando, de la misma manera, puede trabajar con las personas más conocedoras del mundo entero?\\
Para aquellos de nosotros que somos hablantes nativos de inglés, es una suerte que el inglés parezca ser el idioma que las personas usan si quieren llegar a la audiencia más amplia posible. Sin embargo, incluso para los hablantes nativos de inglés, la comunicación intercultural puede ser un problema: solo presencie la incomprensión mutua que a veces puede surgir entre personas de diferentes países de habla inglesa.\\
En este nuevo mundo, una buena comunicación intercultural es imprescindible.
\paragraph{Comprender la diversidad cultural}
Dados diferentes contextos culturales, esto trae nuevos desafíos de comunicación al lugar de trabajo. Incluso cuando los empleados ubicados en diferentes ubicaciones u oficinas hablan el mismo idioma (por ejemplo, las correspondencias entre los angloparlantes en los EE. UU. Y los angloparlantes en el Reino Unido), existen algunas diferencias culturales que deben considerarse en un esfuerzo por optimizar las comunicaciones entre Las dos partes.\\
En tales casos, una estrategia de comunicación efectiva comienza con el entendimiento de que el remitente del mensaje y el receptor del mensaje son de diferentes culturas y orígenes. Por supuesto, esto introduce una cierta cantidad de incertidumbre, haciendo que las comunicaciones sean aún más complejas.\\
Sin entrar en culturas y subculturas, quizás sea más importante que las personas se den cuenta de que una comprensión básica de la diversidad cultural es la clave para una comunicación intercultural efectiva. Sin necesariamente estudiar en detalle las culturas e idiomas individuales, todos debemos aprender a comunicarnos mejor con individuos y grupos cuyo primer idioma, o idioma de elección, no coincida con el nuestro.
\paragraph{Desarrollando la conciencia de las culturas individuales}
Sin embargo, es importante aprender los conceptos básicos sobre cultura y al menos algo sobre el lenguaje de la comunicación en diferentes países. Esto es necesario incluso para el nivel básico de comprensión requerido para participar en saludos apropiados y contacto físico, que puede ser un área difícil interculturalmente. Por ejemplo, besar a un socio comercial no se considera una práctica comercial adecuada en los Estados Unidos, pero en París, un beso en cada mejilla es un saludo aceptable. Y, el firme apretón de manos que es ampliamente aceptado en los Estados Unidos no es reconocido en todas las demás culturas.\\
Si bien muchas empresas ahora ofrecen capacitación en las diferentes culturas donde la empresa realiza negocios, es importante que los empleados que se comunican entre culturas practiquen la paciencia y trabajen para aumentar su conocimiento y comprensión de estas culturas. Esto requiere la capacidad de ver que los propios comportamientos y reacciones de una persona a menudo están impulsados culturalmente y que, si bien pueden no coincidir con los nuestros, son culturalmente apropiados.\\
Si un líder o gerente de un equipo que trabaja en diferentes culturas o incorpora personas que hablan diferentes idiomas, practican diferentes religiones o son miembros de una sociedad que requiere una nueva comprensión, debe trabajar para transmitir esto.\\
Considere cualquier necesidad especial que puedan tener las personas en su equipo. Por ejemplo, pueden observar diferentes días festivos, o incluso tener diferentes horarios de operación. Tenga en cuenta las diferencias de zona horaria y trabaje para mantener a todos los involucrados conscientes y respetuosos de tales diferencias.\\
En términos generales, la paciencia, la cortesía y un poco de curiosidad hacen mucho. Y, si no está seguro de las diferencias que puedan existir, simplemente pregunte a los miembros del equipo. Una vez más, esto puede hacerse mejor en un entorno individual para que nadie se sienta "puesto en el lugar" o cohibido, tal vez incluso avergonzado, sobre discutir sus propias necesidades o diferencias o necesidades.\\
\paragraph{Exigir aceptación mutua}
Luego, cultive y exija aceptación y comprensión mutuas . Al hacer esto, un poco de educación generalmente será suficiente. Explique a los miembros del equipo que la parte del equipo que trabaja fuera de la oficina de Australia, por ejemplo, trabajará en una zona horaria diferente, por lo que las comunicaciones electrónicas y / o las llamadas telefónicas de retorno experimentarán un retraso. Y, los miembros de la oficina de India también observarán diferentes días festivos (como el cumpleaños de Mahatma Gandhi, el 2 de octubre).\\
La mayoría de las personas apreciarán la información y trabajarán arduamente para comprender las diferentes necesidades y los diferentes medios utilizados para alcanzar objetivos comunes. Sin embargo, cuando este no sea el caso, predica con el ejemplo y deja en claro que esperas que te sigan por un camino de mentalidad abierta, comprensión y aceptación.\\
Cuando se trata con personas de una cultura diferente, la cortesía y la buena voluntad también pueden ser muy útiles para garantizar una comunicación exitosa. Nuevamente, esto debe insistirse.\\
Si su punto de partida para resolver problemas es asumir que la comunicación ha fallado, encontrará que muchos problemas se resuelven rápidamente.
\paragraph{Mantenlo simple}
Cuando se comunique, tenga en cuenta que, aunque el inglés se considera el idioma internacional de los negocios, es un error suponer que todos los empresarios hablan un buen inglés. De hecho, solo la mitad de los 800 millones de personas que hablan inglés lo aprendieron como lengua materna. Y, quienes lo hablan como un segundo idioma a menudo son más limitados que los hablantes nativos.\\
Cuando se comunique entre culturas, haga esfuerzos particulares para mantener su comunicación clara, simple y sin ambigüedades.\\
Y (tristemente) evite el humor hasta que sepa que la persona con la que se está comunicando "lo entiende" y no se ofende. El humor es notoriamente específico de la cultura: muchas cosas que pasan por humor en una cultura pueden verse como muy ofensivas en otra.
\paragraph{Y obtenga ayuda si la necesita}
Finalmente, si se presentan barreras idiomáticas, puede ser conveniente para todos contratar a un traductor confiable y con experiencia.\\
Debido a que el inglés no es el primer idioma de muchos empresarios internacionales, su uso del idioma puede estar salpicado de frases en inglés específicas de la cultura o no estándar, lo que puede dificultar el proceso de comunicación. Una vez más, tener un traductor a mano (aunque solo sea durante las fases iniciales del trabajo) puede ser la mejor solución aquí. El traductor puede ayudar a todos los involucrados a reconocer las diferencias culturales y de comunicación y garantizar que todas las partes, independientemente de su ubicación geográfica y antecedentes, se unan y permanezcan juntas hasta la finalización exitosa del proyecto.\\
\section{Preguntas y respuestas}
\begin{enumerate}[\bfseries 1.]
\item ¿Cuál de los siguientes es un ejemplo de comunicación descendente?\\
La comunicación que fluye de un nivel de un grupo u organización a un nivel inferior es comunicación descendente. Los líderes y gerentes de grupo lo usan para asignar objetivos, proporcionar instrucciones de trabajo, explicar políticas y procedimientos, señalar problemas que requieren atención y ofrecer comentarios sobre el desempeño.
\item ¿Cuál de las siguientes afirmaciones es verdadera con respecto a la comunicación oral?\\
Las ventajas de la comunicación oral son la velocidad y la retroalimentación. Podemos transmitir un mensaje verbal y recibir una respuesta en un tiempo mínimo. Si el receptor no está seguro del mensaje, la respuesta rápida le permite al remitente detectarlo y corregirlo rápidamente. La principal desventaja de la comunicación oral aparece cada vez que un mensaje tiene que pasar por varias personas: cuantas más personas, mayor es la distorsión potencial.
\item Genepa Corporation fabrica electrodomésticos y otros productos electrónicos. Genepa planea presentar un nuevo modelo de refrigerador. El gerente de marketing de Genepa ha desarrollado un plan de marketing para este nuevo producto y desea comunicarlo a todos los empleados del departamento de marketing. ¿Cuál de las siguientes opciones es la mejor forma de comunicación para comunicar este plan?\\
La comunicación escrita es a menudo tangible y verificable. Cuando se imprime, tanto el emisor como el receptor tienen un registro de la comunicación; y el mensaje puede almacenarse por un período indefinido. Es probable que el plan de marketing contenga varias tareas repartidas en varios meses. Al ponerlo por escrito, quienes tienen que iniciar el plan pueden consultarlo fácilmente a lo largo de su vida útil.
\item La red de comunicación informal en una organización es caracterizado por la ambigüedad?\\
La red de comunicación informal en un grupo u organización se llama la vid. No está controlado por la gerencia. La mayoría de los empleados lo perciben como más creíble y confiable que los comunicados formales emitidos por la alta gerencia. Los rumores surgen como respuesta a situaciones que son importantes para nosotros, cuando hay ambigüedad y en condiciones que provocan ansiedad.
\item ¿Cuál de los siguientes canales de comunicación proporciona la mayor riqueza de información?\\
Los canales de comunicación enriquecidos pueden:
\begin{itemize}
\item Manejar múltiples señales simultáneamente, 
\item facilitar una retroalimentación rápida y 
\item ser muy personal. 
\end{itemize}
La conversación cara a cara es la más alta en riqueza de canales porque transmite la mayor cantidad de información por episodio de comunicación.
\item ¿Cuál de las siguientes opciones reducirá la efectividad de la comunicación?\\
\begin{itemize}
\item Uso de frases y jerga corporativa.
\item Uso de siglas y abreviaturas.
\item Uso de lenguaje técnico o jerga.
\item Uso del silencio.
\end{itemize}
Cuando nos estamos comunicando en el mismo idioma, las palabras significan cosas diferentes para diferentes personas. Usar jerga, frases, jerga, palabras de jerga corporativa no son deseables. En la medida de lo posible, debemos intentar utilizar un lenguaje uniforme para minimizar las dificultades de comunicación.
\item El correo electrónico es una forma útil de comunicación. Sin embargo, ¿cuál de estos es el mayor problema con los correos electrónicos?\\
Escala conflictos rápidamente.\\
El mayor problema con el correo electrónico es que conduce a una escalada innecesaria de conflictos ya que el tono no se puede interpretar correctamente.
\item El CEO de la compañía dijo: "Queremos construir una cultura de innovación y esto requiere que nuestros empleados se expresen, expresen sus ideas y participen en conversaciones". ¿Qué formas de comunicación permitirán este proceso?\\
Informal de abajo hacia arriba, oral. Como el CEO quiere que los empleados hablen y expresen sus ideas, la comunicación informal, de abajo hacia arriba y oral facilitará el proceso.
\item La comunicación comprometida es Haciendo la pregunta correcta, la paráfrasis y respondiendo.
\item Ruhana estaba en el trabajo cuando Anna, su hermana, llamó para confirmar su plan de cena el siguiente fin de semana en la granja de su tía. Ruhana dijo que llamaría a Sharad, su esposo, y que volvería con ella. Sin embargo, cuando Ruhana le devolvió la llamada, Anna no pudo escuchar la mayor parte de lo que dijo porque Jennifer, la hija de Anna, estaba llorando en el fondo. ¿Quién representa el ruido en el proceso de comunicación representado en el escenario?\\
El ruido representa barreras de comunicación que distorsionan la claridad del mensaje, como problemas de percepción, sobrecarga de información, dificultades semánticas o diferencias culturales. El llanto de Jennifer es el ruido en este escenario.
\end{enumerate}

\chapter{Autogestión}
Esta semana, usted:
\begin{itemize}
\item Obtenga una visión general  de las cinco semanas anteriores.
\item Comprenda  las dimensiones de conocer sus fortalezas y debilidades.
\item Examine  las prácticas que emprende  para conocer sus contribuciones que se vinculan con objetivos organizacionales más grandes.
\item Reflexionar  sobre los valores gerenciales. 
\item Comprender la importancia de invertir en uno mismo.
\end{itemize}
\section{Conociendo sus fortalezas y debilidades}
En esta subsección, cubriremos:
\begin{itemize}
\item Importancia de conocer sus fortalezas y debilidades.
\item Cómo conocer tus  fortalezas y debilidades.
\end{itemize}
\subsection{Importancia de conocer sus fortalezas y debilidades}
$"$¿Cuáles son sus fortalezas y debilidades?$"$\\
Muchas veces las personas te dan puntos fuertes que parecen ser muy comunes. Déjame darte un ejemplo. $"$Soy un comunicador eficaz$"$.\\
Hay varios comunicadores efectivos. Entonces, ¿qué es tan distintivo de tu fuerza? ¿Y cuando pasamos mucho tiempo con él preguntándole qué es lo que hizo tan diferente en la comunicación? ¿Fue en la forma en que habló? ¿Fue en la forma en que pronunció las presentaciones? ¿Era en la forma en que podía influir? ¿Fue en la forma en que podía negociar? ¿Dónde estaba la diferencia? ¿Y cuál es la fuerza?  $"$Mira hacia atrás en tu vida y observa incidentes críticos en los que dirías que fuiste un comunicador efectivo$"$. Y luego parece que el centavo cayó en su lugar, porque se dio la vuelta y dijo: $"$Mi fortaleza es realmente poder comprender las diferentes perspectivas que las personas aportan y poder ver rápidamente los puntos de diferencias y convergencia . $"$ Y fue esta habilidad la que lo ayudó cuando fue identificado muy temprano en la carrera para trabajar con el equipo de finanzas para realizar la debida diligencia en fusiones y adquisiciones. Fue esta habilidad la que más tarde lo ayudó a poder trabajar en la fuerza de trabajo que estaba comprometida con la gestión del cambio dentro de su organización.\\
¿Por qué menciono esto? Porque la mayoría de nosotros cuando hemos pedido identificar fortalezas, identificamos fortalezas muy genéricas. Estas son fortalezas para todos los que están en un lugar de trabajo. Entonces, ¿qué tiene esa fuerza que es distintiva y única para ti?
\subsection{Cómo conocer tus fortalezas y debilidades-Parte 1}
También me preguntan: "¿Cómo llego a esta fortaleza?" "¿Cómo sé cuál es esta fuerza?" Primero, recuerde los tiempos en que ha sido eficaz en el trabajo, en su lugar de trabajo o incluso en su espacio personal.\\
¿Qué fue lo que te hizo efectivo? Si pudieras mirar hacia atrás y hacer una lista, ¿qué fue lo que hiciste? ¿Cómo es que te comportaste? ¿Cómo es que pensaste? Si pudieras identificar esos comportamientos, entonces tienes cierta comprensión de tus fortalezas. La segunda forma de ver las fortalezas es mirar los comentarios que recibe de sus jefes, sus pares, sus informes directos y otras personas importantes en su vida. Por lo general, la gente te felicitará por tus fortalezas. Esa es otra forma de identificar tus fortalezas.\\
Mire cuáles son realmente sus puntos fuertes. Si tiene alguna duda, tal vez quiera caminar y preguntarle a alguien importante, que está cerca, cuáles son sus puntos fuertes. Es posible que la persona se ría porque está preguntando algo que tiene un sentido común pero que vale la pena el esfuerzo porque al menos le ayudará a comprender cuáles son sus puntos fuertes.
\subsection{Cómo saber tu fuerza y debilidades - Parte 2}
Debe aprovechar sus puntos fuertes y, de todos modos, su organización le ofrece suficientes oportunidades para poder aplicar sus puntos fuertes. De hecho, lo que encuentro con mayor frecuencia es que, particularmente entre los gerentes de carrera temprana, que haces lo que eres bueno de manera muy efectiva. Pero, curiosamente, mientras aplicas tus fortalezas, recuerda que tus debilidades deben ser reparadas. De hecho, muchas veces, las carreras de los gerentes jóvenes se rompen debido a la arrogancia y la falta de aceptación de la propia debilidad.\\
¿Cómo sé cuál es mi debilidad? Exactamente la misma respuesta que $"$¿Cómo sé cuáles son mis puntos fuertes?$"$ ¿Estás escuchando cuando las personas te dan comentarios sobre lo que quieren que cambies? ¿Estás mirando hacia atrás en aquellas situaciones en las que no te ha ido bien o las cosas no han salido como esperabas y haciendo esta pregunta $"$¿Por qué fue así?$"$ Nuestra primera reacción es decir que las cosas salieron mal debido a mi gerente, mi compañero, mi reportero directo, porque no teníamos recursos. Pero da un paso atrás y pregúntate qué es lo que podrías haber hecho que podría haber ayudado a que saliera bien.\\
Por lo tanto, si observa sus debilidades, debe enfocarse y observar esos episodios, eventos, incidentes críticos, aquellas acciones que tuvieron consecuencias no del modo previsto. Entonces, como gerente, si usted es sensible a la manera en que se le brindan comentarios, los incorporará y reflexionará automáticamente, y verá si esa es su debilidad. Al final del día, \textbf{la fuerza de una cadena está determinada por su eslabón más débil, no por su eslabón más fuerte}. Así que arregla tu debilidad, pero aprovecha tus fortalezas.
\subsubsection{Análisis DAFO personal}
Es más probable que tenga éxito en la vida si usa sus talentos al máximo. Del mismo modo, sufrirá menos problemas si sabe cuáles son sus debilidades, y si maneja estas debilidades para que no importen en el trabajo que realiza.\\
Entonces, ¿cómo identificas estas fortalezas y debilidades y analizas las oportunidades y amenazas que surgen de ellas? El análisis FODA es una técnica útil que te ayuda a hacer esto.\\
\paragraph{Cómo usar la herramienta}
\subparagraph{Fortalezas}
\begin{center}
\begin{itemize}
\item ¿Qué ventajas tiene que otros no tienen (por ejemplo, habilidades, certificaciones, educación o conexiones)?
\item ¿Qué haces mejor que nadie?
\item ¿A qué recursos personales puede acceder?
\item ¿Qué ven otras personas (y su jefe, en particular) como sus puntos fuertes?
\item ¿De cuáles de tus logros estás más orgulloso?
\item ¿Qué valores crees que otros no pueden exhibir?
\item ¿Eres parte de una red en la que nadie más está involucrado? Si es así, ¿qué conexiones tienes con personas influyentes?
\end{itemize}
\end{center}
Considere esto desde su propia perspectiva, y desde el punto de vista de las personas que lo rodean. Y no seas modesto o tímido, sé tan objetivo como puedas. Conocer y usar tus fortalezas puede hacerte más feliz y más satisfecho en el trabajo.\\
Y si aún tiene dificultades para identificar sus puntos fuertes, escriba una lista de sus características personales.
\textbf{NOTA} Piensa en tus puntos fuertes en relación con las personas que te rodean. Por ejemplo, si eres un gran matemático y las personas que te rodean también son excelentes en matemáticas, entonces es probable que esto no sea una fortaleza en tu rol actual, puede ser una necesidad.
\subparagraph{Debilidades}
\begin{itemize}
\item ¿Qué tareas suele evitar porque no se siente seguro haciéndolas?
\item ¿Qué verán las personas a tu alrededor como tus debilidades?
\item ¿Confía completamente en su educación y capacitación? Si no, ¿dónde estás más débil?
\item ¿Cuáles son sus hábitos de trabajo negativos (por ejemplo, a menudo llega tarde, está desorganizado, tiene mal genio o es pobre para manejar el estrés)?
\item ¿Tiene rasgos de personalidad que lo detienen en su campo? Por ejemplo, si tiene que realizar reuniones de manera regular, el miedo a hablar en público sería una gran debilidad.
\end{itemize}
\subparagraph{Oportunidades}
\begin{itemize}
\item ¿Qué nueva tecnología puede ayudarte? ¿O puede obtener ayuda de otros o de personas a través de Internet? 
\item ¿Su industria está creciendo? Si es así, ¿cómo puede aprovechar el mercado actual?
\item ¿Tiene una red de contactos estratégicos para ayudarlo u ofrecer buenos consejos?
\item ¿Qué tendencias (gerenciales o no) ve en su empresa y cómo puede aprovecharlas?
\item ¿Alguno de sus competidores no está haciendo algo importante? Si es así, ¿puedes aprovechar sus errores?
\item ¿Hay una necesidad en su empresa o industria que nadie esté llenando?
\item ¿Sus clientes o proveedores se quejan de algo en su empresa? Si es así, ¿podría crear una oportunidad ofreciendo una solución?
\end{itemize}
Puede encontrar oportunidades útiles en lo siguiente:
\begin{itemize}
\item Eventos de redes, clases educativas o conferencias.
\item Un colega que tiene una licencia extendida. ¿Podrías asumir algunos de los proyectos de esta persona para ganar experiencia?
\item Un nuevo rol o proyecto que lo obliga a aprender nuevas habilidades, como hablar en público o relaciones internacionales.
\item Una expansión o adquisición de la empresa. ¿Tiene habilidades específicas (como un segundo idioma) que podrían ayudar con el proceso?
\end{itemize}
Además, lo que es más importante, observe sus fortalezas y pregúntese si esto abre alguna oportunidad, y observe sus debilidades y pregúntese si podría abrir oportunidades eliminando esas debilidades.
\paragraph{Amenazas}
\begin{itemize}
\item ¿Qué obstáculos enfrenta actualmente en el trabajo?
\item ¿Alguno de tus colegas compite contigo por proyectos o roles?
\item ¿Está cambiando su trabajo (o la demanda de las cosas que hace)?
\item ¿La tecnología cambiante amenaza su posición?
\item ¿Alguna de tus debilidades podría generar amenazas?
\end{itemize}
La realización de este análisis a menudo proporcionará información clave: puede señalar lo que hay que hacer y poner los problemas en perspectiva.
\section{Conociendo su contribución}
En esta subsección, cubriremos:
\begin{itemize}
\item Claridad sobre el rol de uno.
\item Encajar en el propósito organizacional más grande.
\end{itemize}
\subsection{Claridad sobre el papel de uno}
Una parte importante de la administración de uno mismo es saber: $"$¿Cuál es mi contribución y cómo se vincula mi contribución con el propósito organizacional más amplio y mi aspiración profesional más amplia?$"$\\
¿Cuál es su función y cómo encaja su función con el propósito más amplio de la organización?\\
No entendemos cómo el papel era importante y destacado en el contexto de la organización. De hecho, muchas veces cuando se pregunta: "¿Cuál es el propósito de su trabajo y por qué están allí?" Fue interesante que todo lo que escuché de ellos fue una lista de tareas y actividades que tenían que entregar. ¿Por qué les menciono esto como gerentes de etapa temprana? No estás allí solo para hacer un conjunto de tareas y actividades. Hay un propósito detrás del papel que desempeñas. Si no tiene claro el propósito de su función, le será muy difícil influir en los demás.
\subsection{Encajar en el propósito organizacional más grande}
Supongamos que su departamento fuera cerrado, qué tan seguro está de que podrá mantener una posición en su organización en cualquier departamento, en cualquier división , en alguna parte de la organización más grande?\\
Si está pensando y sigue pensando, haga una pausa. Significa que no tiene claridad sobre cómo su rol actual contribuye al propósito y significado más amplio de la organización. Si no tiene claro qué es lo que es único, lo que distingue su rol y no lo ha pensado en el contexto de la organización más grande, significa que solo está realizando un conjunto de tareas y actividades que pueden ser extremadamente valioso a corto plazo, pero puede no agregarle valor profesional a largo plazo.\\
Mi sugerencia para usted es regresar y obtener claridad sobre cuál es el propósito de su función. ¿Por qué te asignan el conjunto de actividades? Pero, lo que es más importante, si continúa haciendo estas actividades, ¿hacia dónde se dirigirá y cuál es el resultado final que sucederá, tal vez, en dos años o en tres años?\\
Esta es una conversación que debe tener con sus gerentes. Es extremadamente importante porque cada uno de ustedes debe tener claro cuál es su contribución a la organización, no en términos de áreas clave de resultados, no en términos de objetivos, no en términos de entregables sino en términos de mi desarrollo profesional, mi crecimiento profesional , mi contribución a la organización.\\
Asegúrese de comprender cuál es el propósito de su rol y cuál es la contribución que este rol está haciendo a la organización más grande. Esta claridad es extremadamente importante porque no se trata solo del desempeño organizacional, sino más importante, se trata del desarrollo personal y profesional.
\section{Valores gerenciales}
En esta subsección, cubriremos:
\begin{itemize}
\item Importantes valores gerenciales
\item Disfrutando el éxito de los informes directos
\item Integridad visible
\end{itemize}
\subsection{Importantes valores gerenciales}
Hay tres valores importantes que todo gerentes debe demostrar:
\begin{itemize}
\item Hacer las cosas a través de las personas.
\item Disfrutando el éxito de sus informes directos.
\item Demostrando integridad visible.
\end{itemize}
La mayoría de las veces, cuando hablamos de ser un gerente efectivo, tendemos a hablar sobre la competencia y la habilidad necesarias para convertirse en un gerente efectivo. ¿Sabes cómo planificar? ¿Sabes cómo establecer metas? ¿Sabes cómo entrenar? ¿Sabes cómo dar retroalimentación?\\
Pero una de las cosas más importantes de ser gerente también es demostrar ciertos valores. Muy interesante, puedes enseñarle a alguien cómo planificar, puedes enseñarle a alguien cómo establecer metas, pero es muy difícil enseñarle a alguien a demostrar valores. Entonces, ¿cuáles son estos valores únicos que se necesitan para los gerentes?\\
Hay tres valores importantes que los gerentes tienen que demostrar. No importa si usted es un gerente de etapa temprana o si es un gerente experimentado. El primero y el más importante es\textbf{ hacer las cosas a través de las personas}. El segundo aspecto es \textbf{disfrutar el éxito de sus informes directos}. Y tercero, por último pero no menos importante, \textbf{está demostrando integridad visible}.\\
Pasemos unos minutos mirando cada uno de estos. Ahora estamos considerando hacer las cosas a través de otros como un valor. Si recuerdas en la sesión sobre delegación, hablamos de que la delegación requiere confianza. Ahora comienzas a ver por qué mencioné la confianza, porque la confianza es un valor. Entonces, como gerente, cuánto confío en los demás determinará cuánto delego. La cantidad que delegue determinará si estoy haciendo las cosas a través de las personas o si lo estoy haciendo yo mismo. Y uno de los desafíos de ser un gerente por primera vez es que obviamente serás más competente que todos tus colaboradores individuales para hacer las cosas, ¡porque has estado allí y lo has hecho! El más difícil es dejarlo ir. Deja de lado lo que eres bueno y consigue que tus reportes directos contribuyan y trabajen en aquellas áreas en las que eres excepcionalmente bueno.\\
Entonces, ¿qué se necesita para ti? Significa que tienes que renunciar a tus puntos fuertes. Pídales a los demás que lideren las cosas en las que es bueno y, de hecho, asuma más y más responsabilidades en las que no sea tan bueno, pero eso es lo más importante que debe hacer en su función.\\
Entonces, ahora ves porqué hacer las cosas a través de las personas es una habilidad, una capacidad, una competencia, una tarea. Pero lo más importante, es un valor.
\subsection{Disfrutando el éxito de los informes directos}
El segundo valor gerencial importante es el éxito de los informes directos. Déjame pedirte que reflexiones un poco. Necesita sus informes directos más de lo que ellos lo necesitan a usted. Repito, Necesita sus informes directos más de lo que ellos lo necesitan a usted. ¿Por qué? Porque usted es el gerente que necesita hacer más cosas. Usted es un gerente y su jefe lo responsabilizará por el cumplimiento de los entregables. Como contribuyentes individuales, siempre pueden moverse y encontrar otras organizaciones, pero como gerente, es su responsabilidad asegurarse de que su equipo disfrute trabajar con usted.\\
Uno de los aspectos de ser un gerente efectivo es disfrutar el éxito de los informes directos. Debe tener la humildad de reconocer que sus informes directos son un aspecto integral de su desempeño.\\
Entonces, hagamos una pausa por un momento. Realice la autoevaluación y vea por sí mismo lo bueno que es en esta dimensión de disfrutar el éxito de los informes directos. Su éxito a largo plazo depende del tipo de informes directos en los que invierte y crece. Solo puede crecer si puede asignar responsabilidades cada vez más altas a sus informes directos. Si tiene cada vez más personas en las que puede confiar, puede asumir responsabilidades adicionales y esas responsabilidades adicionales, como ya hemos mencionado anteriormente, contribuirán a su crecimiento y desarrollo personal. Por lo tanto, reconocer y disfrutar el éxito de los informes directos es una parte integral de la demostración de valores gerenciales.
\subsection{Integridad visible}
¿Qué significa demostrar integridad visible?\\
El primer aspecto de demostrar la integridad visible es: si hay un proceso o una política en la organización, primero debe seguirlo antes de implementarlo en sus informes directos.\\
En segundo lugar, si la política es ambigua, pregunte si ha utilizado un juicio justo para llegar a una decisión. En tercer lugar, supongamos que aún no está claro y que no puede llegar a este juicio. ¿Realmente vas a hablar con líderes de alto nivel que son creíbles en la organización y realmente les pides consejo? Y, por último, pero no menos importante, si aún no puede evaluar y llegar a un juicio justo, entonces solo hay una prueba, que dejaría que piense, que llamo la prueba espejo.\\
Si tiene un dilema y no está seguro de qué hacer, simplemente vuelva y pregunte: $"$Si tomé esa decisión en particular, ¿podré dormir tranquilo con la conciencia tranquila?$"$ Si tiene claro esto, está demostrando integridad visible. Como profesional, demostrar integridad visible no es más que caminar la charla. ¿Estás caminando la charla? Porque solo si usted camina la charla, sus informes directos lo harán.
\section{Invierta en uno mismo}
En esta subsección, cubriremos:
\begin{itemize}
\item Invertir continuamente en uno mismo.
\item Espacios para el desarrollo.
\end{itemize}
El último y más importante aspecto de la autogestión es invertir continuamente en uno mismo. No puedo enfatizar lo importante que es invertir en la capacidad de uno continuamente.  A menudo, nuestras carreras se han estancado y ni siquiera sabemos por que se han estancado. Y ese es el momento en que vienen a invertir en un programa de educación continua. Si  tiene que invertir proactivamente en usted mismo, uno podría preguntarse: $"$¿Cómo se hace para hacer eso?$"$ Investigaciones previas en el campo del desarrollo profesional y personal indican que una gran parte del desarrollo profesional en realidad ocurre en el lugar de trabajo. De hecho, la regla 70:20:10 del desarrollo profesional y personal establece que el desarrollo profesional y la lectura ocurre de tres maneras. El 70\% de nuestro desarrollo profesional ocurre en el trabajo. El 20\% sucede a través de las conversaciones y las interacciones, observaciones de nuestros compañeros. Y, menos del 10\% ocurre en contextos formales de aprendizaje: un programa de capacitación, un curso de educación continua o algo similar a lo que está haciendo en este momento es inscribirse en este módulo. Si sabemos que el 70\% de nuestro desarrollo profesional ocurre en el lugar de trabajo, la siguiente pregunta sería: $"$¿Dónde sucede específicamente en el lugar de trabajo?$"$ McCall, Lombardo y sus colegas han investigado durante varias décadas y ahora se preguntan: $"$¿Cómo se desarrollan los gerentes?$"$ ¿Cómo se desarrollan los gerentes? A través de su investigación, lo que nos aportan es que hay ciertos contextos dentro de las organizaciones que brindan oportunidades naturales para el desarrollo profesional.
\subsubsection{Lectura adicional}
Durante veintitrés años de mi vida laboral, construí empresas para otras personas. En 1999, junto con algunos profesionales con ideas afines, cofundé MindTree Consulting. Las empresas que construimos a menudo se denominan "empresas". Si piensa un poco más, se dará cuenta de que la vida de cada persona es una empresa. Cada uno de nosotros es dueño de su propia pequeña empresa. A través del nacimiento, crecimiento y decadencia, experimentamos altibajos muy similares a los de las empresas. Como algunas compañías tienen más éxito que otras, algunas vidas tienen más éxito que otras. A medida que las empresas crecen, crean alianzas y redes, al igual que todos nosotros en nuestras propias vidas.\\
Las similitudes no son inusuales porque después de todo, una empresa no es más que una colección de personas, no es una caja negra. No tiene vida propia. Por lo tanto, puede haber cosas interesantes que podemos aprender de la creación y el mantenimiento de empresas ganadoras y extender el aprendizaje a nuestras pequeñas empresas personales cuyos presidentes somos todos. En este artículo, quiero hablar sobre la creación de empresas personales, sacando una hoja de la vida corporativa.\\
Ganar se trata de diferenciación. La diferenciación, como bien sabe, no es una casualidad o un accidente. Tiene que ser cultivado durante un período de tiempo. Hay seis cosas cuando se hace bien y con el tiempo, crear un ganador. Estos son:
\begin{itemize}
\item dominio, 
\item uso de herramientas adecuadas, 
\item metodología, 
\item calidad, 
\item innovación y 
\item marca.
\end{itemize}
A primera vista, pueden parecerle técnicas y aburridas, y usted puede decir que ¡eso no parece mi vida! Espera un momento y te mostraré cómo son realmente sobre la vida.\\
Primero, hablemos del \textbf{dominio}. En el futuro, tanto las empresas exitosas como las personas tendrán que estar estrechamente asociadas a un campo primario de enfoque. No podemos ser todo para todos. No puedo ser excelente para hacer helados y escribir un excelente software. Cada vez más, las personas elegirán trabajar con empresas que hacen algo tan bien que están asociadas con ese dominio. En otras palabras, son especializados, así es como entregan valor y ganan la lealtad del cliente. Son tan buenos en el área específica de su trabajo que a menudo "representan algo". Centrarse en un dominio ofrece la capacidad de representar algo. Para ser de clase mundial hoy en día, las empresas no tienen que ser grandes, tienen que representar algo. Para continuar siendo un ganador, para diferenciarse en la vida, usted también tendrá que defender algo. En un nivel menos importante, puede significar que debe ser un profesional altamente especializado. Sin embargo, eso no es lo que estoy diciendo. Necesita lograr una especialización profesional y ser bueno en lo que hace. Más importante aún, debes desarrollar a una persona que el mundo que te rodea se asocia con algo de una manera fundamental. Cuando defiendes algo, de repente te vuelves visible y poderoso. Al mundo le resulta más fácil tomar nota de las personas que defienden algo.\\
Ahora veamos el segundo aspecto de la \textbf{diferenciación}: este es sobre el uso de las herramientas adecuadas. Las empresas ganadoras, exitosas y diferenciadas utilizan mejores herramientas que sus competidores. Desde que la humanidad ha inventado herramientas, forman una parte muy importante de nuestra existencia. Las herramientas ayudan a la productividad. La elección inteligente y cuidadosa de herramientas decide la diferencia entre la mejor y la segunda mejor. Entre lo mejor y lo segundo mejor en la vida, la diferencia rara vez es su intelecto. Piense en el hogar de ancianos del vecindario y compárelo con el mejor hospital de la ciudad. La diferencia rara vez es el talento. La diferencia está en el uso de herramientas. Cuando un paciente es llevado al mejor hospital, en cuestión de minutos, los médicos pueden usar una tomografía computarizada, un ultrasonido, una docena de pruebas de laboratorio y cero en la entrega de bienestar. A medida que creces, pregúntate, ¿cuáles son mis herramientas? ¿Cómo son diferentes de los demás? Si su herramienta es mejor, es más probable que gane. Todos sabemos cómo los reyes indios perdieron ante los invasores mogoles: los lugareños usaban elefantes, lanzas, arcos y flechas, los retadores usaban cañones. Las herramientas, no el patriotismo innato, decidieron el ganador.\\
Las herramientas pueden ser adquiridas por cualquier persona. Para mantenerse diferenciado, debe ir más allá de las herramientas y centrarse en el tercer aspecto: la \textbf{metodología}. ¿Qué es la metodología? Es tu propia forma especial de hacer las cosas. Las grandes empresas tienen importancia en desarrollar su propia forma de hacer las cosas. Siempre que vea un problema, antes de saltar para resolverlo, pregúntese: ¿cuál es la metodología? Lo se ¿Necesito aprenderlo primero? Las personas, que saltan a resolver problemas sin establecer primero la metodología, a veces incluso pueden tener éxito. Sin embargo, es probable que ese éxito sea un destello en la sartén. No es sostenible. El poder de la metodología está en su capacidad de ofrecer un éxito predecible. En la vida, tanto como en la empresa, a las personas les gusta la previsibilidad. No desea ir a un médico que es impredecible ¿Volarás con un piloto que es impredecible? Buscamos personas que sean predecibles porque usan una metodología sólida. Antes de intentar el siguiente problema complejo, pregúntese: ¿cuál será mi metodología ganadora? Una vez que lo sabes, debes refinarlo, evolucionarlo y mantenerlo afilado.\\
Ahora veamos la cuarta área de enfoque que ayuda a construir la diferenciación. Este es sobre \textbf{calidad}. Nadie quiere tratar con empresas que no ofrecen productos y servicios de calidad. Las mejores empresas del mundo hacen su trabajo tan bien que se dice que ejecutan sus procesos a un asombroso nivel Six Sigma. Significa que cometen 3.4 errores, ¡dado un millón de oportunidades! ¿Cuál es la esencia de la calidad? La calidad no sucede como un accidente. Está cuidadosamente planificado e implementado. Las empresas centradas en la calidad creen que puede ofrecer calidad solo cuando comienza a centrarse en el proceso en lugar de centrarse en el resultado. Como dice el dicho sánscrito "Karmanye badhkia rastu - ma faleshu kadachana". En otras palabras, concéntrate en el proceso y déjame el resto a Mí. \textbf{Los individuos centrados en el proceso son individuos de calidad}. Como un corredor excepcional, estas son personas que toman en serio todos los aspectos de su deporte y prestan toda su atención al acto de entrega sin que la cinta lo desenfoque al final de la carrera.\\
En el mundo competitivo de hoy, la calidad es un hecho y hemos comenzado a darnos cuenta de la importancia de centrarse en la \textbf{innovación} como estrategia para construir la diferenciación. La innovación es su capacidad de encontrar formas y significados nuevos y diferentes. La innovación es mi capacidad de responder: "¿Qué es lo que hago que es nuevo y diferente"? Si no puedo responder esa pregunta satisfactoriamente, el mundo no quiere tratar conmigo. El mundo real está abarrotado de tantos clones y yo también. En un mundo así, los ganadores están invariablemente asociados con su capacidad de pensar diferente, actuar de manera diferente y entregar de manera diferente. La innovación se pensó anteriormente como brujería. Hoy en día, las empresas dicen: queremos practicar la innovación. Usted también debe preguntarse, ¿seguiré el camino trillado o encontraré mi propio espacio? Ser innovador no es un evento casual: la mente puede ser entrenada para pensar en formas más nuevas, más diversas y creativas.\\
Ahora veamos la última área de enfoque para construir una estrategia para la diferenciación continua. Este es sobre la \textbf{marca}. El concepto proviene de los ganaderos que calificaron su rebaño con signos únicos para que pudieran ser fácilmente reconocidos. En los tiempos modernos, el concepto se ha convertido en una ciencia y un arte para ayudar a todo tipo de organizaciones e incluso conceptos a ser reconocidos por encima del estruendo del mercado. Las compañías exitosas y ganadoras son compañías bien calificadas. ¿Cuál es la esencia de la marca? No es publicidad como puede pensar. La esencia de la marca se manifiesta en dos cosas: 
\begin{itemize}
\item El valor innato y
\item  la capacidad de expresar ese valor para que el mundo vea una conexión entre usted y el valor que pueden esperar recibir de usted.
\end{itemize} 
El concepto es muy importante. Sin un gran valor, no se puede construir una gran marca. Si no agrego un valor perceptible a las personas que me rodean, la simple publicidad solo creará distancia entre mi cliente y yo. Cuando Dios te creó, Él te dio tu propio ADN, en cierto sentido, esa es tu marca física. En un sentido mental, ahora debes preguntarte, ¿cuál es mi ADN que estoy desarrollando? En otras palabras, ¿cuál es el valor único que tengo para las personas que me rodean?\\
Cuando miras las marcas, nacen, algunas viven y otras mueren. Los que viven son periódicamente rediseñados. Su valor innato se reconstruye y se actualiza. También es cierto para usted y mi marca. Si no hago nada sobre el valor que puedo ofrecerle, pronto quedaré obsoleto. Tengo que mirar constantemente cuánto valor agrego para seguir siendo útil para el mundo que me rodea. Nuestra capacidad de agregar valor a los demás tiene, según mí, tres estados: puedo ser de valor positivo, de valor neutral o de valor negativo. Si no hago esfuerzos, pasaré fácilmente del primer estado al último. Esto es cierto no solo para las empresas, es cierto para el paradigma de maestro de estudiante, es cierto para una relación padre-hijo, y es cierto para la vida misma.\\
\textbf{Ganar se trata de diferenciación.} La diferenciación ocurre cuando las empresas y los individuos se enfocan en las seis cosas simultáneamente. Estos son: representar algo. Elija herramientas con cuidado. Desarrolla tu propia metodología. Construye calidad en lo que sea que hagas. Innovar: pregúntese qué hay de nuevo y diferente hoy y, finalmente, cree una marca personal exitosa. Estos son como seis caballos que deben tirar del charlot en la misma dirección, con la misma energía, para terminar.\\
\begin{center}
$"$Tú eres lo que tu deseo de conducción profunda es como es tu deseo, así es tu voluntad como es tu voluntad, así es tu acto como es tu acto, así es tu destino$"$
\end{center}
\subsection{Espacios para el desarrollo}
Si tiene que invertir proactivamente en usted mismo, ¿cuáles son los espacios donde puede hacer esto? Quiero decir, ¿ que crees que debería hacer para progresar en mi carrera?. Hay cinco contextos en el trabajo dentro de su contexto laboral que le brindan oportunidades para el desarrollo profesional y personal.\\
La primera y la más importante es la \textbf{puesta en marcha}. Si está iniciando algo nuevo, en realidad requiere capacidades nuevas. ¿No es así?. Hay muy poca capacidad que tiene para iniciar que puede aportar de su antiguo contexto de trabajo. Entonces, tomemos un ejemplo. Su organización quiere comenzar un nuevo proceso y este es un proceso de administración de proveedores. Para iniciar dicho proceso, la organización generalmente lo probará en una parte de la organización. Entonces, ¿hay una oportunidad natural cuando los pilotos están sucediendo en la organización para que usted participe y mejore su desarrollo profesional? ¡Por supuesto! No hace falta decir que los pilotos le brindan oportunidades no amenazantes para probar algo nuevo, experimentar y aportar valor a la organización. Por lo tanto, si hay algo que está comenzando o hay un proyecto piloto en su organización, esa es una oportunidad para el desarrollo profesional.\\
La segunda oportunidad para el desarrollo profesional, tanto profesional como personal, es una \textbf{asignación no estructurada}. Personalmente, he encontrado tareas no estructuradas extremadamente valiosas en mi propio viaje de desarrollo. Déjame darte un ejemplo. Su organización quería mejorar la productividad. El contexto es que la productividad se ha estancado en la organización. Hay varias formas, todos sabemos que hay varias maneras en que se puede mejorar la productividad. Pero, ¿cuál es esa forma adecuada a través de la cual podemos obtener la máxima mejora de la productividad con el mínimo esfuerzo? Ahora, este problema obviamente requería formas muy diferentes de pensar de lo que tradicionalmente haces como parte de tu trabajo. Las tareas no estructuradas le ofrecen oportunidades para adquirir múltiples capacidades. El primero es la tolerancia a vivir con un problema ambiguo con mucha incertidumbre. El segundo es armar una estructura sobre un problema relativamente no estructurado y, finalmente, existen múltiples alternativas, todas ellas igualmente factibles. Entonces, ¿cómo juzga cuál de esas alternativas debe elegir como decisión final? Este tipo de oportunidad para un desarrollo profesional solo puede venir donde hay tareas no estructuradas.\\
La tercera oportunidad para invertir en usted es cuando hay una \textbf{crisis organizacional} y la organización necesita a alguien para cambiar la situación. En tales situaciones, donde hay una crisis o un problema o digamos que algo no va bien. Digamos cliente. Hay una situación de cliente y hay un problema y todos están luchando contra incendios y las cosas simplemente están bajando. En esta situación, interviniendo, recuperando la confianza del cliente, construyéndola y brindando oportunidades y soluciones y cambiando de una crisis a un entorno estable, la gente dice que es una situación muy intensiva en aprendizaje. Todo lo que haya hecho en su contexto normal no funcionará en ese contexto.So, the third and the most important opportunity for personal and professional development is a turnaround opportunity. Por lo tanto, la tercera y más importante oportunidad para el desarrollo personal y profesional es una oportunidad de respuesta.\\
El cuarto es el \textbf{aumento en el alcance de lo que está haciendo}. Aproveche más oportunidades disponibles dentro de su departamento. Pídale a su gerente que le dé.
\section{Preguntas y respuestas}
\begin{enumerate}[\bfseries 1.]
\item Radhika ha estado trabajando durante 5 años con Dravya Bank. Recientemente fue promovida como gerente de sucursal, donde dirige un equipo de 10 empleados. ¿Qué es lo que debe hacer para su propio desarrollo personal?\\
La autoconciencia es el corazón del desarrollo personal. Si bien cuidar a los clientes, hacer que los procesos bancarios sean efectivos y dar retroalimentación son actividades importantes, invertir tiempo en reflexionar sobre nuestros propios comportamientos es muy importante para un crecimiento personal sostenido a largo plazo.
\item Eres Rahul, quien recientemente se hizo cargo de la entrega del producto en Huge Bazar. En una de sus tareas, su equipo no pudo entregar a tiempo. Estás reflexionando sobre cuáles podrían ser las diferentes razones del fracaso.\\
Escuche a los demás y obtenga información sobre por qué el equipo falló y Evalúate a ti mismo en lo que podrías haber hecho mejor para asegurar una entrega oportuna.
\item ¿Cómo debe manejar sus fortalezas y debilidades?\\
Uno debe manejar la fuerza y la debilidad simultáneamente. Uno no puede concentrarse en uno y descuidar al otro, así que desarrolle su fuerza pero al mismo tiempo arregle su debilidad.
\item John ha trabajado para una empresa de hardware durante casi 15 años. Fue entrevistado para un puesto de liderazgo senior por el CEO. El CEO le preguntó: $"$¿Cuál es su papel y cómo se adapta al propósito más amplio de la organización?$"$ John respondió: $“$Administro un equipo de 30 personas. Soy responsable de garantizar la entrega oportuna y el control de calidad de las piezas de hardware a nuestro socio off-shore $"$.\\
Joh n no entiende cómo su rol se ajusta al contexto más amplio de la organización y John ha enumerado un conjunto de tareas y responsabilidades que debe cumplir, pero no el propósito del rol.
\item ¿Cuáles son algunas de las conversaciones clave que necesita tener con sus gerentes con respecto a su crecimiento personal?\\
¿Cómo contribuye mi rol actual a mi desarrollo profesional? y ¿Cómo contribuye mi trabajo al propósito más amplio de la organización?
\item ¿Cuáles son los 5 contextos que discutió el profesor, que proporcionan espacio para el desarrollo profesional y personal?\\
puesta en marcha, asignación no estructurada, oportunidad de respuesta , aumento del alcance del trabajo y ser parte de equipos y grupos de trabajo multifuncionales.
\item Como ingeniero, fuiste excelente para corregir errores en los programas de software. Ahora es un gerente, responsable de un equipo de 25 ingenieros. Hay un miembro del equipo que está tomando más tiempo en comparación con otros en términos de dominar su trabajo. ¿Cuál debería ser tu enfoque hacia él?\\
Identifique lo que impide el rendimiento y ayúdelo a desarrollar su capacidad.
\item ¿Cuál debería ser su enfoque para identificar sus fortalezas y debilidades?\\
Obtenga comentarios de sus seres queridos, colegas, jefes e informes directos. Como han trabajado estrechamente con usted, deberían poder ayudarlo y Mire hacia atrás en su vida y vea qué funcionó y qué no funcionó, qué tareas realizó y por qué pudo / no pudo ejecutarlas de manera efectiva.
\item Como gerente de atención al cliente de una empresa de computadoras portátiles, se le ha informado sobre una queja de un cliente de que la batería se calienta y provoca el apagado del dispositivo. Su compañía no produce baterías y sus términos y condiciones especifican que los clientes tendrán que tratar directamente con el fabricante de la batería en caso de un problema de batería. Su informe directo le ha dicho lo mismo al cliente. Sin embargo, el cliente insiste en que, dado que le compró la computadora portátil, es su responsabilidad solucionar el problema. Te das cuenta de que el punto del cliente es válido. ¿Cómo debe pensar en resolver este problema?\\
Pedir consejo y asesoramiento a sus personas mayores correctamente. Muchos de ustedes habrían mencionado A, que técnicamente es lo que sucede más tiempo. ¿Como solucionar el problema? Hay muchas formas de hacerlo. Puede verificar si los problemas de la batería prevalecen con los modelos específicos y si no es un caso aislado. Si se trata de un caso aislado, ¿es posible hablar con el fabricante? Todo esto requerirá que hable con personas mayores para obtener su cónsul y luego responder.
\item Jane hace los mejores pastelitos en su comunidad. Ella espera comenzar su propio rincón de magdalenas. Todos le aconsejan que se gradúe en restauración. Intuitivamente, Jane sabe que el desarrollo personal y profesional puede no ocurrir en una universidad de catering.\\
El 20\% sucede a través de conversaciones e interacciones con compañeros
\end{enumerate}
\section{Resumen}
Algunos de los puntos clave que discutimos son:
\begin{itemize}
\item  Mientras juegas con tus fortalezas, recuerda que tus debilidades deben ser reparadas.
Una parte importante de la administración de uno mismo es  saber cuál es su contribución y cómo se vincula con el propósito organizacional más amplio y su mayor aspiración profesional. 
\item Sea claro sobre cuál es su contribución a la organización, no en términos de áreas de resultados clave, objetivos o entregables, sino en términos de desarrollo profesional, crecimiento profesional y contribución a la organización.
\item Hay 3 valores importantes que los gerentes tienen que demostrar. No importa si usted es un gerente de "etapa temprana" o "experimentado". El primer aspecto y el más importante es "hacer las cosas a través de las personas" . El segundo aspecto es "disfrutar el éxito de sus informes directos" y el tercero es "demostrar integridad visible".
\item Para invertir continuamente en uno mismo, use la regla 70-20-10 de desarrollo profesional y personal.
\item Los cinco contextos que proporcionan espacios para el desarrollo profesional y personal son: puesta en marcha, asignación no estructurada, oportunidad de respuesta , aumento del alcance del trabajo y ser parte de equipos y grupos de trabajo multifuncionales .
\end{itemize}
\end{document}